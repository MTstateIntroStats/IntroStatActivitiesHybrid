% Options for packages loaded elsewhere
\PassOptionsToPackage{unicode}{hyperref}
\PassOptionsToPackage{hyphens}{url}
%
\documentclass[
]{report}
\usepackage{amsmath,amssymb}
\usepackage{iftex}
\ifPDFTeX
  \usepackage[T1]{fontenc}
  \usepackage[utf8]{inputenc}
  \usepackage{textcomp} % provide euro and other symbols
\else % if luatex or xetex
  \usepackage{unicode-math} % this also loads fontspec
  \defaultfontfeatures{Scale=MatchLowercase}
  \defaultfontfeatures[\rmfamily]{Ligatures=TeX,Scale=1}
\fi
\usepackage{lmodern}
\ifPDFTeX\else
  % xetex/luatex font selection
\fi
% Use upquote if available, for straight quotes in verbatim environments
\IfFileExists{upquote.sty}{\usepackage{upquote}}{}
\IfFileExists{microtype.sty}{% use microtype if available
  \usepackage[]{microtype}
  \UseMicrotypeSet[protrusion]{basicmath} % disable protrusion for tt fonts
}{}
\makeatletter
\@ifundefined{KOMAClassName}{% if non-KOMA class
  \IfFileExists{parskip.sty}{%
    \usepackage{parskip}
  }{% else
    \setlength{\parindent}{0pt}
    \setlength{\parskip}{6pt plus 2pt minus 1pt}}
}{% if KOMA class
  \KOMAoptions{parskip=half}}
\makeatother
\usepackage{xcolor}
\usepackage{color}
\usepackage{fancyvrb}
\newcommand{\VerbBar}{|}
\newcommand{\VERB}{\Verb[commandchars=\\\{\}]}
\DefineVerbatimEnvironment{Highlighting}{Verbatim}{commandchars=\\\{\}}
% Add ',fontsize=\small' for more characters per line
\usepackage{framed}
\definecolor{shadecolor}{RGB}{248,248,248}
\newenvironment{Shaded}{\begin{snugshade}}{\end{snugshade}}
\newcommand{\AlertTok}[1]{\textcolor[rgb]{0.94,0.16,0.16}{#1}}
\newcommand{\AnnotationTok}[1]{\textcolor[rgb]{0.56,0.35,0.01}{\textbf{\textit{#1}}}}
\newcommand{\AttributeTok}[1]{\textcolor[rgb]{0.13,0.29,0.53}{#1}}
\newcommand{\BaseNTok}[1]{\textcolor[rgb]{0.00,0.00,0.81}{#1}}
\newcommand{\BuiltInTok}[1]{#1}
\newcommand{\CharTok}[1]{\textcolor[rgb]{0.31,0.60,0.02}{#1}}
\newcommand{\CommentTok}[1]{\textcolor[rgb]{0.56,0.35,0.01}{\textit{#1}}}
\newcommand{\CommentVarTok}[1]{\textcolor[rgb]{0.56,0.35,0.01}{\textbf{\textit{#1}}}}
\newcommand{\ConstantTok}[1]{\textcolor[rgb]{0.56,0.35,0.01}{#1}}
\newcommand{\ControlFlowTok}[1]{\textcolor[rgb]{0.13,0.29,0.53}{\textbf{#1}}}
\newcommand{\DataTypeTok}[1]{\textcolor[rgb]{0.13,0.29,0.53}{#1}}
\newcommand{\DecValTok}[1]{\textcolor[rgb]{0.00,0.00,0.81}{#1}}
\newcommand{\DocumentationTok}[1]{\textcolor[rgb]{0.56,0.35,0.01}{\textbf{\textit{#1}}}}
\newcommand{\ErrorTok}[1]{\textcolor[rgb]{0.64,0.00,0.00}{\textbf{#1}}}
\newcommand{\ExtensionTok}[1]{#1}
\newcommand{\FloatTok}[1]{\textcolor[rgb]{0.00,0.00,0.81}{#1}}
\newcommand{\FunctionTok}[1]{\textcolor[rgb]{0.13,0.29,0.53}{\textbf{#1}}}
\newcommand{\ImportTok}[1]{#1}
\newcommand{\InformationTok}[1]{\textcolor[rgb]{0.56,0.35,0.01}{\textbf{\textit{#1}}}}
\newcommand{\KeywordTok}[1]{\textcolor[rgb]{0.13,0.29,0.53}{\textbf{#1}}}
\newcommand{\NormalTok}[1]{#1}
\newcommand{\OperatorTok}[1]{\textcolor[rgb]{0.81,0.36,0.00}{\textbf{#1}}}
\newcommand{\OtherTok}[1]{\textcolor[rgb]{0.56,0.35,0.01}{#1}}
\newcommand{\PreprocessorTok}[1]{\textcolor[rgb]{0.56,0.35,0.01}{\textit{#1}}}
\newcommand{\RegionMarkerTok}[1]{#1}
\newcommand{\SpecialCharTok}[1]{\textcolor[rgb]{0.81,0.36,0.00}{\textbf{#1}}}
\newcommand{\SpecialStringTok}[1]{\textcolor[rgb]{0.31,0.60,0.02}{#1}}
\newcommand{\StringTok}[1]{\textcolor[rgb]{0.31,0.60,0.02}{#1}}
\newcommand{\VariableTok}[1]{\textcolor[rgb]{0.00,0.00,0.00}{#1}}
\newcommand{\VerbatimStringTok}[1]{\textcolor[rgb]{0.31,0.60,0.02}{#1}}
\newcommand{\WarningTok}[1]{\textcolor[rgb]{0.56,0.35,0.01}{\textbf{\textit{#1}}}}
\usepackage{longtable,booktabs,array}
\usepackage{calc} % for calculating minipage widths
% Correct order of tables after \paragraph or \subparagraph
\usepackage{etoolbox}
\makeatletter
\patchcmd\longtable{\par}{\if@noskipsec\mbox{}\fi\par}{}{}
\makeatother
% Allow footnotes in longtable head/foot
\IfFileExists{footnotehyper.sty}{\usepackage{footnotehyper}}{\usepackage{footnote}}
\makesavenoteenv{longtable}
\usepackage{graphicx}
\makeatletter
\def\maxwidth{\ifdim\Gin@nat@width>\linewidth\linewidth\else\Gin@nat@width\fi}
\def\maxheight{\ifdim\Gin@nat@height>\textheight\textheight\else\Gin@nat@height\fi}
\makeatother
% Scale images if necessary, so that they will not overflow the page
% margins by default, and it is still possible to overwrite the defaults
% using explicit options in \includegraphics[width, height, ...]{}
\setkeys{Gin}{width=\maxwidth,height=\maxheight,keepaspectratio}
% Set default figure placement to htbp
\makeatletter
\def\fps@figure{htbp}
\makeatother
\setlength{\emergencystretch}{3em} % prevent overfull lines
\providecommand{\tightlist}{%
  \setlength{\itemsep}{0pt}\setlength{\parskip}{0pt}}
\setcounter{secnumdepth}{5}
% definitions for citeproc citations
\NewDocumentCommand\citeproctext{}{}
\NewDocumentCommand\citeproc{mm}{%
  \begingroup\def\citeproctext{#2}\cite{#1}\endgroup}
\makeatletter
 % allow citations to break across lines
 \let\@cite@ofmt\@firstofone
 % avoid brackets around text for \cite:
 \def\@biblabel#1{}
 \def\@cite#1#2{{#1\if@tempswa , #2\fi}}
\makeatother
\newlength{\cslhangindent}
\setlength{\cslhangindent}{1.5em}
\newlength{\csllabelwidth}
\setlength{\csllabelwidth}{3em}
\newenvironment{CSLReferences}[2] % #1 hanging-indent, #2 entry-spacing
 {\begin{list}{}{%
  \setlength{\itemindent}{0pt}
  \setlength{\leftmargin}{0pt}
  \setlength{\parsep}{0pt}
  % turn on hanging indent if param 1 is 1
  \ifodd #1
   \setlength{\leftmargin}{\cslhangindent}
   \setlength{\itemindent}{-1\cslhangindent}
  \fi
  % set entry spacing
  \setlength{\itemsep}{#2\baselineskip}}}
 {\end{list}}
\usepackage{calc}
\newcommand{\CSLBlock}[1]{\hfill\break\parbox[t]{\linewidth}{\strut\ignorespaces#1\strut}}
\newcommand{\CSLLeftMargin}[1]{\parbox[t]{\csllabelwidth}{\strut#1\strut}}
\newcommand{\CSLRightInline}[1]{\parbox[t]{\linewidth - \csllabelwidth}{\strut#1\strut}}
\newcommand{\CSLIndent}[1]{\hspace{\cslhangindent}#1}
\usepackage{booktabs}
\usepackage{geometry}
\usepackage[none]{hyphenat}
\usepackage{titlesec}
\usepackage{longtable}
\usepackage{xcolor}
\usepackage{setspace}
\usepackage{pdfpages}

\pagestyle{plain}

%%%% Set margins
\setlength{\topmargin}{-1cm}
\addtolength{\evensidemargin}{-1cm}
\addtolength{\oddsidemargin}{-1cm}
\addtolength{\textheight}{3cm}
\addtolength{\textwidth}{2cm}

% Spacing for reading guides
\newcommand{\rgs}{\vspace{12pt}} % Vertical space
\newcommand{\rgi}{\hspace{24pt}}  % Indent

\newcommand\latexcode[1]{#1}

% Format chapter titles and spacing
\renewcommand*{\chaptername}{Module}

\titleformat{\chapter}[display]
{\bfseries\Large}
{\filleft\MakeUppercase{\chaptertitlename} \Huge\thechapter}
{3ex}
{\titlerule
\vspace{1.5ex}%
\filright}
[\vspace{1.5ex}%
\titlerule]
\titlespacing*{\chapter}{0pt}{-40pt}{20pt}
\ifLuaTeX
  \usepackage{selnolig}  % disable illegal ligatures
\fi
\usepackage{bookmark}
\IfFileExists{xurl.sty}{\usepackage{xurl}}{} % add URL line breaks if available
\urlstyle{same}
\hypersetup{
  hidelinks,
  pdfcreator={LaTeX via pandoc}}

\title{\textbf{STAT 216 Coursepack}\\
\strut \\
\includegraphics[width=5in,height=\textheight]{images/msu-campus.jpg}}
\usepackage{etoolbox}
\makeatletter
\providecommand{\subtitle}[1]{% add subtitle to \maketitle
  \apptocmd{\@title}{\par {\large #1 \par}}{}{}
}
\makeatother
\subtitle{Spring 2025\\
Montana State University}
\author{Melinda Yager\\
Jade Schmidt\\
Stacey Hancock}
\date{}

\begin{document}
\maketitle

\newpage
\thispagestyle{empty}

This resource was developed by Melinda Yager, Jade Schmidt, and Stacey Hancock in 2021 to accompany the online textbook: Hancock, S., Carnegie, N., Meyer, E., Schmidt, J., and Yager, M. (2021). \emph{Montana State Introductory Statistics with R}. Montana State University. \url{https://mtstateintrostats.github.io/IntroStatTextbook/}.

This resource is released under a \href{https://creativecommons.org/licenses/by-nc-sa/4.0/}{Creative Commons BY-NC-SA 4.0} license unless otherwise noted.

\setcounter{tocdepth}{1}
\addtocontents{toc}{\protect\thispagestyle{empty}}
\tableofcontents
\thispagestyle{empty}

\newpage
\setcounter{page}{1}

\chapter*{Preface}\label{preface}
\addcontentsline{toc}{chapter}{Preface}

This coursepack accompanies the textbook for STAT 216: Montana State Introductory Statistics with R, which can be found at \url{https://mtstateintrostats.github.io/IntroStatTextbook/}. The syllabus for the course (including the course calendar), data sets, and links to D2L Brightspace, Gradescope, and the MSU RStudio server can be found on the course webpage: \url{https://math.montana.edu/courses/s216/}.
Other notes and review materials are linked in D2L.

Each of the activities in this workbook is designed to target specific learning outcomes of the course, giving you practice with important statistical concepts in a group setting with instructor guidance. In addition to the in-class activities for the course, video notes are provided to aid in taking notes while you complete the required videos. Bring this workbook with you to class each class period, and take notes in the workbook as you would your own notes. A well-written completed workbook will provide an optimal study guide for exams!

All activities and labs in this coursepack will be completed during class time. Parts of each lab will be turned in on Gradescope. To aid in your understanding, read through the introduction for each activity before attending class each day.

STAT 216 is a 3-credit in-person course. In our experience, it takes six to nine hours per week outside of class to achieve a good grade in this class. By ``good'' we mean at least a C because a grade of D or below does not count toward fulfilling degree requirements. Many of you set your goals higher than just getting a C, and we fully support that. You need roughly nine hours per week to review past activities, read feedback on previous assignments, complete current assignments, and prepare for the next day's class. A typical week in the life of a STAT 216 student looks like:

\begin{itemize}
\tightlist
\item
  \emph{Prior to class meeting}:

  \begin{itemize}
  \tightlist
  \item
    Read assigned sections of the textbook, using the provided reading guides to take notes on the material.
  \item
    Watch the provided videos, taking notes in the coursepack.
  \item
    Read through the introduction to the day's in-class activity.
  \item
    Read through the week's homework assignment and note any questions you may have on the content.
  \end{itemize}
\item
  \emph{During class meeting}:

  \begin{itemize}
  \tightlist
  \item
    Work through the guided activity, in-class activity or weekly lab with your classmates and instructor, taking detailed notes on your answers to each question in the activity.
  \end{itemize}
\item
  \emph{After class meeting}:

  \begin{itemize}
  \tightlist
  \item
    Complete any parts of the activity you did not complete in class.
  \item
    Review the activity solutions in the Math and Stat Center, and take notes on key points.
  \item
    Complete any remaining assigned readings for the week.
  \item
    Complete the week's homework assignment.
  \end{itemize}
\end{itemize}

\nocite{*}

\chapter{Inference for a Single Quantitative Variable}\label{inference-for-a-single-quantitative-variable}

\section{Vocabulary Review and Key Topics}\label{vocabulary-review-and-key-topics}

Review the Golden Ticket posted in the resources at the end of the coursepack for a summary of a single quantitative variable. Module 6 will cover hypothesis testing using both simulation and theory-based methods.

\begin{itemize}
\item
  The \textbf{summary measure} for one quantitative variable is the \textbf{mean}
\item
  Additionally, we can find the five number summary (min, Q1, median, Q3, max) as well as the sample standard deviation
\item
  R code to find the summary statistics for a quantitative variable

\begin{Shaded}
\begin{Highlighting}[]
\NormalTok{object }\SpecialCharTok{\%\textgreater{}\%} \CommentTok{\# Data set piped into...}
    \FunctionTok{summarise}\NormalTok{(}\FunctionTok{favstats}\NormalTok{(variable))}
\end{Highlighting}
\end{Shaded}
\item
  Quartile 1, \(Q_1\): value at the 25th percentile; approximately 25\% of data values are at and below the value of \(Q_1\)
\item
  Quartile 2, \(Q_3\): value at the 75th percentile; approximately 75\% of data values are at and below the value of \(Q_3\)
\item
  Sample standard deviation, \(s\): on average, each value in the data set is s units from the mean of the data set
\item
  \textbf{Interquartile range}: \(IQR = Q_3-Q_1\)
\end{itemize}

Types of plot for one quantitative variables

\begin{itemize}
\item
  \textbf{Histogram}: sorts a quantitative variable into bins of a certain width
\item
  R code to create a histogram
\end{itemize}

\begin{Shaded}
\begin{Highlighting}[]
\NormalTok{    object }\SpecialCharTok{\%\textgreater{}\%} \CommentTok{\# Data set piped into...}
        \FunctionTok{ggplot}\NormalTok{(}\FunctionTok{aes}\NormalTok{(}\AttributeTok{x =}\NormalTok{ variable)) }\SpecialCharTok{+}   \CommentTok{\# Name variable to plot}
        \FunctionTok{geom\_histogram}\NormalTok{(}\AttributeTok{binwidth =} \DecValTok{10}\NormalTok{) }\SpecialCharTok{+}  \CommentTok{\# Create histogram with specified binwidth}
        \FunctionTok{labs}\NormalTok{(}\AttributeTok{title =} \StringTok{"Don\textquotesingle{}t forget to title the plot!"}\NormalTok{, }\CommentTok{\# Title for plot}
            \AttributeTok{x =} \StringTok{"x{-}axis label"}\NormalTok{, }\CommentTok{\# Label for x axis}
            \AttributeTok{y =} \StringTok{"y{-}axis label"}\NormalTok{) }\CommentTok{\# Label for y axis}
\end{Highlighting}
\end{Shaded}

\begin{itemize}
\item
  \textbf{Boxplot}: plots the values of the five number summary and shows any outliers in the data set
\item
  R code to create a boxplot
\end{itemize}

\begin{Shaded}
\begin{Highlighting}[]
\NormalTok{    object }\SpecialCharTok{\%\textgreater{}\%} \CommentTok{\# Data set piped into...}
        \FunctionTok{ggplot}\NormalTok{(}\FunctionTok{aes}\NormalTok{(}\AttributeTok{x =}\NormalTok{ variable)) }\SpecialCharTok{+} \CommentTok{\# Name variable to plot}
        \FunctionTok{geom\_boxplot}\NormalTok{() }\SpecialCharTok{+} \CommentTok{\# Create boxplot }
        \FunctionTok{labs}\NormalTok{(}\AttributeTok{title =} \StringTok{"Don\textquotesingle{}t forget to title the plot!"}\NormalTok{, }\CommentTok{\# Title for plot}
            \AttributeTok{x =} \StringTok{"x{-}axis label"}\NormalTok{, }\CommentTok{\# Label for x axis}
            \AttributeTok{y =} \StringTok{"y{-}axis label"}\NormalTok{) }\CommentTok{\# Label for y axis}
\end{Highlighting}
\end{Shaded}

\begin{itemize}
\item
  \textbf{Dotplot}: plots each value as a dot along the x-axis
\item
  Four characteristics of boxplots

  \begin{itemize}
  \item
    Shape (symmetric or skewed)
  \item
    Center
  \item
    Spread
  \item
    Outliers?
  \end{itemize}
\end{itemize}

\subsection*{Simulation Hypothesis Testing}\label{simulation-hypothesis-testing}
\addcontentsline{toc}{subsection}{Simulation Hypothesis Testing}

Hypotheses for a single quantitative variable:

\[H_0: \mu = \mu_0\]
\[H_A: \mu\left\{
\begin{array}{ll}
< \\
\ne \\
< \\
\end{array}
\right\}
\mu_0 \]

\begin{itemize}
\item
  R code to use for \textbf{simulation methods} for one quantitative variable to find the p-value, one\_mean\_test, is shown below. Review the comments (instructions after the \#) to see what each should be entered for each line of code.

\begin{Shaded}
\begin{Highlighting}[]
\FunctionTok{one\_mean\_test}\NormalTok{(object}\SpecialCharTok{$}\NormalTok{variable,}\CommentTok{\#Enter the object name and variable}
          \AttributeTok{null\_value =}\NormalTok{ xx, }\CommentTok{\#Enter the null value for the study}
          \AttributeTok{summary\_measure =} \StringTok{"mean"}\NormalTok{,  }\CommentTok{\#Can choose between mean or median}
          \AttributeTok{shift =}\NormalTok{ xx, }\CommentTok{\#Difference between the null value and the sample mean}
          \AttributeTok{as\_extreme\_as =}\NormalTok{ xx, }\CommentTok{\#Value of the summary statistic}
          \AttributeTok{direction =} \StringTok{"xx"}\NormalTok{, }\CommentTok{\#Specify direction of alternative hypothesis}
          \AttributeTok{number\_repetitions =} \DecValTok{10000}\NormalTok{)}
\end{Highlighting}
\end{Shaded}
\end{itemize}

\subsection*{Theory-based Hypothesis Testing}\label{theory-based-hypothesis-testing}
\addcontentsline{toc}{subsection}{Theory-based Hypothesis Testing}

\begin{itemize}
\item
  \textbf{Theory-based methods}: when specific conditions are met, a data can be fit with a theoretical distribution
\item
  \textbf{Conditions for the sampling distribution of \(\bar{x}\) to follow an approximate normal distribution}:

  \begin{itemize}
  \item
    \textbf{Independence}: The sample's observations are independent, e.g., are from a simple random sample. (\emph{Remember}: This also must be true to use simulation methods!)
  \item
    \textbf{Large enough sample size: Normality Condition}: The sample observations come from a normally distributed population. To check use the the following rules of thumb:

    \begin{itemize}
    \item
      \(n < 30\): The distribution of the sample must be approximately normal with no outliers
    \item
      \(30 \ge n < 100\): We can relax the condition a little; the distribution of the sample must have no extreme outliers or skewness
    \item
      \(n > 100\): Can assume the sampling distribution of \(\bar{x}\) is nearly normal, even if the underlying distribuion of individual observationals is not
    \end{itemize}
  \end{itemize}
\item
  \textbf{t-distribution}: a theoretical distribution that is symmetric with a given degrees of freedom (\(n-1\))

  \begin{itemize}
  \tightlist
  \item
    \(t_{n-1}\)
  \end{itemize}
\item
  \textbf{Standardized sample mean}: standardized statistic for a single quantitative variable calculated using:
\end{itemize}

\[
T = \frac{\bar{x} - \mu_0}{SE(\bar{x})},
\]

\begin{itemize}
\tightlist
\item
  \textbf{Standard error of the sample mean assuming the null is true}: measures the how far each possible sample mean is from the true mean, on average and is calculated using the formula below:
\end{itemize}

\[SE(\bar{x})=\frac{s}{\sqrt{n}}\]

\begin{itemize}
\item
  The following R code is used to find the p-value using theory based methods for a single quantitative variables.

  \begin{itemize}
  \item
    pt will give you a p-value using the t-distribution with n-1 df (enter for yy)
  \item
    Enter the value of the standardized statistic for xx
  \item
    If a greater than alternative, change lower.tail = TRUE to FALSE.
  \item
    If a two-sided test, multiply by 2.
  \end{itemize}

\begin{Shaded}
\begin{Highlighting}[]
\FunctionTok{pt}\NormalTok{(xx, }\AttributeTok{df =}\NormalTok{ yy, }\AttributeTok{lower.tail=}\ConstantTok{TRUE}\NormalTok{)}
\end{Highlighting}
\end{Shaded}
\end{itemize}

\subsubsection*{Exploratory data analysis}\label{exploratory-data-analysis}
\addcontentsline{toc}{subsubsection}{Exploratory data analysis}

At the end of this module, you should understand how to calculate a summary statistic and plot a single quantitative variable.

\begin{itemize}
\item
  Notation for a sample mean: \(\bar{x}\)
\item
  Notation for a population mean: \(\mu\)
\item
  Types of plots for a single categorical variable:

  \begin{itemize}
  \item
    Histogram
  \item
    Boxplot
  \item
    Dotplot
  \end{itemize}
\end{itemize}

\newpage

\section{Video Notes: Exploratory Data Analysis of Quantitative Variables}\label{video-notes-exploratory-data-analysis-of-quantitative-variables}

Read Chapters 5 and 17 in the course textbook. Use the following videos to complete the video notes for Module 6.

\subsection{Course Videos}\label{course-videos}

\begin{itemize}
\tightlist
\item
  QuantitativeData
\end{itemize}

*5.2to5.4

\begin{itemize}
\item
  5.5to5.6
\item
  5.7
\end{itemize}

\begin{center}\rule{0.5\linewidth}{0.5pt}\end{center}

\setstretch{1}

\subsection*{Summarizing quantitative data - Videos 5.2to5.4 and 5.5to5.6}\label{summarizing-quantitative-data---videos-5.2to5.4-and-5.5to5.6}
\addcontentsline{toc}{subsection}{Summarizing quantitative data - Videos 5.2to5.4 and 5.5to5.6}

\subsubsection*{Types of plots}\label{types-of-plots}
\addcontentsline{toc}{subsubsection}{Types of plots}

We will revisit the moving to Montana data set and plot the age of the buyers.

Dotplot:

\vspace{0.5in}

\begin{Shaded}
\begin{Highlighting}[]
\NormalTok{moving }\SpecialCharTok{\%\textgreater{}\%}
  \FunctionTok{ggplot}\NormalTok{(}\FunctionTok{aes}\NormalTok{(}\AttributeTok{x =}\NormalTok{ Age))}\SpecialCharTok{+} \CommentTok{\#Enter variable to plot}
  \FunctionTok{geom\_dotplot}\NormalTok{() }\SpecialCharTok{+} 
  \FunctionTok{labs}\NormalTok{(}\AttributeTok{title =} \StringTok{"Dotplot of Age of Buyers from Gallatin }
\StringTok{       County Home Sales"}\NormalTok{, }\CommentTok{\#Title your plot}
       \AttributeTok{x =} \StringTok{"Age"}\NormalTok{, }\CommentTok{\#x{-}axis label}
       \AttributeTok{y =} \StringTok{"Proportion"}\NormalTok{) }\CommentTok{\#y{-}axis label}
\end{Highlighting}
\end{Shaded}

\begin{center}\includegraphics[width=0.75\linewidth]{06-VN06-EDAonemeanSim_files/figure-latex/unnamed-chunk-2-1} \end{center}

\newpage

Histogram:

\vspace{0.2in}

\begin{Shaded}
\begin{Highlighting}[]
\NormalTok{moving }\SpecialCharTok{\%\textgreater{}\%}
  \FunctionTok{ggplot}\NormalTok{(}\FunctionTok{aes}\NormalTok{(}\AttributeTok{x =}\NormalTok{ Age))}\SpecialCharTok{+}
  \FunctionTok{geom\_histogram}\NormalTok{(}\AttributeTok{binwidth =} \DecValTok{7}\NormalTok{) }\SpecialCharTok{+} 
  \FunctionTok{labs}\NormalTok{(}\AttributeTok{title =} \StringTok{"Histogram of Age of Buyers from Gallatin }
\StringTok{       County Home Sales"}\NormalTok{,}
       \CommentTok{\#Title your plot}
       \AttributeTok{x =} \StringTok{"Age"}\NormalTok{,}
       \AttributeTok{y =} \StringTok{"Count"}\NormalTok{)}
\end{Highlighting}
\end{Shaded}

\begin{center}\includegraphics[width=0.7\linewidth]{06-VN06-EDAonemeanSim_files/figure-latex/unnamed-chunk-3-1} \end{center}

\setstretch{1.5}

Quantitative data can be numerically summarized by finding:

Two measures of center:

\begin{itemize}
\item
  Mean: \_\_\_\_\_\_\_\_\_\_\_\_ of all the \_\_\_\_\_\_\_\_\_\_\_\_\_ in the data set.

  \begin{itemize}
  \tightlist
  \item
    Sum the values in the data set and divide
    the sum by the sample size
  \end{itemize}
\item
  Notation used for the population mean:

  \begin{itemize}
  \tightlist
  \item
    Single quantitative variable:
  \end{itemize}
\end{itemize}

\vspace{0.1in}

\rgi \rgi - One categorical and one quantitative variable:

\vspace{0.1in}

\rgi \rgi \rgi - Subscripts represent the \_\_\_\_\_\_\_\_\_\_\_ variable groups

\begin{itemize}
\tightlist
\item
  Notation used for the sample mean:
\end{itemize}

\rgi \rgi - Single quantitative variable:

\vspace{0.1in}

\rgi \rgi - One categorical and one quantitative variable:

\vspace{0.1in}

\begin{itemize}
\item
  Median: Value at the \_\_\_\_\_\_\_\_\_\_\_\_\_ percentile

  \begin{itemize}
  \item
    \_\_\_\_\_\_\_\_\_\_ \% of values are at and \_\_\_\_\_\_\_\_\_\_\_ and at and \_\_\_\_\_\_\_\_\_\_\_ the value of the \_\_\_\_\_\_\_\_\_\_\_\_\_\_.
  \item
    Middle value in a list of ordered values
  \end{itemize}
\end{itemize}

Two measures of spread:

\begin{itemize}
\tightlist
\item
  Standard deviation: Average \_\_\_\_\_\_\_\_\_\_\_\_\_\_\_\_\_\_\_ each data point is from the \_\_\_\_\_\_\_\_\_\_\_\_\_\_ of the data set.
\end{itemize}

\vspace{1mm}

\rgi \rgi - Notation used for the population standard deviation

\vspace{0.2in}

\rgi \rgi - Notation used for the sample standard deviation

\vspace{0.2in}

\begin{itemize}
\tightlist
\item
  Interquartile range: middle 50\% of data values
\end{itemize}

\rgi Formula:

\rgi \rgi Quartile 3 (Q3) - value at the 75th percentile

\rgi \rgi - \_\_\_\_\_\_\_\_\_\_\_\_ \% of values are at and \_\_\_\_\_\_\_\_\_\_\_\_\_ the value of Q3

\rgi \rgi Quartile 1 (Q1) - value at the 25th percentile

\rgi \rgi - \_\_\_\_\_\_\_\_\_\_\_\_\_ \% of values are at and \_\_\_\_\_\_\_\_\_\_\_\_\_ the value of Q1

\vspace{1mm}

\setstretch{1}

\newpage

Boxplot (3rd type of plot for quantitative variables)

\begin{verbatim}
- Five number summary: minimum, Q1, median, Q3, maximum
\end{verbatim}

\vspace{0.3in}

\begin{Shaded}
\begin{Highlighting}[]
\NormalTok{moving }\SpecialCharTok{\%\textgreater{}\%}
  \FunctionTok{ggplot}\NormalTok{(}\FunctionTok{aes}\NormalTok{(}\AttributeTok{x =}\NormalTok{ Age))}\SpecialCharTok{+} \CommentTok{\#Enter variable to plot}
  \FunctionTok{geom\_boxplot}\NormalTok{() }\SpecialCharTok{+} 
  \FunctionTok{labs}\NormalTok{(}\AttributeTok{title =} \StringTok{"Boxplot of Age of Buyers from Gallatin }
\StringTok{       County Home Sales"}\NormalTok{, }\CommentTok{\#Title your plot}
       \AttributeTok{x =} \StringTok{"Age"}\NormalTok{, }\CommentTok{\#x{-}axis label}
       \AttributeTok{y =} \StringTok{""}\NormalTok{) }\CommentTok{\#y{-}axis label}
\end{Highlighting}
\end{Shaded}

\begin{center}\includegraphics[width=0.7\linewidth]{06-VN06-EDAonemeanSim_files/figure-latex/unnamed-chunk-4-1} \end{center}

\begin{Shaded}
\begin{Highlighting}[]
\FunctionTok{favstats}\NormalTok{(moving}\SpecialCharTok{$}\NormalTok{Age)}
\end{Highlighting}
\end{Shaded}

\begin{verbatim}
#>  min Q1 median    Q3 max  mean       sd   n missing
#>   20 28     36 49.25  77 39.77 14.35471 100       0
\end{verbatim}

Interpret the value of \(Q_3\) for the age of buyers.

\vspace{0.5in}

Interpret the value of s for the age of buyers.

\vspace{0.5in}

\newpage

\subsubsection*{Four characteristics of plots for quantitative variables}\label{four-characteristics-of-plots-for-quantitative-variables}
\addcontentsline{toc}{subsubsection}{Four characteristics of plots for quantitative variables}

\begin{itemize}
\tightlist
\item
  Shape: overall pattern of the data
\end{itemize}

\begin{center}\includegraphics[width=0.8\linewidth]{images/shape} \end{center}

\rgi \rgi - What is the shape of the distribution of age of buyers for Gallatin County home sales?

\vspace{0.3in}

\begin{itemize}
\tightlist
\item
  Center:
\end{itemize}

\rgi Mean or Median

\rgi \rgi - Report the measure of center for the boxplot of age of buyers for Gallatin County home sales.

\vspace{0.3in}

\begin{itemize}
\tightlist
\item
  Spread (or variability):
\end{itemize}

\rgi Standard deviation or IQR

\rgi \rgi - Report the IQR for the distribution of age of buyers from Gallatin County home sales.

\vspace{0.3in}

\begin{itemize}
\tightlist
\item
  Outliers?
\end{itemize}

\rgi values \textless{} \(Q_1 - 1.5 \times IQR\)

\rgi values \textgreater{} \(Q_3 + 1.5 \times IQR\)

\rgi \rgi - Use these formulas to show that there are no outliers in the distribution of age of buyers from Gallatin County home sales.

\vspace{0.8in}
\newpage

Let's look at side-by-side boxplot of the variable age by state of origin moved from.

\begin{Shaded}
\begin{Highlighting}[]
\NormalTok{moving }\SpecialCharTok{\%\textgreater{}\%}  \CommentTok{\# Data set piped into...}
  \FunctionTok{ggplot}\NormalTok{(}\FunctionTok{aes}\NormalTok{(}\AttributeTok{y =}\NormalTok{ Age, }\AttributeTok{x =}\NormalTok{ From))}\SpecialCharTok{+}  \CommentTok{\# Identify variables}
  \FunctionTok{geom\_boxplot}\NormalTok{()}\SpecialCharTok{+}  \CommentTok{\# Tell it to make a box plot}
  \FunctionTok{labs}\NormalTok{(}\AttributeTok{title =} \StringTok{"Side by side box plot of Age by State of Origin }
\StringTok{  of Buyers from Gallatin County Home Sales"}\NormalTok{,  }\CommentTok{\# Title}
       \AttributeTok{x =} \StringTok{"State of Origin"}\NormalTok{,    }\CommentTok{\# x{-}axis label}
       \AttributeTok{y =} \StringTok{"Age"}\NormalTok{)  }\CommentTok{\# y{-}axis label}
\end{Highlighting}
\end{Shaded}

\begin{center}\includegraphics[width=0.85\linewidth]{06-VN06-EDAonemeanSim_files/figure-latex/unnamed-chunk-7-1} \end{center}

\begin{itemize}
\tightlist
\item
  Which state of origin had the oldest median age of buyers from sampled home sales?
\end{itemize}

\vspace{0.4in}

\begin{itemize}
\tightlist
\item
  Which state of origin had the most variability in age of buyers from sampled home sales?
\end{itemize}

\vspace{0.4in}

\begin{itemize}
\tightlist
\item
  Which state of origin had the most symmetric distribution of ages of buyers from sampled home sales?
\end{itemize}

\vspace{0.4in}

\begin{itemize}
\tightlist
\item
  Which state of origin had outliers for the age of buyers from sampled home sales?
\end{itemize}

\vspace{0.4in}

\newpage

\subsubsection*{Robust statistics - Video 5.7}\label{robust-statistics---video-5.7}
\addcontentsline{toc}{subsubsection}{Robust statistics - Video 5.7}

Let's review the summary statistics and histogram of age of buyers from sampled home sales.

\begin{center}\includegraphics[width=0.85\linewidth]{06-VN06-EDAonemeanSim_files/figure-latex/unnamed-chunk-8-1} \end{center}

\begin{verbatim}
#>  min Q1 median    Q3 max  mean       sd   n missing
#>   20 28     36 49.25  77 39.77 14.35471 100       0
\end{verbatim}

\setstretch{1.5}

Notice that the \_\_\_\_\_\_\_\_\_\_\_\_\_ has been pulled in the direction of the \_\_\_\_\_\_\_\_\_\_\_\_\_\_\_.

\begin{itemize}
\item
  The \_\_\_\_\_\_\_\_\_\_\_ is a robust measure of center.
\item
  The \_\_\_\_\_\_\_\_\_\_\_ is a robust measure of spread.
\item
  Robust means not \_\_\_\_\_\_\_\_\_\_\_\_\_\_\_\_\_ by outliers.
\end{itemize}

When the distribution is symmetric use the \_\_\_\_\_\_\_\_\_\_\_\_ as the measure of center and the \_\_\_\_\_\_\_\_\_\_\_ as the measure of spread.

When the distribution is skewed with outliers use the \_\_\_\_\_\_\_\_\_\_\_\_\_ as the measure of center and the \_\_\_\_\_\_\_\_\_\_\_\_ as the measure of spread.

\setstretch{1}

\newpage

\subsection{Video notes single quantitative variable inference}\label{video-notes-single-quantitative-variable-inference}

\setstretch{1}

Example: What is the average weight of adult male polar bears? The weight was measured on a representative sample of 83 male polar bears from the Southern Beaufort Sea.

\begin{Shaded}
\begin{Highlighting}[]
\NormalTok{pb }\OtherTok{\textless{}{-}} \FunctionTok{read.csv}\NormalTok{(}\StringTok{"https://math.montana.edu/courses/s216/data/polarbear.csv"}\NormalTok{)}
\end{Highlighting}
\end{Shaded}

Plots of the data:

\begin{Shaded}
\begin{Highlighting}[]
\NormalTok{pb }\SpecialCharTok{\%\textgreater{}\%}
    \FunctionTok{ggplot}\NormalTok{(}\FunctionTok{aes}\NormalTok{(}\AttributeTok{x =}\NormalTok{ Weight)) }\SpecialCharTok{+}   \CommentTok{\# Name variable to plot}
    \FunctionTok{geom\_histogram}\NormalTok{(}\AttributeTok{binwidth =} \DecValTok{10}\NormalTok{) }\SpecialCharTok{+}  \CommentTok{\# Create histogram with specified binwidth}
    \FunctionTok{labs}\NormalTok{(}\AttributeTok{title =} \StringTok{"Histogram of Male Polar Bear Weight"}\NormalTok{, }\CommentTok{\# Title for plot}
       \AttributeTok{x =} \StringTok{"Weight (kg)"}\NormalTok{, }\CommentTok{\# Label for x axis}
       \AttributeTok{y =} \StringTok{"Frequency"}\NormalTok{) }\CommentTok{\# Label for y axis}

\NormalTok{pb }\SpecialCharTok{\%\textgreater{}\%} \CommentTok{\# Data set piped into...}
\FunctionTok{ggplot}\NormalTok{(}\FunctionTok{aes}\NormalTok{(}\AttributeTok{x =}\NormalTok{ Weight)) }\SpecialCharTok{+}   \CommentTok{\# Name variable to plot}
  \FunctionTok{geom\_boxplot}\NormalTok{() }\SpecialCharTok{+}  \CommentTok{\# Create boxplot}
  \FunctionTok{labs}\NormalTok{(}\AttributeTok{title =} \StringTok{"Boxplot of Male Polar Bear Weight"}\NormalTok{, }\CommentTok{\# Title for plot}
       \AttributeTok{x =} \StringTok{"Weight (kg)"}\NormalTok{, }\CommentTok{\# Label for x axis}
       \AttributeTok{y =} \StringTok{"Frequency"}\NormalTok{) }\CommentTok{\# Label for y axis}
\end{Highlighting}
\end{Shaded}

\begin{center}\includegraphics[width=0.6\linewidth]{06-VN06-EDAonemeanSim_files/figure-latex/unnamed-chunk-11-1} \includegraphics[width=0.6\linewidth]{06-VN06-EDAonemeanSim_files/figure-latex/unnamed-chunk-11-2} \end{center}

\newpage

Summary Statistics:

\begin{Shaded}
\begin{Highlighting}[]
\NormalTok{pb }\SpecialCharTok{\%\textgreater{}\%}
  \FunctionTok{summarise}\NormalTok{(}\FunctionTok{favstats}\NormalTok{(Weight)) }\CommentTok{\#Gives the summary statistics}
\CommentTok{\#\textgreater{}     min    Q1 median     Q3   max     mean       sd  n missing}
\CommentTok{\#\textgreater{} 1 104.1 276.3  339.4 382.45 543.6 324.5988 88.32615 83       0}
\end{Highlighting}
\end{Shaded}

\subsection*{Hypothesis testing}\label{hypothesis-testing}
\addcontentsline{toc}{subsection}{Hypothesis testing}

\setstretch{1.5}

\begin{itemize}
\tightlist
\item
  Hypotheses are always written about the \_\_\_\_\_\_\_\_\_\_\_\_\_\_\_\_\_\_\_\_\_\_\_\_\_. For a single mean we will use the notation \_\_\_\_\_\_\_\_\_\_\_.
\end{itemize}

\setstretch{1}

Null Hypothesis:

\(H_0:\)

\vspace{0.2in}

Alternative Hypothesis:

\(H_A:\)

\vspace{0.2in}

\setstretch{1.5}

\begin{itemize}
\tightlist
\item
  Direction of the alternative depends on the \_\_\_\_\_\_\_\_\_\_\_\_\_\_\_\_\_\_
  \_\_\_\_\_\_\_\_\_\_\_\_\_\_\_\_\_\_\_.
\end{itemize}

\setstretch{1}

\subsubsection*{Simulation-based method}\label{simulation-based-method}
\addcontentsline{toc}{subsubsection}{Simulation-based method}

\begin{itemize}
\item
  Simulate many samples assuming \(H_0: \mu = \mu_0\)

  \begin{itemize}
  \item
    Shift the data by the difference between \(\mu_0\) and \(\bar{x}\)
  \item
    Sample with replacement \(n\) times from the shifted data
  \item
    Plot the simulated shifted sample mean from each simulation
  \item
    Repeat 1000 times (simulations) to create the null distribution
  \item
    Find the proportion of simulations at least as extreme as \(\bar{x}\)
  \end{itemize}
\end{itemize}

Example: Is there evidence that male polar bears weigh less than 370kg (previously recorded measure), on average? The weight was measured on a representative sample of 83 male polar bears from the Southern Beaufort Sea.

Hypotheses:

In notation:

\(H_0:\)

\vspace{0.2in}

\(H_A:\)

\vspace{0.2in}

\newpage

In words:

\(H_0:\)

\vspace{0.6in}

\(H_A:\)

\vspace{0.6in}

Reminder of summary statistics:

\begin{Shaded}
\begin{Highlighting}[]
\NormalTok{pb }\SpecialCharTok{\%\textgreater{}\%}
  \FunctionTok{summarise}\NormalTok{(}\FunctionTok{favstats}\NormalTok{(Weight)) }\CommentTok{\#Gives the summary statistics}
\CommentTok{\#\textgreater{}     min    Q1 median     Q3   max     mean       sd  n missing}
\CommentTok{\#\textgreater{} 1 104.1 276.3  339.4 382.45 543.6 324.5988 88.32615 83       0}
\end{Highlighting}
\end{Shaded}

Find the difference:

\(\mu_0 - \bar{x} =\)

\begin{Shaded}
\begin{Highlighting}[]
\FunctionTok{set.seed}\NormalTok{(}\DecValTok{216}\NormalTok{)}
\FunctionTok{one\_mean\_test}\NormalTok{(pb}\SpecialCharTok{$}\NormalTok{Weight,   }\CommentTok{\#Enter the object name and variable}
              \AttributeTok{null\_value =} \DecValTok{370}\NormalTok{, }\CommentTok{\#Enter null value for the study}
              \AttributeTok{summary\_measure =} \StringTok{"mean"}\NormalTok{,  }\CommentTok{\#Can choose between mean or median}
              \AttributeTok{shift =} \FloatTok{45.4}\NormalTok{,   }\CommentTok{\# Shift needed for bootstrap hypothesis test}
              \AttributeTok{as\_extreme\_as =} \FloatTok{324.6}\NormalTok{,  }\CommentTok{\# Observed statistic}
              \AttributeTok{direction =} \StringTok{"less"}\NormalTok{,  }\CommentTok{\# Direction of alternative}
              \AttributeTok{number\_repetitions =} \DecValTok{10000}\NormalTok{)  }\CommentTok{\# Number of simulated samples for null distribution}
\end{Highlighting}
\end{Shaded}

\begin{center}\includegraphics[width=0.6\linewidth]{06-VN06-EDAonemeanSim_files/figure-latex/unnamed-chunk-14-1} \end{center}

\newpage

Interpretation of the p-value:

\begin{itemize}
\item
  Statement about probability or proportion of samples
\item
  Statistic (summary measure and value)
\item
  Direction of the alternative
\item
  Null hypothesis (in context)
\end{itemize}

\vspace{0.8in}

Conclusion:

\begin{itemize}
\item
  Amount of evidence
\item
  Parameter of interest
\item
  Direction of the alternative hypothesis
\end{itemize}

\vspace{0.8in}

\subsubsection*{Theory-based method}\label{theory-based-method}
\addcontentsline{toc}{subsubsection}{Theory-based method}

Conditions for inference using theory-based methods:

\begin{itemize}
\tightlist
\item
  Independence:
\end{itemize}

\vspace{0.2in}

\begin{itemize}
\tightlist
\item
  Large enough sample size:
\end{itemize}

\vspace{0.2in}

\subsection*{T - distribution}\label{t---distribution}
\addcontentsline{toc}{subsection}{T - distribution}

In the theoretical approach, we use the CLT to tell us that the distribution of sample means will be approximately normal, centered at the assumed true mean under \(H_0\) and with standard deviation \(\frac{\sigma}{\sqrt{n}}\).

\[\bar{x} \sim N(\mu_0, \frac{\sigma}{\sqrt{n}})\]
\setstretch{1.5}

\begin{itemize}
\item
  Estimate the population standard deviation, \(\sigma\), with the
  \_\_\_\_\_\_\_\_\_\_\_\_\_\_\_\_\_\_\_\_\_\_\_\_\_\_\_ standard deviation, \_\_\_\_\_\_\_\_.
\item
  For a single quantitative variable we use the \_\_\_\_ - distribution
  with \_\_\_\_\_\_\_\_\_\_\_\_\_\_\_
  degrees of freedom to approximate the sampling distribution.
\end{itemize}

\setstretch{1}

The \(t^*\) multiplier is the value at the given percentile of the t-distribution with \(n - 1\) degrees of freedom.

\begin{center}\includegraphics[width=0.7\linewidth]{06-VN06-EDAonemeanSim_files/figure-latex/tstarpb-1} \end{center}

\begin{itemize}
\item
  Calculate the standardized statistic
\item
  Find the area under the t-distribution with \(n - 1\) df at least as extreme as the standardized statistic
\end{itemize}

Equation for the standard error of the sample mean:

\vspace{0.5in}

Equation for the standardized sample mean:

\vspace{0.5in}

Calculate the standardized sample mean weight of adult male polar bears:

\vspace{0.4in}

\begin{center}\includegraphics[width=0.7\linewidth]{06-VN06-EDAonemeanSim_files/figure-latex/pvaluepb-1} \end{center}

Interpret the standardized sample mean weight:

\vspace{0.8in}

To find the theory-based p-value:

\begin{Shaded}
\begin{Highlighting}[]
\FunctionTok{pt}\NormalTok{(}\SpecialCharTok{{-}}\FloatTok{4.683}\NormalTok{, }\AttributeTok{df=}\DecValTok{82}\NormalTok{, }\AttributeTok{lower.tail=}\ConstantTok{TRUE}\NormalTok{)}
\CommentTok{\#\textgreater{} [1] 5.531605e{-}06}
\end{Highlighting}
\end{Shaded}

\subsection{Concept Check}\label{concept-check}

Be prepared for group discussion in the next class. One member from the table should write the answers to the following on the whiteboard.

\begin{enumerate}
\def\labelenumi{\arabic{enumi}.}
\tightlist
\item
  What plots can be used to summarize quantitative data?
\end{enumerate}

\vspace{0.7in}

\begin{enumerate}
\def\labelenumi{\arabic{enumi}.}
\setcounter{enumi}{1}
\tightlist
\item
  Which measure of center is robust to outliers?
\end{enumerate}

\vspace{0.2in}

\newpage

\section{Activity 11: Summarizing Quantitative Variables}\label{activity-11-summarizing-quantitative-variables}

\setstretch{1}

\subsection{Learning outcomes}\label{learning-outcomes}

\begin{itemize}
\item
  Identify and create appropriate summary statistics and plots given a data set or research question for quantitative data.
\item
  Interpret the following summary statistics in context:
  median, lower quartile, upper quartile,
  standard deviation, interquartile range.
\end{itemize}

\subsection{Terminology review}\label{terminology-review}

In today's activity, we will review summary measures and plots for quantitative variables. Some terms covered in this activity are:

\begin{itemize}
\item
  Two measures of center: mean, median
\item
  Two measures of spread (variability): standard deviation, interquartile range (IQR)
\item
  Plots of quantitative variables: dotplots, boxplots, histograms
\item
  Given a plot or set of plots, describe and compare the distribution(s)
  of quantitative variables
  (center, variability, shape, outliers).
\end{itemize}

To review these concepts, see Chapter 5 in the textbook.

\subsection{The Integrated Postsecondary Education Data System (IPEDS)}\label{the-integrated-postsecondary-education-data-system-ipeds}

These data were collected on a subset of institutions that met the following selection criteria (Education Statistics 2018):

\begin{itemize}
\item
  Degree granting
\item
  United States only
\item
  Title IV participating
\item
  Not for profit
\item
  2-year or 4-year or above
\item
  Has full-time first-time undergraduates
\end{itemize}

Some of the variables collected and their descriptions are below. Note that several variables have missing values for some institutions (denoted by ``NA'').

\begin{longtable}[]{@{}
  >{\raggedright\arraybackslash}p{(\columnwidth - 2\tabcolsep) * \real{0.2353}}
  >{\raggedright\arraybackslash}p{(\columnwidth - 2\tabcolsep) * \real{0.7647}}@{}}
\toprule\noalign{}
\begin{minipage}[b]{\linewidth}\raggedright
\textbf{Variable}
\end{minipage} & \begin{minipage}[b]{\linewidth}\raggedright
\textbf{Description}
\end{minipage} \\
\midrule\noalign{}
\endhead
\bottomrule\noalign{}
\endlastfoot
\texttt{UnitID} & Unique institution identifier \\
\texttt{Name} & Institution name \\
\texttt{State} & State abbreviation \\
\texttt{Sector} & whether public or private \\
\texttt{LandGrant} & Is this a land-grant institution (Yes/No) \\
\texttt{Size} & Institution size category based on total student enrolled for credit, Fall 2018: Under 1,000, 1,000\$-\(4,999, 5,000\)-\(9,999, 10,000\)-\$19,999, 20,000 and above \\
\texttt{Cost\_OutofState} & Cost of attendance for full-time out-of-state undergraduate students \\
\texttt{Cost\_InState} & Cost of attendance for full-time in-state undergraduate students \\
\texttt{Retention} & Retention rate is the percent of the undergraduate students that re-enroll in the next year \\
\texttt{Graduation\_Rate} & 6-year graduation rate for undergraduate students \\
\texttt{SATMath\_75} & 75th percentile Math SAT score \\
\texttt{ACT\_75} & 75th percentile ACT score \\
\end{longtable}

\subsubsection*{Identifying Variables in a data set}\label{identifying-variables-in-a-data-set}
\addcontentsline{toc}{subsubsection}{Identifying Variables in a data set}

Look through the provided chart showing the description of variables measured. The UnitID and Name are identifiers for each observational unit, \emph{US degree granting institutions in 2018}.

\begin{enumerate}
\def\labelenumi{\arabic{enumi}.}
\tightlist
\item
  Identify in the chart which variables collected on the US institutions are categorical (C) and which variables are quantitative (Q).
\end{enumerate}

\subsubsection*{Summarizing quantitative variables}\label{summarizing-quantitative-variables}
\addcontentsline{toc}{subsubsection}{Summarizing quantitative variables}

The \texttt{favstats()} function from the \texttt{mosaic} package gives the summary statistics for a quantitative variable. The \texttt{R} output below provides the summary statistics for the variable \texttt{Graduation\_Rate}. The summary statistics provided are the two measures of center (mean and median) and two measures of spread (standard deviation and the quartile values to calculate the IQR) for IMDb score.

\begin{itemize}
\item
  Highlight and run lines 1 -- 12 in the provided \texttt{R} script file to load the data set. Check that the summary statistics match the output given in the coursepack.
\item
  Notice that the 2-year institutions were removed so the observational units for this study are \textbf{4-year higher education institutions.}
\end{itemize}

\begin{Shaded}
\begin{Highlighting}[]
\NormalTok{IPEDS }\OtherTok{\textless{}{-}} \FunctionTok{read.csv}\NormalTok{(}\StringTok{"https://www.math.montana.edu/courses/s216/data/IPEDS\_2018.csv"}\NormalTok{) }
\NormalTok{IPEDS }\OtherTok{\textless{}{-}}\NormalTok{ IPEDS }\SpecialCharTok{\%\textgreater{}\%}
  \FunctionTok{filter}\NormalTok{(Sector }\SpecialCharTok{!=} \StringTok{"Public 2{-}year"}\NormalTok{) }\CommentTok{\# Filters the data set to remove Public 2{-}year}
\NormalTok{IPEDS }\OtherTok{\textless{}{-}}\NormalTok{ IPEDS }\SpecialCharTok{\%\textgreater{}\%}
  \FunctionTok{filter}\NormalTok{(Sector }\SpecialCharTok{!=} \StringTok{"Private 2{-}year"}\NormalTok{) }\CommentTok{\# Filters the data set to remove Private 2{-}year}
\NormalTok{IPEDS }\SpecialCharTok{\%\textgreater{}\%}
    \FunctionTok{summarize}\NormalTok{(}\FunctionTok{favstats}\NormalTok{(Graduation\_Rate))}
\end{Highlighting}
\end{Shaded}

\begin{verbatim}
#>   min Q1 median Q3 max     mean       sd    n missing
#> 1   0 38     53 67 100 52.48749 20.63192 1918      49
\end{verbatim}

\begin{enumerate}
\def\labelenumi{\arabic{enumi}.}
\setcounter{enumi}{1}
\tightlist
\item
  Report the values for the two measures of center (mean and median).
\end{enumerate}

\vspace{0.5in}

\begin{enumerate}
\def\labelenumi{\arabic{enumi}.}
\setcounter{enumi}{2}
\tightlist
\item
  Calculate the interquartile range (IQR = Q3 \(-\) Q1) of Graduation Rates.
\end{enumerate}

\vspace{0.5in}

\begin{enumerate}
\def\labelenumi{\arabic{enumi}.}
\setcounter{enumi}{3}
\item
  Report the value of the standard deviation and interpret this value in context of the problem.
  \vspace{0.8in}
\item
  Interpret the value of \(Q_3\) in context of the study.
\end{enumerate}

\vspace{0.8in}

\subsubsection*{Displaying a single quantitative variable}\label{displaying-a-single-quantitative-variable}
\addcontentsline{toc}{subsubsection}{Displaying a single quantitative variable}

There are three type of plots used to plot a single quantitative variable: a dotplot, a histogram or a boxplot. A dotplot of graduation rate would plot a dot for the graduation rate for each 4-year US higher education institution.

First, let's create a histogram of the variable \texttt{Graduation\_Rate}.

\begin{itemize}
\item
  Enter the name of the variable in line 19 for \texttt{variable} in the R script file.
\item
  Replace the word title for the plot in line 21 between the quotations with a descriptive title. \textbf{A title should include: type of plot, variable or variables plotted, and observational units.}
\item
  Highlight and run lines 18 -- ?? to create the histogram.
\end{itemize}

\begin{Shaded}
\begin{Highlighting}[]
\NormalTok{IPEDS }\SpecialCharTok{\%\textgreater{}\%} \CommentTok{\# Data set piped into...}
\FunctionTok{ggplot}\NormalTok{(}\FunctionTok{aes}\NormalTok{(}\AttributeTok{x =}\NormalTok{ xx)) }\SpecialCharTok{+}   \CommentTok{\# Name variable to plot}
  \FunctionTok{geom\_histogram}\NormalTok{(}\AttributeTok{binwidth =} \DecValTok{10}\NormalTok{) }\SpecialCharTok{+}  \CommentTok{\# Create histogram with specified binwidth}
  \FunctionTok{labs}\NormalTok{(}\AttributeTok{title =} \StringTok{"Don\textquotesingle{}t forget to title the plot!"}\NormalTok{, }\CommentTok{\# Title for plot}
       \AttributeTok{x =} \StringTok{"Graduation Rate"}\NormalTok{, }\CommentTok{\# Label for x axis}
       \AttributeTok{y =} \StringTok{"Frequency"}\NormalTok{) }\CommentTok{\# Label for y axis}
\end{Highlighting}
\end{Shaded}

Notice that the \textbf{bin width} for the histogram is 10. For example the first bin consists of the number of institutions in the data set with a graduation rate of 0 to 10\%. It is important to note that a graduation rate on the boundary of a bin will fall into the bin above it; for example, 20 would be counted in the bin 20--30.

\begin{enumerate}
\def\labelenumi{\arabic{enumi}.}
\setcounter{enumi}{5}
\tightlist
\item
  Which range of Graduation Rates have the highest frequency?
\end{enumerate}

\vspace{0.2in}

Next we will create a boxplot of the variable \texttt{Graduation\_Rate}.

\begin{itemize}
\item
  Enter the name of the variable in line 19 for \texttt{variable} in the R script file.
\item
  Highlight and run lines\ldots.
\end{itemize}

\begin{Shaded}
\begin{Highlighting}[]
\NormalTok{IPEDS }\SpecialCharTok{\%\textgreater{}\%} \CommentTok{\# Data set piped into...}
\FunctionTok{ggplot}\NormalTok{(}\FunctionTok{aes}\NormalTok{(}\AttributeTok{x =}\NormalTok{ variable)) }\SpecialCharTok{+}   \CommentTok{\# Name variable to plot}
  \FunctionTok{geom\_boxplot}\NormalTok{() }\SpecialCharTok{+}  \CommentTok{\# Create boxplot with specified binwidth}
  \FunctionTok{labs}\NormalTok{(}\AttributeTok{title =} \StringTok{"Boxplot of Graduation Rates for 4{-}year Higher Education Institutions"}\NormalTok{, }\CommentTok{\# Title for plot}
       \AttributeTok{x =} \StringTok{"Graduation\_Rate"}\NormalTok{, }\CommentTok{\# Label for x axis}
       \AttributeTok{y =} \StringTok{""}\NormalTok{) }\SpecialCharTok{+} \CommentTok{\# Remove y axis label}
    \FunctionTok{theme}\NormalTok{(}\AttributeTok{axis.text.y =} \FunctionTok{element\_blank}\NormalTok{(), }
          \AttributeTok{axis.ticks.y =} \FunctionTok{element\_blank}\NormalTok{()) }\CommentTok{\# Removes y{-}axis ticks}
\end{Highlighting}
\end{Shaded}

\begin{enumerate}
\def\labelenumi{\arabic{enumi}.}
\setcounter{enumi}{6}
\tightlist
\item
  Sketch the boxplot created and identify the values of the 5-number summary (minimum value, Q1, median, Q3, maximum value) on the plot. Use the following formulas to find the invisible fence on both ends of the distribution. Draw a dotted line at the invisible fence to show how the outliers were found.
\end{enumerate}

\[\text{Lower Fence: values} \le \text{Q}1 - 1.5\times\text{IQR}\]

\[\text{Upper Fence: values} \ge \text{Q}3 + 1.5\times\text{IQR}\]
\vspace{1.8in}

When describing plots of quantitative variables we discuss the shape (symmetric or skewed), the center (mean or median), spread (standard deviation or IQR), and if there are outliers present.

\begin{enumerate}
\def\labelenumi{\arabic{enumi}.}
\setcounter{enumi}{7}
\tightlist
\item
  What is the shape of the distribution of graduation rates?
\end{enumerate}

\vspace{0.4in}

\begin{enumerate}
\def\labelenumi{\arabic{enumi}.}
\setcounter{enumi}{8}
\tightlist
\item
  From which plot (histogram or boxplot) is it easier to determine the shape of the distribution?
\end{enumerate}

\vspace{0.3in}

\begin{enumerate}
\def\labelenumi{\arabic{enumi}.}
\setcounter{enumi}{9}
\tightlist
\item
  From which plot is it easier to determine if there are outliers?
\end{enumerate}

\vspace{0.3in}

\subsubsection*{Robust Statistics}\label{robust-statistics}
\addcontentsline{toc}{subsubsection}{Robust Statistics}

Let's examine how the presence of outliers affect the values of center and spread. For this part of the activity we will look at the variable retention rate in the IPEDS data set.

\begin{Shaded}
\begin{Highlighting}[]
\NormalTok{IPEDS }\SpecialCharTok{\%\textgreater{}\%} \CommentTok{\# Data set piped into...}
    \FunctionTok{summarise}\NormalTok{(}\FunctionTok{favstats}\NormalTok{(Retention))}
\CommentTok{\#\textgreater{}   min Q1 median Q3 max    mean       sd    n missing}
\CommentTok{\#\textgreater{} 1   0 66     75 83 100 73.8525 15.14323 1817     150}

\NormalTok{IPEDS }\SpecialCharTok{\%\textgreater{}\%} \CommentTok{\# Data set piped into...}
    \FunctionTok{ggplot}\NormalTok{(}\FunctionTok{aes}\NormalTok{(}\AttributeTok{x =}\NormalTok{ Retention)) }\SpecialCharTok{+} \CommentTok{\# Name variable to plot}
    \FunctionTok{geom\_boxplot}\NormalTok{() }\SpecialCharTok{+} \CommentTok{\# Create boxplot }
    \FunctionTok{labs}\NormalTok{(}\AttributeTok{title =} \StringTok{"Boxplot of Retention Rates for 4{-}year Higher Education Institutions"}\NormalTok{, }\CommentTok{\# Title for plot}
         \AttributeTok{x =} \StringTok{"Retention Rates (\%)"}\NormalTok{, }\CommentTok{\# Label for x axis}
         \AttributeTok{y =} \StringTok{"Frequency"}\NormalTok{) }\CommentTok{\# Label for y axis}
\CommentTok{\#\textgreater{} Warning: Removed 150 rows containing non{-}finite outside the scale range}
\CommentTok{\#\textgreater{} (\textasciigrave{}stat\_boxplot()\textasciigrave{}).}
\end{Highlighting}
\end{Shaded}

\begin{center}\includegraphics[width=0.7\linewidth]{06-A11-EDA-quantitative_files/figure-latex/unnamed-chunk-4-1} \end{center}

\begin{enumerate}
\def\labelenumi{\arabic{enumi}.}
\setcounter{enumi}{10}
\tightlist
\item
  Report the two measures of center for these data.
\end{enumerate}

\vspace{0.5in}

\begin{enumerate}
\def\labelenumi{\arabic{enumi}.}
\setcounter{enumi}{11}
\tightlist
\item
  Report the two measures of spread for these data.
\end{enumerate}

\vspace{0.5in}

To show the effect of outliers on the measures of center and spread, the smallest values of retention rate in the
data set were increased by 30\%. This variable is called \texttt{Retention\_Inc}.

\begin{Shaded}
\begin{Highlighting}[]
\NormalTok{IPEDS }\SpecialCharTok{\%\textgreater{}\%} \CommentTok{\# Data set piped into...}
    \FunctionTok{summarise}\NormalTok{(}\FunctionTok{favstats}\NormalTok{(Retention\_Inc))}
\CommentTok{\#\textgreater{}   min Q1 median Q3 max     mean       sd    n missing}
\CommentTok{\#\textgreater{} 1  30 66     75 83 100 74.49642 13.41255 1817     150}

\NormalTok{IPEDS }\SpecialCharTok{\%\textgreater{}\%} \CommentTok{\# Data set piped into...}
    \FunctionTok{ggplot}\NormalTok{(}\FunctionTok{aes}\NormalTok{(}\AttributeTok{x =}\NormalTok{ Retention\_Inc)) }\SpecialCharTok{+} \CommentTok{\# Name variable to plot}
    \FunctionTok{geom\_boxplot}\NormalTok{() }\SpecialCharTok{+} \CommentTok{\# Create histogram }
\FunctionTok{labs}\NormalTok{(}\AttributeTok{title =} \StringTok{"Boxplot of Increased Retention Rates for 4{-}year Higher Education Institutions"}\NormalTok{, }\CommentTok{\# Title for plot}
\AttributeTok{x =} \StringTok{"Retention Rates (\%)"}\NormalTok{, }\CommentTok{\# Label for x axis}
\AttributeTok{y =} \StringTok{"Frequency"}\NormalTok{) }\CommentTok{\# Label for y axis}
\CommentTok{\#\textgreater{} Warning: Removed 150 rows containing non{-}finite outside the scale range}
\CommentTok{\#\textgreater{} (\textasciigrave{}stat\_boxplot()\textasciigrave{}).}
\end{Highlighting}
\end{Shaded}

\begin{center}\includegraphics[width=0.7\linewidth]{06-A11-EDA-quantitative_files/figure-latex/unnamed-chunk-5-1} \end{center}

\begin{enumerate}
\def\labelenumi{\arabic{enumi}.}
\setcounter{enumi}{12}
\item
  Report the two measures of center for this new data set.
  \vspace{0.5in}
\item
  Report the two measures of spread for this new data set.
  \vspace{0.5in}
\item
  Which measure of center is robust to outliers? Explain your answer.
  \vspace{0.8in}
\item
  Which measure of spread is robust to outliers? Explain your answer.
  \vspace{0.8in}
\end{enumerate}

\subsection{Take-home messages}\label{take-home-messages}

\begin{enumerate}
\def\labelenumi{\arabic{enumi}.}
\item
  Histograms, box plots, and dot plots can all be used to graphically display a single quantitative variable.
\item
  The box plot is created using the five number summary: minimum value, quartile 1, median, quartile 3, and maximum value. Values in the data set that are less than \(\text{Q}_1 - 1.5\times \text{IQR}\) and greater than \(\text{Q}_3 + 1.5\times \text{IQR}\) are considered outliers and are graphically represented by a dot outside of the whiskers on the box plot.
\item
  Data should be summarized numerically and displayed graphically to give us information about the study.
\item
  When comparing distributions of quantitative variables we look at the shape, center, spread, and for outliers. There are two measures of center: mean and the median and two measures of spread: standard deviation and the interquartile range, IQR = Q3 \(-\) Q1.
\end{enumerate}

\subsection{Additional notes}\label{additional-notes}

Use this space to summarize your thoughts and take additional notes on today's activity and material covered.

\newpage

\section{Activity 12: Hypothesis Testing of a Single Quantitative Variable}\label{activity-12-hypothesis-testing-of-a-single-quantitative-variable}

\setstretch{1}

\subsection{Learning outcomes}\label{learning-outcomes-1}

\begin{itemize}
\item
  Given a research question involving one quantitative variable, construct the null and alternative hypotheses
  in words and using appropriate statistical symbols.
\item
  Investigate the process of creating a null distribution for one quantitative variable
\item
  Find, evaluate, and interpret a p-value from the null distribution
\end{itemize}

\subsection{Terminology review}\label{terminology-review-1}

In today's activity, we will simulation and theory-based methods to analyze a single quantitative variable. Some terms covered in this activity are:

\begin{itemize}
\item
  Null hypothesis
\item
  Alternative hypothesis
\end{itemize}

To review these concepts, see Chapter 17 in the textbook.

\subsection{College student sleep habits}\label{college-student-sleep-habits}

According to the an article in \emph{Sleep} (Watson 2015), experts recommend adults (\textgreater18) get at least 7 hours of sleep per night. A survey was sent to students in four sections of Stat 216 asking about their sleep habits. Is there evidence that sleep college students get less than the recommended 7 hours of sleep per night, on average?

\subsubsection*{Summarizing quantitative variables}\label{summarizing-quantitative-variables-1}
\addcontentsline{toc}{subsubsection}{Summarizing quantitative variables}

\begin{itemize}
\item
  Download the R script file and data file for this activity
\item
  Upload both files to the RStudio server and open the R script file
\item
  Enter the name of the dataset for datasetname.csv
\item
  Highlight and run lines 1--8 to load the data
\end{itemize}

\begin{Shaded}
\begin{Highlighting}[]
\NormalTok{sleep }\OtherTok{\textless{}{-}} \FunctionTok{read.csv}\NormalTok{(}\StringTok{"datasetname.csv"}\NormalTok{)}
\end{Highlighting}
\end{Shaded}

\subsubsection*{Ask a research question}\label{ask-a-research-question}
\addcontentsline{toc}{subsubsection}{Ask a research question}

\begin{enumerate}
\def\labelenumi{\arabic{enumi}.}
\tightlist
\item
  Write the parameter of interest in context of the study.
\end{enumerate}

\vspace{1in}

\begin{enumerate}
\def\labelenumi{\arabic{enumi}.}
\setcounter{enumi}{1}
\tightlist
\item
  Write the null hypothesis in words in context of the study.
\end{enumerate}

\vspace{1in}

\begin{enumerate}
\def\labelenumi{\arabic{enumi}.}
\setcounter{enumi}{2}
\tightlist
\item
  Write the alternative hypothesis in notation.
\end{enumerate}

\vspace{0.4in}

\subsubsection*{Summarize and visualize the data}\label{summarize-and-visualize-the-data}
\addcontentsline{toc}{subsubsection}{Summarize and visualize the data}

The \texttt{favstats()} function from the \texttt{mosaic} package gives the summary statistics for a quantitative variable.

\begin{itemize}
\item
  Enter the variable name, \texttt{SleepHours} for variable in line 13
\item
  Highlight and run lines 12--13
\end{itemize}

\begin{Shaded}
\begin{Highlighting}[]
\NormalTok{sleep }\SpecialCharTok{\%\textgreater{}\%}
    \FunctionTok{summarize}\NormalTok{(}\FunctionTok{favstats}\NormalTok{(variable))}
\end{Highlighting}
\end{Shaded}

\begin{enumerate}
\def\labelenumi{\arabic{enumi}.}
\setcounter{enumi}{3}
\tightlist
\item
  How far is each number of hours of sleep for a Stat 216 student from the mean number of hours of sleep, on average?
\end{enumerate}

\vspace{0.3in}

Create a boxplot of the variable \texttt{SleepHours}.

\begin{itemize}
\item
  Enter the name of the variable in line 19 for \texttt{variable} in the R script file.
\item
  Enter a title in line 21 for the plot between the quotations
\item
  Highlight and run lines 18 - 25
\end{itemize}

\begin{Shaded}
\begin{Highlighting}[]
\NormalTok{sleep }\SpecialCharTok{\%\textgreater{}\%} \CommentTok{\# Data set piped into...}
    \FunctionTok{ggplot}\NormalTok{(}\FunctionTok{aes}\NormalTok{(}\AttributeTok{x =}\NormalTok{ variable)) }\SpecialCharTok{+}   \CommentTok{\# Name variable to plot}
    \FunctionTok{geom\_boxplot}\NormalTok{() }\SpecialCharTok{+}  \CommentTok{\# Create boxplot with specified binwidth}
    \FunctionTok{labs}\NormalTok{(}\AttributeTok{title =} \StringTok{"Don\textquotesingle{}t forget to title your plot!"}\NormalTok{, }\CommentTok{\# Title for plot}
       \AttributeTok{x =} \StringTok{"Amount of sleep (hrs)"}\NormalTok{, }\CommentTok{\# Label for x axis}
       \AttributeTok{y =} \StringTok{""}\NormalTok{) }\SpecialCharTok{+} \CommentTok{\# Remove y axis label}
    \FunctionTok{theme}\NormalTok{(}\AttributeTok{axis.text.y =} \FunctionTok{element\_blank}\NormalTok{(), }
          \AttributeTok{axis.ticks.y =} \FunctionTok{element\_blank}\NormalTok{()) }\CommentTok{\# Removes y{-}axis ticks}
\end{Highlighting}
\end{Shaded}

\begin{enumerate}
\def\labelenumi{\arabic{enumi}.}
\setcounter{enumi}{4}
\tightlist
\item
  Describe the boxplot using the four characteristics of boxplots.
\end{enumerate}

\vspace{1in}

\subsection*{Simulation methods}\label{simulation-methods}
\addcontentsline{toc}{subsection}{Simulation methods}

To simulate the null distribution of sample means we will use a bootstrapping method. Recall that the null distribution must be created under the assumption that the null hypothesis is true. Therefore, before bootstrapping, we will need to \emph{shift} each data point by the difference \(\mu_0 - \bar{x}\). This will ensure that the mean of the shifted data is \(\mu_0\) (rather than the mean of the original data, \(\bar{x}\)), and that the simulated null distribution will be centered at the null value.

\begin{enumerate}
\def\labelenumi{\arabic{enumi}.}
\setcounter{enumi}{5}
\tightlist
\item
  Calculate the difference \(\mu_0 - \bar{x}\). Will we need to shift the data up or down?
\end{enumerate}

\vspace{0.3in}

\begin{itemize}
\item
  Open the data set (sleep\_college) in Excel
\item
  Create a new column labeled Shift
\item
  In the column, Shift, add the shifted value to each value in the column, SleepHours
\item
  Save the file and upload again to the RStudio server
\item
  Find the favstats of the variable, Shift
\item
  Highlight and run lines 30--32
\end{itemize}

\begin{Shaded}
\begin{Highlighting}[]
\NormalTok{sleep }\OtherTok{\textless{}{-}} \FunctionTok{read.csv}\NormalTok{(}\StringTok{"sleep\_college.csv"}\NormalTok{)}
\NormalTok{sleep }\SpecialCharTok{\%\textgreater{}\%}
    \FunctionTok{summarize}\NormalTok{(}\FunctionTok{favstats}\NormalTok{(Shift))}
\end{Highlighting}
\end{Shaded}

\begin{enumerate}
\def\labelenumi{\arabic{enumi}.}
\setcounter{enumi}{6}
\tightlist
\item
  Report the mean of the Shift variable. Why does it make sense that this value is the same as the null value?
\end{enumerate}

\vspace{0.9in}

\begin{enumerate}
\def\labelenumi{\arabic{enumi}.}
\setcounter{enumi}{7}
\tightlist
\item
  Report the standard deviation of the Shift variable. How does this compare to the standard deviation for the variable SleepHours? Explain why these values are the same?
\end{enumerate}

\vspace{0.9in}

\begin{enumerate}
\def\labelenumi{\arabic{enumi}.}
\setcounter{enumi}{8}
\tightlist
\item
  What inputs should be entered for each of the following to create the simulation?
  \vspace{1mm}
\end{enumerate}

\begin{itemize}
\tightlist
\item
  Null Value (What is the null value for the study?):
\end{itemize}

\vspace{.15in}

\begin{itemize}
\tightlist
\item
  Summary measure (``mean'' or ``median''):
\end{itemize}

\vspace{0.15in}

\begin{itemize}
\tightlist
\item
  Shift (Difference between \(\mu_0 -\bar{x}\)):
\end{itemize}

\vspace{0.15in}

\begin{itemize}
\tightlist
\item
  As extreme as (enter the value for the sample difference in proportions):
\end{itemize}

\vspace{.15in}

\begin{itemize}
\tightlist
\item
  Direction (\texttt{"greater"}, \texttt{"less"}, or \texttt{"two-sided"}):
\end{itemize}

\vspace{.15in}

\begin{itemize}
\tightlist
\item
  Number of repetitions:
\end{itemize}

\vspace{.15in}

Using the R script file for this activity\ldots{}

\begin{itemize}
\item
  Enter your answers for question 9 in place of the \texttt{xx}'s to produce the null distribution with 10000 simulations
\item
  Highlight and run lines 361--42.
\end{itemize}

\begin{Shaded}
\begin{Highlighting}[]
\FunctionTok{one\_mean\_test}\NormalTok{(sleep}\SpecialCharTok{$}\NormalTok{SleepHours,}\CommentTok{\#Enter the object name and variable}
              \AttributeTok{null\_value =}\NormalTok{ xx,}
              \AttributeTok{summary\_measure =} \StringTok{"xx"}\NormalTok{,  }\CommentTok{\#Can choose between mean or median}
              \AttributeTok{shift =}\NormalTok{ xx, }\CommentTok{\#Difference between the null value and the sample mean}
              \AttributeTok{as\_extreme\_as =}\NormalTok{ xx, }\CommentTok{\#Value of the summary statistic}
              \AttributeTok{direction =} \StringTok{"xx"}\NormalTok{, }\CommentTok{\#Specify direction of alternative hypothesis}
              \AttributeTok{number\_repetitions =} \DecValTok{10000}\NormalTok{)}
\end{Highlighting}
\end{Shaded}

\begin{enumerate}
\def\labelenumi{\arabic{enumi}.}
\setcounter{enumi}{9}
\tightlist
\item
  Interpret the p-value of the test in context of the problem.
\end{enumerate}

\vspace{1in}

\begin{enumerate}
\def\labelenumi{\arabic{enumi}.}
\setcounter{enumi}{10}
\tightlist
\item
  Write a conclusion to the test in context of the problem.
\end{enumerate}

\vspace{1in}

\subsection{Take-home messages}\label{take-home-messages-1}

\begin{enumerate}
\def\labelenumi{\arabic{enumi}.}
\item
  Histograms, box plots, and dot plots can all be used to graphically display a single quantitative variable.
\item
  The box plot is created using the five number summary: minimum value, quartile 1, median, quartile 3, and maximum value. Values in the data set that are less than \(\text{Q}_1 - 1.5\times \text{IQR}\) and greater than \(\text{Q}_3 + 1.5\times \text{IQR}\) are considered outliers and are graphically represented by a dot outside of the whiskers on the box plot.
\item
  Data should be summarized numerically and displayed graphically to give us information about the study.
\item
  When comparing distributions of quantitative variables we look at the shape, center, spread, and for outliers. There are two measures of center: mean and the median and two measures of spread: standard deviation and the interquartile range, IQR = Q3 \(-\) Q1.
\end{enumerate}

\subsection{Additional notes}\label{additional-notes-1}

Use this space to summarize your thoughts and take additional notes on today's activity and material covered.

\newpage

\section{Activity 13: Body Temperature}\label{activity-13-body-temperature}

\setstretch{1}

\subsection{Learning outcomes}\label{learning-outcomes-2}

\begin{itemize}
\item
  Given a research question involving a quantitative variable, construct the null and alternative hypotheses
  in words and using appropriate statistical symbols.
\item
  Describe and perform a theory-based hypothesis test for a single mean.
\item
  Interpret and evaluate a p-value for a theory-based hypothesis test for a single mean.
\end{itemize}

\subsection{Terminology review}\label{terminology-review-2}

In today's activity, we will analyze quantitative data using theory-based methods. Some terms covered in this activity are:

\begin{itemize}
\item
  Normality
\item
  \(t\)-distribution
\item
  Degrees of freedom
\item
  T-score
\end{itemize}

To review these concepts, see Chapter 5and? in the textbook.

\subsection{Body Temperature}\label{body-temperature}

It has long been reported that the mean body temperature of adults is \(98.6^{\circ}\)F. There have been a few articles that challenge this assertion. In 2018, a sample of 52 Stat 216 undergraduates, were asked to report their body temperature. Is there evidence that body temperatures of adults differ from the known temperature of \(98.6^{\circ}\)F?

\subsubsection*{Ask a research question}\label{ask-a-research-question-1}
\addcontentsline{toc}{subsubsection}{Ask a research question}

\begin{enumerate}
\def\labelenumi{\arabic{enumi}.}
\tightlist
\item
  Write out the null hypothesis in proper notation for this study.
\end{enumerate}

\vspace{0.8in}

\begin{enumerate}
\def\labelenumi{\arabic{enumi}.}
\setcounter{enumi}{1}
\tightlist
\item
  Write out the null hypothesis in words for this study.
\end{enumerate}

\vspace{0.5in}

In general, the sampling distribution for a sample mean, \(\bar{x}\), based on a sample of size \(n\) from a population with a true mean \(\mu\) and true standard deviation \(\sigma\) can be modeled using a Normal distribution when certain conditions are met.

Conditions for the sampling distribution of \(\bar{x}\) to follow an approximate Normal distribution:

\begin{itemize}
\item
  \textbf{Independence}: The sample's observations are independent. For paired data, that means each pairwise difference should be independent.
\item
  \textbf{Normality}: The data should be approximately normal or the sample size should be large.

  \begin{itemize}
  \item
    \(n < 30\): If the sample size \(n\) is less than 30 and the distribution of the data is approximately normal with no clear outliers in the data, then we typically assume the data come from a nearly normal distribution to satisfy the condition.
  \item
    \(30 \leq n < 100\): If the sample size \(n\) is betwe 30 and 100 and there are no particularly extreme outliers in the data, then we typically assume the sampling distribution of \(\bar{x}\) is nearly normal, even if the underlying distribution of individual observations is not.
  \item
    \(n \geq 100\): If the sample size \(n\) is at least 100 (regardless of the presence of skew or outliers), we typically assume the sampling distribution of \(\bar{x}\) is nearly normal, even if the underlying distribution of individual observations is not.
  \end{itemize}
\end{itemize}

\begin{figure}

{\centering \includegraphics[width=0.7\linewidth]{06-A13-quantitative_theory_files/figure-latex/tdist-1} 

}

\caption{Comparison of the standard Normal vs t-distribution with various degrees of freedom}\label{fig:tdist}
\end{figure}

Like we saw in Chapter \textbf{5}, we will not know the values of the parameters and must use the sample data to estimate them. Unlike with proportions, in which we only needed to estimate the population proportion, \(\pi\), quantitative sample data must be used to estimate both a population mean \(\mu\) and a population standard deviation \(\sigma\). This additional uncertainty will require us to use a theoretical distribution that is just a bit wider than the Normal distribution. Enter the \textbf{\(t\)-distribution}!

As you can seen from Figure \ref{fig:tdist}, the \(t\)-distributions (dashed and dotted lines) are centered at 0 just like a standard Normal distribution (solid line), but are slightly wider. The variability of a \(t\)-distribution depends on its degrees of freedom, which is calculated from the sample size of a study. (For a single sample of \(n\) observations or paired differences, the degrees of freedom is equal to \(n-1\).) Recall from previous classes that larger sample sizes tend to result in narrower sampling distributions. We see that here as well. The larger the sample size, the larger the degrees of freedom, the narrower the \(t\)-distribution. (In fact, a \(t\)-distribution with infinite degrees of freedom actually IS the standard Normal distribution!)

\subsubsection*{Summarize and visualize the data}\label{summarize-and-visualize-the-data-1}
\addcontentsline{toc}{subsubsection}{Summarize and visualize the data}

The following code is used to create a boxplot of the data.

\begin{itemize}
\item
  Download the R script file upload to the R studio server.
\item
  Open the R script file and highlight and run lines 1--14
\end{itemize}

\begin{Shaded}
\begin{Highlighting}[]
\NormalTok{bodytemp }\OtherTok{\textless{}{-}} \FunctionTok{read.csv}\NormalTok{(}\StringTok{"https://math.montana.edu/courses/s216/data/normal\_temperature.csv"}\NormalTok{)}
\NormalTok{bodytemp }\SpecialCharTok{\%\textgreater{}\%}
  \FunctionTok{ggplot}\NormalTok{(}\FunctionTok{aes}\NormalTok{(}\AttributeTok{x =}\NormalTok{ Temp))}\SpecialCharTok{+}
  \FunctionTok{geom\_boxplot}\NormalTok{()}\SpecialCharTok{+}
  \FunctionTok{labs}\NormalTok{(}\AttributeTok{title=}\StringTok{"Boxplot of Body Temperatures for Stat 216 Students"}\NormalTok{,}
       \AttributeTok{x =} \StringTok{"body temperature (*F)"}\NormalTok{) }\SpecialCharTok{+}
        \FunctionTok{theme}\NormalTok{(}\AttributeTok{axis.text.y =} \FunctionTok{element\_blank}\NormalTok{(), }
          \AttributeTok{axis.ticks.y =} \FunctionTok{element\_blank}\NormalTok{()) }\CommentTok{\# Removes y{-}axis ticks}
\end{Highlighting}
\end{Shaded}

\begin{center}\includegraphics[width=0.7\linewidth]{06-A13-quantitative_theory_files/figure-latex/unnamed-chunk-1-1} \end{center}

\begin{itemize}
\tightlist
\item
  Highlight and run lines 17 - 18 to get the summary statistics for the variable Temp.
\end{itemize}

\begin{Shaded}
\begin{Highlighting}[]
\NormalTok{bodytemp }\SpecialCharTok{\%\textgreater{}\%} 
  \FunctionTok{summarise}\NormalTok{(}\FunctionTok{favstats}\NormalTok{(Temp))}
\end{Highlighting}
\end{Shaded}

\begin{verbatim}
#>    min     Q1 median   Q3 max     mean        sd  n missing
#> 1 97.2 97.675   98.2 98.7 100 98.28462 0.6823789 52       0
\end{verbatim}

\subsubsection*{Check theoretical conditions}\label{check-theoretical-conditions}
\addcontentsline{toc}{subsubsection}{Check theoretical conditions}

\begin{enumerate}
\def\labelenumi{\arabic{enumi}.}
\setcounter{enumi}{2}
\tightlist
\item
  Report the sample size of the study. Give appropriate notation.
\end{enumerate}

\vspace{0.2in}

\begin{enumerate}
\def\labelenumi{\arabic{enumi}.}
\setcounter{enumi}{3}
\tightlist
\item
  Report the sample mean of the study. Give appropriate notation.
\end{enumerate}

\vspace{0.2in}

\begin{enumerate}
\def\labelenumi{\arabic{enumi}.}
\setcounter{enumi}{4}
\item
  How do you know the independence condition is met for these data?
  \vspace{0.8in}
\item
  Is the normality condition met to use the theory-based methods for analysis? Explain your answer.
  \vspace{1in}
\end{enumerate}

\subsubsection*{Use statistical inferential methods to draw inferences from the data}\label{use-statistical-inferential-methods-to-draw-inferences-from-the-data}
\addcontentsline{toc}{subsubsection}{Use statistical inferential methods to draw inferences from the data}

To find the standardized statistic for the mean we will use the following formula:

\[T = \frac{\bar{x} - \mu_0}{SE(\bar{x})},\]
where the standard error of the sample mean difference is:

\[SE(\bar{x})=\frac{s}{\sqrt{n}}.\]

\begin{enumerate}
\def\labelenumi{\arabic{enumi}.}
\setcounter{enumi}{6}
\tightlist
\item
  Calculate the standard error of the sample mean.
\end{enumerate}

\vspace{0.5in}

\begin{enumerate}
\def\labelenumi{\arabic{enumi}.}
\setcounter{enumi}{7}
\tightlist
\item
  Interpret the standard error in context of the study.
\end{enumerate}

\vspace{1in}

\begin{enumerate}
\def\labelenumi{\arabic{enumi}.}
\setcounter{enumi}{8}
\tightlist
\item
  Calculate the standardized mean.
\end{enumerate}

\vspace{1in}

\begin{enumerate}
\def\labelenumi{\arabic{enumi}.}
\setcounter{enumi}{9}
\tightlist
\item
  We model a single mean with a t-distribution with \(n-1\) degrees of freedom. Calculate the degrees of freedom for this study.
\end{enumerate}

\vspace{0.2in}

\begin{enumerate}
\def\labelenumi{\arabic{enumi}.}
\setcounter{enumi}{10}
\tightlist
\item
  Mark the value of the standardized statistic on the t-distribution and illustrate how the p-value is found.
\end{enumerate}

\begin{center}\includegraphics[width=0.7\linewidth]{06-A13-quantitative_theory_files/figure-latex/tdistmean-1} \end{center}

To find the p-value for the theory-based test:

\begin{itemize}
\item
  Enter the value for the standardized statistic for xx in the pt function.
\item
  Enter the df for yy in the pt function.
\item
  Highlight and run line 24
\end{itemize}

\begin{Shaded}
\begin{Highlighting}[]
\FunctionTok{pt}\NormalTok{(xx, }\AttributeTok{df=}\NormalTok{yy, }\AttributeTok{lower.tail=}\ConstantTok{FALSE}\NormalTok{)}
\end{Highlighting}
\end{Shaded}

\begin{enumerate}
\def\labelenumi{\arabic{enumi}.}
\setcounter{enumi}{11}
\tightlist
\item
  What does this p-value mean, in the context of the study? Hint: it is the probability of what\ldots assuming what?
  \vspace{1in}
\end{enumerate}

Next we will calculate a theory-based confidence interval. To calculate a theory-based confidence interval for the paired mean difference, use the following formula:

\[\bar{x}\pm t^* \times SE(\bar{x}).\]

We will need to find the \(t^*\) multiplier using the function \texttt{qt()}.

\begin{itemize}
\item
  Enter the appropriate percentile in the R code to find the multiplier for a 90\% confidence interval.
\item
  Enter the df for yy.
\item
  Highlight and run line 30
\end{itemize}

\begin{Shaded}
\begin{Highlighting}[]
\FunctionTok{qt}\NormalTok{(percentile, }\AttributeTok{df =}\NormalTok{ yy, }\AttributeTok{lower.tail=}\ConstantTok{TRUE}\NormalTok{)}
\end{Highlighting}
\end{Shaded}

\begin{enumerate}
\def\labelenumi{\arabic{enumi}.}
\setcounter{enumi}{12}
\tightlist
\item
  Report the \(t^*\) multiplier for the 90\% confidence interval.
\end{enumerate}

\vspace{0.2in}

\begin{enumerate}
\def\labelenumi{\arabic{enumi}.}
\setcounter{enumi}{13}
\tightlist
\item
  Calculate the margin of error for the true mean using theory-based methods.
\end{enumerate}

\vspace{0.6in}

\begin{enumerate}
\def\labelenumi{\arabic{enumi}.}
\setcounter{enumi}{14}
\tightlist
\item
  Calculate the confidence interval.
\end{enumerate}

\vspace{0.6in}

\begin{enumerate}
\def\labelenumi{\arabic{enumi}.}
\setcounter{enumi}{15}
\tightlist
\item
  Interpret the confidence interval in context of the study.
\end{enumerate}

\vspace{1in}

\begin{enumerate}
\def\labelenumi{\arabic{enumi}.}
\setcounter{enumi}{16}
\tightlist
\item
  Write a conclusion to the test in context of the study.
  \vspace{0.6in}
\end{enumerate}

\subsection{Take-home messages}\label{take-home-messages-2}

\begin{enumerate}
\def\labelenumi{\arabic{enumi}.}
\item
  In order to use theory-based methods for dependent groups (paired data), the independent observational units and normality conditions must be met.
\item
  A T-score is compared to a \(t\)-distribution with \(n - 1\) df in order to calculate a one-sided p-value. To find a two-sided p-value using theory-based methods we need to multiply the one-sided p-value by 2.
\item
  A \(t^*\) multiplier is found by obtaining the bounds of the middle X\% (X being the desired confidence level) of a \(t\)-distribution with \(n - 1\) df.
\end{enumerate}

\subsection{Additional notes}\label{additional-notes-2}

Use this space to summarize your thoughts and take additional notes on today's activity and material covered

\newpage

\chapter*{References}\label{references}
\addcontentsline{toc}{chapter}{References}

\phantomsection\label{refs}
\begin{CSLReferences}{1}{0}
\bibitem[\citeproctext]{ref-pga}
{``Average Driving Distance and Fairway Accuracy.''} 2008. \href{https://www.pga.com/\%20and\%20https://www.lpga.com/}{https://www.pga.com/ and https://www.lpga.com/}.

\bibitem[\citeproctext]{ref-banton2022}
Banton, et al, S. 2022. {``Jog with Your Dog: Dog Owner Exercise Routines Predict Dog Exercise Routines and Perception of Ideal Body Weight.''} \emph{PLoS ONE} 17(8).

\bibitem[\citeproctext]{ref-bhavsar2022}
Bhavsar, et al, A. 2022. {``Increased Risk of Herpes Zoster in Adults \(\geq\)50 Years Old Diagnosed with COVID-19 in the United States.''} \emph{Open Forum Infectious Diseases} 9(5).

\bibitem[\citeproctext]{ref-islands}
Bulmer, M. n.d. {``Islands in Schools Project.''} \url{https://sites.google.com/site/islandsinschoolsprojectwebsite/home}.

\bibitem[\citeproctext]{ref-bts}
{``Bureau of Transportation Statistics.''} 2019. \url{https://www.bts.gov/}.

\bibitem[\citeproctext]{ref-babies}
{``Child Health and Development Studies.''} n.d. \url{https://www.chdstudies.org/}.

\bibitem[\citeproctext]{ref-darley1973}
Darley, J. M., and C. D. Batson. 1973. {``"From Jerusalem to Jericho": A Study of Situational and Dispositional Variables in Helping Behavior.''} \emph{Journal of Personality and Social Psychology} 27: 100--108.

\bibitem[\citeproctext]{ref-davis2020}
Davis, Smith, A. K. 2020. {``A Poor Substitute for the Real Thing: Captive-Reared Monarch Butterflies Are Weaker, Paler and Have Less Elongated Wings Than Wild Migrants.''} \emph{Biology Letters} 16.

\bibitem[\citeproctext]{ref-doit2015}
Du Toit, et al, G. 2015. {``Randomized Trial of Peanut Consumption in Infants at Risk for Peanut Allergy.''} \emph{New England Journal of Medicine} 372.

\bibitem[\citeproctext]{ref-edmunds2016}
Edmunds, et al, D. 2016. {``Chronic Wasting Disease Drives Population Decline of White-Tailed Deer.''} \emph{PLoS ONE} 11(8).

\bibitem[\citeproctext]{ref-ipeds}
Education Statistics, National Center for. 2018. {``IPEDS.''} \url{https://nces.ed.gov/ipeds/}.

\bibitem[\citeproctext]{ref-gbmarried}
{``Great Britain Married Couples: Great Britain Office of Population Census and Surveys.''} n.d. \url{https://discovery.nationalarchives.gov.uk/details/r/C13351}.

\bibitem[\citeproctext]{ref-zeitler2012}
Group, TODAY Study. 2012. {``\href{https://www.ncbi.nlm.nih.gov/pubmed/22540912}{A Clinical Trial to Maintain Glycemic Control in Youth with Type 2 Diabetes}.''} \emph{New England Journal of Medicine} 366: 2247--56.

\bibitem[\citeproctext]{ref-hamblin2007}
Hamblin, J. K., K. Wynn, and P. Bloom. 2007. {``Social Evaluation by Preverbal Infants.''} \emph{Nature} 450 (6288): 557--59.

\bibitem[\citeproctext]{ref-hirschfelder2018}
Hirschfelder, A., and P. F. Molin. 2018. {``I Is for Ignoble: Stereotyping Native Americans.''} \href{Retrieved\%20from\%20https://www.ferris.edu/HTMLS/news/jimcrow/native/homepage.htm.}{Retrieved from https://www.ferris.edu/HTMLS/news/jimcrow/native/homepage.htm.}

\bibitem[\citeproctext]{ref-hutchison2013}
Hutchison, R. L., and M. A. Hirthler. 2013. {``\href{https://www.ncbi.nlm.nih.gov/pubmed/23932117}{Upper Extremity Injuies in Homer's Iliad}.''} \emph{Journal of Hand Surgery (American Volume)} 38: 1790--93.

\bibitem[\citeproctext]{ref-imdb}
{``{IMDb} Movies Extensive Dataset.''} 2016. \url{https://kaggle.com/stefanoleone992/imdb-extensive-dataset}.

\bibitem[\citeproctext]{ref-kalra2022}
Kalra, et al., Dl. 2022. {``Trustworthiness of Indian Youtubers.''} Kaggle. \url{https://doi.org/10.34740/KAGGLE/DSV/4426566}.

\bibitem[\citeproctext]{ref-keating2021}
Keating, D., N. Ahmed, F. Nirappil, Stanley-Becker I., and L. Bernstein. 2021. {``Coronavirus Infections Dropping Where People Are Vaccinated, Rising Where They Are Not, Post Analysis Finds.''} \emph{Washington Post}. \url{https://www.washingtonpost.com/health/2021/06/14/covid-cases-vaccination-rates/}.

\bibitem[\citeproctext]{ref-laeng2007}
Laeng, Mathisen, B. 2007. {``Why Do Blue-Eyed Men Prefer Women with the Same Eye Color?''} \emph{Behavioral Ecology and Sociobiology} 61(3).

\bibitem[\citeproctext]{ref-levin2000}
Levin, D. T. 2000. {``Race as a Visual Feature: Using Visual Search and Perceptual Discrimination Tasks to Understand Face Categories and the Cross-Race Recognition Deficit.''} \emph{Journal of Experimental Psychology} 129(4).

\bibitem[\citeproctext]{ref-madden2020}
Madden, et al, J. 2020. {``Ready Student One: Exploring the Predictors of Student Learning in Virtual Reality.''} \emph{PLoS ONE} 15(3).

\bibitem[\citeproctext]{ref-miller1956}
Miller, G. A. 1956. {``The Magical Number Seven, Plus or Minus Two: Some Limits on Our Capacity for Processing Information.''} \emph{Psychological Review} 63(2).

\bibitem[\citeproctext]{ref-becentispeech}
Moquin, W., and C. Van Doren. 1973. {``Great Documents in American Indian History.''} Praeger.

\bibitem[\citeproctext]{ref-pew2022}
{``More Americans Are Joining the 'Cashless' Economy.''} 2022. \url{https://www.pewresearch.org/short-reads/2022/10/05/more-americans-are-joining-the-cashless-economy/.}

\bibitem[\citeproctext]{ref-weather}
National Weather Service Corporate Image Web Team. n.d. {``National Weather Service -- {NWS} Billings.''} \url{https://w2.weather.gov/climate/xmacis.php?wfo=byz}.

\bibitem[\citeproctext]{ref-obrien2019}
O'Brien, Lynch, H. D. 2019. {``Crocodylian Head Width Allometry and Phylogenetic Prediction of Body Size in Extinct Crocodyliforms.''} \emph{Integrative Organismal Biology} 1.

\bibitem[\citeproctext]{ref-ocean}
{``Ocean Temperature and Salinity Study.''} n.d. \url{https://calcofi.org/}.

\bibitem[\citeproctext]{ref-WashPost2022}
{``Older People Who Get Covid Are at Increased Risk of Getting Shingles.''} 2022. \url{https://www.washingtonpost.com/health/2022/04/19/shingles-and-covid-over-50/.}

\bibitem[\citeproctext]{ref-physhealth}
{``Physician's Health Study.''} n.d. \url{https://phs.bwh.harvard.edu/}.

\bibitem[\citeproctext]{ref-porath2017}
Porath, Erez, C. 2017. {``Does Rudeness Really Matter? The Effects of Rudeness on Task Performance and Helpfulness.''} \emph{Academy of Management Journal} 50.

\bibitem[\citeproctext]{ref-quinn1999}
Quinn, G. E., C. H. Shin, M. G. Maguire, and R. A. Stone. 1999. {``Myopia and Ambient Lighting at Night.''} \emph{Nature} 399 (6732): 113--14. \url{https://doi.org/10.1038/20094}.

\bibitem[\citeproctext]{ref-ramachandran2007}
Ramachandran, V. 2007. {``3 Clues to Understanding Your Brain.''} \url{https://www.ted.com/talks/vs_ramachandran_3_clues_to_understanding_your_brain}.

\bibitem[\citeproctext]{ref-cdchospitalization}
{``Rates of Laboratory-Confimed COVID-19 Hospitalizations by Vaccination Status.''} 2021. CDC. \url{https://covid.cdc.gov/covid-data-tracker/\#covidnet-hospitalizations-vaccination}.

\bibitem[\citeproctext]{ref-richardson2019}
Richardson, T., and R. T. Gilman. 2019. {``Left-Handedness Is Associated with Greater Fighting Success in Humans.''} \emph{Scientific Reports} 9 (1): 15402. \url{https://doi.org/10.1038/s41598-019-51975-3}.

\bibitem[\citeproctext]{ref-stephens2020}
Stephens, R., and O. Robertson. 2020. {``Swearing as a Response to Pain: Assessing Hypoalgesic Effects of Novel "Swear" Words.''} \emph{Frontiers in Psychology} 11: 643--62.

\bibitem[\citeproctext]{ref-stewart2014}
Stewart, E. H., B. Davis, B. L. Clemans-Taylor, B. Littenberg, C. A. Estrada, and R. M. Centor. 2014. {``Rapid Antigen Group a Streptococcus Test to Diagnose Pharyngitis: A Systematic Review and Meta-Analysis''} 9 (11). \url{https://doi.org/10.1371/journal.pone.0111727}.

\bibitem[\citeproctext]{ref-stroop1935}
Stroop, J. R. 1935. {``Studies of Interference in Serial Verbal Reactions.''} \emph{Journal of Experimental Psychology} 18: 643--62.

\bibitem[\citeproctext]{ref-subach2022}
Subach, et al, A. 2022. {``Foraging Behaviour, Habitat Use and Population Size of the Desert Horned Viper in the Negev Desert.''} \emph{Soc.Open Sci} 9.

\bibitem[\citeproctext]{ref-sulheim2017}
Sulheim, S., A. Ekeland, I. Holme, and R. Bahr. 2017. {``Helmet Use and Risk of Head Injuries in Alpine Skiers and Snowboarders: Changes After an Interval of One Decade''} 51 (1): 44--50. \url{https://doi.org/10.1136/bjsports-2015-095798}.

\bibitem[\citeproctext]{ref-titanic}
{``Titanic.''} n.d. \url{http://www.encyclopedia-titanica.org}.

\bibitem[\citeproctext]{ref-covidvaccinetracker}
{``US COVID-19 Vaccine Tracker: See Your State's Progress.''} 2021. Mayo Clinic. \url{https://www.mayoclinic.org/coronavirus-covid-19/vaccine-tracker}.

\bibitem[\citeproctext]{ref-usepa2020}
US Environmental Protection Agency. n.d. {``Air Data -- Daily Air Quality Tracker.''} \url{https://www.epa.gov/outdoor-air-quality-data/air-data-daily-air-quality-tracker}.

\bibitem[\citeproctext]{ref-wahlstrom2014}
Wahlstrom, et al, K. 2014. {``Examining the Impact of Later School Start Times on the Health and Academic Performance of High School Students: A Multi-Site Study.''} \emph{Center for Applied Research and Educational Improvement}.

\bibitem[\citeproctext]{ref-watson2015}
Watson, et al., N. 2015. {``Recommended Amount of Sleep for a Heathy Adult: A Joint Consensus Statement of the American Academy of Sleep Medicine and Sleep Research Society.''} \emph{Sleep} 38(6).

\bibitem[\citeproctext]{ref-Weiss1988}
Weiss, R. D. 1988. {``Relapse to Cocaine Abuse After Initiating Desipramine Treatment.''} \emph{JAMA} 260(17).

\bibitem[\citeproctext]{ref-navajo2011}
{``Welcome to the Navajo Nation Government: Official Site of the Navajo Nation.''} 2011.\href{\%20Retrieved\%20from\%20https://www.navajo-nsn.gov/.}{Retrieved from https://www.navajo-nsn.gov/.}

\bibitem[\citeproctext]{ref-wilson2016}
Wilson, Woodruff, J. P. 2016. {``Vertebral Adaptations to Large Body Size in Theropod Dinosaurs.''} \emph{PLoS ONE} 11(7).

\end{CSLReferences}

\end{document}
