% Options for packages loaded elsewhere
\PassOptionsToPackage{unicode}{hyperref}
\PassOptionsToPackage{hyphens}{url}
%
\documentclass[
]{report}
\usepackage{amsmath,amssymb}
\usepackage{iftex}
\ifPDFTeX
  \usepackage[T1]{fontenc}
  \usepackage[utf8]{inputenc}
  \usepackage{textcomp} % provide euro and other symbols
\else % if luatex or xetex
  \usepackage{unicode-math} % this also loads fontspec
  \defaultfontfeatures{Scale=MatchLowercase}
  \defaultfontfeatures[\rmfamily]{Ligatures=TeX,Scale=1}
\fi
\usepackage{lmodern}
\ifPDFTeX\else
  % xetex/luatex font selection
\fi
% Use upquote if available, for straight quotes in verbatim environments
\IfFileExists{upquote.sty}{\usepackage{upquote}}{}
\IfFileExists{microtype.sty}{% use microtype if available
  \usepackage[]{microtype}
  \UseMicrotypeSet[protrusion]{basicmath} % disable protrusion for tt fonts
}{}
\makeatletter
\@ifundefined{KOMAClassName}{% if non-KOMA class
  \IfFileExists{parskip.sty}{%
    \usepackage{parskip}
  }{% else
    \setlength{\parindent}{0pt}
    \setlength{\parskip}{6pt plus 2pt minus 1pt}}
}{% if KOMA class
  \KOMAoptions{parskip=half}}
\makeatother
\usepackage{xcolor}
\usepackage{color}
\usepackage{fancyvrb}
\newcommand{\VerbBar}{|}
\newcommand{\VERB}{\Verb[commandchars=\\\{\}]}
\DefineVerbatimEnvironment{Highlighting}{Verbatim}{commandchars=\\\{\}}
% Add ',fontsize=\small' for more characters per line
\usepackage{framed}
\definecolor{shadecolor}{RGB}{248,248,248}
\newenvironment{Shaded}{\begin{snugshade}}{\end{snugshade}}
\newcommand{\AlertTok}[1]{\textcolor[rgb]{0.94,0.16,0.16}{#1}}
\newcommand{\AnnotationTok}[1]{\textcolor[rgb]{0.56,0.35,0.01}{\textbf{\textit{#1}}}}
\newcommand{\AttributeTok}[1]{\textcolor[rgb]{0.13,0.29,0.53}{#1}}
\newcommand{\BaseNTok}[1]{\textcolor[rgb]{0.00,0.00,0.81}{#1}}
\newcommand{\BuiltInTok}[1]{#1}
\newcommand{\CharTok}[1]{\textcolor[rgb]{0.31,0.60,0.02}{#1}}
\newcommand{\CommentTok}[1]{\textcolor[rgb]{0.56,0.35,0.01}{\textit{#1}}}
\newcommand{\CommentVarTok}[1]{\textcolor[rgb]{0.56,0.35,0.01}{\textbf{\textit{#1}}}}
\newcommand{\ConstantTok}[1]{\textcolor[rgb]{0.56,0.35,0.01}{#1}}
\newcommand{\ControlFlowTok}[1]{\textcolor[rgb]{0.13,0.29,0.53}{\textbf{#1}}}
\newcommand{\DataTypeTok}[1]{\textcolor[rgb]{0.13,0.29,0.53}{#1}}
\newcommand{\DecValTok}[1]{\textcolor[rgb]{0.00,0.00,0.81}{#1}}
\newcommand{\DocumentationTok}[1]{\textcolor[rgb]{0.56,0.35,0.01}{\textbf{\textit{#1}}}}
\newcommand{\ErrorTok}[1]{\textcolor[rgb]{0.64,0.00,0.00}{\textbf{#1}}}
\newcommand{\ExtensionTok}[1]{#1}
\newcommand{\FloatTok}[1]{\textcolor[rgb]{0.00,0.00,0.81}{#1}}
\newcommand{\FunctionTok}[1]{\textcolor[rgb]{0.13,0.29,0.53}{\textbf{#1}}}
\newcommand{\ImportTok}[1]{#1}
\newcommand{\InformationTok}[1]{\textcolor[rgb]{0.56,0.35,0.01}{\textbf{\textit{#1}}}}
\newcommand{\KeywordTok}[1]{\textcolor[rgb]{0.13,0.29,0.53}{\textbf{#1}}}
\newcommand{\NormalTok}[1]{#1}
\newcommand{\OperatorTok}[1]{\textcolor[rgb]{0.81,0.36,0.00}{\textbf{#1}}}
\newcommand{\OtherTok}[1]{\textcolor[rgb]{0.56,0.35,0.01}{#1}}
\newcommand{\PreprocessorTok}[1]{\textcolor[rgb]{0.56,0.35,0.01}{\textit{#1}}}
\newcommand{\RegionMarkerTok}[1]{#1}
\newcommand{\SpecialCharTok}[1]{\textcolor[rgb]{0.81,0.36,0.00}{\textbf{#1}}}
\newcommand{\SpecialStringTok}[1]{\textcolor[rgb]{0.31,0.60,0.02}{#1}}
\newcommand{\StringTok}[1]{\textcolor[rgb]{0.31,0.60,0.02}{#1}}
\newcommand{\VariableTok}[1]{\textcolor[rgb]{0.00,0.00,0.00}{#1}}
\newcommand{\VerbatimStringTok}[1]{\textcolor[rgb]{0.31,0.60,0.02}{#1}}
\newcommand{\WarningTok}[1]{\textcolor[rgb]{0.56,0.35,0.01}{\textbf{\textit{#1}}}}
\usepackage{longtable,booktabs,array}
\usepackage{calc} % for calculating minipage widths
% Correct order of tables after \paragraph or \subparagraph
\usepackage{etoolbox}
\makeatletter
\patchcmd\longtable{\par}{\if@noskipsec\mbox{}\fi\par}{}{}
\makeatother
% Allow footnotes in longtable head/foot
\IfFileExists{footnotehyper.sty}{\usepackage{footnotehyper}}{\usepackage{footnote}}
\makesavenoteenv{longtable}
\usepackage{graphicx}
\makeatletter
\newsavebox\pandoc@box
\newcommand*\pandocbounded[1]{% scales image to fit in text height/width
  \sbox\pandoc@box{#1}%
  \Gscale@div\@tempa{\textheight}{\dimexpr\ht\pandoc@box+\dp\pandoc@box\relax}%
  \Gscale@div\@tempb{\linewidth}{\wd\pandoc@box}%
  \ifdim\@tempb\p@<\@tempa\p@\let\@tempa\@tempb\fi% select the smaller of both
  \ifdim\@tempa\p@<\p@\scalebox{\@tempa}{\usebox\pandoc@box}%
  \else\usebox{\pandoc@box}%
  \fi%
}
% Set default figure placement to htbp
\def\fps@figure{htbp}
\makeatother
\setlength{\emergencystretch}{3em} % prevent overfull lines
\providecommand{\tightlist}{%
  \setlength{\itemsep}{0pt}\setlength{\parskip}{0pt}}
\setcounter{secnumdepth}{5}
% definitions for citeproc citations
\NewDocumentCommand\citeproctext{}{}
\NewDocumentCommand\citeproc{mm}{%
  \begingroup\def\citeproctext{#2}\cite{#1}\endgroup}
\makeatletter
 % allow citations to break across lines
 \let\@cite@ofmt\@firstofone
 % avoid brackets around text for \cite:
 \def\@biblabel#1{}
 \def\@cite#1#2{{#1\if@tempswa , #2\fi}}
\makeatother
\newlength{\cslhangindent}
\setlength{\cslhangindent}{1.5em}
\newlength{\csllabelwidth}
\setlength{\csllabelwidth}{3em}
\newenvironment{CSLReferences}[2] % #1 hanging-indent, #2 entry-spacing
 {\begin{list}{}{%
  \setlength{\itemindent}{0pt}
  \setlength{\leftmargin}{0pt}
  \setlength{\parsep}{0pt}
  % turn on hanging indent if param 1 is 1
  \ifodd #1
   \setlength{\leftmargin}{\cslhangindent}
   \setlength{\itemindent}{-1\cslhangindent}
  \fi
  % set entry spacing
  \setlength{\itemsep}{#2\baselineskip}}}
 {\end{list}}
\usepackage{calc}
\newcommand{\CSLBlock}[1]{\hfill\break\parbox[t]{\linewidth}{\strut\ignorespaces#1\strut}}
\newcommand{\CSLLeftMargin}[1]{\parbox[t]{\csllabelwidth}{\strut#1\strut}}
\newcommand{\CSLRightInline}[1]{\parbox[t]{\linewidth - \csllabelwidth}{\strut#1\strut}}
\newcommand{\CSLIndent}[1]{\hspace{\cslhangindent}#1}
\usepackage{booktabs}
\usepackage{geometry}
\usepackage[none]{hyphenat}
\usepackage{titlesec}
\usepackage{longtable}
\usepackage{xcolor}
\usepackage{setspace}
\usepackage{pdfpages}

\pagestyle{plain}

%%%% Set margins
\setlength{\topmargin}{-1cm}
\addtolength{\evensidemargin}{-1cm}
\addtolength{\oddsidemargin}{-1cm}
\addtolength{\textheight}{3cm}
\addtolength{\textwidth}{2cm}

% Spacing for reading guides
\newcommand{\rgs}{\vspace{12pt}} % Vertical space
\newcommand{\rgi}{\hspace{24pt}}  % Indent

\newcommand\latexcode[1]{#1}

% Format chapter titles and spacing
\renewcommand*{\chaptername}{Module}

\titleformat{\chapter}[display]
{\bfseries\Large}
{\filleft\MakeUppercase{\chaptertitlename} \Huge\thechapter}
{3ex}
{\titlerule
\vspace{1.5ex}%
\filright}
[\vspace{1.5ex}%
\titlerule]
\titlespacing*{\chapter}{0pt}{-40pt}{20pt}
\usepackage{bookmark}
\IfFileExists{xurl.sty}{\usepackage{xurl}}{} % add URL line breaks if available
\urlstyle{same}
\hypersetup{
  hidelinks,
  pdfcreator={LaTeX via pandoc}}

\title{\textbf{STAT 216 Coursepack}\\
\strut \\
\includegraphics[width=5in,height=\textheight,keepaspectratio]{images/msu-campus.jpg}}
\usepackage{etoolbox}
\makeatletter
\providecommand{\subtitle}[1]{% add subtitle to \maketitle
  \apptocmd{\@title}{\par {\large #1 \par}}{}{}
}
\makeatother
\subtitle{Fall 2025\\
Montana State University}
\author{Melinda Yager\\
Jade Schmidt\\
Stacey Hancock}
\date{}

\begin{document}
\maketitle

\chapter*{Preface}\label{preface}
\addcontentsline{toc}{chapter}{Preface}

Placeholder

\chapter{Basics of Data and Sampling Methods}\label{basics-of-data-and-sampling-methods}

Placeholder

\section{Vocabulary Review and Key Topics}\label{vocabulary-review-and-key-topics}

\subsection{Key topics}\label{key-topics}

\subsection{Vocabulary}\label{vocabulary}

\section{Video Notes: Intro to data and Sampling Methods}\label{video-notes-intro-to-data-and-sampling-methods}

\subsection{Course Videos}\label{course-videos}

\subsection*{Data basics: Video 1.2.1and1.2.2}\label{data-basics-video-1.2.1and1.2.2}
\addcontentsline{toc}{subsection}{Data basics: Video 1.2.1and1.2.2}

\subsubsection*{Types of variables}\label{types-of-variables}
\addcontentsline{toc}{subsubsection}{Types of variables}

\subsubsection*{Exploratory data analysis (EDA)}\label{exploratory-data-analysis-eda}
\addcontentsline{toc}{subsubsection}{Exploratory data analysis (EDA)}

\subsection*{Sampling Methods: Video 2.1}\label{sampling-methods-video-2.1}
\addcontentsline{toc}{subsection}{Sampling Methods: Video 2.1}

\subsubsection*{Good vs.~bad sampling}\label{good-vs.-bad-sampling}
\addcontentsline{toc}{subsubsection}{Good vs.~bad sampling}

\subsection*{Types of Sampling Bias}\label{types-of-sampling-bias}
\addcontentsline{toc}{subsection}{Types of Sampling Bias}

\subsubsection*{Optional Notes: Video Example}\label{optional-notes-video-example}
\addcontentsline{toc}{subsubsection}{Optional Notes: Video Example}

\subsection{Concept Check}\label{concept-check}

\section{Activity 1: Intro to Data Analysis and Sampling Bias}\label{activity-1-intro-to-data-analysis-and-sampling-bias}

\subsection{Learning outcomes}\label{learning-outcomes}

\subsection{Terminology review}\label{terminology-review}

\subsection*{Notes on Observational Units and Variables}\label{notes-on-observational-units-and-variables}
\addcontentsline{toc}{subsection}{Notes on Observational Units and Variables}

\subsubsection*{Further analysis of class data set}\label{further-analysis-of-class-data-set}
\addcontentsline{toc}{subsubsection}{Further analysis of class data set}

\subsection*{Notes on Sampling Methods and Types of bias}\label{notes-on-sampling-methods-and-types-of-bias}
\addcontentsline{toc}{subsection}{Notes on Sampling Methods and Types of bias}

\subsubsection*{Types of bias}\label{types-of-bias}
\addcontentsline{toc}{subsubsection}{Types of bias}

\subsection{Take-home messages}\label{take-home-messages}

\subsection{Additional notes}\label{additional-notes}

\section{Activity 2: American Indian Address}\label{activity-2-american-indian-address}

\subsection{Learning outcomes}\label{learning-outcomes-1}

\subsection{Terminology review}\label{terminology-review-1}

\subsection{Class Preparation}\label{class-preparation}

\subsection{American Indian Address}\label{american-indian-address}

\subsubsection*{By eye selection}\label{by-eye-selection}
\addcontentsline{toc}{subsubsection}{By eye selection}

\subsection*{Notes on sampling}\label{notes-on-sampling}
\addcontentsline{toc}{subsection}{Notes on sampling}

\subsection{Class Activity}\label{class-activity}

\subsubsection*{Random selection}\label{random-selection}
\addcontentsline{toc}{subsubsection}{Random selection}

\subsection*{Effect of sample size}\label{effect-of-sample-size}
\addcontentsline{toc}{subsection}{Effect of sample size}

\subsection{Take-home messages}\label{take-home-messages-1}

\subsection{Additional notes}\label{additional-notes-1}

\chapter{Probability}\label{probability}

Placeholder

\section{Vocabulary Review and Key Topics}\label{vocabulary-review-and-key-topics-1}

\subsection{Key topics}\label{key-topics-1}

\subsection{Vocabulary}\label{vocabulary-1}

\section{Video Notes: Probability}\label{video-notes-probability}

\subsection{Course Videos}\label{course-videos-1}

\subsection*{Probability}\label{probability-1}
\addcontentsline{toc}{subsection}{Probability}

\subsubsection*{Creating a hypothetical two-way table}\label{creating-a-hypothetical-two-way-table}
\addcontentsline{toc}{subsubsection}{Creating a hypothetical two-way table}

\subsubsection*{Diagnostic tests}\label{diagnostic-tests}
\addcontentsline{toc}{subsubsection}{Diagnostic tests}

\subsection{Concept Check}\label{concept-check-1}

\section{Activity 3: Probability Studies}\label{activity-3-probability-studies}

\subsection{Learning outcomes}\label{learning-outcomes-2}

\subsection{Terminology review}\label{terminology-review-2}

\subsection*{Notes on probability}\label{notes-on-probability}
\addcontentsline{toc}{subsection}{Notes on probability}

\subsubsection*{Probability notation}\label{probability-notation}
\addcontentsline{toc}{subsubsection}{Probability notation}

\subsubsection*{Probability questions}\label{probability-questions}
\addcontentsline{toc}{subsubsection}{Probability questions}

\subsection*{Calculating probabilities from a two-way table}\label{calculating-probabilities-from-a-two-way-table}
\addcontentsline{toc}{subsection}{Calculating probabilities from a two-way table}

\subsection{Take home messages}\label{take-home-messages-2}

\subsection{Additional notes}\label{additional-notes-2}

\chapter{Exploring Categorical Data: Exploratory Data Analysis and Inference using Simulation-based Methods}\label{exploring-categorical-data-exploratory-data-analysis-and-inference-using-simulation-based-methods}

Placeholder

\section{Vocabulary Review and Key Topics}\label{vocabulary-review-and-key-topics-2}

\subsection{Key topics}\label{key-topics-2}

\subsubsection*{Steps of the statistical investigation process}\label{steps-of-the-statistical-investigation-process}
\addcontentsline{toc}{subsubsection}{Steps of the statistical investigation process}

\subsection{Vocabulary}\label{vocabulary-2}

\subsubsection*{Plotting one categorical variable}\label{plotting-one-categorical-variable}
\addcontentsline{toc}{subsubsection}{Plotting one categorical variable}

\subsubsection*{Inference}\label{inference}
\addcontentsline{toc}{subsubsection}{Inference}

\subsubsection*{Simulation-based inference for a single proportion}\label{simulation-based-inference-for-a-single-proportion}
\addcontentsline{toc}{subsubsection}{Simulation-based inference for a single proportion}

\section{Video Notes: Exploratory Data Analysis of Categorical Variables}\label{video-notes-exploratory-data-analysis-of-categorical-variables}

\subsection{Course Videos}\label{course-videos-2}

\subsection*{Summarizing categorical data - Video 4.1\_OneProp}\label{summarizing-categorical-data---video-4.1_oneprop}
\addcontentsline{toc}{subsection}{Summarizing categorical data - Video 4.1\_OneProp}

\subsubsection*{Optional Notes: Video Example}\label{optional-notes-video-example-1}
\addcontentsline{toc}{subsubsection}{Optional Notes: Video Example}

\subsubsection*{Displaying categorical variables - Video 4.2\_OneProp}\label{displaying-categorical-variables---video-4.2_oneprop}
\addcontentsline{toc}{subsubsection}{Displaying categorical variables - Video 4.2\_OneProp}

\subsection*{Hypothesis Testing - Video Chapter9}\label{hypothesis-testing---video-chapter9}
\addcontentsline{toc}{subsection}{Hypothesis Testing - Video Chapter9}

\subsection*{Hypothesis Testing/Justice System}\label{hypothesis-testingjustice-system}
\addcontentsline{toc}{subsection}{Hypothesis Testing/Justice System}

\subsection*{Hypotheses}\label{hypotheses}
\addcontentsline{toc}{subsection}{Hypotheses}

\subsubsection*{Null hypothesis}\label{null-hypothesis}
\addcontentsline{toc}{subsubsection}{Null hypothesis}

\subsubsection*{Alternative hypothesis}\label{alternative-hypothesis}
\addcontentsline{toc}{subsubsection}{Alternative hypothesis}

\subsection*{Simulation vs.~Theory-based Methods}\label{simulation-vs.-theory-based-methods}
\addcontentsline{toc}{subsection}{Simulation vs.~Theory-based Methods}

\subsubsection*{Simulation-based method}\label{simulation-based-method}
\addcontentsline{toc}{subsubsection}{Simulation-based method}

\subsubsection*{Theory-based method}\label{theory-based-method}
\addcontentsline{toc}{subsubsection}{Theory-based method}

\subsubsection*{P-value}\label{p-value}
\addcontentsline{toc}{subsubsection}{P-value}

\subsubsection*{Hypothesis testing}\label{hypothesis-testing}
\addcontentsline{toc}{subsubsection}{Hypothesis testing}

\subsection*{Confidence interval - Video Chapter10}\label{confidence-interval---video-chapter10}
\addcontentsline{toc}{subsection}{Confidence interval - Video Chapter10}

\subsubsection*{Sampling distribution}\label{sampling-distribution}
\addcontentsline{toc}{subsubsection}{Sampling distribution}

\subsubsection*{Simulation-based methods}\label{simulation-based-methods}
\addcontentsline{toc}{subsubsection}{Simulation-based methods}

\subsubsection*{Optional Notes: Video Example (Video 14.1)}\label{optional-notes-video-example-video-14.1}
\addcontentsline{toc}{subsubsection}{Optional Notes: Video Example (Video 14.1)}

\subsubsection*{Simulation-based method}\label{simulation-based-method-1}
\addcontentsline{toc}{subsubsection}{Simulation-based method}

\subsubsection*{Optional Notes: Video Example (Video 14.2)}\label{optional-notes-video-example-video-14.2}
\addcontentsline{toc}{subsubsection}{Optional Notes: Video Example (Video 14.2)}

\subsection{Concept Check}\label{concept-check-2}

\section{Activity 4: Helper-Hinderer Part 1 --- Simulation-based Hypothesis Test}\label{activity-4-helper-hinderer-part-1-simulation-based-hypothesis-test}

\subsection{Learning outcomes}\label{learning-outcomes-3}

\subsection{Terminology review}\label{terminology-review-3}

\subsection{Steps of the statistical investigation process}\label{steps-of-the-statistical-investigation-process-1}

\subsection*{Notes on one categorical variable}\label{notes-on-one-categorical-variable}
\addcontentsline{toc}{subsection}{Notes on one categorical variable}

\subsection{Helper-Hinderer}\label{helper-hinderer}

\subsubsection*{Ask a research question}\label{ask-a-research-question}
\addcontentsline{toc}{subsubsection}{Ask a research question}

\subsubsection*{Design a study and collect data}\label{design-a-study-and-collect-data}
\addcontentsline{toc}{subsubsection}{Design a study and collect data}

\subsubsection*{R Instructions}\label{r-instructions}
\addcontentsline{toc}{subsubsection}{R Instructions}

\subsubsection*{Summarize and visualize the data}\label{summarize-and-visualize-the-data}
\addcontentsline{toc}{subsubsection}{Summarize and visualize the data}

\subsubsection*{Use statistical analysis methods to draw inferences from the data}\label{use-statistical-analysis-methods-to-draw-inferences-from-the-data}
\addcontentsline{toc}{subsubsection}{Use statistical analysis methods to draw inferences from the data}

\subsubsection*{Notes on hypotheses tests}\label{notes-on-hypotheses-tests}
\addcontentsline{toc}{subsubsection}{Notes on hypotheses tests}

\subsection{Take-home messages}\label{take-home-messages-3}

\subsection{Additional notes}\label{additional-notes-3}

\section{Activity 5: Helper-Hinderer (continued)}\label{activity-5-helper-hinderer-continued}

\subsection{Learning outcomes}\label{learning-outcomes-4}

\subsection{Steps of the statistical investigation process}\label{steps-of-the-statistical-investigation-process-2}

\subsection{Helper-Hinderer}\label{helper-hinderer-1}

\subsection*{Notes on the null distribution}\label{notes-on-the-null-distribution}
\addcontentsline{toc}{subsection}{Notes on the null distribution}

\subsection*{Notes on the p-value}\label{notes-on-the-p-value}
\addcontentsline{toc}{subsection}{Notes on the p-value}

\subsubsection*{Interpret the p-value}\label{interpret-the-p-value}
\addcontentsline{toc}{subsubsection}{Interpret the p-value}

\subsubsection*{Communicate the results and answer the research question}\label{communicate-the-results-and-answer-the-research-question}
\addcontentsline{toc}{subsubsection}{Communicate the results and answer the research question}

\subsubsection*{Generalization}\label{generalization}
\addcontentsline{toc}{subsubsection}{Generalization}

\subsection{Take-home messages}\label{take-home-messages-4}

\subsection{Additional notes}\label{additional-notes-4}

\section{Activity 6: Helper-Hinderer --- Simulation-based Confidence Interval}\label{activity-6-helper-hinderer-simulation-based-confidence-interval}

\subsection{Learning outcomes}\label{learning-outcomes-5}

\subsection{Terminology review}\label{terminology-review-4}

\subsection{Helper-Hinderer}\label{helper-hinderer-2}

\subsubsection*{Activity intro}\label{activity-intro}
\addcontentsline{toc}{subsubsection}{Activity intro}

\subsection*{Notes on Confidence Intervals}\label{notes-on-confidence-intervals}
\addcontentsline{toc}{subsection}{Notes on Confidence Intervals}

\subsubsection*{Use statistical analysis methods to draw inferences from the data}\label{use-statistical-analysis-methods-to-draw-inferences-from-the-data-1}
\addcontentsline{toc}{subsubsection}{Use statistical analysis methods to draw inferences from the data}

\subsubsection*{Notes on the bootstrap distribution}\label{notes-on-the-bootstrap-distribution}
\addcontentsline{toc}{subsubsection}{Notes on the bootstrap distribution}

\subsubsection*{Communicate the results and answer the research question}\label{communicate-the-results-and-answer-the-research-question-1}
\addcontentsline{toc}{subsubsection}{Communicate the results and answer the research question}

\subsubsection*{Effect of confidence level}\label{effect-of-confidence-level}
\addcontentsline{toc}{subsubsection}{Effect of confidence level}

\subsubsection*{Concluding paragraph}\label{concluding-paragraph}
\addcontentsline{toc}{subsubsection}{Concluding paragraph}

\subsection{Take-home messages}\label{take-home-messages-5}

\subsection{Additional notes}\label{additional-notes-5}

\chapter{Inference for a Single Categorical Variable: Theory-based Methods}\label{inference-for-a-single-categorical-variable-theory-based-methods}

Placeholder

\section{Vocabulary Review and Key Topics}\label{vocabulary-review-and-key-topics-3}

\subsection{Key topics}\label{key-topics-3}

\subsection{Vocabulary}\label{vocabulary-3}

\section{Video Notes: Inference for One Categorical Variable using Theory-based Methods}\label{video-notes-inference-for-one-categorical-variable-using-theory-based-methods}

\subsection{Course Videos}\label{course-videos-3}

\subsection*{Theory-based methods}\label{theory-based-methods}
\addcontentsline{toc}{subsection}{Theory-based methods}

\subsubsection*{Central limit theorem - Video Chapter11}\label{central-limit-theorem---video-chapter11}
\addcontentsline{toc}{subsubsection}{Central limit theorem - Video Chapter11}

\subsection*{68-95-99.7 Rule}\label{rule}
\addcontentsline{toc}{subsection}{68-95-99.7 Rule}

\subsubsection*{Theoretical Testing for a Single Proportion - Video 14.3TheoryTests}\label{theoretical-testing-for-a-single-proportion---video-14.3theorytests}
\addcontentsline{toc}{subsubsection}{Theoretical Testing for a Single Proportion - Video 14.3TheoryTests}

\paragraph*{Optional Notes: Video Example (Video 14.3TheoryTests)}\label{optional-notes-video-example-video-14.3theorytests}
\addcontentsline{toc}{paragraph}{Optional Notes: Video Example (Video 14.3TheoryTests)}

\subsubsection*{Theoretical Confidence Intervals for a Single Proportion - Video 14.3TheoryIntervals}\label{theoretical-confidence-intervals-for-a-single-proportion---video-14.3theoryintervals}
\addcontentsline{toc}{subsubsection}{Theoretical Confidence Intervals for a Single Proportion - Video 14.3TheoryIntervals}

\subsubsection*{Theory-based method for a single categorical variable}\label{theory-based-method-for-a-single-categorical-variable}
\addcontentsline{toc}{subsubsection}{Theory-based method for a single categorical variable}

\paragraph*{Optional Notes: Video Example (Video 14.3TheoryIntervals)}\label{optional-notes-video-example-video-14.3theoryintervals}
\addcontentsline{toc}{paragraph}{Optional Notes: Video Example (Video 14.3TheoryIntervals)}

\subsubsection*{Interpreting confidence level - Video 14.3TheoryIntervals}\label{interpreting-confidence-level---video-14.3theoryintervals}
\addcontentsline{toc}{subsubsection}{Interpreting confidence level - Video 14.3TheoryIntervals}

\subsection{Concept Check}\label{concept-check-3}

\section{Activity 7: Handedness of Male Boxers}\label{activity-7-handedness-of-male-boxers}

\subsection{Learning outcomes}\label{learning-outcomes-6}

\subsection{Terminology review}\label{terminology-review-5}

\subsection{Handedness of male boxers}\label{handedness-of-male-boxers}

\subsection*{R Instructions}\label{r-instructions-1}
\addcontentsline{toc}{subsection}{R Instructions}

\subsection*{Hypotheses and summary statistics}\label{hypotheses-and-summary-statistics}
\addcontentsline{toc}{subsection}{Hypotheses and summary statistics}

\subsection*{Theory-based methods}\label{theory-based-methods-1}
\addcontentsline{toc}{subsection}{Theory-based methods}

\subsection*{Additional notes on Theory-based methods}\label{additional-notes-on-theory-based-methods}
\addcontentsline{toc}{subsection}{Additional notes on Theory-based methods}

\subsection*{Impacts on the p-value}\label{impacts-on-the-p-value}
\addcontentsline{toc}{subsection}{Impacts on the p-value}

\subsection{Take-home messages}\label{take-home-messages-6}

\subsection{Additional notes}\label{additional-notes-6}

\section{Activity 8: Confidence intervals and what confidence means}\label{activity-8-confidence-intervals-and-what-confidence-means}

\subsection{Learning outcomes}\label{learning-outcomes-7}

\subsection{Terminology review}\label{terminology-review-6}

\subsection{Handedness of male boxers continued}\label{handedness-of-male-boxers-continued}

\subsection*{\texorpdfstring{What does \emph{confidence} mean?}{What does confidence mean?}}\label{what-does-confidence-mean}
\addcontentsline{toc}{subsection}{What does \emph{confidence} mean?}

\subsubsection*{Notes on theory-based confidence intervals}\label{notes-on-theory-based-confidence-intervals}
\addcontentsline{toc}{subsubsection}{Notes on theory-based confidence intervals}

\subsubsection*{Simulation methods}\label{simulation-methods}
\addcontentsline{toc}{subsubsection}{Simulation methods}

\subsection*{Effect of sample size on the width of the confidence interval}\label{effect-of-sample-size-on-the-width-of-the-confidence-interval}
\addcontentsline{toc}{subsection}{Effect of sample size on the width of the confidence interval}

\subsection{Take-home messages}\label{take-home-messages-7}

\subsection{Additional notes}\label{additional-notes-7}

\section{Module 3 and 4 Lab: Mixed Breed Dogs in the U.S.}\label{module-3-and-4-lab-mixed-breed-dogs-in-the-u.s.}

\subsection{Learning outcomes}\label{learning-outcomes-8}

\subsection{Mixed Breed Dogs in the U.S.}\label{mixed-breed-dogs-in-the-u.s.}

\subsubsection*{R Instructions}\label{r-instructions-2}
\addcontentsline{toc}{subsubsection}{R Instructions}

\subsubsection*{Use statistical analysis methods to draw inferences from the data}\label{use-statistical-analysis-methods-to-draw-inferences-from-the-data-2}
\addcontentsline{toc}{subsubsection}{Use statistical analysis methods to draw inferences from the data}

\subsubsection*{Null Distribution}\label{null-distribution}
\addcontentsline{toc}{subsubsection}{Null Distribution}

\subsubsection*{Bootstrap distribution}\label{bootstrap-distribution}
\addcontentsline{toc}{subsubsection}{Bootstrap distribution}

\subsubsection*{Summarize the results of the study}\label{summarize-the-results-of-the-study}
\addcontentsline{toc}{subsubsection}{Summarize the results of the study}

\chapter{Unit 1 Review}\label{unit-1-review}

The following module contains both a list of key topics covered in Unit 1 as well as Module Review Worksheets that will be covered in Weekly Review Sessions.

\subsection{Key Topics}\label{key-topics-4}

Review the key topics for Unit 1 prior to the first exams. All of these topics will be covered in Modules 1--4.

\subsection{Module Review}\label{module-review}

\setstretch{1}

The following worksheets review each of the modules. These worksheets will be completed during Melinda's Study Sessions each week. Solutions will be posted on Canvas in the Unit 1 Review folder after the study sessions.

\newpage

\section{Key Topics Exam 1}\label{key-topics-exam-1}

\subsection*{Descriptive statistics and study design}\label{descriptive-statistics-and-study-design}
\addcontentsline{toc}{subsection}{Descriptive statistics and study design}

\subsection*{Hypothesis testing}\label{hypothesis-testing-1}
\addcontentsline{toc}{subsection}{Hypothesis testing}

\subsection{Confidence intervals}\label{confidence-intervals}

\subsection{Probability}\label{probability-2}

\section{Module 1 Review - Sampling Methods}\label{module-1-review---sampling-methods}

\section{Module 2 Review - Probability}\label{module-2-review---probability}

\section{Module 3 Review - Simulation Methods for a Single Proportion}\label{module-3-review---simulation-methods-for-a-single-proportion}

\section{Module 4 Review - Theory-based Methods for a Single Proportion}\label{module-4-review---theory-based-methods-for-a-single-proportion}

\section{Group Exam 1 Review}\label{group-exam-1-review}

\chapter{Exploring Quantitative Data: Exploratory Data Analysis and Inference for a Single Quantitative Variable - Simulation-based Methods}\label{exploring-quantitative-data-exploratory-data-analysis-and-inference-for-a-single-quantitative-variable---simulation-based-methods}

Placeholder

\section{Vocabulary Review and Key Topics}\label{vocabulary-review-and-key-topics-4}

\subsection{Key topics}\label{key-topics-5}

\subsection{Vocabulary}\label{vocabulary-4}

\subsubsection*{Sample statistics for a single quantitative variable}\label{sample-statistics-for-a-single-quantitative-variable}
\addcontentsline{toc}{subsubsection}{Sample statistics for a single quantitative variable}

\subsubsection*{Plotting one quantitative variable}\label{plotting-one-quantitative-variable}
\addcontentsline{toc}{subsubsection}{Plotting one quantitative variable}

\subsubsection*{Hypothesis testing for a single mean}\label{hypothesis-testing-for-a-single-mean}
\addcontentsline{toc}{subsubsection}{Hypothesis testing for a single mean}

\subsubsection*{Simulation-based hypothesis testing}\label{simulation-based-hypothesis-testing}
\addcontentsline{toc}{subsubsection}{Simulation-based hypothesis testing}

\subsubsection*{Simulation-based confidence interval}\label{simulation-based-confidence-interval}
\addcontentsline{toc}{subsubsection}{Simulation-based confidence interval}

\section{Video Notes: Exploratory Data Analysis and Hypothesis Testing of Quantitative Variables}\label{video-notes-exploratory-data-analysis-and-hypothesis-testing-of-quantitative-variables}

\subsection{Course Videos}\label{course-videos-4}

\subsection*{Summarizing quantitative data - Video 5.2to5.4}\label{summarizing-quantitative-data---video-5.2to5.4}
\addcontentsline{toc}{subsection}{Summarizing quantitative data - Video 5.2to5.4}

\subsubsection*{Types of plots}\label{types-of-plots}
\addcontentsline{toc}{subsubsection}{Types of plots}

\subsection*{Summarizing quantitative data - Video 5.5}\label{summarizing-quantitative-data---video-5.5}
\addcontentsline{toc}{subsection}{Summarizing quantitative data - Video 5.5}

\subsubsection*{Four characteristics of plots for quantitative variables}\label{four-characteristics-of-plots-for-quantitative-variables}
\addcontentsline{toc}{subsubsection}{Four characteristics of plots for quantitative variables}

\subsection*{Robust statistics - Video 5.7}\label{robust-statistics---video-5.7}
\addcontentsline{toc}{subsection}{Robust statistics - Video 5.7}

\subsection*{Simulation-based Testing for a Single Mean - Video 17.2}\label{simulation-based-testing-for-a-single-mean---video-17.2}
\addcontentsline{toc}{subsection}{Simulation-based Testing for a Single Mean - Video 17.2}

\subsubsection*{Hypothesis testing}\label{hypothesis-testing-2}
\addcontentsline{toc}{subsubsection}{Hypothesis testing}

\subsubsection*{Simulation-based method}\label{simulation-based-method-2}
\addcontentsline{toc}{subsubsection}{Simulation-based method}

\subsection*{Optional Notes: Video Example (Video 17.2)}\label{optional-notes-video-example-video-17.2}
\addcontentsline{toc}{subsection}{Optional Notes: Video Example (Video 17.2)}

\subsection*{Simuation-based Confidence Intervals for a Single Mean - Video 17.1}\label{simuation-based-confidence-intervals-for-a-single-mean---video-17.1}
\addcontentsline{toc}{subsection}{Simuation-based Confidence Intervals for a Single Mean - Video 17.1}

\subsection*{Confidence interval}\label{confidence-interval}
\addcontentsline{toc}{subsection}{Confidence interval}

\subsubsection*{Simulation-based method}\label{simulation-based-method-3}
\addcontentsline{toc}{subsubsection}{Simulation-based method}

\subsection{Concept Check}\label{concept-check-4}

\section{Activity 9: Summarizing Quantitative Variables}\label{activity-9-summarizing-quantitative-variables}

\subsection{Learning outcomes}\label{learning-outcomes-9}

\subsection{Terminology review}\label{terminology-review-7}

\subsection{The Integrated Postsecondary Education Data System (IPEDS)}\label{the-integrated-postsecondary-education-data-system-ipeds}

\subsubsection*{Identifying variables in a data set}\label{identifying-variables-in-a-data-set}
\addcontentsline{toc}{subsubsection}{Identifying variables in a data set}

\subsubsection*{Notes on Summarizing Quantitative Variables:}\label{notes-on-summarizing-quantitative-variables}
\addcontentsline{toc}{subsubsection}{Notes on Summarizing Quantitative Variables:}

\subsubsection*{R Instructions}\label{r-instructions-3}
\addcontentsline{toc}{subsubsection}{R Instructions}

\subsubsection*{Displaying a single quantitative variable}\label{displaying-a-single-quantitative-variable}
\addcontentsline{toc}{subsubsection}{Displaying a single quantitative variable}

\subsubsection*{Robust statistics}\label{robust-statistics}
\addcontentsline{toc}{subsubsection}{Robust statistics}

\subsection{Take-home messages}\label{take-home-messages-8}

\subsection{Additional notes}\label{additional-notes-8}

\section{Activity 10: Inference for a Single Quantitative Variable: Simulation Methods}\label{activity-10-inference-for-a-single-quantitative-variable-simulation-methods}

\subsection{Learning outcomes}\label{learning-outcomes-10}

\subsection{Terminology review}\label{terminology-review-8}

\subsection{College student sleep habits}\label{college-student-sleep-habits}

\subsubsection*{R Instructions}\label{r-instructions-4}
\addcontentsline{toc}{subsubsection}{R Instructions}

\subsubsection*{Ask a research question}\label{ask-a-research-question-1}
\addcontentsline{toc}{subsubsection}{Ask a research question}

\subsubsection*{Summarize and visualize the data}\label{summarize-and-visualize-the-data-1}
\addcontentsline{toc}{subsubsection}{Summarize and visualize the data}

\subsection*{Simulation methods}\label{simulation-methods-1}
\addcontentsline{toc}{subsection}{Simulation methods}

\subsubsection*{Notes on simulation methods for a single mean}\label{notes-on-simulation-methods-for-a-single-mean}
\addcontentsline{toc}{subsubsection}{Notes on simulation methods for a single mean}

\subsubsection*{Notes on the null distribution}\label{notes-on-the-null-distribution-1}
\addcontentsline{toc}{subsubsection}{Notes on the null distribution}

\subsection*{Simulation methods to create a confidence interval}\label{simulation-methods-to-create-a-confidence-interval}
\addcontentsline{toc}{subsection}{Simulation methods to create a confidence interval}

\subsubsection*{Notes on the bootstrap distribution}\label{notes-on-the-bootstrap-distribution-1}
\addcontentsline{toc}{subsubsection}{Notes on the bootstrap distribution}

\subsection{Take-home messages}\label{take-home-messages-9}

\subsection{Additional notes}\label{additional-notes-9}

\chapter{Exploring Quantitative Data: Inference for a Single Quantitative Variable - Theory-based Methods}\label{exploring-quantitative-data-inference-for-a-single-quantitative-variable---theory-based-methods}

Placeholder

\section{Vocabulary Review and Key Topics}\label{vocabulary-review-and-key-topics-5}

\subsection{Key topics}\label{key-topics-6}

\subsubsection*{Theory-based hypothesis testing}\label{theory-based-hypothesis-testing}
\addcontentsline{toc}{subsubsection}{Theory-based hypothesis testing}

\subsubsection*{Theory-based confidence interval}\label{theory-based-confidence-interval}
\addcontentsline{toc}{subsubsection}{Theory-based confidence interval}

\subsection*{Vocabulary}\label{vocabulary-5}
\addcontentsline{toc}{subsection}{Vocabulary}

\section{Video Notes: Theory-based Inference for a single quantitative variable}\label{video-notes-theory-based-inference-for-a-single-quantitative-variable}

\subsection{Course Videos}\label{course-videos-5}

\subsection*{Theory-based Testing for a Single Mean - Video 17.3TheoryTests}\label{theory-based-testing-for-a-single-mean---video-17.3theorytests}
\addcontentsline{toc}{subsection}{Theory-based Testing for a Single Mean - Video 17.3TheoryTests}

\subsubsection*{\texorpdfstring{\(t\)-distribution}{t-distribution}}\label{t-distribution}
\addcontentsline{toc}{subsubsection}{\(t\)-distribution}

\subsection*{Optional Notes: Video Example (Video 17.3TheoryTests)}\label{optional-notes-video-example-video-17.3theorytests}
\addcontentsline{toc}{subsection}{Optional Notes: Video Example (Video 17.3TheoryTests)}

\subsection*{Theory-based Confidence Interval for a Single Mean - Video 17.3TheoryIntervals}\label{theory-based-confidence-interval-for-a-single-mean---video-17.3theoryintervals}
\addcontentsline{toc}{subsection}{Theory-based Confidence Interval for a Single Mean - Video 17.3TheoryIntervals}

\subsection*{Decisions, Errors, and Power - Video Chapter12}\label{decisions-errors-and-power---video-chapter12}
\addcontentsline{toc}{subsection}{Decisions, Errors, and Power - Video Chapter12}

\subsection{Concept Check}\label{concept-check-5}

\section{Activity 11: Body Temperature}\label{activity-11-body-temperature}

\subsection{Learning outcomes}\label{learning-outcomes-11}

\subsection{Terminology review}\label{terminology-review-9}

\subsection{Body Temperature}\label{body-temperature}

\subsubsection*{Ask a research question}\label{ask-a-research-question-2}
\addcontentsline{toc}{subsubsection}{Ask a research question}

\subsubsection*{Summarize and visualize the data}\label{summarize-and-visualize-the-data-2}
\addcontentsline{toc}{subsubsection}{Summarize and visualize the data}

\subsubsection*{Check theoretical conditions}\label{check-theoretical-conditions}
\addcontentsline{toc}{subsubsection}{Check theoretical conditions}

\subsubsection*{Use statistical inferential methods to draw inferences from the data}\label{use-statistical-inferential-methods-to-draw-inferences-from-the-data}
\addcontentsline{toc}{subsubsection}{Use statistical inferential methods to draw inferences from the data}

\subsection*{Theory-based methods to create a confidence interval}\label{theory-based-methods-to-create-a-confidence-interval}
\addcontentsline{toc}{subsection}{Theory-based methods to create a confidence interval}

\subsection{Take-home messages}\label{take-home-messages-10}

\subsection{Additional notes}\label{additional-notes-10}

\section{Activity 12: Errors and Power}\label{activity-12-errors-and-power}

\subsection{Learning outcomes}\label{learning-outcomes-12}

\subsection{Terminology review}\label{terminology-review-10}

\subsection*{Notes on types of errors and power}\label{notes-on-types-of-errors-and-power}
\addcontentsline{toc}{subsection}{Notes on types of errors and power}

\subsection{College textbook cost}\label{college-textbook-cost}

\subsection{Take-home messages}\label{take-home-messages-11}

\subsection{Additional notes}\label{additional-notes-11}

\section{Module 6 and 7 Lab: Arsenic}\label{module-6-and-7-lab-arsenic}

\subsection{Learning outcomes}\label{learning-outcomes-13}

\subsection{Arsenic}\label{arsenic}

\subsubsection*{R Instructions}\label{r-instructions-5}
\addcontentsline{toc}{subsubsection}{R Instructions}

\subsection*{Use statistical inferential methods to draw inferences from the data}\label{use-statistical-inferential-methods-to-draw-inferences-from-the-data-1}
\addcontentsline{toc}{subsection}{Use statistical inferential methods to draw inferences from the data}

\subsection*{Hypothesis test}\label{hypothesis-test}
\addcontentsline{toc}{subsection}{Hypothesis test}

\subsection*{Communicate the results and answer the research question}\label{communicate-the-results-and-answer-the-research-question-2}
\addcontentsline{toc}{subsection}{Communicate the results and answer the research question}

\subsection*{Confidence interval}\label{confidence-interval-1}
\addcontentsline{toc}{subsection}{Confidence interval}

\chapter{Exploratory Data Analysis and Simulation-based Inference for Two Categorical Variables}\label{exploratory-data-analysis-and-simulation-based-inference-for-two-categorical-variables}

Placeholder

\section{Vocabulary Review and Key Topics}\label{vocabulary-review-and-key-topics-6}

\subsection{Key topics}\label{key-topics-7}

\subsection{Vocabulary}\label{vocabulary-6}

\subsubsection*{Sample statistics for two categorical variables}\label{sample-statistics-for-two-categorical-variables}
\addcontentsline{toc}{subsubsection}{Sample statistics for two categorical variables}

\subsubsection*{Plotting two categorical variables}\label{plotting-two-categorical-variables}
\addcontentsline{toc}{subsubsection}{Plotting two categorical variables}

\subsubsection*{Hypotheses}\label{hypotheses-1}
\addcontentsline{toc}{subsubsection}{Hypotheses}

\subsubsection*{Simulation-based hypothesis testing for a difference in proportions}\label{simulation-based-hypothesis-testing-for-a-difference-in-proportions}
\addcontentsline{toc}{subsubsection}{Simulation-based hypothesis testing for a difference in proportions}

\subsubsection*{Simulation-based confidence interval}\label{simulation-based-confidence-interval-1}
\addcontentsline{toc}{subsubsection}{Simulation-based confidence interval}

\subsubsection*{Study design}\label{study-design}
\addcontentsline{toc}{subsubsection}{Study design}

\subsubsection*{Scope of inference}\label{scope-of-inference}
\addcontentsline{toc}{subsubsection}{Scope of inference}

\section{Video Notes: Inference for Two Categorical Variables using Simulation-based Methods}\label{video-notes-inference-for-two-categorical-variables-using-simulation-based-methods}

\subsection{Course Videos}\label{course-videos-6}

\subsection*{Relationships between variables - Video 1.2.3to1.2.5}\label{relationships-between-variables---video-1.2.3to1.2.5}
\addcontentsline{toc}{subsection}{Relationships between variables - Video 1.2.3to1.2.5}

\subsubsection*{Relationships between variables}\label{relationships-between-variables}
\addcontentsline{toc}{subsubsection}{Relationships between variables}

\subsection*{Observational studies, experiments, and scope of inference: Video 2.2to2.4}\label{observational-studies-experiments-and-scope-of-inference-video-2.2to2.4}
\addcontentsline{toc}{subsection}{Observational studies, experiments, and scope of inference: Video 2.2to2.4}

\subsubsection*{Study design}\label{study-design-1}
\addcontentsline{toc}{subsubsection}{Study design}

\subsection*{Optional Notes: Video Examples (Video 2.2to2.4)}\label{optional-notes-video-examples-video-2.2to2.4}
\addcontentsline{toc}{subsection}{Optional Notes: Video Examples (Video 2.2to2.4)}

\subsubsection*{Scope of Inference}\label{scope-of-inference-1}
\addcontentsline{toc}{subsubsection}{Scope of Inference}

\subsection*{Summarizing two categorical variables - Video 4.1\_TwoProp}\label{summarizing-two-categorical-variables---video-4.1_twoprop}
\addcontentsline{toc}{subsection}{Summarizing two categorical variables - Video 4.1\_TwoProp}

\subsection*{Plots for two categorical variables - Video 4.2\_TwoProp}\label{plots-for-two-categorical-variables---video-4.2_twoprop}
\addcontentsline{toc}{subsection}{Plots for two categorical variables - Video 4.2\_TwoProp}

\subsection*{Simpson's paradox - Video 4.4}\label{simpsons-paradox---video-4.4}
\addcontentsline{toc}{subsection}{Simpson's paradox - Video 4.4}

\subsection*{Simulation Testing for a Difference in Proportions - Video 15.1}\label{simulation-testing-for-a-difference-in-proportions---video-15.1}
\addcontentsline{toc}{subsection}{Simulation Testing for a Difference in Proportions - Video 15.1}

\subsubsection*{Hypothesis Testing}\label{hypothesis-testing-3}
\addcontentsline{toc}{subsubsection}{Hypothesis Testing}

\subsection*{Optional Notes: Video Example (Video 15.1)}\label{optional-notes-video-example-video-15.1}
\addcontentsline{toc}{subsection}{Optional Notes: Video Example (Video 15.1)}

\subsubsection*{Summary statistics and plot}\label{summary-statistics-and-plot}
\addcontentsline{toc}{subsubsection}{Summary statistics and plot}

\subsubsection*{Simulation-based method}\label{simulation-based-method-4}
\addcontentsline{toc}{subsubsection}{Simulation-based method}

\subsection*{Confidence interval for a Difference in Proportion - Video 15.2}\label{confidence-interval-for-a-difference-in-proportion---video-15.2}
\addcontentsline{toc}{subsection}{Confidence interval for a Difference in Proportion - Video 15.2}

\subsubsection*{Simulation-based method}\label{simulation-based-method-5}
\addcontentsline{toc}{subsubsection}{Simulation-based method}

\subsection*{Optional Notes: Video Example (Video 15.2)}\label{optional-notes-video-example-video-15.2}
\addcontentsline{toc}{subsection}{Optional Notes: Video Example (Video 15.2)}

\subsection*{Relative Risk - Video RelativeRisk}\label{relative-risk---video-relativerisk}
\addcontentsline{toc}{subsection}{Relative Risk - Video RelativeRisk}

\subsection*{Optional Notes: Video Example (Video RelativeRisk)}\label{optional-notes-video-example-video-relativerisk}
\addcontentsline{toc}{subsection}{Optional Notes: Video Example (Video RelativeRisk)}

\subsubsection*{Relative risk in the news}\label{relative-risk-in-the-news}
\addcontentsline{toc}{subsubsection}{Relative risk in the news}

\subsubsection*{Testing Relative Risk}\label{testing-relative-risk}
\addcontentsline{toc}{subsubsection}{Testing Relative Risk}

\subsection{Concept Check}\label{concept-check-6}

\section{Activity 13: Study Design}\label{activity-13-study-design}

\subsection{Learning outcomes}\label{learning-outcomes-14}

\subsection{Terminology review}\label{terminology-review-11}

\subsection*{Notes on study design}\label{notes-on-study-design}
\addcontentsline{toc}{subsection}{Notes on study design}

\subsection{Atrial fibrillation}\label{atrial-fibrillation}

\subsubsection*{Notes on scope of inference}\label{notes-on-scope-of-inference}
\addcontentsline{toc}{subsubsection}{Notes on scope of inference}

\subsection{Scope of Inference}\label{scope-of-inference-2}

\subsection{Take-home messages}\label{take-home-messages-12}

\subsection{Additional notes}\label{additional-notes-12}

\section{Activity 14: Summarizing Two Categorical Variables}\label{activity-14-summarizing-two-categorical-variables}

\subsection{Learning outcomes}\label{learning-outcomes-15}

\subsection{Terminology review}\label{terminology-review-12}

\subsection{Graphing categorical variables}\label{graphing-categorical-variables}

\subsection*{Nightlight use and myopia}\label{nightlight-use-and-myopia}
\addcontentsline{toc}{subsection}{Nightlight use and myopia}

\subsubsection*{Notes on two categorical variables}\label{notes-on-two-categorical-variables}
\addcontentsline{toc}{subsubsection}{Notes on two categorical variables}

\subsubsection*{R Instructions}\label{r-instructions-6}
\addcontentsline{toc}{subsubsection}{R Instructions}

\subsubsection*{Summarizing two categorical variables}\label{summarizing-two-categorical-variables}
\addcontentsline{toc}{subsubsection}{Summarizing two categorical variables}

\subsubsection*{Displaying two categorical variables}\label{displaying-two-categorical-variables}
\addcontentsline{toc}{subsubsection}{Displaying two categorical variables}

\subsubsection*{Relative Risk}\label{relative-risk}
\addcontentsline{toc}{subsubsection}{Relative Risk}

\subsection{Take-home messages}\label{take-home-messages-13}

\subsection{Additional notes}\label{additional-notes-13}

\section{Activity 15: The Good Samaritan}\label{activity-15-the-good-samaritan}

\subsection{Learning outcomes}\label{learning-outcomes-16}

\subsection{Terminology review}\label{terminology-review-13}

\subsection{The Good Samaritan}\label{the-good-samaritan}

\subsubsection*{R Instructions}\label{r-instructions-7}
\addcontentsline{toc}{subsubsection}{R Instructions}

\subsubsection*{Ask a research question}\label{ask-a-research-question-3}
\addcontentsline{toc}{subsubsection}{Ask a research question}

\subsubsection*{Summarize and visualize the data}\label{summarize-and-visualize-the-data-3}
\addcontentsline{toc}{subsubsection}{Summarize and visualize the data}

\subsubsection*{Hypothesis Test}\label{hypothesis-test-1}
\addcontentsline{toc}{subsubsection}{Hypothesis Test}

\subsubsection*{Notes on the null distribution}\label{notes-on-the-null-distribution-2}
\addcontentsline{toc}{subsubsection}{Notes on the null distribution}

\subsubsection*{Confidence interval}\label{confidence-interval-2}
\addcontentsline{toc}{subsubsection}{Confidence interval}

\subsubsection*{Bootstrap distribution}\label{bootstrap-distribution-1}
\addcontentsline{toc}{subsubsection}{Bootstrap distribution}

\subsubsection*{Notes on the bootstrap distribution}\label{notes-on-the-bootstrap-distribution-2}
\addcontentsline{toc}{subsubsection}{Notes on the bootstrap distribution}

\subsection{Take-home messages}\label{take-home-messages-14}

\subsection{Additional notes}\label{additional-notes-14}

\chapter{Theory-based Hypothesis Testing and Confidence Intervals for Two Categorical Variables:}\label{theory-based-hypothesis-testing-and-confidence-intervals-for-two-categorical-variables}

Placeholder

\section{Vocabulary Review and Key Topics}\label{vocabulary-review-and-key-topics-7}

\subsection{Key topics}\label{key-topics-8}

\subsection{Vocabulary}\label{vocabulary-7}

\subsubsection*{Theory-based inference}\label{theory-based-inference}
\addcontentsline{toc}{subsubsection}{Theory-based inference}

\section{Video Notes: Theoretical Inference for Two Categorical Variables}\label{video-notes-theoretical-inference-for-two-categorical-variables}

\subsection{Course Videos}\label{course-videos-7}

\subsection*{Theoretical Testing for a Difference in Proportion - Video 15.4TheoryTests}\label{theoretical-testing-for-a-difference-in-proportion---video-15.4theorytests}
\addcontentsline{toc}{subsection}{Theoretical Testing for a Difference in Proportion - Video 15.4TheoryTests}

\subsection*{Optional Notes: Video Example (Video 15.3TheoryTests)}\label{optional-notes-video-example-video-15.3theorytests}
\addcontentsline{toc}{subsection}{Optional Notes: Video Example (Video 15.3TheoryTests)}

\subsection*{Theoretical Confidence Interval for a Difference in Proportion - Video 15.3TheoryIntervals}\label{theoretical-confidence-interval-for-a-difference-in-proportion---video-15.3theoryintervals}
\addcontentsline{toc}{subsection}{Theoretical Confidence Interval for a Difference in Proportion - Video 15.3TheoryIntervals}

\subsubsection*{Theory-based method for a two categorical variables}\label{theory-based-method-for-a-two-categorical-variables}
\addcontentsline{toc}{subsubsection}{Theory-based method for a two categorical variables}

\subsection*{Optional Notes: Video Example (Video 15.3TheoryIntervals)}\label{optional-notes-video-example-video-15.3theoryintervals}
\addcontentsline{toc}{subsection}{Optional Notes: Video Example (Video 15.3TheoryIntervals)}

\subsection{Concept Check}\label{concept-check-7}

\section{Activity 16: Winter Sports Helmet Use and Head Injuries --- Theory-based Methods}\label{activity-16-winter-sports-helmet-use-and-head-injuries-theory-based-methods}

\subsection{Learning outcomes}\label{learning-outcomes-17}

\subsection{Terminology review}\label{terminology-review-14}

\subsection{Winter sports helmet use and head injury}\label{winter-sports-helmet-use-and-head-injury}

\subsubsection*{R Instructions}\label{r-instructions-8}
\addcontentsline{toc}{subsubsection}{R Instructions}

\subsubsection*{Hypothesis test}\label{hypothesis-test-2}
\addcontentsline{toc}{subsubsection}{Hypothesis test}

\subsubsection*{Use statistical analysis methods to draw inferences from the data}\label{use-statistical-analysis-methods-to-draw-inferences-from-the-data-3}
\addcontentsline{toc}{subsubsection}{Use statistical analysis methods to draw inferences from the data}

\subsubsection*{Confidence Interval}\label{confidence-interval-3}
\addcontentsline{toc}{subsubsection}{Confidence Interval}

\subsection{Effect of sample size}\label{effect-of-sample-size-1}

\subsection{Take-home messages}\label{take-home-messages-15}

\subsection{Additional notes}\label{additional-notes-15}

\section{Module 8 and 9 Lab: Poisonous Mushrooms}\label{module-8-and-9-lab-poisonous-mushrooms}

\subsection{Learning outcomes}\label{learning-outcomes-18}

\subsection{Poisonous Mushrooms}\label{poisonous-mushrooms}

\subsubsection*{R Instructions}\label{r-instructions-9}
\addcontentsline{toc}{subsubsection}{R Instructions}

\chapter{Unit 2 Review}\label{unit-2-review}

The following section contains both a list of key topics covered in Unit 2 as well as Module Review Worksheets.

\subsection{Key Topics}\label{key-topics-9}

Review the key topics for Unit 2 to review prior to the exams. All of these topics will be covered in Modules 6--9.

\subsection{Module Review}\label{module-review-1}

\setstretch{1}

The following worksheets review each of the modules. These worksheets will be completed during Melinda's Study Sessions each week. Solutions will be posted on Canvas in the Unit 2 Review folder after the study sessions.

\newpage

\section{Key Topics Exam 2}\label{key-topics-exam-2}

\subsection*{Descriptive statistics and study design}\label{descriptive-statistics-and-study-design-1}
\addcontentsline{toc}{subsection}{Descriptive statistics and study design}

\subsection*{Hypothesis testing}\label{hypothesis-testing-4}
\addcontentsline{toc}{subsection}{Hypothesis testing}

\subsection*{Confidence interval}\label{confidence-interval-4}
\addcontentsline{toc}{subsection}{Confidence interval}

\section{Module 6 Review - Simulation Methods - One Mean}\label{module-6-review---simulation-methods---one-mean}

\section{Module 7 Review - Theory-based Methods - One mean}\label{module-7-review---theory-based-methods---one-mean}

\section{Module 7 and 8 Review}\label{module-7-and-8-review}

\section{Module 8 and 9 Review}\label{module-8-and-9-review}

\section{Group Exam 2 Review}\label{group-exam-2-review}

\chapter{Exploratory Data Analysis and Inference for a Quantitative Response with Independent Samples}\label{exploratory-data-analysis-and-inference-for-a-quantitative-response-with-independent-samples}

Placeholder

\section{Vocabulary Review and Key Topics}\label{vocabulary-review-and-key-topics-8}

\subsection{Key topics}\label{key-topics-10}

\subsection{Vocabulary}\label{vocabulary-8}

\subsubsection*{Plotting a quantitative response with independent groups}\label{plotting-a-quantitative-response-with-independent-groups}
\addcontentsline{toc}{subsubsection}{Plotting a quantitative response with independent groups}

\subsubsection*{Hypotheses}\label{hypotheses-2}
\addcontentsline{toc}{subsubsection}{Hypotheses}

\subsubsection*{Simulation-based inference for a difference in means}\label{simulation-based-inference-for-a-difference-in-means}
\addcontentsline{toc}{subsubsection}{Simulation-based inference for a difference in means}

\subsubsection*{Theory-based inference for a difference in means}\label{theory-based-inference-for-a-difference-in-means}
\addcontentsline{toc}{subsubsection}{Theory-based inference for a difference in means}

\section{Video Notes: Inference for Independent Samples}\label{video-notes-inference-for-independent-samples}

\subsection{Course Videos}\label{course-videos-8}

\subsection*{Theory-based method - Video 19.3TheoryTests}\label{theory-based-method---video-19.3theorytests}
\addcontentsline{toc}{subsection}{Theory-based method - Video 19.3TheoryTests}

\subsection*{Optional Notes: Video Example (Video 19.3TheoryTests)}\label{optional-notes-video-example-video-19.3theorytests}
\addcontentsline{toc}{subsection}{Optional Notes: Video Example (Video 19.3TheoryTests)}

\subsection*{Confidence Interval - Video 19.3TheoryIntervals}\label{confidence-interval---video-19.3theoryintervals}
\addcontentsline{toc}{subsection}{Confidence Interval - Video 19.3TheoryIntervals}

\subsection*{Optional Notes: Video Example (Video 19.3TheoryIntervals)}\label{optional-notes-video-example-video-19.3theoryintervals}
\addcontentsline{toc}{subsection}{Optional Notes: Video Example (Video 19.3TheoryIntervals)}

\subsection*{Optional Notes: Simulation Testing for a Difference in Means: Video 19.1}\label{optional-notes-simulation-testing-for-a-difference-in-means-video-19.1}
\addcontentsline{toc}{subsection}{Optional Notes: Simulation Testing for a Difference in Means: Video 19.1}

\subsubsection*{Hypothesis Testing}\label{hypothesis-testing-5}
\addcontentsline{toc}{subsubsection}{Hypothesis Testing}

\subsubsection*{Simulation-based method}\label{simulation-based-method-6}
\addcontentsline{toc}{subsubsection}{Simulation-based method}

\subsubsection*{Confidence interval}\label{confidence-interval-5}
\addcontentsline{toc}{subsubsection}{Confidence interval}

\subsection*{Optional Notes: Simulation Confidence Interval for a Difference in Means - Video 19.2}\label{optional-notes-simulation-confidence-interval-for-a-difference-in-means---video-19.2}
\addcontentsline{toc}{subsection}{Optional Notes: Simulation Confidence Interval for a Difference in Means - Video 19.2}

\subsection{Concept Check}\label{concept-check-8}

\section{Activity 17: Does behavior impact performance?}\label{activity-17-does-behavior-impact-performance}

\subsection{Learning outcomes}\label{learning-outcomes-19}

\subsection{Terminology review}\label{terminology-review-15}

\subsection{Behavior and Performance}\label{behavior-and-performance}

\subsubsection*{R instructions}\label{r-instructions-10}
\addcontentsline{toc}{subsubsection}{R instructions}

\subsubsection*{Quantitative variables review}\label{quantitative-variables-review}
\addcontentsline{toc}{subsubsection}{Quantitative variables review}

\subsubsection*{Ask a research question}\label{ask-a-research-question-4}
\addcontentsline{toc}{subsubsection}{Ask a research question}

\subsubsection*{Numerically Summarize the data}\label{numerically-summarize-the-data}
\addcontentsline{toc}{subsubsection}{Numerically Summarize the data}

\subsubsection*{Use statistical inferential methods to draw inferences from the data}\label{use-statistical-inferential-methods-to-draw-inferences-from-the-data-2}
\addcontentsline{toc}{subsubsection}{Use statistical inferential methods to draw inferences from the data}

\paragraph*{Hypothesis test}\label{hypothesis-test-3}
\addcontentsline{toc}{paragraph}{Hypothesis test}

\subsubsection*{Notes on the null distribution}\label{notes-on-the-null-distribution-3}
\addcontentsline{toc}{subsubsection}{Notes on the null distribution}

\paragraph*{Confidence interval}\label{confidence-interval-6}
\addcontentsline{toc}{paragraph}{Confidence interval}

\subsubsection*{Notes on the bootstrap distribution}\label{notes-on-the-bootstrap-distribution-3}
\addcontentsline{toc}{subsubsection}{Notes on the bootstrap distribution}

\subsection{Take-home messages}\label{take-home-messages-16}

\subsection{Additional notes}\label{additional-notes-16}

\section{Activity 18: Moon Phases and Virtual Reality}\label{activity-18-moon-phases-and-virtual-reality}

\subsection{Learning outcomes}\label{learning-outcomes-20}

\subsection{Terminology review}\label{terminology-review-16}

\subsection{Moon Phases and Virtual Reality}\label{moon-phases-and-virtual-reality}

\subsubsection*{R instructions}\label{r-instructions-11}
\addcontentsline{toc}{subsubsection}{R instructions}

\subsubsection*{Use statistical inferential methods to draw inferences from the data}\label{use-statistical-inferential-methods-to-draw-inferences-from-the-data-3}
\addcontentsline{toc}{subsubsection}{Use statistical inferential methods to draw inferences from the data}

\subsection{Take-home messages}\label{take-home-messages-17}

\subsection{Additional notes}\label{additional-notes-17}

\section{Module 11 Lab: Dinosaurs}\label{module-11-lab-dinosaurs}

\subsection{Learning outcomes}\label{learning-outcomes-21}

\subsection{Dinosaurs}\label{dinosaurs}

\subsubsection*{R instructions}\label{r-instructions-12}
\addcontentsline{toc}{subsubsection}{R instructions}

\subsection*{Use statistical inferential methods to draw inferences from the data}\label{use-statistical-inferential-methods-to-draw-inferences-from-the-data-4}
\addcontentsline{toc}{subsection}{Use statistical inferential methods to draw inferences from the data}

\subsection*{Hypothesis test}\label{hypothesis-test-4}
\addcontentsline{toc}{subsection}{Hypothesis test}

\subsection*{Communicate the results and answer the research question}\label{communicate-the-results-and-answer-the-research-question-3}
\addcontentsline{toc}{subsection}{Communicate the results and answer the research question}

\subsection*{Confidence interval}\label{confidence-interval-7}
\addcontentsline{toc}{subsection}{Confidence interval}

\chapter{Exploratory Data Analysis and Inference for Two Quantitative Variables}\label{exploratory-data-analysis-and-inference-for-two-quantitative-variables}

Placeholder

\section{Vocabulary Review and Key Topics}\label{vocabulary-review-and-key-topics-9}

\subsection{Key topics}\label{key-topics-11}

\subsection{Vocabulary}\label{vocabulary-9}

\subsubsection*{Plotting two quantitative variables}\label{plotting-two-quantitative-variables}
\addcontentsline{toc}{subsubsection}{Plotting two quantitative variables}

\subsubsection*{Sample statistics for two quantitative variables}\label{sample-statistics-for-two-quantitative-variables}
\addcontentsline{toc}{subsubsection}{Sample statistics for two quantitative variables}

\subsubsection*{Hypotheses}\label{hypotheses-3}
\addcontentsline{toc}{subsubsection}{Hypotheses}

\subsubsection*{Simulation-based inference for two quantitative variables}\label{simulation-based-inference-for-two-quantitative-variables}
\addcontentsline{toc}{subsubsection}{Simulation-based inference for two quantitative variables}

\subsubsection*{Theory-based methods for two quantitative variables}\label{theory-based-methods-for-two-quantitative-variables}
\addcontentsline{toc}{subsubsection}{Theory-based methods for two quantitative variables}

\section{Video Notes: Regression and Correlation}\label{video-notes-regression-and-correlation}

\subsection{Course Videos}\label{course-videos-9}

\subsection*{Summary measures and plots for two quantitative variables - Videos 6.1 - 6.3}\label{summary-measures-and-plots-for-two-quantitative-variables---videos-6.1---6.3}
\addcontentsline{toc}{subsection}{Summary measures and plots for two quantitative variables - Videos 6.1 - 6.3}

\subsubsection*{Type of plot}\label{type-of-plot}
\addcontentsline{toc}{subsubsection}{Type of plot}

\subsubsection*{Correlation}\label{correlation}
\addcontentsline{toc}{subsubsection}{Correlation}

\subsubsection*{Slope}\label{slope}
\addcontentsline{toc}{subsubsection}{Slope}

\subsubsection*{Coefficient of Determination}\label{coefficient-of-determination}
\addcontentsline{toc}{subsubsection}{Coefficient of Determination}

\subsection*{Multivariable plots - Video Chapter7}\label{multivariable-plots---video-chapter7}
\addcontentsline{toc}{subsection}{Multivariable plots - Video Chapter7}

\subsection{Concept Check}\label{concept-check-9}

\subsection*{Theoretical Testing for Slope - Video 21.4to21.5TheoryTests}\label{theoretical-testing-for-slope---video-21.4to21.5theorytests}
\addcontentsline{toc}{subsection}{Theoretical Testing for Slope - Video 21.4to21.5TheoryTests}

\subsection*{Optional Notes: Video Example (Video 21.4TheoryTests)}\label{optional-notes-video-example-video-21.4theorytests}
\addcontentsline{toc}{subsection}{Optional Notes: Video Example (Video 21.4TheoryTests)}

\subsection*{Theoretical Confidence Interval for Slope - Video 21.4TheoryInterval}\label{theoretical-confidence-interval-for-slope---video-21.4theoryinterval}
\addcontentsline{toc}{subsection}{Theoretical Confidence Interval for Slope - Video 21.4TheoryInterval}

\subsection*{Optional Notes: Video Example (Video 21.4TheoryInterval)}\label{optional-notes-video-example-video-21.4theoryinterval}
\addcontentsline{toc}{subsection}{Optional Notes: Video Example (Video 21.4TheoryInterval)}

\subsection*{Video Notes: Inference for Two Quantitative Variables}\label{video-notes-inference-for-two-quantitative-variables}
\addcontentsline{toc}{subsection}{Video Notes: Inference for Two Quantitative Variables}

\subsection*{Hypothesis Testing - Video 21.1}\label{hypothesis-testing---video-21.1}
\addcontentsline{toc}{subsection}{Hypothesis Testing - Video 21.1}

\subsubsection*{Simulation-based method}\label{simulation-based-method-7}
\addcontentsline{toc}{subsubsection}{Simulation-based method}

\subsection*{Confidence interval - Video 21.3}\label{confidence-interval---video-21.3}
\addcontentsline{toc}{subsection}{Confidence interval - Video 21.3}

\subsubsection*{Simulation-based method}\label{simulation-based-method-8}
\addcontentsline{toc}{subsubsection}{Simulation-based method}

\subsection{Concept Check}\label{concept-check-10}

\section{Activity 19: Moneyball --- Linear Regression}\label{activity-19-moneyball-linear-regression}

\subsection{Learning outcomes}\label{learning-outcomes-22}

\subsection{Terminology review}\label{terminology-review-17}

\subsection{Moneyball}\label{moneyball}

\subsection*{Notes on two quantitative variables}\label{notes-on-two-quantitative-variables}
\addcontentsline{toc}{subsection}{Notes on two quantitative variables}

\subsubsection*{R Instructions}\label{r-instructions-13}
\addcontentsline{toc}{subsubsection}{R Instructions}

\subsubsection*{Slope}\label{slope-1}
\addcontentsline{toc}{subsubsection}{Slope}

\subsubsection*{Residuals}\label{residuals}
\addcontentsline{toc}{subsubsection}{Residuals}

\subsubsection*{Correlation}\label{correlation-1}
\addcontentsline{toc}{subsubsection}{Correlation}

\subsection{Take-home messages}\label{take-home-messages-18}

\subsection{Additional notes}\label{additional-notes-18}

\section{Activity 20: IPEDS (continued)}\label{activity-20-ipeds-continued}

\setstretch{1}

\subsection{Learning outcomes}\label{learning-outcomes-23}

\begin{itemize}
\item
  Identify and create appropriate summary statistics and plots
  given a data set with two quantitative variables.
\item
  Find the estimated line of regression using summary statistics and \texttt{R} linear model (\texttt{lm()}) output.
\item
  Interpret the slope coefficient in context of the problem.
\item
  Calculate and interpret \(r^2\), the coefficient of determination, in context of the problem.
\item
  Find the correlation coefficient from \texttt{R} output or from \(r^2\) and the sign of the slope.
\end{itemize}

\subsection{Terminology review}\label{terminology-review-18}

In today's activity, we will review summary measures and plots for two quantitative variables. Some terms covered in this activity are:

\begin{itemize}
\item
  Least-squares line of regression
\item
  Slope and \(y\)-intercept
\item
  Residuals
\item
  Correlation (\(r\))
\item
  Coefficient of determination (\(r\)-squared)
\end{itemize}

To review these concepts, see Chapter 6 in the textbook.

\subsection{The Integrated Postsecondary Education Data System (IPEDS)}\label{the-integrated-postsecondary-education-data-system-ipeds-1}

We will continue to assess the IPEDS data set collected on a subset of institutions that met the following selection criteria (Education Statistics 2018):

\begin{itemize}
\item
  Degree granting
\item
  United States only
\item
  Title IV participating
\item
  Not for profit
\item
  2-year or 4-year or above
\item
  Has full-time first-time undergraduates
\end{itemize}

Some of the variables collected and their descriptions are below. Note that several variables have missing values for some institutions (denoted by ``NA'').

\begin{longtable}[]{@{}
  >{\raggedright\arraybackslash}p{(\linewidth - 2\tabcolsep) * \real{0.2353}}
  >{\raggedright\arraybackslash}p{(\linewidth - 2\tabcolsep) * \real{0.7647}}@{}}
\toprule\noalign{}
\begin{minipage}[b]{\linewidth}\raggedright
\textbf{Variable}
\end{minipage} & \begin{minipage}[b]{\linewidth}\raggedright
\textbf{Description}
\end{minipage} \\
\midrule\noalign{}
\endhead
\bottomrule\noalign{}
\endlastfoot
\texttt{UnitID} & Unique institution identifier \\
\texttt{Name} & Institution name \\
\texttt{State} & State abbreviation \\
\texttt{Sector} & whether public or private \\
\texttt{LandGrant} & Is this a land-grant institution (Yes/No) \\
\texttt{Size} & Institution size category based on total student enrolled for credit, Fall 2018: Under 1,000, 1,000 - 4,999, 5,000 - 9,999, 10,000 - 19,999, 20,000 and above \\
\texttt{Cost\_OutofState} & Cost of attendance for full-time out-of-state undergraduate students \\
\texttt{Cost\_InState} & Cost of attendance for full-time in-state undergraduate students \\
\texttt{Retention} & Retention rate is the percent of the undergraduate students that re-enroll in the next year \\
\texttt{Graduation\_Rate} & 6-year graduation rate for undergraduate students \\
\texttt{SATMath\_75} & 75th percentile Math SAT score \\
\texttt{ACT\_75} & 75th percentile ACT score \\
\end{longtable}

The code below reads in the needed data set, IPEDS\_2018.csv, and filters out the 2-year institutions.

\begin{itemize}
\tightlist
\item
  Highlight and run lines 1--11 to load the data set and filter out the 2-year institutions.
\end{itemize}

\begin{Shaded}
\begin{Highlighting}[]
\NormalTok{IPEDS }\OtherTok{\textless{}{-}} \FunctionTok{read.csv}\NormalTok{(}\StringTok{"https://www.math.montana.edu/courses/s216/data/IPEDS\_2018.csv"}\NormalTok{) }
\NormalTok{IPEDS }\OtherTok{\textless{}{-}}\NormalTok{ IPEDS }\SpecialCharTok{\%\textgreater{}\%}
  \FunctionTok{filter}\NormalTok{(Sector }\SpecialCharTok{!=} \StringTok{"Public 2{-}year"}\NormalTok{) }\CommentTok{\#Filters the data set to remove Public 2{-}year}
\NormalTok{IPEDS }\OtherTok{\textless{}{-}}\NormalTok{ IPEDS }\SpecialCharTok{\%\textgreater{}\%}
  \FunctionTok{filter}\NormalTok{(Sector }\SpecialCharTok{!=} \StringTok{"Private 2{-}year"}\NormalTok{) }\CommentTok{\#Filters the data set to remove Private 2{-}year}
\NormalTok{IPEDS }\OtherTok{\textless{}{-}} \FunctionTok{na.omit}\NormalTok{(IPEDS)}
\end{Highlighting}
\end{Shaded}

To create a scatterplot of the 75th percentile Math SAT score by retention rate for 4-year US Higher Education Institutions\ldots{}

\begin{itemize}
\item
  Enter the variable \texttt{SATMath\_75} for explanatory and \texttt{Retention} for response in line 16.
\item
  Highlight and run lines 15--21.
\end{itemize}

\begin{Shaded}
\begin{Highlighting}[]
\NormalTok{IPEDS }\SpecialCharTok{\%\textgreater{}\%} \CommentTok{\# Data sest pipes into...}
    \FunctionTok{ggplot}\NormalTok{(}\FunctionTok{aes}\NormalTok{(}\AttributeTok{x =}\NormalTok{ SATMath\_75, }\AttributeTok{y =}\NormalTok{ Retention))}\SpecialCharTok{+}  \CommentTok{\# Specify variables}
    \FunctionTok{geom\_point}\NormalTok{(}\AttributeTok{alpha=}\FloatTok{0.5}\NormalTok{) }\SpecialCharTok{+}  \CommentTok{\# Add scatterplot of points}
    \FunctionTok{labs}\NormalTok{(}\AttributeTok{x =} \StringTok{"75th Percentile SAT Math Score"}\NormalTok{,  }\CommentTok{\# Label x{-}axis}
       \AttributeTok{y =} \StringTok{"Retention Rate (\%)"}\NormalTok{,  }\CommentTok{\# Label y{-}axis}
       \AttributeTok{title =} \StringTok{"Scatterplot of SAT Math Score vs. Retention Rate for }
\StringTok{       4{-}year US Higher Education Institutions"}\NormalTok{) }\SpecialCharTok{+} 
    \CommentTok{\# Be sure to title your plots with the type of plot, observational units, variable(s)}
    \FunctionTok{geom\_smooth}\NormalTok{(}\AttributeTok{method =} \StringTok{"lm"}\NormalTok{, }\AttributeTok{se =} \ConstantTok{FALSE}\NormalTok{) }\SpecialCharTok{+} \CommentTok{\# Add regression line}
    \FunctionTok{theme\_bw}\NormalTok{()}
\end{Highlighting}
\end{Shaded}

\begin{center}\includegraphics[width=0.7\linewidth]{12-A20-EDA-two-quantitative-corr_files/figure-latex/unnamed-chunk-2-1} \end{center}

\begin{enumerate}
\def\labelenumi{\arabic{enumi}.}
\tightlist
\item
  Describe the relationship, using the four characteristics of scatterplots, between 75th percentile SAT Math score and retention rate.
\end{enumerate}

\vspace{1in}

\subsubsection*{Slope of the Least Squares Linear Regression Line}\label{slope-of-the-least-squares-linear-regression-line}
\addcontentsline{toc}{subsubsection}{Slope of the Least Squares Linear Regression Line}

There are three summary measures calculated from two quantitative variables: slope, correlation, and the coefficient of determination. We will first assess the slope of the least squares regression line between 75th percentile SAT Math score and retention rate.

\begin{itemize}
\item
  Enter \texttt{Retention} for response and \texttt{SATMath\_75} for explanatory in line 25
\item
  Highlight and run lines 25--26 to fit the linear model.
\end{itemize}

\begin{Shaded}
\begin{Highlighting}[]
\CommentTok{\# Fit linear model: y \textasciitilde{} x}
\NormalTok{IPEDSLM }\OtherTok{\textless{}{-}} \FunctionTok{lm}\NormalTok{(Retention}\SpecialCharTok{\textasciitilde{}}\NormalTok{SATMath\_75, }\AttributeTok{data=}\NormalTok{IPEDS)}
\FunctionTok{round}\NormalTok{(}\FunctionTok{summary}\NormalTok{(IPEDSLM)}\SpecialCharTok{$}\NormalTok{coefficients,}\DecValTok{3}\NormalTok{) }\CommentTok{\# Display coefficient summary}
\end{Highlighting}
\end{Shaded}

\begin{verbatim}
#>             Estimate Std. Error t value Pr(>|t|)
#> (Intercept)    0.059      1.898   0.031    0.975
#> SATMath_75     0.125      0.003  40.485    0.000
\end{verbatim}

\begin{enumerate}
\def\labelenumi{\arabic{enumi}.}
\setcounter{enumi}{1}
\tightlist
\item
  Write out the least squares regression line using the summary statistics from the R output in context of the problem.
\end{enumerate}

\vspace{0.5in}

\begin{enumerate}
\def\labelenumi{\arabic{enumi}.}
\setcounter{enumi}{2}
\tightlist
\item
  Interpret the value of slope.
\end{enumerate}

\vspace{0.8in}

\begin{enumerate}
\def\labelenumi{\arabic{enumi}.}
\setcounter{enumi}{3}
\tightlist
\item
  Predict the retention rate for a 4-year US higher education institution with a 75th percentile SAT Math score of 440.
\end{enumerate}

\vspace{0.4in}

\begin{enumerate}
\def\labelenumi{\arabic{enumi}.}
\setcounter{enumi}{4}
\tightlist
\item
  Calculate the residual for a 4-year US higher education institution with a 75th percentile SAT Math score of 440 and a retention rate of 24\%.
\end{enumerate}

\vspace{0.4in}

\subsubsection*{Correlation}\label{correlation-2}
\addcontentsline{toc}{subsubsection}{Correlation}

Correlation measures the strength and the direction of the linear relationship between two quantitative variables. The closer the value of correlation to \(+1\) or \(-1\), the stronger the linear relationship. Values close to zero indicate a very weak linear relationship between the two variables.

The following output creates a correlation matrix between several pairs of quantitative variables.

\begin{Shaded}
\begin{Highlighting}[]
\NormalTok{IPEDS }\SpecialCharTok{\%\textgreater{}\%}  \CommentTok{\# Data set pipes into}
  \FunctionTok{select}\NormalTok{(}\FunctionTok{c}\NormalTok{(}\StringTok{"Retention"}\NormalTok{, }\StringTok{"Cost\_InState"}\NormalTok{, }
           \StringTok{"Graduation\_Rate"}\NormalTok{, }\StringTok{"Salary"}\NormalTok{, }
           \StringTok{"SATMath\_75"}\NormalTok{, }\StringTok{"ACT\_75"}\NormalTok{)) }\SpecialCharTok{\%\textgreater{}\%}
  \FunctionTok{cor}\NormalTok{(}\AttributeTok{use=}\StringTok{"pairwise.complete.obs"}\NormalTok{) }\SpecialCharTok{\%\textgreater{}\%}
  \FunctionTok{round}\NormalTok{(}\DecValTok{3}\NormalTok{)}
\end{Highlighting}
\end{Shaded}

\begin{verbatim}
#>                 Retention Cost_InState Graduation_Rate Salary SATMath_75 ACT_75
#> Retention           1.000        0.388           0.832  0.698      0.767  0.768
#> Cost_InState        0.388        1.000           0.563  0.365      0.502  0.514
#> Graduation_Rate     0.832        0.563           1.000  0.683      0.817  0.833
#> Salary              0.698        0.365           0.683  1.000      0.747  0.706
#> SATMath_75          0.767        0.502           0.817  0.747      1.000  0.920
#> ACT_75              0.768        0.514           0.833  0.706      0.920  1.000
\end{verbatim}

\begin{enumerate}
\def\labelenumi{\arabic{enumi}.}
\setcounter{enumi}{5}
\tightlist
\item
  What is the value of correlation between SATMath\_75 and Retention?
\end{enumerate}

\vspace{0.3in}

\subsubsection*{Coefficient of determination (squared correlation)}\label{coefficient-of-determination-squared-correlation}
\addcontentsline{toc}{subsubsection}{Coefficient of determination (squared correlation)}

Another summary measure used to explain the linear relationship between two quantitative variables is the coefficient of determination (\(r^2\)). The coefficient of determination, \(r^2\), can also be used to describe the strength of the linear relationship between two quantitative variables. The value of \(r^2\) (a value between 0 and 1) represents the \textbf{proportion of variation in the response that is explained by the least squares line with the explanatory variable}. There are two ways to calculate the coefficient of determination:

~~~Square the correlation coefficient: \(r^2 = (r)^2\)

~~~Use the variances of the response and the residuals: \(r^2 = \dfrac{s_y^2 - s_{RES}^2}{s_y^2} = \dfrac{SST - SSE}{SST}\)

\begin{enumerate}
\def\labelenumi{\arabic{enumi}.}
\setcounter{enumi}{6}
\tightlist
\item
  Use the correlation, \(r\), found in question 6, to calculate the coefficient of determination between SATMath\_75 and Retention, \(r^2\).
\end{enumerate}

\vspace{.4in}

The variance of the response variable, Retention (\%), is \(s_{Retention}^2 = 138.386\) \(\%^2\) and the variability in the residuals is \(s_{RES}^2 = 56.934\) \%\(^2\). Use these values to calculate the coefficient of determination.

\vspace{1in}

In the next part of the activity we will explore what the coefficient of determination measures.

In the first scatterplot, we see the data plotted with a horizontal line. Note that the regression line in this plot has a slope of zero; this assumes there is no relationship between SATMath\_75 and Retention. The value of the y-intercept, 76.387, is the mean of the response variable when there is no relationship between the two variables. To find the sum of squares total (SST) we find the residual (\(residual = y - \hat{y}\)) for each response value from the horizontal line (from the value of 76.387). Each residual is squared and the sum of the squared values is calculated. The SST gives the \textbf{total variability in the response variable, Retention}.

\begin{center}\includegraphics[width=0.7\linewidth]{12-A20-EDA-two-quantitative-corr_files/figure-latex/unnamed-chunk-5-1} \end{center}

The calculated value for the SST is 158451.8.

This next scatterplot, shows the plotted data with the best fit regression line. This is the line of best fit between budget and revenue and has the smallest sum of squares error (SSE). The SSE is calculated by finding the residual from each response value to the regression line. Each residual is squared and the sum of the squared values is calculated.

\begin{center}\includegraphics[width=0.7\linewidth]{12-A20-EDA-two-quantitative-corr_files/figure-latex/unnamed-chunk-6-1} \end{center}

The calculated value for the SSE is 65133.022.

\textbf{Calculate the value for \(r^2\) using the values for SST and SSE provided below each of the previous graphs.}

\vspace{0.6in}

\begin{enumerate}
\def\labelenumi{\arabic{enumi}.}
\setcounter{enumi}{7}
\tightlist
\item
  Write a sentence interpreting the coefficient of determination in context of the problem.
\end{enumerate}

\newpage

\subsubsection*{Multivariable plots}\label{multivariable-plots}
\addcontentsline{toc}{subsubsection}{Multivariable plots}

When adding another categorical predictor, we can add that variable as shape or color to the plot. In the following code we have added the variable \texttt{Size}.

\begin{Shaded}
\begin{Highlighting}[]
\NormalTok{IPEDS}\SpecialCharTok{$}\NormalTok{Size }\OtherTok{\textless{}{-}} \FunctionTok{factor}\NormalTok{(IPEDS}\SpecialCharTok{$}\NormalTok{Size, }\AttributeTok{levels =} \FunctionTok{c}\NormalTok{(}\StringTok{"\textless{} 1000"}\NormalTok{, }\StringTok{"1,000{-}4,999"}\NormalTok{, }\StringTok{"5,000{-}9,999"}\NormalTok{,}
                                            \StringTok{"10,000{-}19,999"}\NormalTok{, }\StringTok{"20,000 and above"}\NormalTok{))}
\NormalTok{IPEDS }\SpecialCharTok{\%\textgreater{}\%} \CommentTok{\# Data set pipes into...}
    \FunctionTok{ggplot}\NormalTok{(}\FunctionTok{aes}\NormalTok{(}\AttributeTok{x =}\NormalTok{ SATMath\_75, }\AttributeTok{y =}\NormalTok{ Retention, }\AttributeTok{shape =}\NormalTok{ Size, }\AttributeTok{color=}\NormalTok{Size))}\SpecialCharTok{+}  \CommentTok{\# Specify variables}
    \FunctionTok{geom\_point}\NormalTok{(}\AttributeTok{alpha=}\FloatTok{0.5}\NormalTok{) }\SpecialCharTok{+}  \CommentTok{\# Add scatterplot of points}
    \FunctionTok{labs}\NormalTok{(}\AttributeTok{x =} \StringTok{"75th Percentile SAT Math Score"}\NormalTok{,  }\CommentTok{\# Label x{-}axis}
       \AttributeTok{y =} \StringTok{"Retention Rate (\%)"}\NormalTok{,  }\CommentTok{\# Label y{-}axis}
       \AttributeTok{title =} \StringTok{"Scatterplot of SAT Math Score vs. Retention Rate for }
\StringTok{       4{-}year US Higher Education Institutions by Size"}\NormalTok{) }\SpecialCharTok{+} 
    \CommentTok{\# Be sure to title your plots with the type of plot, observational units, variable(s)}
    \FunctionTok{geom\_smooth}\NormalTok{(}\AttributeTok{method =} \StringTok{"lm"}\NormalTok{, }\AttributeTok{se =} \ConstantTok{FALSE}\NormalTok{) }\SpecialCharTok{+} \CommentTok{\# Add regression line}
    \FunctionTok{scale\_color\_grey}\NormalTok{()}
\end{Highlighting}
\end{Shaded}

\begin{center}\includegraphics[width=0.7\linewidth]{12-A20-EDA-two-quantitative-corr_files/figure-latex/unnamed-chunk-7-1} \end{center}

\begin{enumerate}
\def\labelenumi{\arabic{enumi}.}
\setcounter{enumi}{8}
\tightlist
\item
  Does the relationship between 75th percentile SAT math score and retention rate of 4-year institutions change depending on the level of size?
\end{enumerate}

\vspace{0.8in}

\begin{center}\includegraphics[width=0.7\linewidth]{12-A20-EDA-two-quantitative-corr_files/figure-latex/unnamed-chunk-8-1} \includegraphics[width=0.7\linewidth]{12-A20-EDA-two-quantitative-corr_files/figure-latex/unnamed-chunk-8-2} \end{center}

\begin{enumerate}
\def\labelenumi{\arabic{enumi}.}
\setcounter{enumi}{9}
\tightlist
\item
  Is size of the higher education institution associated with retention rate? Is size of the higher education institution associated with 75th percentile SAT Math Score?
\end{enumerate}

\vspace{1in}

\newpage

\subsection{Take-home messages}\label{take-home-messages-19}

\begin{enumerate}
\def\labelenumi{\arabic{enumi}.}
\item
  The sign of correlation and the sign of the slope will always be the same. The closer the value of correlation is to \(-1\) or \(+1\), the stronger the linear relationship between the explanatory and the response variable.
\item
  The coefficient of determination multiplied by 100 (\(r^2 \times 100\)) measures the percent of variation in the response variable that is explained by the relationship with the explanatory variable. The closer the value of the coefficient of determination is to 100\%, the stronger the relationship.
\item
  We can use the line of regression to predict values of the response variable for values of the explanatory variable. Do not use values of the explanatory variable that are outside of the range of values in the data set to predict values of the response variable (reflect on why this is true.). This is called \textbf{extrapolation}.
\end{enumerate}

\subsection{Additional notes}\label{additional-notes-19}

Use this space to summarize your thoughts and take additional notes on today's activity and material covered.

\newpage

\section{Activity 21: Golf Driving Distance}\label{activity-21-golf-driving-distance}

\subsection{Learning outcomes}\label{learning-outcomes-24}

\subsection{Terminology review}\label{terminology-review-19}

\subsection{Golf driving distance}\label{golf-driving-distance}

\subsection*{R Instructions}\label{r-instructions-14}
\addcontentsline{toc}{subsection}{R Instructions}

\subsubsection*{Plot review.}\label{plot-review.}
\addcontentsline{toc}{subsubsection}{Plot review.}

\subsubsection*{Conditions for the least squares line}\label{conditions-for-the-least-squares-line}
\addcontentsline{toc}{subsubsection}{Conditions for the least squares line}

\subsubsection*{Ask a research question}\label{ask-a-research-question-5}
\addcontentsline{toc}{subsubsection}{Ask a research question}

\subsubsection*{Summarize and visualize the data}\label{summarize-and-visualize-the-data-4}
\addcontentsline{toc}{subsubsection}{Summarize and visualize the data}

\subsubsection*{Use statistical inferential methods to draw inferences from the data}\label{use-statistical-inferential-methods-to-draw-inferences-from-the-data-5}
\addcontentsline{toc}{subsubsection}{Use statistical inferential methods to draw inferences from the data}

\paragraph*{Hypothesis test}\label{hypothesis-test-5}
\addcontentsline{toc}{paragraph}{Hypothesis test}

\paragraph*{Confidence interval}\label{confidence-interval-8}
\addcontentsline{toc}{paragraph}{Confidence interval}

\subsubsection*{Communicate the results and answer the research question}\label{communicate-the-results-and-answer-the-research-question-4}
\addcontentsline{toc}{subsubsection}{Communicate the results and answer the research question}

\subsubsection*{Simulation-based hypothesis test}\label{simulation-based-hypothesis-test}
\addcontentsline{toc}{subsubsection}{Simulation-based hypothesis test}

\subsubsection*{Simulation-based confidence interval}\label{simulation-based-confidence-interval-2}
\addcontentsline{toc}{subsubsection}{Simulation-based confidence interval}

\subsection*{Multivariable plots}\label{multivariable-plots-1}
\addcontentsline{toc}{subsection}{Multivariable plots}

\subsection{Take-home messages}\label{take-home-messages-20}

\subsection{Additional notes}\label{additional-notes-20}

\section{Module 12 Lab: Big Mac Index}\label{module-12-lab-big-mac-index}

\subsection{Learning outcomes}\label{learning-outcomes-25}

\subsection{Big Mac Index}\label{big-mac-index}

\subsubsection*{Summarize and visualize the data}\label{summarize-and-visualize-the-data-5}
\addcontentsline{toc}{subsubsection}{Summarize and visualize the data}

\subsubsection*{Conditions for the least squares line}\label{conditions-for-the-least-squares-line-1}
\addcontentsline{toc}{subsubsection}{Conditions for the least squares line}

\subsubsection*{Ask a research question}\label{ask-a-research-question-6}
\addcontentsline{toc}{subsubsection}{Ask a research question}

\subsubsection*{Use statistical inferential methods to draw inferences from the data}\label{use-statistical-inferential-methods-to-draw-inferences-from-the-data-6}
\addcontentsline{toc}{subsubsection}{Use statistical inferential methods to draw inferences from the data}

\subsubsection*{Hypothesis test}\label{hypothesis-test-6}
\addcontentsline{toc}{subsubsection}{Hypothesis test}

\subsubsection*{Simulation-based confidence interval}\label{simulation-based-confidence-interval-3}
\addcontentsline{toc}{subsubsection}{Simulation-based confidence interval}

\subsubsection*{Communicate the results and answer the research question}\label{communicate-the-results-and-answer-the-research-question-5}
\addcontentsline{toc}{subsubsection}{Communicate the results and answer the research question}

\chapter{Exploratory Data Analysis and Inference for a Quantitative Response with Paired Samples}\label{exploratory-data-analysis-and-inference-for-a-quantitative-response-with-paired-samples}

Placeholder

\section{Vocabulary Review and Key Topics}\label{vocabulary-review-and-key-topics-10}

\subsection{Key topics}\label{key-topics-12}

\subsection{Vocabulary}\label{vocabulary-10}

\subsubsection*{Simulation-based inference for a paired mean difference}\label{simulation-based-inference-for-a-paired-mean-difference}
\addcontentsline{toc}{subsubsection}{Simulation-based inference for a paired mean difference}

\subsubsection*{Theory-based inference for a paired mean difference}\label{theory-based-inference-for-a-paired-mean-difference}
\addcontentsline{toc}{subsubsection}{Theory-based inference for a paired mean difference}

\section{Video Notes: Inference for Paired Data}\label{video-notes-inference-for-paired-data}

\subsection{Course Videos}\label{course-videos-10}

\subsection*{Single categorical, single quantitative variables - Video Paired\_Data}\label{single-categorical-single-quantitative-variables---video-paired_data}
\addcontentsline{toc}{subsection}{Single categorical, single quantitative variables - Video Paired\_Data}

\subsection*{Paired vs.~Independent Samples}\label{paired-vs.-independent-samples}
\addcontentsline{toc}{subsection}{Paired vs.~Independent Samples}

\subsection*{Theory-based method - Video 18.3}\label{theory-based-method---video-18.3}
\addcontentsline{toc}{subsection}{Theory-based method - Video 18.3}

\subsubsection*{t-distribution}\label{t-distribution-1}
\addcontentsline{toc}{subsubsection}{t-distribution}

\subsection*{Optional Notes: Video Example (Video 18.3)}\label{optional-notes-video-example-video-18.3}
\addcontentsline{toc}{subsection}{Optional Notes: Video Example (Video 18.3)}

\subsection*{Optional Notes: Simulation Inference for a Mean Difference - Video 18.1and18.2}\label{optional-notes-simulation-inference-for-a-mean-difference---video-18.1and18.2}
\addcontentsline{toc}{subsection}{Optional Notes: Simulation Inference for a Mean Difference - Video 18.1and18.2}

\subsubsection*{Hypothesis testing}\label{hypothesis-testing-6}
\addcontentsline{toc}{subsubsection}{Hypothesis testing}

\subsubsection*{Simulation-based method}\label{simulation-based-method-9}
\addcontentsline{toc}{subsubsection}{Simulation-based method}

\subsection*{Confidence interval}\label{confidence-interval-9}
\addcontentsline{toc}{subsection}{Confidence interval}

\subsubsection*{Simulation-based method}\label{simulation-based-method-10}
\addcontentsline{toc}{subsubsection}{Simulation-based method}

\subsection{Concept Check}\label{concept-check-11}

\section{Activity 22: Paired vs.~Independent Samples}\label{activity-22-paired-vs.-independent-samples}

\subsection{Learning outcomes}\label{learning-outcomes-26}

\subsection{Terminology review}\label{terminology-review-20}

\subsection*{Notes on paired data}\label{notes-on-paired-data}
\addcontentsline{toc}{subsection}{Notes on paired data}

\subsection{Paired vs.~Independent Samples}\label{paired-vs.-independent-samples-1}

\subsection{Tattoo Effect on Sweat Rate}\label{tattoo-effect-on-sweat-rate}

\subsection{Exploring Paired Data}\label{exploring-paired-data}

\subsubsection*{R Instructions}\label{r-instructions-15}
\addcontentsline{toc}{subsubsection}{R Instructions}

\subsection{Take home messages}\label{take-home-messages-21}

\subsection{Additional notes}\label{additional-notes-21}

\section{Activity 23: Color Interference}\label{activity-23-color-interference}

\subsection{Learning outcomes}\label{learning-outcomes-27}

\subsection{Terminology review}\label{terminology-review-21}

\subsection{Color Interference}\label{color-interference}

\subsubsection*{Identify the scenario}\label{identify-the-scenario}
\addcontentsline{toc}{subsubsection}{Identify the scenario}

\subsubsection*{Ask a research question}\label{ask-a-research-question-7}
\addcontentsline{toc}{subsubsection}{Ask a research question}

\subsubsection*{Summarize and visualize the data}\label{summarize-and-visualize-the-data-6}
\addcontentsline{toc}{subsubsection}{Summarize and visualize the data}

\subsubsection*{Check theoretical conditions}\label{check-theoretical-conditions-1}
\addcontentsline{toc}{subsubsection}{Check theoretical conditions}

\subsubsection*{Use statistical inferential methods to draw inferences from the data}\label{use-statistical-inferential-methods-to-draw-inferences-from-the-data-7}
\addcontentsline{toc}{subsubsection}{Use statistical inferential methods to draw inferences from the data}

\subsection*{Simulation Hypothesis Test}\label{simulation-hypothesis-test}
\addcontentsline{toc}{subsection}{Simulation Hypothesis Test}

\subsection*{Simulation confidence interval}\label{simulation-confidence-interval}
\addcontentsline{toc}{subsection}{Simulation confidence interval}

\subsection{Take-home messages}\label{take-home-messages-22}

\subsection{Additional notes}\label{additional-notes-22}

\section{Module 13 Lab: Swearing}\label{module-13-lab-swearing}

\subsection{Learning outcomes}\label{learning-outcomes-28}

\subsection{Swearing}\label{swearing}

\subsubsection*{R instructions}\label{r-instructions-16}
\addcontentsline{toc}{subsubsection}{R instructions}

\subsection*{Use statistical inferential methods to draw inferences from the data}\label{use-statistical-inferential-methods-to-draw-inferences-from-the-data-8}
\addcontentsline{toc}{subsection}{Use statistical inferential methods to draw inferences from the data}

\subsection*{Hypothesis test}\label{hypothesis-test-7}
\addcontentsline{toc}{subsection}{Hypothesis test}

\subsection*{Communicate the results and answer the research question}\label{communicate-the-results-and-answer-the-research-question-6}
\addcontentsline{toc}{subsection}{Communicate the results and answer the research question}

\subsection*{Confidence interval}\label{confidence-interval-10}
\addcontentsline{toc}{subsection}{Confidence interval}

\chapter{Unit 3 Review}\label{unit-3-review}

The following section contains both a list of key topics covered in Unit 3 as well as Module Review Worksheets.

\subsection{Key Topics}\label{key-topics-13}

Review the key topics for Unit 3 to review prior to the exams. All of these topics will be covered in Modules 11--13.

\subsection{Module Review}\label{module-review-2}

\setstretch{1}

The following worksheets review each of the modules. These worksheets will be completed during Melinda's Study Sessions each week. Solutions will be posted on Canvas in the Unit 3 Review folder after the study sessions.

\newpage

\section{Key Topics Exam 3}\label{key-topics-exam-3}

\subsection*{Descriptive statistics and study design}\label{descriptive-statistics-and-study-design-2}
\addcontentsline{toc}{subsection}{Descriptive statistics and study design}

\subsection*{Hypothesis testing}\label{hypothesis-testing-7}
\addcontentsline{toc}{subsection}{Hypothesis testing}

\subsection*{Confidence interval}\label{confidence-interval-11}
\addcontentsline{toc}{subsection}{Confidence interval}

\section{Module 11 Review - Independent Samples}\label{module-11-review---independent-samples}

\section{Module 12 Review - Regression}\label{module-12-review---regression}

\section{Module 13 Review - Paired Data}\label{module-13-review---paired-data}

\chapter{Semester Review}\label{semester-review}

Placeholder

\section{Group Final Exam Review}\label{group-final-exam-review}

\section{Golden Ticket to Descriptive and Inferential Statistical Methods}\label{golden-ticket-to-descriptive-and-inferential-statistical-methods}

In this course, we have covered descriptive (summary statistics and plots) and inferential (hypothesis tests and confidence intervals) methods for five different scenarios:

\begin{itemize}
\tightlist
\item
  one categorical response variable (Module 3 \& 4)
\item
  one quantitative response variable (Module 6 \& 7) or paired differences in a quantitative variable (Module 13)
\item
  two categorical variables (Module 8 \& 9)
\item
  one quantitative response variable and one categorical explanatory variable (Module 11)
\item
  two quantitative variables (Module 12)
\end{itemize}

The ``golden ticket'' shown on the next page presents a visual summary of the similarities and differences across these five scenarios.

\includepdf[landscape=true]{GoldenTicket_S25.pdf}

\chapter*{References}\label{references}
\addcontentsline{toc}{chapter}{References}

\phantomsection\label{refs}
\begin{CSLReferences}{1}{0}
\bibitem[\citeproctext]{ref-pga}
{``Average Driving Distance and Fairway Accuracy.''} 2008. \href{https://www.pga.com/\%20and\%20https://www.lpga.com/}{https://www.pga.com/ and https://www.lpga.com/}.

\bibitem[\citeproctext]{ref-banton2022}
Banton, et al, S. 2022. {``Jog with Your Dog: Dog Owner Exercise Routines Predict Dog Exercise Routines and Perception of Ideal Body Weight.''} \emph{PLoS ONE} 17(8).

\bibitem[\citeproctext]{ref-bhavsar2022}
Bhavsar, et al, A. 2022. {``Increased Risk of Herpes Zoster in Adults \(\geq\)50 Years Old Diagnosed with COVID-19 in the United States.''} \emph{Open Forum Infectious Diseases} 9(5).

\bibitem[\citeproctext]{ref-islands}
Bulmer, M. n.d. {``Islands in Schools Project.''} \url{https://sites.google.com/site/islandsinschoolsprojectwebsite/home}.

\bibitem[\citeproctext]{ref-bts}
{``Bureau of Transportation Statistics.''} 2019. \url{https://www.bts.gov/}.

\bibitem[\citeproctext]{ref-babies}
{``Child Health and Development Studies.''} n.d. \url{https://www.chdstudies.org/}.

\bibitem[\citeproctext]{ref-darley1973}
Darley, J. M., and C. D. Batson. 1973. {``"From Jerusalem to Jericho": A Study of Situational and Dispositional Variables in Helping Behavior.''} \emph{Journal of Personality and Social Psychology} 27: 100--108.

\bibitem[\citeproctext]{ref-davis2020}
Davis, Smith, A. K. 2020. {``A Poor Substitute for the Real Thing: Captive-Reared Monarch Butterflies Are Weaker, Paler and Have Less Elongated Wings Than Wild Migrants.''} \emph{Biology Letters} 16.

\bibitem[\citeproctext]{ref-doit2015}
Du Toit, et al, G. 2015. {``Randomized Trial of Peanut Consumption in Infants at Risk for Peanut Allergy.''} \emph{New England Journal of Medicine} 372.

\bibitem[\citeproctext]{ref-edmunds2016}
Edmunds, et al, D. 2016. {``Chronic Wasting Disease Drives Population Decline of White-Tailed Deer.''} \emph{PLoS ONE} 11(8).

\bibitem[\citeproctext]{ref-ipeds}
Education Statistics, National Center for. 2018. {``IPEDS.''} \url{https://nces.ed.gov/ipeds/}.

\bibitem[\citeproctext]{ref-gbmarried}
{``Great Britain Married Couples: Great Britain Office of Population Census and Surveys.''} n.d. \url{https://discovery.nationalarchives.gov.uk/details/r/C13351}.

\bibitem[\citeproctext]{ref-zeitler2012}
Group, TODAY Study. 2012. {``\href{https://www.ncbi.nlm.nih.gov/pubmed/22540912}{A Clinical Trial to Maintain Glycemic Control in Youth with Type 2 Diabetes}.''} \emph{New England Journal of Medicine} 366: 2247--56.

\bibitem[\citeproctext]{ref-hamblin2007}
Hamblin, J. K., K. Wynn, and P. Bloom. 2007. {``Social Evaluation by Preverbal Infants.''} \emph{Nature} 450 (6288): 557--59.

\bibitem[\citeproctext]{ref-hirschfelder2018}
Hirschfelder, A., and P. F. Molin. 2018. {``I Is for Ignoble: Stereotyping Native Americans.''} \href{Retrieved\%20from\%20https://www.ferris.edu/HTMLS/news/jimcrow/native/homepage.htm.}{Retrieved from https://www.ferris.edu/HTMLS/news/jimcrow/native/homepage.htm.}

\bibitem[\citeproctext]{ref-hutchison2013}
Hutchison, R. L., and M. A. Hirthler. 2013. {``\href{https://www.ncbi.nlm.nih.gov/pubmed/23932117}{Upper Extremity Injuies in Homer's Iliad}.''} \emph{Journal of Hand Surgery (American Volume)} 38: 1790--93.

\bibitem[\citeproctext]{ref-imdb}
{``{IMDb} Movies Extensive Dataset.''} 2016. \url{https://kaggle.com/stefanoleone992/imdb-extensive-dataset}.

\bibitem[\citeproctext]{ref-kalra2022}
Kalra, et al., Dl. 2022. {``Trustworthiness of Indian Youtubers.''} Kaggle. \url{https://doi.org/10.34740/KAGGLE/DSV/4426566}.

\bibitem[\citeproctext]{ref-keating2021}
Keating, D., N. Ahmed, F. Nirappil, Stanley-Becker I., and L. Bernstein. 2021. {``Coronavirus Infections Dropping Where People Are Vaccinated, Rising Where They Are Not, Post Analysis Finds.''} \emph{Washington Post}. \url{https://www.washingtonpost.com/health/2021/06/14/covid-cases-vaccination-rates/}.

\bibitem[\citeproctext]{ref-laeng2007}
Laeng, Mathisen, B. 2007. {``Why Do Blue-Eyed Men Prefer Women with the Same Eye Color?''} \emph{Behavioral Ecology and Sociobiology} 61(3).

\bibitem[\citeproctext]{ref-levin2000}
Levin, D. T. 2000. {``Race as a Visual Feature: Using Visual Search and Perceptual Discrimination Tasks to Understand Face Categories and the Cross-Race Recognition Deficit.''} \emph{Journal of Experimental Psychology} 129(4).

\bibitem[\citeproctext]{ref-luetkemeier2017}
LUETKEMEIER, et al., M. 2017. {``Skin Tattoos Alter Sweat Rate and Na+ Concentration.''} \emph{Medicine and Science in Sports and Exercise} 49(7).

\bibitem[\citeproctext]{ref-madden2020}
Madden, et al, J. 2020. {``Ready Student One: Exploring the Predictors of Student Learning in Virtual Reality.''} \emph{PLoS ONE} 15(3).

\bibitem[\citeproctext]{ref-miller1956}
Miller, G. A. 1956. {``The Magical Number Seven, Plus or Minus Two: Some Limits on Our Capacity for Processing Information.''} \emph{Psychological Review} 63(2).

\bibitem[\citeproctext]{ref-becentispeech}
Moquin, W., and C. Van Doren. 1973. {``Great Documents in American Indian History.''} Praeger.

\bibitem[\citeproctext]{ref-pew2022}
{``More Americans Are Joining the 'Cashless' Economy.''} 2022. \url{https://www.pewresearch.org/short-reads/2022/10/05/more-americans-are-joining-the-cashless-economy/.}

\bibitem[\citeproctext]{ref-weather}
National Weather Service Corporate Image Web Team. n.d. {``National Weather Service -- {NWS} Billings.''} \url{https://w2.weather.gov/climate/xmacis.php?wfo=byz}.

\bibitem[\citeproctext]{ref-obrien2019}
O'Brien, Lynch, H. D. 2019. {``Crocodylian Head Width Allometry and Phylogenetic Prediction of Body Size in Extinct Crocodyliforms.''} \emph{Integrative Organismal Biology} 1.

\bibitem[\citeproctext]{ref-ocean}
{``Ocean Temperature and Salinity Study.''} n.d. \url{https://calcofi.org/}.

\bibitem[\citeproctext]{ref-WashPost2022}
{``Older People Who Get Covid Are at Increased Risk of Getting Shingles.''} 2022. \url{https://www.washingtonpost.com/health/2022/04/19/shingles-and-covid-over-50/.}

\bibitem[\citeproctext]{ref-physhealth}
{``Physician's Health Study.''} n.d. \url{https://phs.bwh.harvard.edu/}.

\bibitem[\citeproctext]{ref-porath2017}
Porath, Erez, C. 2017. {``Does Rudeness Really Matter? The Effects of Rudeness on Task Performance and Helpfulness.''} \emph{Academy of Management Journal} 50.

\bibitem[\citeproctext]{ref-quinn1999}
Quinn, G. E., C. H. Shin, M. G. Maguire, and R. A. Stone. 1999. {``Myopia and Ambient Lighting at Night.''} \emph{Nature} 399 (6732): 113--14. \url{https://doi.org/10.1038/20094}.

\bibitem[\citeproctext]{ref-ramachandran2007}
Ramachandran, V. 2007. {``3 Clues to Understanding Your Brain.''} \url{https://www.ted.com/talks/vs_ramachandran_3_clues_to_understanding_your_brain}.

\bibitem[\citeproctext]{ref-cdchospitalization}
{``Rates of Laboratory-Confimed COVID-19 Hospitalizations by Vaccination Status.''} 2021. CDC. \url{https://covid.cdc.gov/covid-data-tracker/\#covidnet-hospitalizations-vaccination}.

\bibitem[\citeproctext]{ref-richardson2019}
Richardson, T., and R. T. Gilman. 2019. {``Left-Handedness Is Associated with Greater Fighting Success in Humans.''} \emph{Scientific Reports} 9 (1): 15402. \url{https://doi.org/10.1038/s41598-019-51975-3}.

\bibitem[\citeproctext]{ref-stephens2020}
Stephens, R., and O. Robertson. 2020. {``Swearing as a Response to Pain: Assessing Hypoalgesic Effects of Novel "Swear" Words.''} \emph{Frontiers in Psychology} 11: 643--62.

\bibitem[\citeproctext]{ref-stewart2014}
Stewart, E. H., B. Davis, B. L. Clemans-Taylor, B. Littenberg, C. A. Estrada, and R. M. Centor. 2014. {``Rapid Antigen Group a Streptococcus Test to Diagnose Pharyngitis: A Systematic Review and Meta-Analysis''} 9 (11). \url{https://doi.org/10.1371/journal.pone.0111727}.

\bibitem[\citeproctext]{ref-stroop1935}
Stroop, J. R. 1935. {``Studies of Interference in Serial Verbal Reactions.''} \emph{Journal of Experimental Psychology} 18: 643--62.

\bibitem[\citeproctext]{ref-subach2022}
Subach, et al, A. 2022. {``Foraging Behaviour, Habitat Use and Population Size of the Desert Horned Viper in the Negev Desert.''} \emph{Soc.Open Sci} 9.

\bibitem[\citeproctext]{ref-sulheim2017}
Sulheim, S., A. Ekeland, I. Holme, and R. Bahr. 2017. {``Helmet Use and Risk of Head Injuries in Alpine Skiers and Snowboarders: Changes After an Interval of One Decade''} 51 (1): 44--50. \url{https://doi.org/10.1136/bjsports-2015-095798}.

\bibitem[\citeproctext]{ref-titanic}
{``Titanic.''} n.d. \url{http://www.encyclopedia-titanica.org}.

\bibitem[\citeproctext]{ref-covidvaccinetracker}
{``US COVID-19 Vaccine Tracker: See Your State's Progress.''} 2021. Mayo Clinic. \url{https://www.mayoclinic.org/coronavirus-covid-19/vaccine-tracker}.

\bibitem[\citeproctext]{ref-usepa2020}
US Environmental Protection Agency. n.d. {``Air Data -- Daily Air Quality Tracker.''} \url{https://www.epa.gov/outdoor-air-quality-data/air-data-daily-air-quality-tracker}.

\bibitem[\citeproctext]{ref-wahlstrom2014}
Wahlstrom, et al, K. 2014. {``Examining the Impact of Later School Start Times on the Health and Academic Performance of High School Students: A Multi-Site Study.''} \emph{Center for Applied Research and Educational Improvement}.

\bibitem[\citeproctext]{ref-watson2015}
Watson, et al., N. 2015. {``Recommended Amount of Sleep for a Heathy Adult: A Joint Consensus Statement of the American Academy of Sleep Medicine and Sleep Research Society.''} \emph{Sleep} 38(6).

\bibitem[\citeproctext]{ref-Weiss1988}
Weiss, R. D. 1988. {``Relapse to Cocaine Abuse After Initiating Desipramine Treatment.''} \emph{JAMA} 260(17).

\bibitem[\citeproctext]{ref-navajo2011}
{``Welcome to the Navajo Nation Government: Official Site of the Navajo Nation.''} 2011.\href{\%20Retrieved\%20from\%20https://www.navajo-nsn.gov/.}{Retrieved from https://www.navajo-nsn.gov/.}

\bibitem[\citeproctext]{ref-wilson2016}
Wilson, Woodruff, J. P. 2016. {``Vertebral Adaptations to Large Body Size in Theropod Dinosaurs.''} \emph{PLoS ONE} 11(7).

\end{CSLReferences}

\end{document}
