% Options for packages loaded elsewhere
\PassOptionsToPackage{unicode}{hyperref}
\PassOptionsToPackage{hyphens}{url}
%
\documentclass[
]{report}
\usepackage{amsmath,amssymb}
\usepackage{iftex}
\ifPDFTeX
  \usepackage[T1]{fontenc}
  \usepackage[utf8]{inputenc}
  \usepackage{textcomp} % provide euro and other symbols
\else % if luatex or xetex
  \usepackage{unicode-math} % this also loads fontspec
  \defaultfontfeatures{Scale=MatchLowercase}
  \defaultfontfeatures[\rmfamily]{Ligatures=TeX,Scale=1}
\fi
\usepackage{lmodern}
\ifPDFTeX\else
  % xetex/luatex font selection
\fi
% Use upquote if available, for straight quotes in verbatim environments
\IfFileExists{upquote.sty}{\usepackage{upquote}}{}
\IfFileExists{microtype.sty}{% use microtype if available
  \usepackage[]{microtype}
  \UseMicrotypeSet[protrusion]{basicmath} % disable protrusion for tt fonts
}{}
\makeatletter
\@ifundefined{KOMAClassName}{% if non-KOMA class
  \IfFileExists{parskip.sty}{%
    \usepackage{parskip}
  }{% else
    \setlength{\parindent}{0pt}
    \setlength{\parskip}{6pt plus 2pt minus 1pt}}
}{% if KOMA class
  \KOMAoptions{parskip=half}}
\makeatother
\usepackage{xcolor}
\usepackage{color}
\usepackage{fancyvrb}
\newcommand{\VerbBar}{|}
\newcommand{\VERB}{\Verb[commandchars=\\\{\}]}
\DefineVerbatimEnvironment{Highlighting}{Verbatim}{commandchars=\\\{\}}
% Add ',fontsize=\small' for more characters per line
\usepackage{framed}
\definecolor{shadecolor}{RGB}{248,248,248}
\newenvironment{Shaded}{\begin{snugshade}}{\end{snugshade}}
\newcommand{\AlertTok}[1]{\textcolor[rgb]{0.94,0.16,0.16}{#1}}
\newcommand{\AnnotationTok}[1]{\textcolor[rgb]{0.56,0.35,0.01}{\textbf{\textit{#1}}}}
\newcommand{\AttributeTok}[1]{\textcolor[rgb]{0.13,0.29,0.53}{#1}}
\newcommand{\BaseNTok}[1]{\textcolor[rgb]{0.00,0.00,0.81}{#1}}
\newcommand{\BuiltInTok}[1]{#1}
\newcommand{\CharTok}[1]{\textcolor[rgb]{0.31,0.60,0.02}{#1}}
\newcommand{\CommentTok}[1]{\textcolor[rgb]{0.56,0.35,0.01}{\textit{#1}}}
\newcommand{\CommentVarTok}[1]{\textcolor[rgb]{0.56,0.35,0.01}{\textbf{\textit{#1}}}}
\newcommand{\ConstantTok}[1]{\textcolor[rgb]{0.56,0.35,0.01}{#1}}
\newcommand{\ControlFlowTok}[1]{\textcolor[rgb]{0.13,0.29,0.53}{\textbf{#1}}}
\newcommand{\DataTypeTok}[1]{\textcolor[rgb]{0.13,0.29,0.53}{#1}}
\newcommand{\DecValTok}[1]{\textcolor[rgb]{0.00,0.00,0.81}{#1}}
\newcommand{\DocumentationTok}[1]{\textcolor[rgb]{0.56,0.35,0.01}{\textbf{\textit{#1}}}}
\newcommand{\ErrorTok}[1]{\textcolor[rgb]{0.64,0.00,0.00}{\textbf{#1}}}
\newcommand{\ExtensionTok}[1]{#1}
\newcommand{\FloatTok}[1]{\textcolor[rgb]{0.00,0.00,0.81}{#1}}
\newcommand{\FunctionTok}[1]{\textcolor[rgb]{0.13,0.29,0.53}{\textbf{#1}}}
\newcommand{\ImportTok}[1]{#1}
\newcommand{\InformationTok}[1]{\textcolor[rgb]{0.56,0.35,0.01}{\textbf{\textit{#1}}}}
\newcommand{\KeywordTok}[1]{\textcolor[rgb]{0.13,0.29,0.53}{\textbf{#1}}}
\newcommand{\NormalTok}[1]{#1}
\newcommand{\OperatorTok}[1]{\textcolor[rgb]{0.81,0.36,0.00}{\textbf{#1}}}
\newcommand{\OtherTok}[1]{\textcolor[rgb]{0.56,0.35,0.01}{#1}}
\newcommand{\PreprocessorTok}[1]{\textcolor[rgb]{0.56,0.35,0.01}{\textit{#1}}}
\newcommand{\RegionMarkerTok}[1]{#1}
\newcommand{\SpecialCharTok}[1]{\textcolor[rgb]{0.81,0.36,0.00}{\textbf{#1}}}
\newcommand{\SpecialStringTok}[1]{\textcolor[rgb]{0.31,0.60,0.02}{#1}}
\newcommand{\StringTok}[1]{\textcolor[rgb]{0.31,0.60,0.02}{#1}}
\newcommand{\VariableTok}[1]{\textcolor[rgb]{0.00,0.00,0.00}{#1}}
\newcommand{\VerbatimStringTok}[1]{\textcolor[rgb]{0.31,0.60,0.02}{#1}}
\newcommand{\WarningTok}[1]{\textcolor[rgb]{0.56,0.35,0.01}{\textbf{\textit{#1}}}}
\usepackage{longtable,booktabs,array}
\usepackage{calc} % for calculating minipage widths
% Correct order of tables after \paragraph or \subparagraph
\usepackage{etoolbox}
\makeatletter
\patchcmd\longtable{\par}{\if@noskipsec\mbox{}\fi\par}{}{}
\makeatother
% Allow footnotes in longtable head/foot
\IfFileExists{footnotehyper.sty}{\usepackage{footnotehyper}}{\usepackage{footnote}}
\makesavenoteenv{longtable}
\usepackage{graphicx}
\makeatletter
\def\maxwidth{\ifdim\Gin@nat@width>\linewidth\linewidth\else\Gin@nat@width\fi}
\def\maxheight{\ifdim\Gin@nat@height>\textheight\textheight\else\Gin@nat@height\fi}
\makeatother
% Scale images if necessary, so that they will not overflow the page
% margins by default, and it is still possible to overwrite the defaults
% using explicit options in \includegraphics[width, height, ...]{}
\setkeys{Gin}{width=\maxwidth,height=\maxheight,keepaspectratio}
% Set default figure placement to htbp
\makeatletter
\def\fps@figure{htbp}
\makeatother
\setlength{\emergencystretch}{3em} % prevent overfull lines
\providecommand{\tightlist}{%
  \setlength{\itemsep}{0pt}\setlength{\parskip}{0pt}}
\setcounter{secnumdepth}{5}
% definitions for citeproc citations
\NewDocumentCommand\citeproctext{}{}
\NewDocumentCommand\citeproc{mm}{%
  \begingroup\def\citeproctext{#2}\cite{#1}\endgroup}
\makeatletter
 % allow citations to break across lines
 \let\@cite@ofmt\@firstofone
 % avoid brackets around text for \cite:
 \def\@biblabel#1{}
 \def\@cite#1#2{{#1\if@tempswa , #2\fi}}
\makeatother
\newlength{\cslhangindent}
\setlength{\cslhangindent}{1.5em}
\newlength{\csllabelwidth}
\setlength{\csllabelwidth}{3em}
\newenvironment{CSLReferences}[2] % #1 hanging-indent, #2 entry-spacing
 {\begin{list}{}{%
  \setlength{\itemindent}{0pt}
  \setlength{\leftmargin}{0pt}
  \setlength{\parsep}{0pt}
  % turn on hanging indent if param 1 is 1
  \ifodd #1
   \setlength{\leftmargin}{\cslhangindent}
   \setlength{\itemindent}{-1\cslhangindent}
  \fi
  % set entry spacing
  \setlength{\itemsep}{#2\baselineskip}}}
 {\end{list}}
\usepackage{calc}
\newcommand{\CSLBlock}[1]{\hfill\break\parbox[t]{\linewidth}{\strut\ignorespaces#1\strut}}
\newcommand{\CSLLeftMargin}[1]{\parbox[t]{\csllabelwidth}{\strut#1\strut}}
\newcommand{\CSLRightInline}[1]{\parbox[t]{\linewidth - \csllabelwidth}{\strut#1\strut}}
\newcommand{\CSLIndent}[1]{\hspace{\cslhangindent}#1}
\usepackage{booktabs}
\usepackage{geometry}
\usepackage[none]{hyphenat}
\usepackage{titlesec}
\usepackage{longtable}
\usepackage{xcolor}
\usepackage{setspace}
\usepackage{pdfpages}

\pagestyle{plain}

%%%% Set margins
\setlength{\topmargin}{-1cm}
\addtolength{\evensidemargin}{-1cm}
\addtolength{\oddsidemargin}{-1cm}
\addtolength{\textheight}{3cm}
\addtolength{\textwidth}{2cm}

% Spacing for reading guides
\newcommand{\rgs}{\vspace{12pt}} % Vertical space
\newcommand{\rgi}{\hspace{24pt}}  % Indent

\newcommand\latexcode[1]{#1}

% Format chapter titles and spacing
\renewcommand*{\chaptername}{Module}

\titleformat{\chapter}[display]
{\bfseries\Large}
{\filleft\MakeUppercase{\chaptertitlename} \Huge\thechapter}
{3ex}
{\titlerule
\vspace{1.5ex}%
\filright}
[\vspace{1.5ex}%
\titlerule]
\titlespacing*{\chapter}{0pt}{-40pt}{20pt}
\ifLuaTeX
  \usepackage{selnolig}  % disable illegal ligatures
\fi
\usepackage{bookmark}
\IfFileExists{xurl.sty}{\usepackage{xurl}}{} % add URL line breaks if available
\urlstyle{same}
\hypersetup{
  hidelinks,
  pdfcreator={LaTeX via pandoc}}

\title{\textbf{STAT 216 Coursepack}\\
\strut \\
\includegraphics[width=5in,height=\textheight]{images/msu-campus.jpg}}
\usepackage{etoolbox}
\makeatletter
\providecommand{\subtitle}[1]{% add subtitle to \maketitle
  \apptocmd{\@title}{\par {\large #1 \par}}{}{}
}
\makeatother
\subtitle{Spring 2025\\
Montana State University}
\author{Melinda Yager\\
Jade Schmidt\\
Stacey Hancock}
\date{}

\begin{document}
\maketitle

\newpage
\thispagestyle{empty}

This resource was developed by Melinda Yager, Jade Schmidt, and Stacey Hancock in 2021 to accompany the online textbook: Hancock, S., Carnegie, N., Meyer, E., Schmidt, J., and Yager, M. (2021). \emph{Montana State Introductory Statistics with R}. Montana State University. \url{https://mtstateintrostats.github.io/IntroStatTextbook/}.

This resource is released under a \href{https://creativecommons.org/licenses/by-nc-sa/4.0/}{Creative Commons BY-NC-SA 4.0} license unless otherwise noted.

\setcounter{tocdepth}{1}
\addtocontents{toc}{\protect\thispagestyle{empty}}
\tableofcontents
\thispagestyle{empty}

\newpage
\setcounter{page}{1}

\chapter*{Preface}\label{preface}
\addcontentsline{toc}{chapter}{Preface}

This coursepack accompanies the textbook for STAT 216: Montana State Introductory Statistics with R, which can be found at \url{https://mtstateintrostats.github.io/IntroStatTextbook/}. The syllabus for the course (including the course calendar), data sets, and links to D2L Brightspace, Gradescope, and the MSU RStudio server can be found on the course webpage: \url{https://math.montana.edu/courses/s216/}.
Other notes and review materials are linked in D2L.

Each of the activities in this workbook is designed to target specific learning outcomes of the course, giving you practice with important statistical concepts in a group setting with instructor guidance. In addition to the in-class activities for the course, video notes are provided to aid in taking notes while you complete the required videos. Bring this workbook with you to class each class period, and take notes in the workbook as you would your own notes. A well-written completed workbook will provide an optimal study guide for exams!

All activities and labs in this coursepack will be completed during class time. Parts of each lab will be turned in on Gradescope. To aid in your understanding, read through the introduction for each activity before attending class each day.

STAT 216 is a 3-credit in-person course. In our experience, it takes six to nine hours per week outside of class to achieve a good grade in this class. By ``good'' we mean at least a C because a grade of D or below does not count toward fulfilling degree requirements. Many of you set your goals higher than just getting a C, and we fully support that. You need roughly nine hours per week to review past activities, read feedback on previous assignments, complete current assignments, and prepare for the next day's class. A typical week in the life of a STAT 216 student looks like:

\begin{itemize}
\tightlist
\item
  \emph{Prior to class meeting}:

  \begin{itemize}
  \tightlist
  \item
    Read assigned sections of the textbook, using the provided reading guides to take notes on the material.
  \item
    Watch the provided videos, taking notes in the coursepack.
  \item
    Read through the introduction to the day's in-class activity.
  \item
    Read through the week's homework assignment and note any questions you may have on the content.
  \end{itemize}
\item
  \emph{During class meeting}:

  \begin{itemize}
  \tightlist
  \item
    Work through the guided activity, in-class activity or weekly lab with your classmates and instructor, taking detailed notes on your answers to each question in the activity.
  \end{itemize}
\item
  \emph{After class meeting}:

  \begin{itemize}
  \tightlist
  \item
    Complete any parts of the activity you did not complete in class.
  \item
    Review the activity solutions in the Math and Stat Center, and take notes on key points.
  \item
    Complete any remaining assigned readings for the week.
  \item
    Complete the week's homework assignment.
  \end{itemize}
\end{itemize}

\nocite{*}

\chapter{Exploratory Data Analysis and Inference for Two Quantitative Variables}\label{exploratory-data-analysis-and-inference-for-two-quantitative-variables}

\section{Vocabulary Review and Key Topics}\label{vocabulary-review-and-key-topics}

Review the Golden Ticket posted in the resources at the end of the coursepack for a summary of two quantitative variables.

\subsection{Key topics}\label{key-topics}

Module 13 will cover exploratory data analysis and both simulation-based and theory-based methods of inference for two quantitative variables. The \textbf{summary measure} for two quantitative variables is either the \textbf{slope} of a regression line or the \textbf{correlation} between the two variables.

\begin{itemize}
\item
  Notation for a sample regression slope: \(b_1\)
\item
  Notation for a population regression slope: \(\beta_1\)
\item
  Notation for a sample correlation: \(r\)
\item
  Notation for a population correlation: \(\rho\)
\end{itemize}

Types of plots for two quantitative variables:

\begin{itemize}
\tightlist
\item
  Scatterplot
\end{itemize}

\subsection{Vocabulary}\label{vocabulary}

\subsubsection*{Plotting two quantitative variables}\label{plotting-two-quantitative-variables}
\addcontentsline{toc}{subsubsection}{Plotting two quantitative variables}

\begin{itemize}
\item
  \textbf{Scatterplot}: plots \((x,y)\) pairs of observations with the explanatory variable on the \(x\)-axis and the response variable on the \(y\)-axis. R code to create a scatterplot:

\begin{Shaded}
\begin{Highlighting}[]
\NormalTok{object }\SpecialCharTok{\%\textgreater{}\%} \CommentTok{\# Pipe data set into...}
\FunctionTok{ggplot}\NormalTok{(}\FunctionTok{aes}\NormalTok{(}\AttributeTok{x =}\NormalTok{ explanatory, }\AttributeTok{y =}\NormalTok{ response))}\SpecialCharTok{+}  \CommentTok{\# Specify variables}
  \FunctionTok{geom\_point}\NormalTok{(}\AttributeTok{alpha=}\FloatTok{0.5}\NormalTok{) }\SpecialCharTok{+}  \CommentTok{\# Add scatterplot of points}
  \FunctionTok{labs}\NormalTok{(}\AttributeTok{x =} \StringTok{"x{-}axis label"}\NormalTok{,  }\CommentTok{\# Label x{-}axis}
   \AttributeTok{y =} \StringTok{"y{-}axis lable"}\NormalTok{,  }\CommentTok{\# Label y{-}axis}
   \AttributeTok{title =} \StringTok{"Don\textquotesingle{}t forget to add a title!"}\NormalTok{) }\SpecialCharTok{+} 
           \CommentTok{\# Be sure to tile your plots}
  \FunctionTok{geom\_smooth}\NormalTok{(}\AttributeTok{method =} \StringTok{"lm"}\NormalTok{, }\AttributeTok{se =} \ConstantTok{FALSE}\NormalTok{)  }\CommentTok{\# Add regression line}
\end{Highlighting}
\end{Shaded}

  \begin{itemize}
  \tightlist
  \item
    If there is a third categorical variable, you can use color or shape to include the third variable on the scatterplot.
  \end{itemize}
\item
  Four characteristics of scatterplots:

  \begin{itemize}
  \item
    Form (linear or non-linear)
  \item
    Direction (positive or negative)
  \item
    Strength (weak, moderate, or strong)
  \item
    Outliers?
  \end{itemize}
\end{itemize}

\subsubsection*{Sample statistics for two quantitative variables}\label{sample-statistics-for-two-quantitative-variables}
\addcontentsline{toc}{subsubsection}{Sample statistics for two quantitative variables}

\begin{itemize}
\item
  \textbf{Least-squares regression line}: a line fit to the data which minimizes the squared vertical distances from the observed \(y\)-value to the line

  \begin{itemize}
  \item
    \textbf{Notation for the fitted least-squares regression line}:
    \[\hat{y} = b_0 + b_1 \times x\]
    or
    \[\widehat{response} = b_0 + b_1 \times explanatory\]
    To write the equation of the regression line in context of the problem, include descriptive names of the response and explanatory variables for ``\(y\)/response'' and ``\(x\)/explanatory'' above.
  \item
    \(b_0\) is the \textbf{\(y\)-intercept} of the regression line: the \emph{predicted} value of the response variable when the explanatory variable is equal to zero.
  \item
    \(b_1\) is the \textbf{slope} of the regression line: the \emph{predicted} increase/decrease in the response variable associated with a one-unit increase in the explanatory variable.
  \item
    The distance from an observation's \(y\)-value (observed response) to its fitted value, \(\hat{y}\) (the value on the line) is called a \textbf{residual}:
    \[ \text{residual} = \text{observed} - \text{fitted} = y - \hat{y}\]
  \item
    A least-squares regression line is a special case of a \textbf{linear model}.
  \item
    R code to find the least-squares regression line (fit the linear model):
  \end{itemize}

\begin{Shaded}
\begin{Highlighting}[]
\NormalTok{linearmodel }\OtherTok{\textless{}{-}} \FunctionTok{lm}\NormalTok{(response}\SpecialCharTok{\textasciitilde{}}\NormalTok{explanatory, }\AttributeTok{data=}\NormalTok{object)}
\FunctionTok{round}\NormalTok{(}\FunctionTok{summary}\NormalTok{(linearmodel)}\SpecialCharTok{$}\NormalTok{coefficients,}\DecValTok{3}\NormalTok{) }\CommentTok{\# Display coefficients}
\end{Highlighting}
\end{Shaded}
\item
  \textbf{Correlation}: measures the magnitude and direction of the linear relationship between two quantitative variables.

  \begin{itemize}
  \item
    Parameter notation: \(\rho\)
  \item
    Sample notation: \(r\)
  \item
    R code to find the \textbf{correlation} matrix between variables:
  \end{itemize}

\begin{Shaded}
\begin{Highlighting}[]
\CommentTok{\# Melinda add code}
\end{Highlighting}
\end{Shaded}
\item
  \textbf{Coefficient of determination}: measures the proportion of total variability in the response variable that is explained by the linear relationship with the explanatory variable. The coefficient of determination can be calculated in three ways:
  \[r^2 = (r)^2 = \frac{SST - SSE}{SST} = \frac{s^2_y - s^2_{residual}}{s^2_y}\]
\end{itemize}

\subsubsection*{Hypotheses}\label{hypotheses}
\addcontentsline{toc}{subsubsection}{Hypotheses}

Hypotheses involving two quantitative variables can be expressed either in terms of the slope or the correlation. When either the slope or correlation is equal to zero, there is no linear relationship between the two quantitative variables (the null hypothesis).

\begin{itemize}
\tightlist
\item
  \textbf{Hypotheses in notation for slope}:
\end{itemize}

\[H_0: \beta_1 = 0\]

\[H_A: \beta_1 \left\{
\begin{array}{ll}
< \\
\ne \\
< \\
\end{array}
\right\}
0\]

\begin{itemize}
\tightlist
\item
  \textbf{Hypotheses in notation for correlation}:
\end{itemize}

\[H_0: \rho = 0\]
\[H_A: \rho \left\{
\begin{array}{ll}
< \\
\ne \\
< \\
\end{array}
\right\}
0\]

\subsubsection*{Simulation-based inference for two quantitative variables}\label{simulation-based-inference-for-two-quantitative-variables}
\addcontentsline{toc}{subsubsection}{Simulation-based inference for two quantitative variables}

\begin{itemize}
\item
  \textbf{Conditions necessary to use simulation-based methods for inference for two quantitative variables}:

  \begin{itemize}
  \item
    \textbf{Independence}: observational units (the \((x, y)\) pairs) must be independent of one another.
  \item
    \textbf{Linearity}: the form of the relationship (if any) between the two variables must be linear.
  \end{itemize}
\item
  \textbf{Simulation-based methods to create the null distribution}: R code for simulation methods to find the p-value using the \texttt{regression\_test} function in the \texttt{catstats} package.

\begin{Shaded}
\begin{Highlighting}[]
\FunctionTok{regression\_test}\NormalTok{(response}\SpecialCharTok{\textasciitilde{}}\NormalTok{explanatory, }\CommentTok{\# response \textasciitilde{} explanatory}
           \AttributeTok{data =}\NormalTok{ object, }\CommentTok{\# Name of data set}
           \AttributeTok{direction =} \StringTok{"xx"}\NormalTok{, }\CommentTok{\# Sign in alternative ("greater", "less", "two{-}sided")}
           \AttributeTok{summary\_measure =} \StringTok{"xx"}\NormalTok{, }\CommentTok{\# "slope" or "correlation"}
           \AttributeTok{as\_extreme\_as =}\NormalTok{ xx, }\CommentTok{\# Observed slope or correlation}
           \AttributeTok{number\_repetitions =} \DecValTok{10000}\NormalTok{) }\CommentTok{\# Number of simulated samples for null distribution}
\end{Highlighting}
\end{Shaded}
\item
  \textbf{Simulation-based methods to create the bootstrap distribution}: R code to find the simulation-based confidence interval using the \texttt{regression\_bootstrap\_CI} function from the \texttt{catstats} package.

\begin{Shaded}
\begin{Highlighting}[]
\FunctionTok{regression\_bootstrap\_CI}\NormalTok{(response}\SpecialCharTok{\textasciitilde{}}\NormalTok{explanatory, }\CommentTok{\# response \textasciitilde{} explanatory}
   \AttributeTok{data =}\NormalTok{ object, }\CommentTok{\# Name of data set}
   \AttributeTok{confidence\_level =}\NormalTok{ xx, }\CommentTok{\# Confidence level as decimal}
   \AttributeTok{summary\_measure =} \StringTok{"xx"}\NormalTok{, }\CommentTok{\# Slope or correlation}
   \AttributeTok{number\_repetitions =} \DecValTok{10000}\NormalTok{) }\CommentTok{\# Number of simulated samples for bootstrap distribution}
\end{Highlighting}
\end{Shaded}
\end{itemize}

\subsubsection{Theory-based methods for two quantitative variables}\label{theory-based-methods-for-two-quantitative-variables}

\begin{itemize}
\item
  \textbf{Conditions necessary to use theory-based methods for inference for two quantitative variables}:

  \begin{itemize}
  \item
    \textbf{Independence} (for both simulation-based and theory-based methods): observational units (the \((x, y)\) pairs) must be independent of one another.
    \textbar{} - Check this assumption by investigating the sampling method and determining if the observational units are related in any way.
  \item
    \textbf{Linearity} (for both simulation-based and theory-based methods): the form of the relationship (if any) between the two variables must be linear.
    \textbar{} - Check this assumption by examining the scatterplot of the two variables, and a scatterplot of the residuals (on the \(y\)-axis) versus the fitted values (on the \(x\)-axis). The pattern in the residuals vs.~fitted plot should display a horizontal line.
  \item
    \textbf{Constant variability} (for theory-based methods only): the variability of points around the least squares line remains roughly constant
    \textbar{} - Check this assumption by examining a scatterplot of the residuals (on the \(y\)-axis) versus the fitted values (on the \(x\)-axis). The variability in the residuals around zero should be approximately the same for all fitted values.
  \item
    \textbf{Nearly normal residuals} (for theory-based methods only): residuals must be nearly normal.
    \textbar{} - Check this assumption by examining a histogram of the residuals, which should appear approximately normal.
  \end{itemize}
\item
  \textbf{Standard error of the slope of the least-squares regression line} (\(SE(b_1)\)): obtain the value of the standard error of the slope from the linear model (\texttt{lm}) R output.
\item
  \textbf{Standardized slope}:
  \[
  T = \frac{\mbox{slope estimate}-null value}{SE} = \frac{b_1-0}{SE(b_1)}.
  \]

  \begin{itemize}
  \tightlist
  \item
    The p-value can be found from the linear model (\texttt{lm}) R output or by using the \texttt{pt} function in R to find the area under a \(t\)-distribution with \(n-2\) degrees of freedom where \(T\) is as or more extreme as the value observed (in the direction of \(H_A\)).
  \end{itemize}
\item
  \textbf{Margin of error}: half the width of the confidence interval. For a regression slope, the margin of error is:
  \[ME = t^* \times SE(b_1)\]
  where \(t^*\) is the \textbf{multiplier}, corresponding to the desired confidence level found from a \(t\)-distribution with \(n-2\) degrees of freedom.

  \begin{itemize}
  \item
    Use the \texttt{qt} function in R to find the \(t^*\) multiplier with \(n-2\) degrees of freedom.
  \item
    To find the endpoints of a confidence interval, add and subtract the margin of error to the sample statistic. The confidence interval for a population slope is:
    \[b_1 \pm ME\]
  \end{itemize}
\end{itemize}

\newpage

\section{Video Notes: Regression and Correlation}\label{video-notes-regression-and-correlation}

Read Chapters 6, 7, 8, 21, and 22 in the course textbook. Use the following videos to complete the video notes for Module 13.

\subsection{Course Videos}\label{course-videos}

\begin{itemize}
\item
  6.1
\item
  6.2
\item
  6.3
\item
  Ch 7
\item
  21.1
\item
  21.3
\item
  21.4TheoryTests
\item
  21.4TheoryIntervals
\end{itemize}

\setstretch{1}

\subsection*{Summary measures and plots for two quantitative variables - Videos 6.1 - 6.3}\label{summary-measures-and-plots-for-two-quantitative-variables---videos-6.1---6.3}
\addcontentsline{toc}{subsection}{Summary measures and plots for two quantitative variables - Videos 6.1 - 6.3}

Example: Data were collected from 1236 births between 1960 and 1967 in the San Francisco East Bay area to better understand what variables contributed to child birthweight, as children with low birthweight often suffer from an array of complications later in life ({``Child Health and Development Studies,''} n.d.). There were some missing values in the study and with those observations removed we have a total of 1223 births.

\begin{Shaded}
\begin{Highlighting}[]
\NormalTok{babies}\OtherTok{\textless{}{-}}\FunctionTok{read.csv}\NormalTok{(}\StringTok{"data/babies.csv"}\NormalTok{) }\SpecialCharTok{\%\textgreater{}\%}
    \FunctionTok{drop\_na}\NormalTok{(bwt) }\SpecialCharTok{\%\textgreater{}\%}
    \FunctionTok{drop\_na}\NormalTok{(gestation)}
\FunctionTok{glimpse}\NormalTok{(babies)}
\CommentTok{\#\textgreater{} Rows: 1,223}
\CommentTok{\#\textgreater{} Columns: 8}
\CommentTok{\#\textgreater{} $ case      \textless{}int\textgreater{} 1, 2, 3, 5, 6, 7, 8, 9, 10, 11, 12, 13, 14, 15, 16, 17, 18, \textasciitilde{}}
\CommentTok{\#\textgreater{} $ bwt       \textless{}int\textgreater{} 120, 113, 128, 108, 136, 138, 132, 120, 143, 140, 144, 141, \textasciitilde{}}
\CommentTok{\#\textgreater{} $ gestation \textless{}int\textgreater{} 284, 282, 279, 282, 286, 244, 245, 289, 299, 351, 282, 279, \textasciitilde{}}
\CommentTok{\#\textgreater{} $ parity    \textless{}int\textgreater{} 0, 0, 0, 0, 0, 0, 0, 0, 0, 0, 0, 0, 0, 0, 0, 0, 0, 0, 0, 0, \textasciitilde{}}
\CommentTok{\#\textgreater{} $ age       \textless{}int\textgreater{} 27, 33, 28, 23, 25, 33, 23, 25, 30, 27, 32, 23, 36, 30, 38, \textasciitilde{}}
\CommentTok{\#\textgreater{} $ height    \textless{}int\textgreater{} 62, 64, 64, 67, 62, 62, 65, 62, 66, 68, 64, 63, 61, 63, 63, \textasciitilde{}}
\CommentTok{\#\textgreater{} $ weight    \textless{}int\textgreater{} 100, 135, 115, 125, 93, 178, 140, 125, 136, 120, 124, 128, 9\textasciitilde{}}
\CommentTok{\#\textgreater{} $ smoke     \textless{}int\textgreater{} 0, 0, 1, 1, 0, 0, 0, 0, 1, 0, 1, 1, 1, 0, 0, 1, 1, 0, 1, 0, \textasciitilde{}}
\end{Highlighting}
\end{Shaded}

Here you see a glimpse of the data. The 1223 rows correspond to the sample size. The case variable is labeling each pregnancy 1 through 1223. Then 7 variables are recorded. birthweight (bwt), length of gestation in days, parity is called an indicator variable telling us if the pregnancy was a first pregnancy (labeled as 0) or not (labeled as 1) were recorded about the child and pregnancy. The age, height, and weight were recorded for the mother giving birth, as was smoke, another indicator variable where 0 means the mother did not smoke during pregnancy, and 1 indicates that she did smoke while pregnant.

\setstretch{1.5}

\subsubsection*{Type of plot}\label{type-of-plot}
\addcontentsline{toc}{subsubsection}{Type of plot}

A \_\_\_\_\_\_\_\_\_\_\_\_\_\_\_\_\_\_ is used to display the relationship
between two \_\_\_\_\_\_\_\_\_\_\_\_\_\_\_\_\_\_\_ variables.

\setstretch{1}
\newpage

Four characteristics of the scatterplot:

\begin{itemize}
\tightlist
\item
  Form:
\end{itemize}

\vspace{0.2in}

\begin{itemize}
\tightlist
\item
  Direction:
\end{itemize}

\vspace{0.2in}

\begin{itemize}
\tightlist
\item
  Strength:
\end{itemize}

\vspace{0.2in}

\begin{itemize}
\tightlist
\item
  Outliers:
\end{itemize}

\vspace{0.2in}

\rgi \rgi - Influential points: outliers that change the regression line; far from the line of regression

\rgi \rgi - High leverage points: outliers that are extreme in the x- axis; far from the mean of the x-axis

The following shows a scatterplot of length of gestation as a predictor of birthweight.

\begin{Shaded}
\begin{Highlighting}[]
\NormalTok{babies }\SpecialCharTok{\%\textgreater{}\%} \CommentTok{\# Data set pipes into...}
\FunctionTok{ggplot}\NormalTok{(}\FunctionTok{aes}\NormalTok{(}\AttributeTok{x =}\NormalTok{ gestation, }\AttributeTok{y =}\NormalTok{ bwt))}\SpecialCharTok{+}  \CommentTok{\# Specify variables}
  \FunctionTok{geom\_point}\NormalTok{(}\AttributeTok{alpha=}\FloatTok{0.5}\NormalTok{) }\SpecialCharTok{+}  \CommentTok{\# Add scatterplot of points}
  \FunctionTok{labs}\NormalTok{(}\AttributeTok{x =} \StringTok{"number of days of gestation"}\NormalTok{,  }\CommentTok{\# Label x{-}axis}
       \AttributeTok{y =} \StringTok{"birthweight (oz)"}\NormalTok{,  }\CommentTok{\# Label y{-}axis}
       \AttributeTok{title =} \StringTok{"Scatterplot of Gestation vs. Birthweight for Births}
\StringTok{       between 1960 and 1967 in San Francisco"}\NormalTok{) }\SpecialCharTok{+} 
    \CommentTok{\# Be sure to title your plots with the type of plot, observational units, variable(s)}
  \FunctionTok{geom\_smooth}\NormalTok{(}\AttributeTok{method =} \StringTok{"lm"}\NormalTok{, }\AttributeTok{se =} \ConstantTok{FALSE}\NormalTok{) }\SpecialCharTok{+} \CommentTok{\# Add regression line}
    \FunctionTok{theme\_bw}\NormalTok{()}
\end{Highlighting}
\end{Shaded}

\begin{center}\includegraphics[width=0.8\linewidth]{13-VN13-regression_files/figure-latex/unnamed-chunk-2-1} \end{center}

Describe the scatterplot using the four characteristics of a scatterplot.

\vspace{1in}

\setstretch{1.5}

The summary measures for two quantitative variables are:

\begin{itemize}
\item
  \begin{center}\rule{0.5\linewidth}{0.5pt}\end{center}
\item
  \begin{center}\rule{0.5\linewidth}{0.5pt}\end{center}
\item
  \begin{center}\rule{0.5\linewidth}{0.5pt}\end{center}
\end{itemize}

\setstretch{1}

Notation:

\begin{itemize}
\item
  Population slope:
\item
  Population correlation:
\item
  Sample slope:
\item
  Sample correlation:
\end{itemize}

\subsubsection*{Correlation}\label{correlation}
\addcontentsline{toc}{subsubsection}{Correlation}

Correlation is always between the values of \_\_\_\_\_\_\_ and \_\_\_\_\_\_\_\_.

\begin{itemize}
\item
  Measures the \_\_\_\_\_\_\_\_\_\_\_\_\_ and \_\_\_\_\_\_\_\_\_\_\_\_\_\_ of the linear relationship between two quantitative variables.
\item
  The stronger the relationship between the variables the closer the value of \_\_\_\_\_\_\_\_\_\_\_\_\_\_\_ is to \_\_\_\_\_\_\_\_ or \_\_\_\_\_\_\_\_.
\item
  The sign gives the \_\_\_\_\_\_\_\_\_\_\_\_\_\_\_\_\_.
\end{itemize}

The following code creates a correlation matrix between different quantitative variables in the data set.

\begin{Shaded}
\begin{Highlighting}[]
\NormalTok{babies }\SpecialCharTok{\%\textgreater{}\%}
    \FunctionTok{select}\NormalTok{(}\FunctionTok{c}\NormalTok{(}\StringTok{"gestation"}\NormalTok{, }\StringTok{"age"}\NormalTok{, }\StringTok{"height"}\NormalTok{, }\StringTok{"weight"}\NormalTok{, }\StringTok{"bwt"}\NormalTok{)) }\SpecialCharTok{\%\textgreater{}\%}
    \FunctionTok{cor}\NormalTok{(}\AttributeTok{use=}\StringTok{"pairwise.complete.obs"}\NormalTok{) }\SpecialCharTok{\%\textgreater{}\%}
    \FunctionTok{round}\NormalTok{(}\DecValTok{3}\NormalTok{)}
\end{Highlighting}
\end{Shaded}

\begin{verbatim}
#>           gestation    age height weight   bwt
#> gestation     1.000 -0.056  0.064  0.022 0.408
#> age          -0.056  1.000 -0.005  0.147 0.029
#> height        0.064 -0.005  1.000  0.436 0.201
#> weight        0.022  0.147  0.436  1.000 0.154
#> bwt           0.408  0.029  0.201  0.154 1.000
\end{verbatim}

\setstretch{1.5}

The value of correlation between gestation and birthweight is \_\_\_\_\_\_\_\_\_\_\_\_\_\_. This shows a \_\_\_\_\_\_\_\_\_\_\_, \_\_\_\_\_\_\_\_\_\_\_\_\_ relationship between gestation and birthweight.

\setstretch{1}

\subsubsection*{Slope}\label{slope}
\addcontentsline{toc}{subsubsection}{Slope}

\begin{itemize}
\item
  Least-squares regression line: \(\hat{y}=b_0+b_1\times x\) (put y and x in the context of the problem) or \(\widehat{response}=b_0+b_1 \times \text{explanatory}\)
\item
  \(\hat{y}\) or \(\widehat{\text{response}}\) is
\end{itemize}

\vspace{0.1in}

\begin{itemize}
\tightlist
\item
  \(b_0\) is
\end{itemize}

\vspace{0.1in}

\begin{itemize}
\tightlist
\item
  \(b_1\) is
\end{itemize}

\vspace{0.1in}

\begin{itemize}
\tightlist
\item
  \(x\) or explanatory is
\end{itemize}

\vspace{0.1in}

\setstretch{1.5}

\begin{itemize}
\item
  The estimates for the linear model output will give the value of the \_\_\_\_\_\_\_\_\_\_\_\_\_\_\_\_\_\_\_ and the \_\_\_\_\_\_\_\_\_\_\_\_\_\_.
\item
  Interpretation of slope: an increase in the \_\_\_\_\_\_\_\_\_\_\_\_\_ variable of 1 unit is associated with an increase/decrease in the \_\_\_\_\_\_\_\_\_\_\_\_\_\_\_\_ variable by the value of slope, on average.
\item
  Interpretation of the y-intercept: for a value of 0 for the \_\_\_\_\_\_\_\_\_\_\_\_\_ variable, the predicted value for the \_\_\_\_\_\_\_\_\_\_ variable would be the value of y-intercept.
\item
  We can predict values of the \_\_\_\_\_\_\_\_\_\_\_ variable by plugging in a given \_\_\_\_\_\_\_\_\_\_ variable value using the least squares equation line.
\item
  A prediction of a response variable value for an explanatory value outside the range of x values is called \_\_\_\_\_\_\_\_\_\_\_\_\_\_\_.
\item
  To find how far the predicted value deviates from the actual value we find the \_\_\_\_\_\_\_\_\_\_\_\_.
\end{itemize}

\vspace{0.3in}

\begin{itemize}
\item
  To find the least squares regression line the line with the \_\_\_\_\_\_\_\_\_\_ SSE is found.

  SSE = sum of squared errors

  \begin{itemize}
  \tightlist
  \item
    To find SSE, the residual for each data point is found, squared and all the squared residuals are summed together
  \end{itemize}
\end{itemize}

The linear model output for this study is given below:

\begin{Shaded}
\begin{Highlighting}[]
\CommentTok{\# Fit linear model: y \textasciitilde{} x}
\NormalTok{babiesLM }\OtherTok{\textless{}{-}} \FunctionTok{lm}\NormalTok{(bwt }\SpecialCharTok{\textasciitilde{}}\NormalTok{ gestation, }\AttributeTok{data=}\NormalTok{babies)}
\FunctionTok{round}\NormalTok{(}\FunctionTok{summary}\NormalTok{(babiesLM)}\SpecialCharTok{$}\NormalTok{coefficients,}\DecValTok{3}\NormalTok{) }\CommentTok{\# Display coefficient summary}
\end{Highlighting}
\end{Shaded}

\begin{verbatim}
#>             Estimate Std. Error t value Pr(>|t|)
#> (Intercept)  -10.064      8.322  -1.209    0.227
#> gestation      0.464      0.030  15.609    0.000
\end{verbatim}

Write the least squares equation of the line.

\vspace{0.6in}

Interpret the slope in context of the problem.

\vspace{0.6in}

Interpret the y-intercept in context of the problem.

\vspace{0.6in}

Predict the birthweight for a birth with a baby born at 310 days gestation.

\vspace{0.5in}

Calculate the residual for a birth of a baby with a birthweight of 151 ounces and born at 310 days gestation.

\vspace{0.5in}

Is this value (310, 151) above or below the line of regression? Did the line of regression overestimate or underestimate the birthweight?

\vspace{0.2in}

\subsubsection*{Coefficient of Determination}\label{coefficient-of-determination}
\addcontentsline{toc}{subsubsection}{Coefficient of Determination}

The coefficient of determination can be found by squaring the value of correlation, using the variances for each variable or using the SSE (sum of squares error) and SST (sum of squares total)

\begin{itemize}
\item
  \(r^2 = (r)^2 = \frac{SST - SSE}{SST} = \frac{s^2_y - s^2_{residual}}{s^2_y}\)
\item
  The coefficient of determination measures the \_\_\_\_\_\_\_\_\_\_\_\_ of total variation in the \_\_\_\_\_\_\_\_\_\_\_ variable that is explained by the changes in the \_\_\_\_\_\_\_\_\_\_\_\_\_ variable.
\end{itemize}

\setstretch{1}

\begin{center}\includegraphics[width=0.7\linewidth]{13-VN13-regression_files/figure-latex/unnamed-chunk-6-1} \end{center}

The value for SST was calculated as 406753.48. The value for SSE was calculated as 339092.13.

Calculate the coefficient of determination between gestation and birthweight.

\vspace{0.3in}

Interpret the coefficient of determination between gestation and birthweight.

\vspace{0.5in}

\newpage

\subsubsection*{Multivariable plots - Video Chapter7}\label{multivariable-plots---video-chapter7}
\addcontentsline{toc}{subsubsection}{Multivariable plots - Video Chapter7}

Aesthetics: visual property of the objects in your plot

\setstretch{1.5}

\begin{itemize}
\item
  Position on the axes: groups for \_\_\_\_\_\_\_\_\_\_\_\_\_\_\_ variables, or a number line if the variable is \_\_\_\_\_\_\_\_\_\_\_\_\_\_\_\_\_
\item
  Color or shape - to represent \_\_\_\_\_\_\_\_\_\_\_\_\_\_\_ variables
\item
  Size - to represent \_\_\_\_\_\_\_\_\_\_\_\_\_\_\_\_ variables
\end{itemize}

\setstretch{1}

Adding the quantitative variable maternal age to the scatterplot between gestation and birthweight.

\begin{Shaded}
\begin{Highlighting}[]
\NormalTok{babies }\SpecialCharTok{\%\textgreater{}\%} \CommentTok{\# Data set pipes into...}
\FunctionTok{ggplot}\NormalTok{(}\FunctionTok{aes}\NormalTok{(}\AttributeTok{x =}\NormalTok{ gestation, }\AttributeTok{y =}\NormalTok{ bwt))}\SpecialCharTok{+}  \CommentTok{\# Specify variables}
  \FunctionTok{geom\_point}\NormalTok{(}\AttributeTok{alpha=}\FloatTok{0.5}\NormalTok{, }\AttributeTok{shape=}\DecValTok{1}\NormalTok{, }\FunctionTok{aes}\NormalTok{(}\AttributeTok{size=}\NormalTok{age)) }\SpecialCharTok{+}  \CommentTok{\# Add scatterplot of points}
  \FunctionTok{labs}\NormalTok{(}\AttributeTok{x =} \StringTok{"number of days of gestation"}\NormalTok{,  }\CommentTok{\# Label x{-}axis}
       \AttributeTok{y =} \StringTok{"birthweight (oz)"}\NormalTok{,  }\CommentTok{\# Label y{-}axis}
       \AttributeTok{title =} \StringTok{"Scatterplot of Gestation vs. Birthweight by Age }
\StringTok{       for Births between 1960 and 1967 in San Francisco"}\NormalTok{) }\SpecialCharTok{+} 
    \CommentTok{\# Be sure to title your plots}
  \FunctionTok{geom\_smooth}\NormalTok{(}\AttributeTok{method =} \StringTok{"lm"}\NormalTok{, }\AttributeTok{se =} \ConstantTok{FALSE}\NormalTok{)  }\CommentTok{\# Add regression line}
\end{Highlighting}
\end{Shaded}

\begin{center}\includegraphics[width=0.8\linewidth]{13-VN13-regression_files/figure-latex/unnamed-chunk-7-1} \end{center}

\newpage

Let's add the categorical variable, whether a mother smoked, to the scatterplot between gestation and birthweight.

\begin{Shaded}
\begin{Highlighting}[]
\NormalTok{babies }\OtherTok{\textless{}{-}}\NormalTok{ babies }\SpecialCharTok{\%\textgreater{}\%} 
    \FunctionTok{mutate}\NormalTok{(}\AttributeTok{smoke =} \FunctionTok{factor}\NormalTok{(smoke)) }\SpecialCharTok{\%\textgreater{}\%}
    \FunctionTok{na.omit}\NormalTok{()}
           
\NormalTok{babies }\SpecialCharTok{\%\textgreater{}\%} \CommentTok{\# Data set pipes into...}
    \FunctionTok{ggplot}\NormalTok{(}\FunctionTok{aes}\NormalTok{(}\AttributeTok{x =}\NormalTok{ gestation, }\AttributeTok{y =}\NormalTok{ bwt, }\AttributeTok{color =}\NormalTok{ smoke))}\SpecialCharTok{+}  \CommentTok{\#Specify variables}
    \FunctionTok{geom\_point}\NormalTok{(}\FunctionTok{aes}\NormalTok{(}\AttributeTok{shape =}\NormalTok{ smoke), }\AttributeTok{size =} \DecValTok{2}\NormalTok{) }\SpecialCharTok{+}  \CommentTok{\#Add scatterplot of points}
    \FunctionTok{labs}\NormalTok{(}\AttributeTok{x =} \StringTok{"number of days of gestation"}\NormalTok{,  }\CommentTok{\#Label x{-}axis}
         \AttributeTok{y =} \StringTok{"birthweight (oz)"}\NormalTok{,  }\CommentTok{\#Label y{-}axis}
         \AttributeTok{title =} \StringTok{"Scatterplot of Gestation vs. Birthweight by }
\StringTok{         Smoking Status for Births between 1960 and 1967 }
\StringTok{         in San Francisco"}\NormalTok{) }\SpecialCharTok{+} 
    \CommentTok{\#Be sure to title your plots}
    \FunctionTok{geom\_smooth}\NormalTok{(}\AttributeTok{method =} \StringTok{"lm"}\NormalTok{, }\AttributeTok{se =} \ConstantTok{FALSE}\NormalTok{) }\SpecialCharTok{+} \CommentTok{\#Add regression line}
    \FunctionTok{scale\_color\_grey}\NormalTok{()}
\end{Highlighting}
\end{Shaded}

\begin{center}\includegraphics[width=0.8\linewidth]{13-VN13-regression_files/figure-latex/unnamed-chunk-8-1} \end{center}

Does the relationship between length of gestation and birthweight appear to depend upon maternal smoking status?

\vspace{1in}

Is the variable smoking status a potential confounding variable?

\vspace{1in}

Adding a categorical predictor:

\setstretch{1.5}

\begin{itemize}
\item
  Look at the regression line for each level of the \_\_\_\_\_\_\_\_\_\_\_\_\_\_
\item
  If the slopes are \_\_\_\_\_\_\_\_\_\_\_\_\_\_\_\_, the two predictor variables do not \_\_\_\_\_\_\_\_\_\_\_\_\_\_\_ to help explain the response
\item
  If the slopes \_\_\_\_\_\_\_\_\_\_\_\_\_\_\_\_\_, there is an interaction between the categorical predictor and the relationship between the two quantitative variables.
\end{itemize}

\setstretch{1}

\subsection{Concept Check}\label{concept-check}

Be prepared for group discussion in the next class. One member from the table should write the answers to the following on the whiteboard.

\begin{enumerate}
\def\labelenumi{\arabic{enumi}.}
\tightlist
\item
  What are the three summary measures for two quantitative variables?
\end{enumerate}

\vspace{0.5in}

\begin{enumerate}
\def\labelenumi{\arabic{enumi}.}
\setcounter{enumi}{1}
\tightlist
\item
  What are the four characteristics used to describe a scatterplot?
\end{enumerate}

\vspace{0.5in}

\begin{enumerate}
\def\labelenumi{\arabic{enumi}.}
\setcounter{enumi}{2}
\tightlist
\item
  When we add a categorical predictor variable to a scatterplot of two quantitative variables, what summary measure will we compare across the categories to assess the change in the relationship between the two quantitative variables.
\end{enumerate}

\vspace{0.2in}
\newpage

\subsection{Video Notes: Inference for Two Quantitative Variables}\label{video-notes-inference-for-two-quantitative-variables}

\setstretch{1}

Example: Oceanic temperature is important for sea life. The California Cooperative Oceanic Fisheries Investigations has measured several variables on the Pacific Ocean for more than 70 years hoping to better understand weather patterns and impacts on ocean life. ({``Ocean Temperature and Salinity Study,''} n.d.) For this example, we will look at the most recent 100 measurements of salt water salinity (measured in PSUs or practical salinity units) and the temperature of the ocean measured in degrees Celsius. Is there evidence that water temperature in the Pacific Ocean tends to decrease with higher levels of salinity?

\subsection*{Hypothesis Testing - Video 21.1}\label{hypothesis-testing---video-21.1}
\addcontentsline{toc}{subsection}{Hypothesis Testing - Video 21.1}

Null hypothesis assumes ``no effect'', ``no difference'', ``nothing interesting happening'', etc.

\rgi Always of form: ``parameter'' = null value

\(H_0:\)

\vspace{0.5in}

\(H_A:\)

\vspace{0.5in}

\begin{itemize}
\tightlist
\item
  Research question determines the alternative hypothesis.
\end{itemize}

Write the null and alternative for the ocean study:

In notation:

\(H_0:\)

\vspace{0.2in}

\(H_A:\)

\vspace{0.2in}

\begin{Shaded}
\begin{Highlighting}[]
\NormalTok{water }\SpecialCharTok{\%\textgreater{}\%} \CommentTok{\# Pipe data set into...}
\FunctionTok{ggplot}\NormalTok{(}\FunctionTok{aes}\NormalTok{(}\AttributeTok{x =}\NormalTok{ Salnty, }\AttributeTok{y =}\NormalTok{ T\_degC))}\SpecialCharTok{+}  \CommentTok{\# Specify variables}
  \FunctionTok{geom\_point}\NormalTok{(}\AttributeTok{alpha=}\FloatTok{0.5}\NormalTok{) }\SpecialCharTok{+}  \CommentTok{\# Add scatterplot of points}
  \FunctionTok{labs}\NormalTok{(}\AttributeTok{x =} \StringTok{"salinity (PSUs)"}\NormalTok{,  }\CommentTok{\# Label x{-}axis}
       \AttributeTok{y =} \StringTok{"temperature (C)"}\NormalTok{,  }\CommentTok{\# Label y{-}axis}
       \AttributeTok{title =} \StringTok{"Scatterplot of Pacific Ocean Salinity vs Temperature"}\NormalTok{) }\SpecialCharTok{+}
               \CommentTok{\# Be sure to title your plots}
  \FunctionTok{geom\_smooth}\NormalTok{(}\AttributeTok{method =} \StringTok{"lm"}\NormalTok{, }\AttributeTok{se =} \ConstantTok{FALSE}\NormalTok{)  }\CommentTok{\# Add regression line}
\end{Highlighting}
\end{Shaded}

\begin{center}\includegraphics[width=0.7\linewidth]{13-VN13-regression_files/figure-latex/unnamed-chunk-10-1} \end{center}

Describe the four characteristics of the scatterplot:

\vspace{1in}

Linear model output:

\begin{Shaded}
\begin{Highlighting}[]
\NormalTok{lm.water }\OtherTok{\textless{}{-}} \FunctionTok{lm}\NormalTok{(T\_degC}\SpecialCharTok{\textasciitilde{}}\NormalTok{Salnty, }\AttributeTok{data=}\NormalTok{water) }\CommentTok{\# lm(response\textasciitilde{}explanatory)}
\FunctionTok{round}\NormalTok{(}\FunctionTok{summary}\NormalTok{(lm.water)}\SpecialCharTok{$}\NormalTok{coefficients, }\DecValTok{3}\NormalTok{)}
\end{Highlighting}
\end{Shaded}

\begin{verbatim}
#>             Estimate Std. Error t value Pr(>|t|)
#> (Intercept)  197.156     21.478    9.18        0
#> Salnty        -5.514      0.636   -8.67        0
\end{verbatim}

Correlation:

\begin{Shaded}
\begin{Highlighting}[]
\FunctionTok{cor}\NormalTok{(T\_degC}\SpecialCharTok{\textasciitilde{}}\NormalTok{Salnty, }\AttributeTok{data=}\NormalTok{water)}
\end{Highlighting}
\end{Shaded}

\begin{verbatim}
#> [1] -0.6588365
\end{verbatim}

Write the least squares equation of the line in context of the problem:

\vspace{0.5in}

Interpret the value of slope in the context of the problem:

\vspace{0.5in}

Report and describe the correlation value:

\vspace{0.5in}

Calculate and interpret the coefficient of determination:

\vspace{0.8in}

\subsubsection*{Simulation-based method}\label{simulation-based-method}
\addcontentsline{toc}{subsubsection}{Simulation-based method}

Conditions:

\begin{itemize}
\item
  Independence: the response for one observational unit will not influence another observational unit
\item
  Linear relationship:
\end{itemize}

\vspace{0.3in}

\begin{itemize}
\item
  Simulate many samples assuming \(H_0: \beta_1 = 0\) or \(H_0: \rho =0\)

  \begin{itemize}
  \item
    Write the response variable values on cards
  \item
    Hold the explanatory variable values constant
  \item
    Shuffle a new response variable to an explanatory variable
  \item
    Plot the shuffled data points to find the least squares line of regression
  \item
    Calculate and plot the simulated slope or correlation from each simulation
  \item
    Repeat 1000 times (simulations) to create the null distribution
  \item
    Find the proportion of simulations at least as extreme as \(b_1\) or \(r\)
  \end{itemize}
\end{itemize}

To test slope:

\begin{Shaded}
\begin{Highlighting}[]
\FunctionTok{set.seed}\NormalTok{(}\DecValTok{216}\NormalTok{)}
\FunctionTok{regression\_test}\NormalTok{(T\_degC }\SpecialCharTok{\textasciitilde{}}\NormalTok{ Salnty, }\CommentTok{\# response \textasciitilde{} explanatory}
               \AttributeTok{data =}\NormalTok{ water, }\CommentTok{\# Name of data set}
               \AttributeTok{direction =} \StringTok{"less"}\NormalTok{, }\CommentTok{\# Sign in alternative ("greater", "less", "two{-}sided")}
               \AttributeTok{summary\_measure =} \StringTok{"slope"}\NormalTok{, }\CommentTok{\# "slope" or "correlation"}
               \AttributeTok{as\_extreme\_as =} \SpecialCharTok{{-}}\FloatTok{5.514}\NormalTok{, }\CommentTok{\# Observed slope or correlation}
               \AttributeTok{number\_repetitions =} \DecValTok{10000}\NormalTok{) }\CommentTok{\# Number of simulated samples for null distribution}
\end{Highlighting}
\end{Shaded}

\begin{center}\includegraphics[width=0.7\linewidth]{13-VN13-regression_files/figure-latex/unnamed-chunk-13-1} \end{center}

\newpage

To test correlation:

\begin{Shaded}
\begin{Highlighting}[]
\FunctionTok{set.seed}\NormalTok{(}\DecValTok{216}\NormalTok{)}
\FunctionTok{regression\_test}\NormalTok{(T\_degC}\SpecialCharTok{\textasciitilde{}}\NormalTok{Salnty, }\CommentTok{\# response \textasciitilde{} explanatory}
               \AttributeTok{data =}\NormalTok{ water, }\CommentTok{\# Name of data set}
               \AttributeTok{direction =} \StringTok{"less"}\NormalTok{, }\CommentTok{\# Sign in alternative ("greater", "less", "two{-}sided")}
               \AttributeTok{summary\_measure =} \StringTok{"correlation"}\NormalTok{, }\CommentTok{\# "slope" or "correlation"}
               \AttributeTok{as\_extreme\_as =} \SpecialCharTok{{-}}\FloatTok{0.659}\NormalTok{, }\CommentTok{\# Observed slope or correlation}
               \AttributeTok{number\_repetitions =} \DecValTok{10000}\NormalTok{) }\CommentTok{\# Number of simulated samples for null distribution}
\end{Highlighting}
\end{Shaded}

\begin{center}\includegraphics[width=0.7\linewidth]{13-VN13-regression_files/figure-latex/unnamed-chunk-14-1} \end{center}

Explain why the null distribution is centered at the value of zero:

\vspace{0.5in}

Interpretation of the p-value:

\begin{itemize}
\item
  Statement about probability or proportion of samples
\item
  Statistic (summary measure and value)
\item
  Direction of the alternative
\item
  Null hypothesis (in context)
\end{itemize}

\vspace{0.8in}

Conclusion:

\begin{itemize}
\item
  Amount of evidence
\item
  Parameter of interest
\item
  Direction of the alternative hypothesis
\end{itemize}

\vspace{0.6in}

\subsection*{Confidence interval - Video 21.3}\label{confidence-interval---video-21.3}
\addcontentsline{toc}{subsection}{Confidence interval - Video 21.3}

To estimate the true slope (or true correlation) we will create a confidence interval.

\subsubsection*{Simulation-based method}\label{simulation-based-method-1}
\addcontentsline{toc}{subsubsection}{Simulation-based method}

\begin{itemize}
\item
  Write the explanatory and response value pairs on cards
\item
  Sample pairs with replacement \(n\) times
\item
  Plot the resampled data points to find the least squares line of regression
\item
  Calculate and plot the simulated slope (or correlation) from each simulation
\item
  Repeat 1000 times (simulations) to create the bootstrap distribution
\item
  Find the cut-offs for the middle X\% (confidence level) in a bootstrap distribution.
\end{itemize}

Returning to the ocean example, we will estimate the true slope between salinity and temperature of the Pacific Ocean.

\begin{Shaded}
\begin{Highlighting}[]
\FunctionTok{set.seed}\NormalTok{(}\DecValTok{216}\NormalTok{)}
\FunctionTok{regression\_bootstrap\_CI}\NormalTok{(T\_degC}\SpecialCharTok{\textasciitilde{}}\NormalTok{Salnty, }\CommentTok{\# response \textasciitilde{} explanatory}
   \AttributeTok{data =}\NormalTok{ water, }\CommentTok{\# Name of data set}
   \AttributeTok{confidence\_level =} \FloatTok{0.95}\NormalTok{, }\CommentTok{\# Confidence level as decimal}
   \AttributeTok{summary\_measure =} \StringTok{"slope"}\NormalTok{, }\CommentTok{\# Slope or correlation}
   \AttributeTok{number\_repetitions =} \DecValTok{10000}\NormalTok{) }\CommentTok{\# Number of simulated samples for bootstrap distribution}
\end{Highlighting}
\end{Shaded}

\begin{center}\includegraphics[width=0.7\linewidth]{13-VN13-regression_files/figure-latex/unnamed-chunk-15-1} \end{center}

Confidence interval interpretation:

\begin{itemize}
\item
  How confident you are (e.g., 90\%, 95\%, 98\%, 99\%)
\item
  Parameter of interest
\item
  Calculated interval
\item
  Order of subtraction when comparing two groups
\end{itemize}

\vspace{0.8in}

Now we will estimate the true correlation between salinity and temperature of the Pacific Ocean.

\begin{Shaded}
\begin{Highlighting}[]
\FunctionTok{set.seed}\NormalTok{(}\DecValTok{216}\NormalTok{)}
\FunctionTok{regression\_bootstrap\_CI}\NormalTok{(T\_degC}\SpecialCharTok{\textasciitilde{}}\NormalTok{Salnty, }\CommentTok{\# response \textasciitilde{} explanatory}
   \AttributeTok{data =}\NormalTok{ water, }\CommentTok{\# Name of data set}
   \AttributeTok{confidence\_level =} \FloatTok{0.95}\NormalTok{, }\CommentTok{\# Confidence level as decimal}
   \AttributeTok{summary\_measure =} \StringTok{"correlation"}\NormalTok{, }\CommentTok{\# Slope or correlation}
   \AttributeTok{number\_repetitions =} \DecValTok{10000}\NormalTok{) }\CommentTok{\# Number of simulated samples for bootstrap distribution}
\end{Highlighting}
\end{Shaded}

\begin{center}\includegraphics[width=0.7\linewidth]{13-VN13-regression_files/figure-latex/unnamed-chunk-16-1} \end{center}

Confidence interval interpretation:

\begin{itemize}
\item
  How confident you are (e.g., 90\%, 95\%, 98\%, 99\%)
\item
  Parameter of interest
\item
  Calculated interval
\item
  Order of subtraction when comparing two groups
\end{itemize}

\vspace{0.8in}

\subsubsection*{Theory-based method - Video 21.4to21.5TheoryTests}\label{theory-based-method---video-21.4to21.5theorytests}
\addcontentsline{toc}{subsubsection}{Theory-based method - Video 21.4to21.5TheoryTests}

Conditions:

\setstretch{1.5}

\begin{itemize}
\item
  Linearity (for both simulation-based and theory-based methods): the data should follow a linear trend.

  \begin{itemize}
  \tightlist
  \item
    Check this assumption by examining the \_\_\_\_\_\_\_\_\_\_\_\_\_\_\_\_\_\_\_\_\_\_\_\_\_\_\_\_ of the two variables, and \_\_\_\_\_\_\_\_\_\_\_\_\_\_\_\_\_\_\_\_\_\_\_\_\_\_\_\_\_\_\_\_\_\_\_\_\_\_\_\_\_\_\_\_. The pattern in the residual plot should display a horizontal line.
  \end{itemize}
\end{itemize}

\newpage

\begin{itemize}
\item
  Independence (for both simulation-based and theory-based methods)

  \begin{itemize}
  \tightlist
  \item
    One\_\_\_\_\_\_\_\_\_\_\_\_\_\_\_\_\_\_\_\_\_\_\_\_\_\_\_\_\_\_for an observational unit has no impact on \_\_\_\_\_\_\_\_\_\_\_\_\_\_\_\_\_\_\_\_\_\_\_\_\_\_\_\_\_\_\_\_.
  \end{itemize}
\item
  Constant variability (for theory-based methods only): the variability of points around the least squares line remains roughly constant

  \begin{itemize}
  \tightlist
  \item
    Check this assumption by examining the \_\_\_\_\_\_\_\_\_\_\_\_\_\_\_\_\_\_\_\_\_\_\_\_\_\_\_\_\_\_\_\_. The variability in the residuals around zero should be approximately the same for all fitted values.
  \end{itemize}
\item
  Nearly normal residuals (for theory-based methods only): residuals must be nearly normal

  \begin{itemize}
  \tightlist
  \item
    Check this assumption by examining a \_\_\_\_\_\_\_\_\_\_\_\_\_\_\_\_\_\_\_\_\_\_\_\_\_\_\_\_\_\_\_\_\_, which should appear approximately normal
  \end{itemize}
\end{itemize}

\setstretch{1}

Example:

It is a generally accepted fact that the more carats a diamond has, the more expensive that diamond will be. The question is, how much more expensive? Data on thousands of diamonds were collected for this data set. We will only look at one type of cut (``Ideal'') and diamonds less than 1 carat. Does the association between carat size and price have a linear relationship for these types of diamonds? What can we state about the association between carat size and price?

Scatterplot:

\begin{Shaded}
\begin{Highlighting}[]
\NormalTok{Diamonds }\SpecialCharTok{\%\textgreater{}\%} \CommentTok{\# Pipe data set into...}
    \FunctionTok{ggplot}\NormalTok{(}\FunctionTok{aes}\NormalTok{(}\AttributeTok{x =}\NormalTok{ carat, }\AttributeTok{y =}\NormalTok{ price))}\SpecialCharTok{+}  \CommentTok{\# Specify variables}
    \FunctionTok{geom\_point}\NormalTok{(}\AttributeTok{alpha=}\FloatTok{0.5}\NormalTok{) }\SpecialCharTok{+}  \CommentTok{\# Add scatterplot of points}
    \FunctionTok{labs}\NormalTok{(}\AttributeTok{x =} \StringTok{"carat"}\NormalTok{,  }\CommentTok{\# Label x{-}axis}
       \AttributeTok{y =} \StringTok{"price ($)"}\NormalTok{,  }\CommentTok{\# Label y{-}axis}
       \AttributeTok{title =} \StringTok{"Scatterplot of Diamonds Carats vs Price"}\NormalTok{) }\SpecialCharTok{+}
               \CommentTok{\# Be sure to title your plots}
    \FunctionTok{geom\_smooth}\NormalTok{(}\AttributeTok{method =} \StringTok{"lm"}\NormalTok{, }\AttributeTok{se =} \ConstantTok{FALSE}\NormalTok{)  }\CommentTok{\# Add regression line}
\end{Highlighting}
\end{Shaded}

\begin{center}\includegraphics[width=0.7\linewidth]{13-VN13-regression_files/figure-latex/unnamed-chunk-18-1} \end{center}

\newpage

Diagnostic plots:

\begin{center}\includegraphics[width=0.7\linewidth]{13-VN13-regression_files/figure-latex/unnamed-chunk-19-1} \end{center}

Check the conditions for the ocean data:

Scatterplot:

\begin{Shaded}
\begin{Highlighting}[]
\NormalTok{water }\SpecialCharTok{\%\textgreater{}\%} \CommentTok{\# Pipe data set into...}
\FunctionTok{ggplot}\NormalTok{(}\FunctionTok{aes}\NormalTok{(}\AttributeTok{x =}\NormalTok{ Salnty, }\AttributeTok{y =}\NormalTok{ T\_degC))}\SpecialCharTok{+}  \CommentTok{\# Specify variables}
  \FunctionTok{geom\_point}\NormalTok{(}\AttributeTok{alpha=}\FloatTok{0.5}\NormalTok{) }\SpecialCharTok{+}  \CommentTok{\# Add scatterplot of points}
  \FunctionTok{labs}\NormalTok{(}\AttributeTok{x =} \StringTok{"salinity (PSUs)"}\NormalTok{,  }\CommentTok{\# Label x{-}axis}
       \AttributeTok{y =} \StringTok{"temperature (C)"}\NormalTok{,  }\CommentTok{\# Label y{-}axis}
       \AttributeTok{title =} \StringTok{"Scatterplot of Pacific Ocean Salinity vs Temperature"}\NormalTok{) }\SpecialCharTok{+} 
               \CommentTok{\# Be sure to title your plots}
  \FunctionTok{geom\_smooth}\NormalTok{(}\AttributeTok{method =} \StringTok{"lm"}\NormalTok{, }\AttributeTok{se =} \ConstantTok{FALSE}\NormalTok{)  }\CommentTok{\# Add regression line}
\end{Highlighting}
\end{Shaded}

\begin{center}\includegraphics[width=0.7\linewidth]{13-VN13-regression_files/figure-latex/unnamed-chunk-20-1} \end{center}

\newpage

Diagnostic plots:

\begin{center}\includegraphics[width=0.7\linewidth]{13-VN13-regression_files/figure-latex/unnamed-chunk-21-1} \end{center}

Like with paired data the \(t\)-distribution can be used to model slope and correlation.

\setstretch{1.5}

\begin{itemize}
\tightlist
\item
  For two quantitative variables we use the \_\_\_\_\_\_-distribution
  with \_\_\_\_\_\_\_\_\_\_\_\_\_\_\_\_\_\_\_\_\_ degrees of freedom to approximate the sampling distribution.
\end{itemize}

\setstretch{1}

Theory-based test:

\begin{itemize}
\item
  Calculate the standardized statistic
\item
  Find the area under the \(t\)-distribution with \(n - 2\) df at least as extreme as the standardized statistic
\end{itemize}

Equation for the standardized slope:

\vspace{0.8in}

Calculate the standardized slope for the ocean data

\begin{Shaded}
\begin{Highlighting}[]
\NormalTok{lm.water }\OtherTok{\textless{}{-}} \FunctionTok{lm}\NormalTok{(T\_degC}\SpecialCharTok{\textasciitilde{}}\NormalTok{Salnty, }\AttributeTok{data=}\NormalTok{water) }\CommentTok{\# lm(response\textasciitilde{}explanatory)}
\FunctionTok{round}\NormalTok{(}\FunctionTok{summary}\NormalTok{(lm.water)}\SpecialCharTok{$}\NormalTok{coefficients,}\DecValTok{3}\NormalTok{)}
\end{Highlighting}
\end{Shaded}

\begin{verbatim}
#>             Estimate Std. Error t value Pr(>|t|)
#> (Intercept)  197.156     21.478    9.18        0
#> Salnty        -5.514      0.636   -8.67        0
\end{verbatim}

\vspace{1in}

\begin{center}\includegraphics[width=0.7\linewidth]{13-VN13-regression_files/figure-latex/pvalueoce-1} \end{center}

Interpret the standardized statistic:

\vspace{0.8in}

To find the theory-based p-value:

\begin{Shaded}
\begin{Highlighting}[]
\NormalTok{lm.water }\OtherTok{\textless{}{-}} \FunctionTok{lm}\NormalTok{(T\_degC}\SpecialCharTok{\textasciitilde{}}\NormalTok{Salnty, }\AttributeTok{data=}\NormalTok{water) }\CommentTok{\# lm(response\textasciitilde{}explanatory)}
\FunctionTok{round}\NormalTok{(}\FunctionTok{summary}\NormalTok{(lm.water)}\SpecialCharTok{$}\NormalTok{coefficients,}\DecValTok{3}\NormalTok{)}
\end{Highlighting}
\end{Shaded}

\begin{verbatim}
#>             Estimate Std. Error t value Pr(>|t|)
#> (Intercept)  197.156     21.478    9.18        0
#> Salnty        -5.514      0.636   -8.67        0
\end{verbatim}

or

\begin{Shaded}
\begin{Highlighting}[]
\FunctionTok{pt}\NormalTok{(}\SpecialCharTok{{-}}\FloatTok{8.670}\NormalTok{, }\AttributeTok{df =} \DecValTok{98}\NormalTok{, }\AttributeTok{lower.tail=}\ConstantTok{TRUE}\NormalTok{)}
\CommentTok{\#\textgreater{} [1] 4.623445e{-}14}
\end{Highlighting}
\end{Shaded}

\newpage

\subsubsection*{Theory-based method}\label{theory-based-method}
\addcontentsline{toc}{subsubsection}{Theory-based method}

\begin{itemize}
\tightlist
\item
  Calculate the interval centered at the sample statistic
\end{itemize}

\rgi \(\text{statistic} \pm \text{margin of error}\)

\vspace{0.6in}

\begin{Shaded}
\begin{Highlighting}[]
\NormalTok{lm.water }\OtherTok{\textless{}{-}} \FunctionTok{lm}\NormalTok{(T\_degC}\SpecialCharTok{\textasciitilde{}}\NormalTok{Salnty, }\AttributeTok{data=}\NormalTok{water) }\CommentTok{\# lm(response\textasciitilde{}explanatory)}
\FunctionTok{round}\NormalTok{(}\FunctionTok{summary}\NormalTok{(lm.water)}\SpecialCharTok{$}\NormalTok{coefficients, }\DecValTok{3}\NormalTok{)}
\end{Highlighting}
\end{Shaded}

\begin{verbatim}
#>             Estimate Std. Error t value Pr(>|t|)
#> (Intercept)  197.156     21.478    9.18        0
#> Salnty        -5.514      0.636   -8.67        0
\end{verbatim}

Using the ocean data, calculate a 95\% confidence interval for the true slope.

\begin{itemize}
\tightlist
\item
  Need the \(t^*\) multiplier for a 95\% confidence interval from a t-distribution with \_\_\_\_\_\_\_\_\_ df.
\end{itemize}

\begin{Shaded}
\begin{Highlighting}[]
\FunctionTok{qt}\NormalTok{(}\FloatTok{0.975}\NormalTok{, }\AttributeTok{df=}\DecValTok{98}\NormalTok{, }\AttributeTok{lower.tail =} \ConstantTok{TRUE}\NormalTok{)}
\end{Highlighting}
\end{Shaded}

\begin{verbatim}
#> [1] 1.984467
\end{verbatim}

\vspace{1in}

\subsection{Concept Check}\label{concept-check-1}

Be prepared for group discussion in the next class. One member from the table should write the answers to the following on the whiteboard.

\begin{enumerate}
\def\labelenumi{\arabic{enumi}.}
\tightlist
\item
  Explain why theory-based methods should not be used to analyze the salinity study?
\end{enumerate}

\vspace{0.6in}

\begin{enumerate}
\def\labelenumi{\arabic{enumi}.}
\setcounter{enumi}{1}
\tightlist
\item
  What is the proper notation for the population slope? Population correlation?
\end{enumerate}

\vspace{0.4in}

\newpage

\section{Activity 26: Moneyball --- Linear Regression}\label{activity-26-moneyball-linear-regression}

\setstretch{1}

\subsection{Learning outcomes}\label{learning-outcomes}

\begin{itemize}
\item
  Identify and create appropriate summary statistics and plots
  given a data set with two quantitative variables.
\item
  Use scatterplots to assess the relationship between two quantitative variables.
\item
  Find the estimated line of regression using summary statistics and \texttt{R} linear model (\texttt{lm()}) output.
\item
  Interpret the slope coefficient in context of the problem.
\end{itemize}

\subsection{Terminology review}\label{terminology-review}

In today's activity, we will review summary measures and plots for two quantitative variables. Some terms covered in this activity are:

\begin{itemize}
\item
  Scatterplot
\item
  Least-squares line of regression
\item
  Slope and \(y\)-intercept
\item
  Residuals
\end{itemize}

To review these concepts, see Chapter 6 \& 7 in the textbook.

\subsection{Moneyball}\label{moneyball}

The goal of a Major League baseball team is to make the playoffs. In 2002, the manager of the Oakland A's, Billy Bean, with the help of Paul DePodesta began to use statistics to determine which players to choose for their season. Based on past data, DePodesta determined that to make it to the playoffs, the A's would need to win at least 95 games in the regular season. In order to win more games, they would need to score more runs than they allowed. The Oakland A's won 20 consecutive games and a total of 103 games for the season. The success of this use of sports analytics was portrayed by the 2011 movie, Moneyball. In this study, we will see if there is evidence of a positive linear relationship between the difference in the number of runs scored minus the number of runs allowed (\texttt{RD}) and the number of wins for Major League baseball teams in the years before 2002. Some of the variables collected in the data set baseball consist of the following:

\begin{longtable}[]{@{}ll@{}}
\toprule\noalign{}
\textbf{Variable} & \textbf{Description} \\
\midrule\noalign{}
\endhead
\bottomrule\noalign{}
\endlastfoot
\texttt{RA} & Runs allowed \\
\texttt{RS} & Runs scored \\
\texttt{OBP} & On-base percentage \\
\texttt{SLG} & Slugging percentage \\
\texttt{BA} & Batting average \\
\texttt{OOBP} & Opponent's on-base percentage \\
\texttt{OSLG} & Opponent's slugging percentage \\
\texttt{W} & Number of wins in the season \\
\texttt{RD} & Difference of runs scored minus runs allowed \\
\end{longtable}

\begin{Shaded}
\begin{Highlighting}[]
\NormalTok{moneyball }\OtherTok{\textless{}{-}} \FunctionTok{read.csv}\NormalTok{(}\StringTok{"data/baseball.csv"}\NormalTok{) }\CommentTok{\# Reads in data set }
\NormalTok{moneyball}\SpecialCharTok{$}\NormalTok{RD }\OtherTok{\textless{}{-}}\NormalTok{ moneyball}\SpecialCharTok{$}\NormalTok{RS }\SpecialCharTok{{-}}\NormalTok{ moneyball}\SpecialCharTok{$}\NormalTok{RA}
\NormalTok{moneyball }\OtherTok{\textless{}{-}} 
\NormalTok{    moneyball }\SpecialCharTok{\%\textgreater{}\%} \CommentTok{\# Pipe data set into}
    \FunctionTok{subset}\NormalTok{(Year }\SpecialCharTok{\textless{}} \DecValTok{2002}\NormalTok{) }\CommentTok{\# Select only years before 2002}
\end{Highlighting}
\end{Shaded}

\subsubsection*{Vocabulary review}\label{vocabulary-review}
\addcontentsline{toc}{subsubsection}{Vocabulary review}

\begin{itemize}
\item
  Use the provided R script file to create a scatterplot to examine the relationship between the difference in number of runs scored minus number of runs allowed and the number of wins by filling in the variable names (RD and W) for explanatory and response in line 14. Note, we are using the difference in runs scores minus runs allowed to predict the number of season wins.
\item
  Highlight and run lines 1--20.
\end{itemize}

\begin{Shaded}
\begin{Highlighting}[]
\NormalTok{moneyball }\SpecialCharTok{\%\textgreater{}\%} \CommentTok{\# Data set pipes into...}
    \FunctionTok{ggplot}\NormalTok{(}\FunctionTok{aes}\NormalTok{(}\AttributeTok{x =}\NormalTok{ explanatory, }\AttributeTok{y =}\NormalTok{ response))}\SpecialCharTok{+} \CommentTok{\# Specify variables}
    \FunctionTok{geom\_point}\NormalTok{() }\SpecialCharTok{+} \CommentTok{\# Add scatterplot of points}
    \FunctionTok{labs}\NormalTok{(}\AttributeTok{x =} \StringTok{"Difference in number of runs"}\NormalTok{, }\CommentTok{\# Label x{-}axis}
         \AttributeTok{y =} \StringTok{"Number of Season wins"}\NormalTok{, }\CommentTok{\# Label y{-}axis}
         \AttributeTok{title =} \StringTok{"Scatterplot of Run Difference vs. Number of Season Wins for MLB Teams"}\NormalTok{) }\SpecialCharTok{+}
\CommentTok{\# Be sure to tile your plots}
\FunctionTok{geom\_smooth}\NormalTok{(}\AttributeTok{method =} \StringTok{"lm"}\NormalTok{, }\AttributeTok{se =} \ConstantTok{FALSE}\NormalTok{) }\CommentTok{\# Add regression line}
\end{Highlighting}
\end{Shaded}

\begin{enumerate}
\def\labelenumi{\arabic{enumi}.}
\tightlist
\item
  Assess the four features of the scatterplot that describe this relationship.
\end{enumerate}

\begin{itemize}
\tightlist
\item
  Form (linear, non-linear)
\end{itemize}

\vspace{.075in}

\begin{itemize}
\tightlist
\item
  Direction (positive, negative)
\end{itemize}

\vspace{.075in}

\begin{itemize}
\tightlist
\item
  Strength
\end{itemize}

\vspace{.075in}

\begin{itemize}
\tightlist
\item
  Unusual observations or outliers
\end{itemize}

\vspace{.075in}

\begin{enumerate}
\def\labelenumi{\arabic{enumi}.}
\setcounter{enumi}{1}
\tightlist
\item
  Based on the plot, does there appear to be an association between run difference and number of season wins? Explain your answer.
\end{enumerate}

\vspace{1in}

\subsubsection*{Slope}\label{slope-1}
\addcontentsline{toc}{subsubsection}{Slope}

The linear model function in R (\texttt{lm()}) gives us the summary for the least squares regression line. The estimate for \texttt{(Intercept)} is the \(y\)-intercept for the line of least squares, and the estimate for \texttt{budget\_mil} (the \(x\)-variable name) is the value of \(b_1\), the slope.

\begin{itemize}
\tightlist
\item
  Run lines 24--25 in the R script file to reproduce the linear model output found in the coursepack.
\end{itemize}

\begin{Shaded}
\begin{Highlighting}[]
\CommentTok{\# Fit linear model: y \textasciitilde{} x}
\NormalTok{moneyballLM }\OtherTok{\textless{}{-}} \FunctionTok{lm}\NormalTok{(W}\SpecialCharTok{\textasciitilde{}}\NormalTok{RD, }\AttributeTok{data=}\NormalTok{moneyball)}
\FunctionTok{round}\NormalTok{(}\FunctionTok{summary}\NormalTok{(moneyballLM)}\SpecialCharTok{$}\NormalTok{coefficients, }\DecValTok{3}\NormalTok{) }\CommentTok{\# Display coefficient summary}
\end{Highlighting}
\end{Shaded}

\begin{verbatim}
#>             Estimate Std. Error t value Pr(>|t|)
#> (Intercept)   80.881      0.131 616.675        0
#> RD             0.106      0.001  81.554        0
\end{verbatim}

\begin{enumerate}
\def\labelenumi{\arabic{enumi}.}
\setcounter{enumi}{2}
\item
  Write out the least squares regression line using the summary statistics provided above in context of the problem.
  \vspace{0.8in}
\item
  Interpret the value of slope in context of the problem.
\end{enumerate}

\vspace{.8in}

\begin{enumerate}
\def\labelenumi{\arabic{enumi}.}
\setcounter{enumi}{4}
\tightlist
\item
  Using the least squares line from question 3, predict the number of season wins for a MLB team that has a run difference of -66 runs.
\end{enumerate}

\vspace{.6in}

\begin{enumerate}
\def\labelenumi{\arabic{enumi}.}
\setcounter{enumi}{5}
\tightlist
\item
  Predict the number of season wins for a MLB team that has a run difference of 400 runs.
\end{enumerate}

\vspace{0.8in}

\begin{enumerate}
\def\labelenumi{\arabic{enumi}.}
\setcounter{enumi}{6}
\tightlist
\item
  The prediction in question 6 is an example of what?
\end{enumerate}

\vspace{0.3in}

\subsubsection*{Residuals}\label{residuals}
\addcontentsline{toc}{subsubsection}{Residuals}

The model we are using assumes the relationship between the two variables follows a straight line. The residuals are the errors, or the variability in the response that hasn't been modeled by the regression line.

\begin{center}

$\implies$ Residual = actual y value $-$ predicted y value

$e=y-\hat{y}$
\end{center}

\begin{enumerate}
\def\labelenumi{\arabic{enumi}.}
\setcounter{enumi}{7}
\tightlist
\item
  The MLB team \emph{Florida Marlins} had a run difference of -66 runs and 79 wins for the season. Find the residual for this MLB team.
\end{enumerate}

\vspace{.8in}

\begin{enumerate}
\def\labelenumi{\arabic{enumi}.}
\setcounter{enumi}{8}
\tightlist
\item
  Did the line of regression overestimate or underestimate the number of wins for the season for this team?
\end{enumerate}

\vspace{.2in}

\newpage

\subsubsection*{Correlation}\label{correlation-1}
\addcontentsline{toc}{subsubsection}{Correlation}

The following output shows a correlation matrix between several pairs of quantitative variables.

\begin{itemize}
\tightlist
\item
  Highlight and run lines 29--33 to produce the same table as below.
\end{itemize}

\begin{Shaded}
\begin{Highlighting}[]
\NormalTok{moneyball }\SpecialCharTok{\%\textgreater{}\%}  \CommentTok{\# Data set pipes into}
  \FunctionTok{select}\NormalTok{(}\FunctionTok{c}\NormalTok{(}\StringTok{"RD"}\NormalTok{, }\StringTok{"BA"}\NormalTok{, }
           \StringTok{"SLG"}\NormalTok{, }\StringTok{"W"}\NormalTok{)) }\SpecialCharTok{\%\textgreater{}\%}
  \FunctionTok{cor}\NormalTok{(}\AttributeTok{use=}\StringTok{"pairwise.complete.obs"}\NormalTok{) }\SpecialCharTok{\%\textgreater{}\%}
  \FunctionTok{round}\NormalTok{(}\DecValTok{3}\NormalTok{)}
\end{Highlighting}
\end{Shaded}

\begin{verbatim}
#>        RD    BA   SLG     W
#> RD  1.000 0.442 0.428 0.939
#> BA  0.442 1.000 0.814 0.416
#> SLG 0.428 0.814 1.000 0.406
#> W   0.939 0.416 0.406 1.000
\end{verbatim}

\begin{enumerate}
\def\labelenumi{\arabic{enumi}.}
\setcounter{enumi}{9}
\tightlist
\item
  Report the value of correlation between the run difference and the number of season wins.
\end{enumerate}

\vspace{0.3in}

\begin{enumerate}
\def\labelenumi{\arabic{enumi}.}
\setcounter{enumi}{10}
\tightlist
\item
  Calculate the coefficient of determination between the run difference and the number of season wins.
\end{enumerate}

\vspace{0.5in}

\begin{enumerate}
\def\labelenumi{\arabic{enumi}.}
\setcounter{enumi}{11}
\tightlist
\item
  Interpret the value of coefficient of determination in context of the study.
\end{enumerate}

\vspace{0.7in}

\subsection{Take-home messages}\label{take-home-messages}

\begin{enumerate}
\def\labelenumi{\arabic{enumi}.}
\item
  Two quantitative variables are graphically displayed in a scatterplot. The explanatory variable is on the \(x\)-axis and the response variable is on the \(y\)-axis. When describing the relationship between two quantitative variables we look at the form (linear or non-linear), direction (positive or negative), strength, and for the presence of outliers.
\item
  There are three summary statistics used to summarize the relationship between two quantitative variables: correlation (\(r\)), slope of the regression line (\(b_1\)), and the coefficient of determination (\(r^2\)).
\end{enumerate}

\subsection{Additional notes}\label{additional-notes}

Use this space to summarize your thoughts and take additional notes on today's activity and material covered.

\newpage

\section{Activity 27: IPEDS (continued)}\label{activity-27-ipeds-continued}

\setstretch{1}

\subsection{Learning outcomes}\label{learning-outcomes-1}

\begin{itemize}
\item
  Identify and create appropriate summary statistics and plots
  given a data set with two quantitative variables.
\item
  Find the estimated line of regression using summary statistics and \texttt{R} linear model (\texttt{lm()}) output.
\item
  Interpret the slope coefficient in context of the problem.
\item
  Calculate and interpret \(r^2\), the coefficient of determination, in context of the problem.
\item
  Find the correlation coefficient from \texttt{R} output or from \(r^2\) and the sign of the slope.
\end{itemize}

\subsection{Terminology review}\label{terminology-review-1}

In today's activity, we will review summary measures and plots for two quantitative variables. Some terms covered in this activity are:

\begin{itemize}
\item
  Least-squares line of regression
\item
  Slope and \(y\)-intercept
\item
  Residuals
\item
  Correlation (\(r\))
\item
  Coefficient of determination (\(r\)-squared)
\end{itemize}

To review these concepts, see Chapter 6 in the textbook.

\subsection{The Integrated Postsecondary Education Data System (IPEDS)}\label{the-integrated-postsecondary-education-data-system-ipeds}

We will continue to assess the IPEDS data set collected on a subset of institutions that met the following selection criteria (Education Statistics 2018):

\begin{itemize}
\item
  Degree granting
\item
  United States only
\item
  Title IV participating
\item
  Not for profit
\item
  2-year or 4-year or above
\item
  Has full-time first-time undergraduates
\end{itemize}

Some of the variables collected and their descriptions are below. Note that several variables have missing values for some institutions (denoted by ``NA'').

\begin{longtable}[]{@{}
  >{\raggedright\arraybackslash}p{(\columnwidth - 2\tabcolsep) * \real{0.2353}}
  >{\raggedright\arraybackslash}p{(\columnwidth - 2\tabcolsep) * \real{0.7647}}@{}}
\toprule\noalign{}
\begin{minipage}[b]{\linewidth}\raggedright
\textbf{Variable}
\end{minipage} & \begin{minipage}[b]{\linewidth}\raggedright
\textbf{Description}
\end{minipage} \\
\midrule\noalign{}
\endhead
\bottomrule\noalign{}
\endlastfoot
\texttt{UnitID} & Unique institution identifier \\
\texttt{Name} & Institution name \\
\texttt{State} & State abbreviation \\
\texttt{Sector} & whether public or private \\
\texttt{LandGrant} & Is this a land-grant institution (Yes/No) \\
\texttt{Size} & Institution size category based on total student enrolled for credit, Fall 2018: Under 1,000, 1,000 - 4,999, 5,000 - 9,999, 10,000 - 19,999, 20,000 and above \\
\texttt{Cost\_OutofState} & Cost of attendance for full-time out-of-state undergraduate students \\
\texttt{Cost\_InState} & Cost of attendance for full-time in-state undergraduate students \\
\texttt{Retention} & Retention rate is the percent of the undergraduate students that re-enroll in the next year \\
\texttt{Graduation\_Rate} & 6-year graduation rate for undergraduate students \\
\texttt{SATMath\_75} & 75th percentile Math SAT score \\
\texttt{ACT\_75} & 75th percentile ACT score \\
\end{longtable}

The code below reads in the needed data set, IPEDS\_2018.csv, and filters out the 2-year institutions.

\begin{itemize}
\tightlist
\item
  Highlight and run lines 1 -- 11 to load the data set and filter out the 2-year institutions.
\end{itemize}

\begin{Shaded}
\begin{Highlighting}[]
\NormalTok{IPEDS }\OtherTok{\textless{}{-}} \FunctionTok{read.csv}\NormalTok{(}\StringTok{"https://www.math.montana.edu/courses/s216/data/IPEDS\_2018.csv"}\NormalTok{) }
\NormalTok{IPEDS }\OtherTok{\textless{}{-}}\NormalTok{ IPEDS }\SpecialCharTok{\%\textgreater{}\%}
  \FunctionTok{filter}\NormalTok{(Sector }\SpecialCharTok{!=} \StringTok{"Public 2{-}year"}\NormalTok{) }\CommentTok{\#Filters the data set to remove Public 2{-}year}
\NormalTok{IPEDS }\OtherTok{\textless{}{-}}\NormalTok{ IPEDS }\SpecialCharTok{\%\textgreater{}\%}
  \FunctionTok{filter}\NormalTok{(Sector }\SpecialCharTok{!=} \StringTok{"Private 2{-}year"}\NormalTok{) }\CommentTok{\#Filters the data set to remove Private 2{-}year}
\NormalTok{IPEDS }\OtherTok{\textless{}{-}} \FunctionTok{na.omit}\NormalTok{(IPEDS)}
\end{Highlighting}
\end{Shaded}

To create a scatterplot of the 75th percentile Math SAT score by retention rate for 4-year US Higher Education Institutions\ldots{}

\begin{itemize}
\item
  Enter the variable \texttt{SATMath\_75} for explanatory and \texttt{Retention} for response in line 16.
\item
  Highlight and run lines 15--21.
\end{itemize}

\begin{Shaded}
\begin{Highlighting}[]
\NormalTok{IPEDS }\SpecialCharTok{\%\textgreater{}\%} \CommentTok{\# Data sest pipes into...}
    \FunctionTok{ggplot}\NormalTok{(}\FunctionTok{aes}\NormalTok{(}\AttributeTok{x =}\NormalTok{ SATMath\_75, }\AttributeTok{y =}\NormalTok{ Retention))}\SpecialCharTok{+}  \CommentTok{\# Specify variables}
    \FunctionTok{geom\_point}\NormalTok{(}\AttributeTok{alpha=}\FloatTok{0.5}\NormalTok{) }\SpecialCharTok{+}  \CommentTok{\# Add scatterplot of points}
    \FunctionTok{labs}\NormalTok{(}\AttributeTok{x =} \StringTok{"75th Percentile SAT Math Score"}\NormalTok{,  }\CommentTok{\# Label x{-}axis}
       \AttributeTok{y =} \StringTok{"Retention Rate (\%)"}\NormalTok{,  }\CommentTok{\# Label y{-}axis}
       \AttributeTok{title =} \StringTok{"Scatterplot of SAT Math Score vs. Retention Rate for }
\StringTok{       4{-}year US Higher Education Institutions"}\NormalTok{) }\SpecialCharTok{+} 
    \CommentTok{\# Be sure to title your plots with the type of plot, observational units, variable(s)}
    \FunctionTok{geom\_smooth}\NormalTok{(}\AttributeTok{method =} \StringTok{"lm"}\NormalTok{, }\AttributeTok{se =} \ConstantTok{FALSE}\NormalTok{) }\SpecialCharTok{+} \CommentTok{\# Add regression line}
    \FunctionTok{theme\_bw}\NormalTok{()}
\end{Highlighting}
\end{Shaded}

\begin{center}\includegraphics[width=0.7\linewidth]{13-A27-EDA-two-quantitative-corr_files/figure-latex/unnamed-chunk-2-1} \end{center}

\begin{enumerate}
\def\labelenumi{\arabic{enumi}.}
\tightlist
\item
  Describe the relationship between 75th percentile SAT Math score and retention rate.
\end{enumerate}

\vspace{1in}

\subsubsection*{Slope of the Least Squares Linear Regression Line}\label{slope-of-the-least-squares-linear-regression-line}
\addcontentsline{toc}{subsubsection}{Slope of the Least Squares Linear Regression Line}

There are three summary measures calculated from two quantitative variables: slope, correlation, and the coefficient of determination. We will first assess the slope of the least squares regression line between 75th percentile SAT Math score and retention rate.

\begin{itemize}
\item
  Enter \texttt{Retention} for response and \texttt{SATMath\_75} for explanatory in line 25
\item
  Highlight and run lines 25--26 to fit the linear model.
\end{itemize}

\begin{Shaded}
\begin{Highlighting}[]
\CommentTok{\# Fit linear model: y \textasciitilde{} x}
\NormalTok{IPEDSLM }\OtherTok{\textless{}{-}} \FunctionTok{lm}\NormalTok{(Retention}\SpecialCharTok{\textasciitilde{}}\NormalTok{SATMath\_75, }\AttributeTok{data=}\NormalTok{IPEDS)}
\FunctionTok{round}\NormalTok{(}\FunctionTok{summary}\NormalTok{(IPEDSLM)}\SpecialCharTok{$}\NormalTok{coefficients,}\DecValTok{3}\NormalTok{) }\CommentTok{\# Display coefficient summary}
\end{Highlighting}
\end{Shaded}

\begin{verbatim}
#>             Estimate Std. Error t value Pr(>|t|)
#> (Intercept)    0.059      1.898   0.031    0.975
#> SATMath_75     0.125      0.003  40.485    0.000
\end{verbatim}

\begin{enumerate}
\def\labelenumi{\arabic{enumi}.}
\setcounter{enumi}{1}
\tightlist
\item
  Write out the least squares regression line using the summary statistics from the R output in context of the problem.
\end{enumerate}

\vspace{0.5in}

\textbf{Slope Interpretation}: An increase of one point in SAT Math 75th percentile score is associated with an increase in retention rate, on average, of 0.125 percentage points for 4-year higher education institutions.

\begin{enumerate}
\def\labelenumi{\arabic{enumi}.}
\setcounter{enumi}{2}
\tightlist
\item
  Predict the retention rate for a 4-year US higher education institution with a 75th percentile SAT Math score of 440.
\end{enumerate}

\vspace{0.4in}

\begin{enumerate}
\def\labelenumi{\arabic{enumi}.}
\setcounter{enumi}{3}
\tightlist
\item
  Calculate the residual for a 4-year US higher education institution with a 75th percentile SAT Math score of 440 and a retention rate of 24\%.
\end{enumerate}

\vspace{0.4in}

\subsubsection*{Correlation}\label{correlation-2}
\addcontentsline{toc}{subsubsection}{Correlation}

Correlation measures the strength and the direction of the linear relationship between two quantitative variables. The closer the value of correlation to \(+1\) or \(-1\), the stronger the linear relationship. Values close to zero indicate a very weak linear relationship between the two variables.

The following output creates a correlation matrix between several pairs of quantitative variables.

\begin{Shaded}
\begin{Highlighting}[]
\NormalTok{IPEDS }\SpecialCharTok{\%\textgreater{}\%}  \CommentTok{\# Data set pipes into}
  \FunctionTok{select}\NormalTok{(}\FunctionTok{c}\NormalTok{(}\StringTok{"Retention"}\NormalTok{, }\StringTok{"Cost\_InState"}\NormalTok{, }
           \StringTok{"Graduation\_Rate"}\NormalTok{, }\StringTok{"Salary"}\NormalTok{, }
           \StringTok{"SATMath\_75"}\NormalTok{, }\StringTok{"ACT\_75"}\NormalTok{)) }\SpecialCharTok{\%\textgreater{}\%}
  \FunctionTok{cor}\NormalTok{(}\AttributeTok{use=}\StringTok{"pairwise.complete.obs"}\NormalTok{) }\SpecialCharTok{\%\textgreater{}\%}
  \FunctionTok{round}\NormalTok{(}\DecValTok{3}\NormalTok{)}
\end{Highlighting}
\end{Shaded}

\begin{verbatim}
#>                 Retention Cost_InState Graduation_Rate Salary SATMath_75 ACT_75
#> Retention           1.000        0.388           0.832  0.698      0.767  0.768
#> Cost_InState        0.388        1.000           0.563  0.365      0.502  0.514
#> Graduation_Rate     0.832        0.563           1.000  0.683      0.817  0.833
#> Salary              0.698        0.365           0.683  1.000      0.747  0.706
#> SATMath_75          0.767        0.502           0.817  0.747      1.000  0.920
#> ACT_75              0.768        0.514           0.833  0.706      0.920  1.000
\end{verbatim}

\begin{enumerate}
\def\labelenumi{\arabic{enumi}.}
\setcounter{enumi}{4}
\tightlist
\item
  What is the value of correlation between SATMath\_75 and Retention?
\end{enumerate}

\vspace{0.3in}

\subsubsection*{Coefficient of determination (squared correlation)}\label{coefficient-of-determination-squared-correlation}
\addcontentsline{toc}{subsubsection}{Coefficient of determination (squared correlation)}

Another summary measure used to explain the linear relationship between two quantitative variables is the coefficient of determination (\(r^2\)). The coefficient of determination, \(r^2\), can also be used to describe the strength of the linear relationship between two quantitative variables. The value of \(r^2\) (a value between 0 and 1) represents the \textbf{proportion of variation in the response that is explained by the least squares line with the explanatory variable}. There are two ways to calculate the coefficient of determination:

~~~Square the correlation coefficient: \(r^2 = (r)^2\)

~~~Use the variances of the response and the residuals: \(r^2 = \dfrac{s_y^2 - s_{RES}^2}{s_y^2} = \dfrac{SST - SSE}{SST}\)

\begin{enumerate}
\def\labelenumi{\arabic{enumi}.}
\setcounter{enumi}{5}
\tightlist
\item
  Use the correlation, \(r\), found in question 5, to calculate the coefficient of determination between SATMath\_75 and Retention, \(r^2\).
\end{enumerate}

\vspace{.4in}

\begin{enumerate}
\def\labelenumi{\arabic{enumi}.}
\setcounter{enumi}{6}
\tightlist
\item
  The variance of the response variable, Retention in \$MM, is \(s_{Retention}^2 = 138.386\) \(\%^2\) and the variability in the residuals is \(s_{RES}^2 = 56.934\) \%\(^2\). Use these values to calculate the coefficient of determination.
\end{enumerate}

\vspace{1in}

In the next part of the activity we will explore what the coefficient of determination measures.

In the first scatterplot, we see the data plotted with a horizontal line. Note that the regression line in this plot has a slope of zero; this assumes there is no relationship between SATMath\_75 and Retention. The value of the y-intercept, 76.387, is the mean of the response variable when there is no relationship between the two variables. To find the sum of squares total (SST) we find the residual (\(residual = y - \hat{y}\)) for each response value from the horizontal line (from the value of 76.387). Each residual is squared and the sum of the squared values is calculated. The SST gives the \textbf{total variability in the response variable, Retention}.

\begin{center}\includegraphics[width=0.7\linewidth]{13-A27-EDA-two-quantitative-corr_files/figure-latex/unnamed-chunk-5-1} \end{center}

The calculated value for the SST is 158451.8.

This next scatterplot, shows the plotted data with the best fit regression line. This is the line of best fit between budget and revenue and has the smallest sum of squares error (SSE). The SSE is calculated by finding the residual from each response value to the regression line. Each residual is squared and the sum of the squared values is calculated.

\begin{center}\includegraphics[width=0.7\linewidth]{13-A27-EDA-two-quantitative-corr_files/figure-latex/unnamed-chunk-6-1} \end{center}

The calculated value for the SSE is 65133.022.

\begin{enumerate}
\def\labelenumi{\arabic{enumi}.}
\setcounter{enumi}{7}
\tightlist
\item
  Calculate the value for \(r^2\) using the values for SST and SSE provided below each of the previous graphs.
\end{enumerate}

\vspace{0.8in}

\begin{enumerate}
\def\labelenumi{\arabic{enumi}.}
\setcounter{enumi}{8}
\tightlist
\item
  Write a sentence interpreting the coefficient of determination in context of the problem.
\end{enumerate}

\newpage

\subsubsection*{Multivariable plots}\label{multivariable-plots}
\addcontentsline{toc}{subsubsection}{Multivariable plots}

When adding another categorical predictor, we can add that variable as shape or color to the plot. In the following code we have added the variable \texttt{Sector}, whether the 4-year institution is public or private.

\begin{Shaded}
\begin{Highlighting}[]
\NormalTok{IPEDS }\SpecialCharTok{\%\textgreater{}\%} \CommentTok{\# Data sest pipes into...}
    \FunctionTok{ggplot}\NormalTok{(}\FunctionTok{aes}\NormalTok{(}\AttributeTok{x =}\NormalTok{ SATMath\_75, }\AttributeTok{y =}\NormalTok{ Retention, }\AttributeTok{shape =}\NormalTok{ Sector, }\AttributeTok{color=}\NormalTok{Sector))}\SpecialCharTok{+}  \CommentTok{\# Specify variables}
    \FunctionTok{geom\_point}\NormalTok{(}\AttributeTok{alpha=}\FloatTok{0.5}\NormalTok{) }\SpecialCharTok{+}  \CommentTok{\# Add scatterplot of points}
    \FunctionTok{labs}\NormalTok{(}\AttributeTok{x =} \StringTok{"75th Percentile SAT Math Score"}\NormalTok{,  }\CommentTok{\# Label x{-}axis}
       \AttributeTok{y =} \StringTok{"Retention Rate (\%)"}\NormalTok{,  }\CommentTok{\# Label y{-}axis}
       \AttributeTok{title =} \StringTok{"Scatterplot of SAT Math Score vs. Retention Rate for }
\StringTok{       4{-}year US Higher Education Institutions"}\NormalTok{) }\SpecialCharTok{+} 
    \CommentTok{\# Be sure to title your plots with the type of plot, observational units, variable(s)}
    \FunctionTok{geom\_smooth}\NormalTok{(}\AttributeTok{method =} \StringTok{"lm"}\NormalTok{, }\AttributeTok{se =} \ConstantTok{FALSE}\NormalTok{) }\SpecialCharTok{+} \CommentTok{\# Add regression line}
    \FunctionTok{scale\_color\_grey}\NormalTok{()}
\end{Highlighting}
\end{Shaded}

\begin{center}\includegraphics[width=0.7\linewidth]{13-A27-EDA-two-quantitative-corr_files/figure-latex/unnamed-chunk-7-1} \end{center}

\begin{enumerate}
\def\labelenumi{\arabic{enumi}.}
\setcounter{enumi}{9}
\tightlist
\item
  Does the relationship between 75th percentile SAT math score and retention rate of 4-year institutions change depending on the level of sector?
\end{enumerate}

\vspace{0.8in}

\newpage

\subsection{Take-home messages}\label{take-home-messages-1}

\begin{enumerate}
\def\labelenumi{\arabic{enumi}.}
\item
  The sign of correlation and the sign of the slope will always be the same. The closer the value of correlation is to \(-1\) or \(+1\), the stronger the linear relationship between the explanatory and the response variable.
\item
  The coefficient of determination multiplied by 100 (\(r^2 \times 100\)) measures the percent of variation in the response variable that is explained by the relationship with the explanatory variable. The closer the value of the coefficient of determination is to 100\%, the stronger the relationship.
\item
  We can use the line of regression to predict values of the response variable for values of the explanatory variable. Do not use values of the explanatory variable that are outside of the range of values in the data set to predict values of the response variable (reflect on why this is true.). This is called \textbf{extrapolation}.
\end{enumerate}

\subsection{Additional notes}\label{additional-notes-1}

Use this space to summarize your thoughts and take additional notes on today's activity and material covered.

\newpage

\section{Activity 28: Prediction of Crocodilian Body Size}\label{activity-28-prediction-of-crocodilian-body-size}

\setstretch{1}

\subsection{Learning outcomes}\label{learning-outcomes-2}

\begin{itemize}
\item
  Given a research question involving two quantitative variables, construct the null and alternative hypotheses
  in words and using appropriate statistical symbols.
\item
  Describe and perform a simulation-based hypothesis test for slope or correlation.
\item
  Interpret and evaluate a p-value for a simulation-based hypothesis test for a slope or correlation.
\item
  Use bootstrapping to find a confidence interval for the slope or correlation.
\item
  Interpret a confidence interval for a slope or correlation.
\end{itemize}

\subsection{Terminology review}\label{terminology-review-2}

In today's activity, we will use simulation-based methods for hypothesis tests and confidence intervals for a linear regression slope or correlation. Some terms covered in this activity are:

\begin{itemize}
\item
  Correlation
\item
  Slope
\item
  Regression line
\end{itemize}

To review these concepts, see Chapter 21 in the textbook.

\subsection{Crocodilian Body Size}\label{crocodilian-body-size}

Much research surrounds using measurements of animals to estimate body-size of extinct animals. Many challenges exist in making accurate estimates for extinct crocodilians. The term crocodilians refers to all members of the family Crocodylidae (``true'' crocodiles), family Alligatoridae (alligators and caimans) and family Gavialidae (gharial, Tomistoma). The researchers in this study (O'Brien 2019) state, ``Among extinct crocodilians and their precursors (e.g., suchians), several methods have been developed to predict body size from suites of hard-tissue proxies. Nevertheless, many have limited applications due to the disparity of some major suchian groups and biases in the fossil record. Here, we test the utility of head width (HW) as a broadly applicable body-size estimator in living and fossil suchians.'' Data were collected on 76 male and female individuals of different species. Is there evidence that head width (measured in cm) is a good predictor of total body length (measured in cm) for crocodilians?

\begin{itemize}
\item
  Download the R script file from D2L and upload to the RStudio server
\item
  Open the file and run lines 1 - 8 to load the dataset
\end{itemize}

\begin{Shaded}
\begin{Highlighting}[]
\CommentTok{\# Read in data set}
\NormalTok{croc }\OtherTok{\textless{}{-}} \FunctionTok{read.csv}\NormalTok{(}\StringTok{"https://math.montana.edu/courses/s216/data/Crocodylian\_headwidth.csv"}\NormalTok{)}
\NormalTok{croc }\OtherTok{\textless{}{-}}\NormalTok{ croc }\SpecialCharTok{\%\textgreater{}\%}
    \FunctionTok{na.omit}\NormalTok{()}
\end{Highlighting}
\end{Shaded}

To create a scatterplot to examine the relationship between head width and total body length we will use \texttt{HW\_cm} as the explanatory variable and \texttt{TL\_cm} as the response variable.

\begin{itemize}
\item
  Enter the name of the explanatory and response variable in line 14
\item
  Highlight and run lines 13 - 20
\end{itemize}

\newpage

\begin{Shaded}
\begin{Highlighting}[]
\NormalTok{croc }\SpecialCharTok{\%\textgreater{}\%} \CommentTok{\# Pipe data set into...}
\FunctionTok{ggplot}\NormalTok{(}\FunctionTok{aes}\NormalTok{(}\AttributeTok{x =}\NormalTok{ explanatory, }\AttributeTok{y =}\NormalTok{ response))}\SpecialCharTok{+}  \CommentTok{\# Specify variables}
  \FunctionTok{geom\_point}\NormalTok{(}\AttributeTok{alpha=}\FloatTok{0.5}\NormalTok{) }\SpecialCharTok{+}  \CommentTok{\# Add scatterplot of points}
  \FunctionTok{labs}\NormalTok{(}\AttributeTok{x =} \StringTok{"head width (cm)"}\NormalTok{,  }\CommentTok{\# Label x{-}axis}
       \AttributeTok{y =} \StringTok{"total length (cm)"}\NormalTok{,  }\CommentTok{\# Label y{-}axis}
       \AttributeTok{title =} \StringTok{"Scatterplot of Crocodilian Head Width vs. Total Length"}\NormalTok{) }\SpecialCharTok{+} 
    \CommentTok{\# Be sure to title your plots}
  \FunctionTok{geom\_smooth}\NormalTok{(}\AttributeTok{method =} \StringTok{"lm"}\NormalTok{, }\AttributeTok{se =} \ConstantTok{FALSE}\NormalTok{)  }\CommentTok{\# Add regression line}
\end{Highlighting}
\end{Shaded}

\begin{enumerate}
\def\labelenumi{\arabic{enumi}.}
\tightlist
\item
  Describe the features of the plot, addressing all four characteristics of a scatterplot.
\end{enumerate}

\vspace{1.2in}

~~~~~~~If you indicated there are potential outliers, which points are they?

\vspace{0.5in}

\subsubsection*{Hypotheses}\label{hypotheses-1}
\addcontentsline{toc}{subsubsection}{Hypotheses}

When analyzing two quantitative variables we can either test regression slope or correlation. In both cases, we are testing that there is a linear relationship between variables.

\begin{enumerate}
\def\labelenumi{\arabic{enumi}.}
\setcounter{enumi}{1}
\tightlist
\item
  Write the null hypothesis in words.
\end{enumerate}

\vspace{0.6in}

\begin{enumerate}
\def\labelenumi{\arabic{enumi}.}
\setcounter{enumi}{2}
\tightlist
\item
  Write the null hypothesis to test slope in notation.
\end{enumerate}

\vspace{0.4in}

\begin{enumerate}
\def\labelenumi{\arabic{enumi}.}
\setcounter{enumi}{3}
\tightlist
\item
  Write the null hypothesis to test correlation in notation.
\end{enumerate}

\vspace{0.4in}

\begin{enumerate}
\def\labelenumi{\arabic{enumi}.}
\setcounter{enumi}{4}
\tightlist
\item
  Write the alternative hypothesis in words.
\end{enumerate}

\vspace{0.6in}

\newpage

\subsubsection*{Summarize and visualize the data}\label{summarize-and-visualize-the-data}
\addcontentsline{toc}{subsubsection}{Summarize and visualize the data}

To create the linear model output and find the value of correlation for the linear relationship\ldots{}

\begin{itemize}
\item
  Enter the the name of the explanatory and response in line 25
\item
  Highlight and run lines 25 - 27
\end{itemize}

\begin{Shaded}
\begin{Highlighting}[]
\CommentTok{\#Linear model}
\NormalTok{lm.croc }\OtherTok{\textless{}{-}} \FunctionTok{lm}\NormalTok{(response}\SpecialCharTok{\textasciitilde{}}\NormalTok{explanatory, }\AttributeTok{data=}\NormalTok{croc) }\CommentTok{\#lm(response\textasciitilde{}explanatory)}
\FunctionTok{round}\NormalTok{(}\FunctionTok{summary}\NormalTok{(lm.croc)}\SpecialCharTok{$}\NormalTok{coefficients, }\DecValTok{5}\NormalTok{)}
\CommentTok{\#Correlation}
\FunctionTok{cor}\NormalTok{(croc}\SpecialCharTok{$}\NormalTok{HW\_cm, croc}\SpecialCharTok{$}\NormalTok{TL\_cm)}
\end{Highlighting}
\end{Shaded}

\begin{enumerate}
\def\labelenumi{\arabic{enumi}.}
\setcounter{enumi}{5}
\tightlist
\item
  Using the output from the evaluated R code, write the equation of the regression line in the context of the problem using appropriate statistical notation.
\end{enumerate}

\vspace{1in}

\begin{enumerate}
\def\labelenumi{\arabic{enumi}.}
\setcounter{enumi}{6}
\tightlist
\item
  Interpret the estimated slope in context of the problem.
\end{enumerate}

\vspace{1in}

\subsubsection*{Use statistical inferential methods to draw inferences from the data}\label{use-statistical-inferential-methods-to-draw-inferences-from-the-data}
\addcontentsline{toc}{subsubsection}{Use statistical inferential methods to draw inferences from the data}

In this activity, we will focus on using simulation-based methods for inference in regression.

\subsubsection*{Simulation-based hypothesis test}\label{simulation-based-hypothesis-test}
\addcontentsline{toc}{subsubsection}{Simulation-based hypothesis test}

Let's start by thinking about how one simulation would be created on the null distribution using cards. First, we would write the values for the response variable, total length, on each card. Next, we would shuffle these \(y\) values while keeping the \(x\) values (explanatory variable) in the same order. Then, find the line of regression for the shuffled \((x, y)\) pairs and calculate either the slope or correlation of the shuffled sample.

We will use the \texttt{regression\_test()} function in R (in the \texttt{catstats} package) to simulate the null distribution of shuffled slopes (or shuffled correlations) and compute a p-value. We will need to enter the response variable name and the explanatory variable name for the formula, the data set name (identified above as \texttt{croc}), the summary measure for the test (either slope or correlation), number of repetitions, the sample statistic (value of slope or correlation), and the direction of the alternative hypothesis.

The response variable name is \texttt{TL\_cm} and the explanatory variable name is \texttt{HW\_cm} for these data.

\begin{enumerate}
\def\labelenumi{\arabic{enumi}.}
\setcounter{enumi}{7}
\tightlist
\item
  What inputs should be entered for each of the following to create the simulation to test regression slope?
\end{enumerate}

\vspace{.5 mm}

\begin{itemize}
\tightlist
\item
  Direction (\texttt{"greater"}, \texttt{"less"}, or \texttt{"two-sided"}):
\end{itemize}

\vspace{.2in}

\begin{itemize}
\tightlist
\item
  Summary measure (choose \texttt{"slope"} or \texttt{"correlation"}):
\end{itemize}

\vspace{.2in}

\begin{itemize}
\tightlist
\item
  As extreme as (enter the value for the sample slope):
\end{itemize}

\vspace{0.2in}

\begin{itemize}
\tightlist
\item
  Number of repetitions:
\end{itemize}

\vspace{.2in}

Using the R script file for this activity\ldots{}

\begin{itemize}
\item
  Enter your answers for question 8 in place of the \texttt{xx}'s to produce the null distribution with 1000 simulations.
\item
  Highlight and run lines 32--37.
\end{itemize}

\begin{Shaded}
\begin{Highlighting}[]
\FunctionTok{regression\_test}\NormalTok{(TL\_cm}\SpecialCharTok{\textasciitilde{}}\NormalTok{HW\_cm, }\CommentTok{\# response \textasciitilde{} explanatory}
               \AttributeTok{data =}\NormalTok{ croc, }\CommentTok{\# Name of data set}
               \AttributeTok{direction =} \StringTok{"xx"}\NormalTok{, }\CommentTok{\# Sign in alternative ("greater", "less", "two{-}sided")}
               \AttributeTok{summary\_measure =} \StringTok{"xx"}\NormalTok{, }\CommentTok{\# "slope" or "correlation"}
               \AttributeTok{as\_extreme\_as =}\NormalTok{ xx, }\CommentTok{\# Observed slope or correlation}
               \AttributeTok{number\_repetitions =} \DecValTok{10000}\NormalTok{) }\CommentTok{\# Number of simulated samples for null distribution}
\end{Highlighting}
\end{Shaded}

\begin{enumerate}
\def\labelenumi{\arabic{enumi}.}
\setcounter{enumi}{8}
\item
  Report the p-value from the R output.
  \vspace{0.5in}
\item
  Suppose we wanted to complete the simulation test using correlation as the summary measure, instead of slope. Which two inputs in \#8 would need to be changed to test for correlation? What inputs should you use instead?
  \vspace{0.75in}
\item
  Change the inputs in lines 32--37 to test for correlation instead of slope. Highlight and run those lines, then report the new p-value of the test.
  \vspace{0.5in}
\item
  The p-values from the test of slope (\#9) and the test of correlation (\#11) should be similar. Explain why the two p-values should match. \emph{Hint: think about the relationship between slope and correlation!}
  \vspace{1in}
\end{enumerate}

\subsubsection*{Simulation-based confidence interval}\label{simulation-based-confidence-interval}
\addcontentsline{toc}{subsubsection}{Simulation-based confidence interval}

We will use the \texttt{regression\_bootstrap\_CI()} function in R (in the \texttt{catstats} package) to simulate the bootstrap distribution of sample slopes (or sample correlations) and calculate a confidence interval.

\begin{itemize}
\item
  Fill in the missing values in the provided R script file to find a 95\% confidence interval for slope.
\item
  Highlight and run lines 42--46.
\end{itemize}

\begin{Shaded}
\begin{Highlighting}[]
\FunctionTok{regression\_bootstrap\_CI}\NormalTok{(response}\SpecialCharTok{\textasciitilde{}}\NormalTok{explanatory, }\CommentTok{\# response \textasciitilde{} explanatory}
   \AttributeTok{data =}\NormalTok{ croc, }\CommentTok{\# Name of data set}
   \AttributeTok{confidence\_level =}\NormalTok{ xx, }\CommentTok{\# Confidence level as decimal}
   \AttributeTok{summary\_measure =} \StringTok{"xx"}\NormalTok{, }\CommentTok{\# Slope or correlation}
   \AttributeTok{number\_repetitions =} \DecValTok{10000}\NormalTok{) }\CommentTok{\# Number of simulated samples for bootstrap distribution}
\end{Highlighting}
\end{Shaded}

\begin{enumerate}
\def\labelenumi{\arabic{enumi}.}
\setcounter{enumi}{12}
\item
  Report the bootstrap 95\% confidence interval in interval notation.\\
  \vspace{0.5in}
\item
  Interpret the interval in question 13 in context of the problem. \emph{Hint: use the interpretation of slope in your confidence interval interpretation.}
\end{enumerate}

\vspace{0.8in}

\subsubsection*{Communicate the results and answer the research question}\label{communicate-the-results-and-answer-the-research-question}
\addcontentsline{toc}{subsubsection}{Communicate the results and answer the research question}

\begin{enumerate}
\def\labelenumi{\arabic{enumi}.}
\setcounter{enumi}{14}
\tightlist
\item
  Based on the p-value and confidence interval, write a conclusion in context of the problem.
\end{enumerate}

\vspace{.8in}

\subsection{Take-home messages}\label{take-home-messages-2}

\begin{enumerate}
\def\labelenumi{\arabic{enumi}.}
\item
  The p-value for a test for correlation should be approximately the same as the p-value for the test of slope. In the simulation test, we just change the statistic type from slope to correlation and use the appropriate sample statistic value.
\item
  To interpret a confidence interval for the slope, think about how to interpret the sample slope and use that information in the confidence interval interpretation for slope.
\item
  To create one simulated sample on the null distribution when testing for a relationship between two quantitative variables, hold the \(x\) values constant and shuffle the \(y\) values to new \(x\) values. Find the regression line for the shuffled data and plot the slope or the correlation for the shuffled data.
\item
  To create one simulated sample on the bootstrap distribution when assessing two quantitative variables, label \(n\) cards with the original (response, explanatory) values. Randomly draw with replacement \(n\) times. Find the regression line for the resampled data and plot the resampled slope or correlation.
\end{enumerate}

\subsection{Additional notes}\label{additional-notes-2}

Use this space to summarize your thoughts and take additional notes on today's activity and material covered.

\newpage

\section{Activity 29: Golf Driving Distance}\label{activity-29-golf-driving-distance}

\setstretch{1}

\subsection{Learning outcomes}\label{learning-outcomes-3}

\begin{itemize}
\item
  Given a research question involving two quantitative variables, construct the null and alternative hypotheses
  in words and using appropriate statistical symbols.
\item
  Assess the conditions to use the normal distribution model for a slope.
\item
  Find the T test statistic (T-score) for a slope based off of \texttt{lm()} output in R.
\item
  Find, interpret, and evaluate the p-value for a theory-based hypothesis test for a slope.
\item
  Create and interpret a theory-based confidence interval for a slope.
\item
  Use a confidence interval to determine the conclusion of a hypothesis test.
\end{itemize}

\subsection{Terminology review}\label{terminology-review-3}

In this week's in-class activity, we will use theory-based methods for hypothesis tests and confidence intervals for a linear regression slope. Some terms covered in this activity are:

\begin{itemize}
\item
  Slope
\item
  Regression line
\end{itemize}

To review these concepts, see Chapter 21 in the textbook.

\subsection{Golf driving distance}\label{golf-driving-distance}

In golf the goal is to complete a hole with as few strokes as possible. A long driving distance to start a hole can help minimize the strokes necessary to complete the hole, as long as that drive stays on the fairway. Data were collected on 354 PGA and LPGA players in 2008 ({``Average Driving Distance and Fairway Accuracy''} 2008). For each player, the average driving distance (yards), fairway accuracy (percentage), and sex was measured. Use these data to assess, ``Does a professional golfer give up accuracy when they hit the ball farther?''

\begin{itemize}
\tightlist
\item
  Download the R script file from D2L and open in the RStudio server
\end{itemize}

\begin{Shaded}
\begin{Highlighting}[]
\CommentTok{\# Read in data set}
\NormalTok{golf }\OtherTok{\textless{}{-}} \FunctionTok{read.csv}\NormalTok{(}\StringTok{"https://math.montana.edu/courses/s216/data/golf.csv"}\NormalTok{)}
\end{Highlighting}
\end{Shaded}

\subsubsection*{Plot review.}\label{plot-review.}
\addcontentsline{toc}{subsubsection}{Plot review.}

To create a scatterplot showing the relationship between the driving distance and percent accuracy for professional golfers:

\begin{itemize}
\item
  Enter the name of the explanatory and response in line 10
\item
  Highlight and run lines 1 - 16
\end{itemize}

\begin{Shaded}
\begin{Highlighting}[]
\NormalTok{golf }\SpecialCharTok{\%\textgreater{}\%} \CommentTok{\# Pipe data set into...}
\FunctionTok{ggplot}\NormalTok{(}\FunctionTok{aes}\NormalTok{(}\AttributeTok{x =}\NormalTok{ explanatory, }\AttributeTok{y =}\NormalTok{ response))}\SpecialCharTok{+}  \CommentTok{\# Specify variables}
  \FunctionTok{geom\_point}\NormalTok{(}\AttributeTok{alpha=}\FloatTok{0.5}\NormalTok{) }\SpecialCharTok{+}  \CommentTok{\# Add scatterplot of points}
  \FunctionTok{labs}\NormalTok{(}\AttributeTok{x =} \StringTok{"Driving Distance (yards)"}\NormalTok{,  }\CommentTok{\# Label x{-}axis}
       \AttributeTok{y =} \StringTok{"Percent Accuracy"}\NormalTok{,  }\CommentTok{\# Label y{-}axis}
       \AttributeTok{title =} \StringTok{"Scatterplot of Driving Distance by Percent Accuracy}
\StringTok{       for Professional Golfers"}\NormalTok{) }\SpecialCharTok{+} 
               \CommentTok{\# Be sure to tile your plots}
  \FunctionTok{geom\_smooth}\NormalTok{(}\AttributeTok{method =} \StringTok{"lm"}\NormalTok{, }\AttributeTok{se =} \ConstantTok{FALSE}\NormalTok{)  }\CommentTok{\# Add regression line}
\end{Highlighting}
\end{Shaded}

\subsubsection*{Conditions for the least squares line}\label{conditions-for-the-least-squares-line}
\addcontentsline{toc}{subsubsection}{Conditions for the least squares line}

When performing inference on a least squares line, the follow conditions are generally required:

\begin{itemize}
\tightlist
\item
  \emph{Independent observations} (for both simulation-based and theory-based methods): individual data points must be independent.

  \begin{itemize}
  \tightlist
  \item
    Check this assumption by investigating the sampling method and determining if the observational units are related in any way.
  \end{itemize}
\item
  \emph{Linearity} (for both simulation-based and theory-based methods): the data should follow a linear trend.

  \begin{itemize}
  \tightlist
  \item
    Check this assumption by examining the scatterplot of the two variables, and a scatterplot of the residuals (on the \(y\)-axis) versus the fitted values (on the \(x\)-axis). The pattern in the residual plot should display a horizontal line.
  \end{itemize}
\item
  \emph{Constant variability} (for theory-based methods only): the variability of points around the least squares line remains roughly constant

  \begin{itemize}
  \tightlist
  \item
    Check this assumption by examining a scatterplot of the residuals (on the \(y\)-axis) versus the fitted values (on the \(x\)-axis). The variability in the residuals around zero should be approximately the same for all fitted values.
  \end{itemize}
\item
  \emph{Nearly normal residuals} (for theory-based methods only: residuals must be nearly normal.

  \begin{itemize}
  \tightlist
  \item
    Check this assumption by examining a histogram of the residuals, which should appear approximately normal.
  \end{itemize}
\end{itemize}

The scatterplot generated earlier and the residual plots shown below will be used to assess these conditions for approximating the data with the \(t\)-distribution.

\begin{center}\includegraphics[width=0.7\linewidth]{13-A29-regression-theory_files/figure-latex/unnamed-chunk-3-1} \end{center}

\begin{enumerate}
\def\labelenumi{\arabic{enumi}.}
\tightlist
\item
  Are the conditions met to use the \(t\)-distribution to approximate the sampling distribution of the standardized statistic? Justify your answer.
\end{enumerate}

\vspace{3in}

\subsubsection*{Ask a research question}\label{ask-a-research-question}
\addcontentsline{toc}{subsubsection}{Ask a research question}

\begin{enumerate}
\def\labelenumi{\arabic{enumi}.}
\setcounter{enumi}{1}
\tightlist
\item
  Write out the null hypothesis in words to test the slope.
\end{enumerate}

\vspace{1in}

\begin{enumerate}
\def\labelenumi{\arabic{enumi}.}
\setcounter{enumi}{2}
\tightlist
\item
  Using the research question, write the alternative hypothesis in notation to test the slope.
\end{enumerate}

\vspace{0.4in}

\subsubsection*{Summarize and visualize the data}\label{summarize-and-visualize-the-data-1}
\addcontentsline{toc}{subsubsection}{Summarize and visualize the data}

The linear model output for this study is shown below.

\begin{Shaded}
\begin{Highlighting}[]
\NormalTok{lm.golf }\OtherTok{\textless{}{-}} \FunctionTok{lm}\NormalTok{(Percent\_Accuracy}\SpecialCharTok{\textasciitilde{}}\NormalTok{Driving\_Distance, }\AttributeTok{data=}\NormalTok{golf) }\CommentTok{\# lm(response\textasciitilde{}explanatory)}
\FunctionTok{round}\NormalTok{(}\FunctionTok{summary}\NormalTok{(lm.golf)}\SpecialCharTok{$}\NormalTok{coefficients, }\DecValTok{3}\NormalTok{)}
\end{Highlighting}
\end{Shaded}

\begin{verbatim}
#>                  Estimate Std. Error t value Pr(>|t|)
#> (Intercept)       103.586      3.329  31.119        0
#> Driving_Distance   -0.142      0.012 -11.553        0
\end{verbatim}

\begin{enumerate}
\def\labelenumi{\arabic{enumi}.}
\setcounter{enumi}{3}
\tightlist
\item
  Report the summary statistic (sample slope) for the linear relationship between driving distance and percent accuracy of golfers. Use proper notation.
\end{enumerate}

\vspace{0.3in}

\subsubsection*{Use statistical inferential methods to draw inferences from the data}\label{use-statistical-inferential-methods-to-draw-inferences-from-the-data-1}
\addcontentsline{toc}{subsubsection}{Use statistical inferential methods to draw inferences from the data}

\paragraph*{Hypothesis test}\label{hypothesis-test}
\addcontentsline{toc}{paragraph}{Hypothesis test}

To find the value of the standardized statistic to test the slope we will use,

\[
T = \frac{\mbox{slope estimate}-null value}{SE} = \frac{b_1-0}{SE(b_1)}.
\]

We will use the linear model R output above to get the estimate for slope and the standard error of the slope.

\begin{enumerate}
\def\labelenumi{\arabic{enumi}.}
\setcounter{enumi}{4}
\tightlist
\item
  Calculate the standardized statistic for slope. Identify where this calculated value is in the linear model R output.
\end{enumerate}

\vspace{0.7in}

\begin{enumerate}
\def\labelenumi{\arabic{enumi}.}
\setcounter{enumi}{5}
\tightlist
\item
  The p-value in the linear model R output is the two-sided p-value for the test of significance for slope. Report the p-value to answer the research question.
\end{enumerate}

\vspace{0.5in}

\begin{enumerate}
\def\labelenumi{\arabic{enumi}.}
\setcounter{enumi}{6}
\tightlist
\item
  Based on the p-value, how much evidence is there against the null hypothesis?
\end{enumerate}

\vspace{0.5in}

\paragraph*{Confidence interval}\label{confidence-interval}
\addcontentsline{toc}{paragraph}{Confidence interval}

Recall that a confidence interval is calculated by adding and subtracting the margin of error to the point estimate.\\
\[\mbox{point estimate}\pm t^*\times SE(\mbox{estimate}).\]
When the point estimate is a regression slope, this formula becomes
\[b_1 \pm t^* \times SE(b_1).\]

The \(t^*\) multiplier comes from a \(t\)-distribution with \(n-2\) degrees of freedom. The sample size for this study is 354 so we will use the degrees of freedom 352 (\(n-2\)).

\begin{itemize}
\item
  Enter the percentile needed to find the multiplier for a 95\% confidence interval for xx
\item
  Enter the degrees of freedom for yy
\item
  Highlight and run line 35
\end{itemize}

\begin{Shaded}
\begin{Highlighting}[]
\FunctionTok{qt}\NormalTok{(xx, yy, }\AttributeTok{lower.tail =} \ConstantTok{TRUE}\NormalTok{) }\CommentTok{\# 95\% t* multiplier }
\end{Highlighting}
\end{Shaded}

\begin{enumerate}
\def\labelenumi{\arabic{enumi}.}
\setcounter{enumi}{7}
\item
  Calculate the 95\% confidence interval for the true slope.
  \vspace{0.8in}
\item
  Interpret the 95\% confidence interval in context of the problem.
\end{enumerate}

\vspace{.8in}

\subsubsection*{Communicate the results and answer the research question}\label{communicate-the-results-and-answer-the-research-question-1}
\addcontentsline{toc}{subsubsection}{Communicate the results and answer the research question}

\begin{enumerate}
\def\labelenumi{\arabic{enumi}.}
\setcounter{enumi}{9}
\tightlist
\item
  Write a conclusion to answer the research question in context of the problem.
\end{enumerate}

\vspace{.8in}

\subsection*{Multivariable plots}\label{multivariable-plots-1}
\addcontentsline{toc}{subsection}{Multivariable plots}

Another variable that may affect the percent accuracy is the which league the golfer is part of. We will look at how this variable may change the relationship between driving distance and percent accuracy.

\begin{Shaded}
\begin{Highlighting}[]
\NormalTok{golf }\SpecialCharTok{\%\textgreater{}\%}
  \FunctionTok{ggplot}\NormalTok{(}\FunctionTok{aes}\NormalTok{(}\AttributeTok{x =}\NormalTok{ Driving\_Distance, }\AttributeTok{y =}\NormalTok{ Percent\_Accuracy, }\AttributeTok{color=}\NormalTok{League))}\SpecialCharTok{+}  \CommentTok{\# Specify variables}
  \FunctionTok{geom\_point}\NormalTok{(}\FunctionTok{aes}\NormalTok{(}\AttributeTok{shape =}\NormalTok{ League), }\AttributeTok{size =} \DecValTok{2}\NormalTok{, }\AttributeTok{alpha=}\FloatTok{0.5}\NormalTok{) }\SpecialCharTok{+}  \CommentTok{\# Add scatterplot of points}
  \FunctionTok{labs}\NormalTok{(}\AttributeTok{x =} \StringTok{"Driving Distance (yards)"}\NormalTok{,  }\CommentTok{\# Label x{-}axis}
       \AttributeTok{y =} \StringTok{"Percent Accuracy"}\NormalTok{,  }\CommentTok{\# Label y{-}axis}
       \AttributeTok{color =} \StringTok{"League"}\NormalTok{, }\AttributeTok{shape =} \StringTok{"League"}\NormalTok{,}
       \AttributeTok{title =} \StringTok{"Scatterplot of Golf Driving Distance and Percent }
\StringTok{       Accuracy by League for Professional Golfers"}\NormalTok{) }\SpecialCharTok{+} \CommentTok{\# Be sure to title your plots}
  \FunctionTok{geom\_smooth}\NormalTok{(}\AttributeTok{method =} \StringTok{"lm"}\NormalTok{, }\AttributeTok{se =} \ConstantTok{FALSE}\NormalTok{) }\SpecialCharTok{+} \CommentTok{\# Add regression line}
    \FunctionTok{scale\_color\_grey}\NormalTok{()}
\end{Highlighting}
\end{Shaded}

\begin{center}\includegraphics[width=0.6\linewidth]{13-A29-regression-theory_files/figure-latex/unnamed-chunk-6-1} \end{center}

\begin{enumerate}
\def\labelenumi{\arabic{enumi}.}
\setcounter{enumi}{10}
\item
  Does the association between driving distance and percent accuracy change depending on which league the golfer is a part of? Explain your answer.\\
  \vspace{1in}
\item
  Explain the association between league and each of the other two variables. Use the following plots in addition to the scatterplot from Q11 to explain your answer.
\end{enumerate}

\begin{center}\includegraphics[width=0.7\linewidth]{13-A29-regression-theory_files/figure-latex/unnamed-chunk-7-1} \includegraphics[width=0.7\linewidth]{13-A29-regression-theory_files/figure-latex/unnamed-chunk-7-2} \end{center}

\vspace{0.8in}

\subsection{Take-home messages}\label{take-home-messages-3}

\begin{enumerate}
\def\labelenumi{\arabic{enumi}.}
\item
  To check the validity conditions for using theory-based methods we must use the residual diagnostic plots to check for normality of residuals and constant variability, and the scatterplot to check for linearity.
\item
  To interpret a confidence interval for the slope, think about how to interpret the sample slope and use that information in the confidence interval interpretation for slope.
\item
  Use the explanatory variable row in the linear model R output to obtain the slope estimate (\texttt{estimate} column) and standard error of the slope (\texttt{Std.\ Error} column) to calculate the standardized slope, or T-score. The calculated T-score should match the \texttt{t\ value} column in the explanatory variable row. The standardized slope tells the number of standard errors the observed slope is above or below 0.
\item
  The explanatory variable row in the linear model R output provides a \textbf{two-sided} p-value under the \texttt{Pr(\textgreater{}\textbar{}t\textbar{})} column.
\item
  The standardized slope is compared to a \(t\)-distribution with \(n-2\) degrees of freedom in order to obtain a p-value. The \(t\)-distribution with \(n-2\) degrees of freedom is also used to find the appropriate multiplier for a given confidence level.
\end{enumerate}

\subsection{Additional notes}\label{additional-notes-3}

Use this space to summarize your thoughts and take additional notes on this week's activity and material covered.

\newpage

\section{Module 13 Lab: Big Mac Index}\label{module-13-lab-big-mac-index}

\setstretch{1}

\subsection{Learning outcomes}\label{learning-outcomes-4}

\begin{itemize}
\item
  Given a research question involving two quantitative variables, construct the null and alternative hypotheses
  in words and using appropriate statistical symbols.
\item
  Assess the conditions to determine in theory or simulation-based methods should be used.
\item
  Find, interpret, and evaluate the p-value for a hypothesis test for a slope or correlation.
\item
  Create and interpret a confidence interval for a slope or correlation.
\end{itemize}

\subsection{Big Mac Index}\label{big-mac-index}

Can the relative cost of a Big Mac across different countries be used to predict the Gross Domestic Product (GDP) per person for that country? The log GDP per person and the adjusted dollar equivalent to purchase a Big Mac was found on a random sample of 55 countries in January of 2022. The cost of a Big Mac in each country was adjusted to US dollars based on current exchange rates. Is there evidence of a positive relationship between Big Mac cost (\texttt{dollar\_price}) and the log GDP per person (\texttt{log\_GDP})?

\begin{itemize}
\item
  Upload and open the R script file for Week 13 lab.
\item
  Upload the csv file, \texttt{big\_mac\_adjusted\_index\_S22.csv}.
\item
  Enter the name of the data set for datasetname in the R script file in line 9.
\item
  Highlight and run lines 1--9 to load the data.
\end{itemize}

\begin{Shaded}
\begin{Highlighting}[]
\CommentTok{\# Read in data set }
\NormalTok{mac }\OtherTok{\textless{}{-}} \FunctionTok{read.csv}\NormalTok{(}\StringTok{"datasetname"}\NormalTok{)}
\end{Highlighting}
\end{Shaded}

\subsubsection*{Summarize and visualize the data}\label{summarize-and-visualize-the-data-2}
\addcontentsline{toc}{subsubsection}{Summarize and visualize the data}

\begin{itemize}
\tightlist
\item
  To find the correlation between the variables, \texttt{log\_GDP} and \texttt{dollar\_price} highlight and run lines 13--16 in the R script file.
\end{itemize}

\begin{Shaded}
\begin{Highlighting}[]
\NormalTok{mac }\SpecialCharTok{\%\textgreater{}\%} 
  \FunctionTok{select}\NormalTok{(}\FunctionTok{c}\NormalTok{(}\StringTok{"log\_GDP"}\NormalTok{, }\StringTok{"dollar\_price"}\NormalTok{)) }\SpecialCharTok{\%\textgreater{}\%}
  \FunctionTok{cor}\NormalTok{(}\AttributeTok{use=}\StringTok{"pairwise.complete.obs"}\NormalTok{) }\SpecialCharTok{\%\textgreater{}\%}
  \FunctionTok{round}\NormalTok{(}\DecValTok{3}\NormalTok{)}
\end{Highlighting}
\end{Shaded}

\begin{enumerate}
\def\labelenumi{\arabic{enumi}.}
\item
  Report the value of correlation between the variables.
  \vspace{0.2in}
\item
  \textbf{Calculate the value of the coefficient of determination between \texttt{log\_GDP} and \texttt{dollar\_price}.}
  \vspace{0.4in}
\item
  Interpret the value of the coefficient of determination in context of the problem.
  \vspace{0.6in}
\end{enumerate}

In the next part of the activity we will assess the linear model between Big Mac cost and log GDP.

\begin{itemize}
\item
  Enter the variable \texttt{log\_GDP} for \texttt{response} and the variable \texttt{dollar\_price} for \texttt{explanatory} in line 22.
\item
  Highlight and run lines 22--23 to get the linear model output.
\end{itemize}

\begin{Shaded}
\begin{Highlighting}[]
\CommentTok{\# Fit linear model: y \textasciitilde{} x}
\NormalTok{bigmacLM }\OtherTok{\textless{}{-}} \FunctionTok{lm}\NormalTok{(response}\SpecialCharTok{\textasciitilde{}}\NormalTok{explanatory, }\AttributeTok{data=}\NormalTok{mac)}
\FunctionTok{round}\NormalTok{(}\FunctionTok{summary}\NormalTok{(bigmacLM)}\SpecialCharTok{$}\NormalTok{coefficients,}\DecValTok{3}\NormalTok{) }\CommentTok{\# Display coefficient summary}
\end{Highlighting}
\end{Shaded}

\begin{enumerate}
\def\labelenumi{\arabic{enumi}.}
\setcounter{enumi}{3}
\tightlist
\item
  Give the value of the slope of the regression line. Interpret this value in context of the problem.
  \vspace{0.6in}
\end{enumerate}

\subsubsection*{Conditions for the least squares line}\label{conditions-for-the-least-squares-line-1}
\addcontentsline{toc}{subsubsection}{Conditions for the least squares line}

\begin{enumerate}
\def\labelenumi{\arabic{enumi}.}
\setcounter{enumi}{4}
\tightlist
\item
  Is there independence between the responses for the observational units? Justify your answer.
\end{enumerate}

\vspace{0.3in}

\begin{itemize}
\tightlist
\item
  Highlight and run lines 28--33 to create the scatterplot to check for linearity.
\end{itemize}

\begin{Shaded}
\begin{Highlighting}[]
\CommentTok{\#Scatterplot}
\NormalTok{mac }\SpecialCharTok{\%\textgreater{}\%} \CommentTok{\# Pipe data set into...}
  \FunctionTok{ggplot}\NormalTok{(}\FunctionTok{aes}\NormalTok{(}\AttributeTok{x =}\NormalTok{ dollar\_price, }\AttributeTok{y =}\NormalTok{ log\_GDP))}\SpecialCharTok{+}  \CommentTok{\# Specify variables}
  \FunctionTok{geom\_point}\NormalTok{(}\AttributeTok{alpha=}\FloatTok{0.5}\NormalTok{) }\SpecialCharTok{+}  \CommentTok{\# Add scatterplot of points}
  \FunctionTok{labs}\NormalTok{(}\AttributeTok{x =} \StringTok{"Big Mac Cost"}\NormalTok{,  }\CommentTok{\# Label x{-}axis}
       \AttributeTok{y =} \StringTok{"log GDP"}\NormalTok{,  }\CommentTok{\# Label y{-}axis}
       \AttributeTok{title =} \StringTok{"Scatterplot of Big Mac Cost vs. log GDP per person}
\StringTok{       for Countries in 2022"}\NormalTok{) }\SpecialCharTok{+}  \CommentTok{\# Be sure to tile your plots}
  \FunctionTok{geom\_smooth}\NormalTok{(}\AttributeTok{method =} \StringTok{"lm"}\NormalTok{, }\AttributeTok{se =} \ConstantTok{FALSE}\NormalTok{)  }\CommentTok{\# Add regression line}
\end{Highlighting}
\end{Shaded}

\begin{enumerate}
\def\labelenumi{\arabic{enumi}.}
\setcounter{enumi}{5}
\tightlist
\item
  Is the linearity condition met to use regression methods to analyze the data? Justify your answer.
\end{enumerate}

\vspace{0.3in}

\begin{itemize}
\tightlist
\item
  Highlight and run lines 38--42 to produce the diagnostic plots needed to assess conditions to use theory-based methods.
\end{itemize}

\begin{Shaded}
\begin{Highlighting}[]
\CommentTok{\#Diagnostic plots}
\NormalTok{bigmacLM }\OtherTok{\textless{}{-}} \FunctionTok{lm}\NormalTok{(log\_GDP}\SpecialCharTok{\textasciitilde{}}\NormalTok{dollar\_price, }\AttributeTok{data =}\NormalTok{ mac) }\CommentTok{\# Fit linear regression model}
\FunctionTok{par}\NormalTok{(}\AttributeTok{mfrow=}\FunctionTok{c}\NormalTok{(}\DecValTok{1}\NormalTok{,}\DecValTok{2}\NormalTok{)) }\CommentTok{\# Set graphics parameters to plot 2 plots in 1 row}
\FunctionTok{plot}\NormalTok{(bigmacLM, }\AttributeTok{which=}\DecValTok{1}\NormalTok{) }\CommentTok{\# Residual vs fitted values}
\FunctionTok{hist}\NormalTok{(bigmacLM}\SpecialCharTok{$}\NormalTok{resid, }\AttributeTok{xlab=}\StringTok{"Residuals"}\NormalTok{, }\AttributeTok{ylab=}\StringTok{"Frequency"}\NormalTok{,}
     \AttributeTok{main =} \StringTok{"Histogram of Residuals"}\NormalTok{) }\CommentTok{\# Histogram of residuals}
\end{Highlighting}
\end{Shaded}

\begin{enumerate}
\def\labelenumi{\arabic{enumi}.}
\setcounter{enumi}{6}
\tightlist
\item
  \textbf{Are the conditions met to use the \(t\)-distribution to approximate the sampling distribution of the standardized statistic? Justify your answer.}
\end{enumerate}

\vspace{1.5in}

\newpage

\subsubsection*{Ask a research question}\label{ask-a-research-question-1}
\addcontentsline{toc}{subsubsection}{Ask a research question}

\begin{enumerate}
\def\labelenumi{\arabic{enumi}.}
\setcounter{enumi}{7}
\tightlist
\item
  Write out the null and alternative hypotheses in notation to test \emph{correlation} between Big Mac cost and log GDP.
\end{enumerate}

\vspace{.2in}

~~~\(H_0:\)

\vspace{.2in}

~~~\(H_A:\)

\vspace{.2in}

\subsubsection*{Use statistical inferential methods to draw inferences from the data}\label{use-statistical-inferential-methods-to-draw-inferences-from-the-data-2}
\addcontentsline{toc}{subsubsection}{Use statistical inferential methods to draw inferences from the data}

\subsubsection*{Hypothesis test}\label{hypothesis-test-1}
\addcontentsline{toc}{subsubsection}{Hypothesis test}

Use the \texttt{regression\_test()} function in R (in the \texttt{catstats} package) to simulate the null distribution of sample \textbf{correlations} and compute a p-value. We will need to enter the response variable name and the explanatory variable name for the formula, the data set name (identified above as \texttt{mac}), the summary measure used for the test, number of repetitions, the sample statistic (value of correlation), and the direction of the alternative hypothesis.

The response variable name is \texttt{log\_GDP} and the explanatory variable name is \texttt{dollar\_price}.

\begin{enumerate}
\def\labelenumi{\arabic{enumi}.}
\setcounter{enumi}{8}
\tightlist
\item
  What inputs should be entered for each of the following to create the simulation to test correlation?
\end{enumerate}

\vspace{.5 mm}

\begin{itemize}
\tightlist
\item
  Direction (\texttt{"greater"}, \texttt{"less"}, or \texttt{"two-sided"}):
\end{itemize}

\vspace{.2in}

\begin{itemize}
\tightlist
\item
  Summary measure (choose \texttt{"slope"} or \texttt{"correlation"}):
\end{itemize}

\vspace{.2in}

\begin{itemize}
\tightlist
\item
  As extreme as (enter the value for the sample correlation):
\end{itemize}

\vspace{0.2in}

\begin{itemize}
\tightlist
\item
  Number of repetitions:
\end{itemize}

\vspace{.2in}

Using the R script file for this activity, enter your answers for question 9 in place of the \texttt{xx}'s to produce the null distribution with 10000 simulations.

\begin{itemize}
\item
  Highlight and run lines 47--53.
\item
  \textbf{Upload a copy of your plot showing the p-value to Gradescope for your group.}
\end{itemize}

\begin{Shaded}
\begin{Highlighting}[]
\FunctionTok{regression\_test}\NormalTok{(log\_GDP}\SpecialCharTok{\textasciitilde{}}\NormalTok{dollar\_price, }\CommentTok{\# response \textasciitilde{} explanatory}
               \AttributeTok{data =}\NormalTok{ mac, }\CommentTok{\# Name of data set}
               \AttributeTok{direction =} \StringTok{"xx"}\NormalTok{, }\CommentTok{\# Sign in alternative ("greater", "less", "two{-}sided")}
               \AttributeTok{summary\_measure  =} \StringTok{"xx"}\NormalTok{, }\CommentTok{\# "slope" or "correlation"}
               \AttributeTok{as\_extreme\_as =}\NormalTok{ xx, }\CommentTok{\# Observed slope or correlation}
               \AttributeTok{number\_repetitions =} \DecValTok{10000}\NormalTok{) }\CommentTok{\# Number of simulated samples for null distribution}
\end{Highlighting}
\end{Shaded}

\begin{enumerate}
\def\labelenumi{\arabic{enumi}.}
\setcounter{enumi}{9}
\tightlist
\item
  Report the p-value from the R output.
  \vspace{0.3in}
\end{enumerate}

\newpage

\subsubsection*{Simulation-based confidence interval}\label{simulation-based-confidence-interval-1}
\addcontentsline{toc}{subsubsection}{Simulation-based confidence interval}

We will use the \texttt{regression\_bootstrap\_CI()} function in R (in the \texttt{catstats} package) to simulate the bootstrap distribution of sample \textbf{correlations} and calculate a confidence interval.

\begin{itemize}
\item
  Fill in the \texttt{xx}'s in the the provided R script file to find a 90\% confidence interval.
\item
  Highlight and run lines 58--62.
\end{itemize}

\begin{Shaded}
\begin{Highlighting}[]
\FunctionTok{regression\_bootstrap\_CI}\NormalTok{(log\_GDP}\SpecialCharTok{\textasciitilde{}}\NormalTok{dollar\_price, }\CommentTok{\# response \textasciitilde{} explanatory}
   \AttributeTok{data =}\NormalTok{ mac, }\CommentTok{\# Name of data set}
   \AttributeTok{confidence\_level =}\NormalTok{ xx, }\CommentTok{\# Confidence level as decimal}
   \AttributeTok{summary\_measure =} \StringTok{"xx"}\NormalTok{, }\CommentTok{\# Slope or correlation}
   \AttributeTok{number\_repetitions =} \DecValTok{10000}\NormalTok{) }\CommentTok{\# Number of simulated samples for bootstrap distribution}
\end{Highlighting}
\end{Shaded}

\begin{enumerate}
\def\labelenumi{\arabic{enumi}.}
\setcounter{enumi}{10}
\tightlist
\item
  Report the bootstrap 90\% confidence interval in interval notation.\\
  \vspace{0.5in}
\end{enumerate}

\subsubsection*{Communicate the results and answer the research question}\label{communicate-the-results-and-answer-the-research-question-2}
\addcontentsline{toc}{subsubsection}{Communicate the results and answer the research question}

\begin{enumerate}
\def\labelenumi{\arabic{enumi}.}
\setcounter{enumi}{11}
\item
  Using a significance level of 0.1, what decision would you make?
  \vspace{0.2in}
\item
  What type of error is possible?
  \vspace{0.3in}
\item
  Interpret this error in context of the problem.
  \vspace{0.8in}
\item
  Write a paragraph summarizing the results of the study as if you are reporting these results in your local newspaper. \textbf{Upload a copy of your paragraph to Gradescope for your group.} Be sure to describe:
\end{enumerate}

\begin{itemize}
\item
  Summary statistic and interpretation

  \begin{itemize}
  \item
    Summary measure (in context)
  \item
    Value of the statistic
  \item
    Order of subtraction when comparing two groups
  \end{itemize}
\item
  P-value and interpretation

  \begin{itemize}
  \item
    Statement about probability or proportion of samples
  \item
    Statistic (summary measure and value)
  \item
    Direction of the alternative
  \item
    Null hypothesis (in context)
  \end{itemize}
\item
  Confidence interval and interpretation

  \begin{itemize}
  \item
    How confident you are (e.g., 90\%, 95\%, 98\%, 99\%)
  \item
    Parameter of interest
  \item
    Calculated interval
  \item
    Order of subtraction when comparing two groups
  \end{itemize}
\item
  Conclusion (written to answer the research question)

  \begin{itemize}
  \item
    Amount of evidence
  \item
    Parameter of interest
  \item
    Direction of the alternative hypothesis
  \end{itemize}
\item
  Scope of inference

  \begin{itemize}
  \item
    To what group of observational units do the results apply (target population or observational units similar to the sample)?
  \item
    What type of inference is appropriate (causal or non-causal)?
  \end{itemize}
\end{itemize}

\newpage

\phantomsection\label{refs}
\begin{CSLReferences}{1}{0}
\bibitem[\citeproctext]{ref-pga}
{``Average Driving Distance and Fairway Accuracy.''} 2008. \href{https://www.pga.com/\%20and\%20https://www.lpga.com/}{https://www.pga.com/ and https://www.lpga.com/}.

\bibitem[\citeproctext]{ref-banton2022}
Banton, et al, S. 2022. {``Jog with Your Dog: Dog Owner Exercise Routines Predict Dog Exercise Routines and Perception of Ideal Body Weight.''} \emph{PLoS ONE} 17(8).

\bibitem[\citeproctext]{ref-bhavsar2022}
Bhavsar, et al, A. 2022. {``Increased Risk of Herpes Zoster in Adults \(\geq\)50 Years Old Diagnosed with COVID-19 in the United States.''} \emph{Open Forum Infectious Diseases} 9(5).

\bibitem[\citeproctext]{ref-islands}
Bulmer, M. n.d. {``Islands in Schools Project.''} \url{https://sites.google.com/site/islandsinschoolsprojectwebsite/home}.

\bibitem[\citeproctext]{ref-bts}
{``Bureau of Transportation Statistics.''} 2019. \url{https://www.bts.gov/}.

\bibitem[\citeproctext]{ref-babies}
{``Child Health and Development Studies.''} n.d. \url{https://www.chdstudies.org/}.

\bibitem[\citeproctext]{ref-darley1973}
Darley, J. M., and C. D. Batson. 1973. {``"From Jerusalem to Jericho": A Study of Situational and Dispositional Variables in Helping Behavior.''} \emph{Journal of Personality and Social Psychology} 27: 100--108.

\bibitem[\citeproctext]{ref-davis2020}
Davis, Smith, A. K. 2020. {``A Poor Substitute for the Real Thing: Captive-Reared Monarch Butterflies Are Weaker, Paler and Have Less Elongated Wings Than Wild Migrants.''} \emph{Biology Letters} 16.

\bibitem[\citeproctext]{ref-doit2015}
Du Toit, et al, G. 2015. {``Randomized Trial of Peanut Consumption in Infants at Risk for Peanut Allergy.''} \emph{New England Journal of Medicine} 372.

\bibitem[\citeproctext]{ref-edmunds2016}
Edmunds, et al, D. 2016. {``Chronic Wasting Disease Drives Population Decline of White-Tailed Deer.''} \emph{PLoS ONE} 11(8).

\bibitem[\citeproctext]{ref-ipeds}
Education Statistics, National Center for. 2018. {``IPEDS.''} \url{https://nces.ed.gov/ipeds/}.

\bibitem[\citeproctext]{ref-gbmarried}
{``Great Britain Married Couples: Great Britain Office of Population Census and Surveys.''} n.d. \url{https://discovery.nationalarchives.gov.uk/details/r/C13351}.

\bibitem[\citeproctext]{ref-zeitler2012}
Group, TODAY Study. 2012. {``\href{https://www.ncbi.nlm.nih.gov/pubmed/22540912}{A Clinical Trial to Maintain Glycemic Control in Youth with Type 2 Diabetes}.''} \emph{New England Journal of Medicine} 366: 2247--56.

\bibitem[\citeproctext]{ref-hamblin2007}
Hamblin, J. K., K. Wynn, and P. Bloom. 2007. {``Social Evaluation by Preverbal Infants.''} \emph{Nature} 450 (6288): 557--59.

\bibitem[\citeproctext]{ref-hirschfelder2018}
Hirschfelder, A., and P. F. Molin. 2018. {``I Is for Ignoble: Stereotyping Native Americans.''} \href{Retrieved\%20from\%20https://www.ferris.edu/HTMLS/news/jimcrow/native/homepage.htm.}{Retrieved from https://www.ferris.edu/HTMLS/news/jimcrow/native/homepage.htm.}

\bibitem[\citeproctext]{ref-hutchison2013}
Hutchison, R. L., and M. A. Hirthler. 2013. {``\href{https://www.ncbi.nlm.nih.gov/pubmed/23932117}{Upper Extremity Injuies in Homer's Iliad}.''} \emph{Journal of Hand Surgery (American Volume)} 38: 1790--93.

\bibitem[\citeproctext]{ref-imdb}
{``{IMDb} Movies Extensive Dataset.''} 2016. \url{https://kaggle.com/stefanoleone992/imdb-extensive-dataset}.

\bibitem[\citeproctext]{ref-kalra2022}
Kalra, et al., Dl. 2022. {``Trustworthiness of Indian Youtubers.''} Kaggle. \url{https://doi.org/10.34740/KAGGLE/DSV/4426566}.

\bibitem[\citeproctext]{ref-keating2021}
Keating, D., N. Ahmed, F. Nirappil, Stanley-Becker I., and L. Bernstein. 2021. {``Coronavirus Infections Dropping Where People Are Vaccinated, Rising Where They Are Not, Post Analysis Finds.''} \emph{Washington Post}. \url{https://www.washingtonpost.com/health/2021/06/14/covid-cases-vaccination-rates/}.

\bibitem[\citeproctext]{ref-laeng2007}
Laeng, Mathisen, B. 2007. {``Why Do Blue-Eyed Men Prefer Women with the Same Eye Color?''} \emph{Behavioral Ecology and Sociobiology} 61(3).

\bibitem[\citeproctext]{ref-levin2000}
Levin, D. T. 2000. {``Race as a Visual Feature: Using Visual Search and Perceptual Discrimination Tasks to Understand Face Categories and the Cross-Race Recognition Deficit.''} \emph{Journal of Experimental Psychology} 129(4).

\bibitem[\citeproctext]{ref-luetkemeier2017}
LUETKEMEIER, et al., M. 2017. {``Skin Tattoos Alter Sweat Rate and Na+ Concentration.''} \emph{Medicine and Science in Sports and Exercise} 49(7).

\bibitem[\citeproctext]{ref-madden2020}
Madden, et al, J. 2020. {``Ready Student One: Exploring the Predictors of Student Learning in Virtual Reality.''} \emph{PLoS ONE} 15(3).

\bibitem[\citeproctext]{ref-miller1956}
Miller, G. A. 1956. {``The Magical Number Seven, Plus or Minus Two: Some Limits on Our Capacity for Processing Information.''} \emph{Psychological Review} 63(2).

\bibitem[\citeproctext]{ref-becentispeech}
Moquin, W., and C. Van Doren. 1973. {``Great Documents in American Indian History.''} Praeger.

\bibitem[\citeproctext]{ref-pew2022}
{``More Americans Are Joining the 'Cashless' Economy.''} 2022. \url{https://www.pewresearch.org/short-reads/2022/10/05/more-americans-are-joining-the-cashless-economy/.}

\bibitem[\citeproctext]{ref-weather}
National Weather Service Corporate Image Web Team. n.d. {``National Weather Service -- {NWS} Billings.''} \url{https://w2.weather.gov/climate/xmacis.php?wfo=byz}.

\bibitem[\citeproctext]{ref-obrien2019}
O'Brien, Lynch, H. D. 2019. {``Crocodylian Head Width Allometry and Phylogenetic Prediction of Body Size in Extinct Crocodyliforms.''} \emph{Integrative Organismal Biology} 1.

\bibitem[\citeproctext]{ref-ocean}
{``Ocean Temperature and Salinity Study.''} n.d. \url{https://calcofi.org/}.

\bibitem[\citeproctext]{ref-WashPost2022}
{``Older People Who Get Covid Are at Increased Risk of Getting Shingles.''} 2022. \url{https://www.washingtonpost.com/health/2022/04/19/shingles-and-covid-over-50/.}

\bibitem[\citeproctext]{ref-physhealth}
{``Physician's Health Study.''} n.d. \url{https://phs.bwh.harvard.edu/}.

\bibitem[\citeproctext]{ref-porath2017}
Porath, Erez, C. 2017. {``Does Rudeness Really Matter? The Effects of Rudeness on Task Performance and Helpfulness.''} \emph{Academy of Management Journal} 50.

\bibitem[\citeproctext]{ref-quinn1999}
Quinn, G. E., C. H. Shin, M. G. Maguire, and R. A. Stone. 1999. {``Myopia and Ambient Lighting at Night.''} \emph{Nature} 399 (6732): 113--14. \url{https://doi.org/10.1038/20094}.

\bibitem[\citeproctext]{ref-ramachandran2007}
Ramachandran, V. 2007. {``3 Clues to Understanding Your Brain.''} \url{https://www.ted.com/talks/vs_ramachandran_3_clues_to_understanding_your_brain}.

\bibitem[\citeproctext]{ref-cdchospitalization}
{``Rates of Laboratory-Confimed COVID-19 Hospitalizations by Vaccination Status.''} 2021. CDC. \url{https://covid.cdc.gov/covid-data-tracker/\#covidnet-hospitalizations-vaccination}.

\bibitem[\citeproctext]{ref-richardson2019}
Richardson, T., and R. T. Gilman. 2019. {``Left-Handedness Is Associated with Greater Fighting Success in Humans.''} \emph{Scientific Reports} 9 (1): 15402. \url{https://doi.org/10.1038/s41598-019-51975-3}.

\bibitem[\citeproctext]{ref-stephens2020}
Stephens, R., and O. Robertson. 2020. {``Swearing as a Response to Pain: Assessing Hypoalgesic Effects of Novel "Swear" Words.''} \emph{Frontiers in Psychology} 11: 643--62.

\bibitem[\citeproctext]{ref-stewart2014}
Stewart, E. H., B. Davis, B. L. Clemans-Taylor, B. Littenberg, C. A. Estrada, and R. M. Centor. 2014. {``Rapid Antigen Group a Streptococcus Test to Diagnose Pharyngitis: A Systematic Review and Meta-Analysis''} 9 (11). \url{https://doi.org/10.1371/journal.pone.0111727}.

\bibitem[\citeproctext]{ref-stroop1935}
Stroop, J. R. 1935. {``Studies of Interference in Serial Verbal Reactions.''} \emph{Journal of Experimental Psychology} 18: 643--62.

\bibitem[\citeproctext]{ref-subach2022}
Subach, et al, A. 2022. {``Foraging Behaviour, Habitat Use and Population Size of the Desert Horned Viper in the Negev Desert.''} \emph{Soc.Open Sci} 9.

\bibitem[\citeproctext]{ref-sulheim2017}
Sulheim, S., A. Ekeland, I. Holme, and R. Bahr. 2017. {``Helmet Use and Risk of Head Injuries in Alpine Skiers and Snowboarders: Changes After an Interval of One Decade''} 51 (1): 44--50. \url{https://doi.org/10.1136/bjsports-2015-095798}.

\bibitem[\citeproctext]{ref-titanic}
{``Titanic.''} n.d. \url{http://www.encyclopedia-titanica.org}.

\bibitem[\citeproctext]{ref-covidvaccinetracker}
{``US COVID-19 Vaccine Tracker: See Your State's Progress.''} 2021. Mayo Clinic. \url{https://www.mayoclinic.org/coronavirus-covid-19/vaccine-tracker}.

\bibitem[\citeproctext]{ref-usepa2020}
US Environmental Protection Agency. n.d. {``Air Data -- Daily Air Quality Tracker.''} \url{https://www.epa.gov/outdoor-air-quality-data/air-data-daily-air-quality-tracker}.

\bibitem[\citeproctext]{ref-wahlstrom2014}
Wahlstrom, et al, K. 2014. {``Examining the Impact of Later School Start Times on the Health and Academic Performance of High School Students: A Multi-Site Study.''} \emph{Center for Applied Research and Educational Improvement}.

\bibitem[\citeproctext]{ref-watson2015}
Watson, et al., N. 2015. {``Recommended Amount of Sleep for a Heathy Adult: A Joint Consensus Statement of the American Academy of Sleep Medicine and Sleep Research Society.''} \emph{Sleep} 38(6).

\bibitem[\citeproctext]{ref-Weiss1988}
Weiss, R. D. 1988. {``Relapse to Cocaine Abuse After Initiating Desipramine Treatment.''} \emph{JAMA} 260(17).

\bibitem[\citeproctext]{ref-navajo2011}
{``Welcome to the Navajo Nation Government: Official Site of the Navajo Nation.''} 2011.\href{\%20Retrieved\%20from\%20https://www.navajo-nsn.gov/.}{Retrieved from https://www.navajo-nsn.gov/.}

\bibitem[\citeproctext]{ref-wilson2016}
Wilson, Woodruff, J. P. 2016. {``Vertebral Adaptations to Large Body Size in Theropod Dinosaurs.''} \emph{PLoS ONE} 11(7).

\end{CSLReferences}

\end{document}
