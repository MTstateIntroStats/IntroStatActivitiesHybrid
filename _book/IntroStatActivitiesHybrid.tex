% Options for packages loaded elsewhere
\PassOptionsToPackage{unicode}{hyperref}
\PassOptionsToPackage{hyphens}{url}
%
\documentclass[
]{report}
\usepackage{amsmath,amssymb}
\usepackage{iftex}
\ifPDFTeX
  \usepackage[T1]{fontenc}
  \usepackage[utf8]{inputenc}
  \usepackage{textcomp} % provide euro and other symbols
\else % if luatex or xetex
  \usepackage{unicode-math} % this also loads fontspec
  \defaultfontfeatures{Scale=MatchLowercase}
  \defaultfontfeatures[\rmfamily]{Ligatures=TeX,Scale=1}
\fi
\usepackage{lmodern}
\ifPDFTeX\else
  % xetex/luatex font selection
\fi
% Use upquote if available, for straight quotes in verbatim environments
\IfFileExists{upquote.sty}{\usepackage{upquote}}{}
\IfFileExists{microtype.sty}{% use microtype if available
  \usepackage[]{microtype}
  \UseMicrotypeSet[protrusion]{basicmath} % disable protrusion for tt fonts
}{}
\makeatletter
\@ifundefined{KOMAClassName}{% if non-KOMA class
  \IfFileExists{parskip.sty}{%
    \usepackage{parskip}
  }{% else
    \setlength{\parindent}{0pt}
    \setlength{\parskip}{6pt plus 2pt minus 1pt}}
}{% if KOMA class
  \KOMAoptions{parskip=half}}
\makeatother
\usepackage{xcolor}
\usepackage{color}
\usepackage{fancyvrb}
\newcommand{\VerbBar}{|}
\newcommand{\VERB}{\Verb[commandchars=\\\{\}]}
\DefineVerbatimEnvironment{Highlighting}{Verbatim}{commandchars=\\\{\}}
% Add ',fontsize=\small' for more characters per line
\usepackage{framed}
\definecolor{shadecolor}{RGB}{248,248,248}
\newenvironment{Shaded}{\begin{snugshade}}{\end{snugshade}}
\newcommand{\AlertTok}[1]{\textcolor[rgb]{0.94,0.16,0.16}{#1}}
\newcommand{\AnnotationTok}[1]{\textcolor[rgb]{0.56,0.35,0.01}{\textbf{\textit{#1}}}}
\newcommand{\AttributeTok}[1]{\textcolor[rgb]{0.13,0.29,0.53}{#1}}
\newcommand{\BaseNTok}[1]{\textcolor[rgb]{0.00,0.00,0.81}{#1}}
\newcommand{\BuiltInTok}[1]{#1}
\newcommand{\CharTok}[1]{\textcolor[rgb]{0.31,0.60,0.02}{#1}}
\newcommand{\CommentTok}[1]{\textcolor[rgb]{0.56,0.35,0.01}{\textit{#1}}}
\newcommand{\CommentVarTok}[1]{\textcolor[rgb]{0.56,0.35,0.01}{\textbf{\textit{#1}}}}
\newcommand{\ConstantTok}[1]{\textcolor[rgb]{0.56,0.35,0.01}{#1}}
\newcommand{\ControlFlowTok}[1]{\textcolor[rgb]{0.13,0.29,0.53}{\textbf{#1}}}
\newcommand{\DataTypeTok}[1]{\textcolor[rgb]{0.13,0.29,0.53}{#1}}
\newcommand{\DecValTok}[1]{\textcolor[rgb]{0.00,0.00,0.81}{#1}}
\newcommand{\DocumentationTok}[1]{\textcolor[rgb]{0.56,0.35,0.01}{\textbf{\textit{#1}}}}
\newcommand{\ErrorTok}[1]{\textcolor[rgb]{0.64,0.00,0.00}{\textbf{#1}}}
\newcommand{\ExtensionTok}[1]{#1}
\newcommand{\FloatTok}[1]{\textcolor[rgb]{0.00,0.00,0.81}{#1}}
\newcommand{\FunctionTok}[1]{\textcolor[rgb]{0.13,0.29,0.53}{\textbf{#1}}}
\newcommand{\ImportTok}[1]{#1}
\newcommand{\InformationTok}[1]{\textcolor[rgb]{0.56,0.35,0.01}{\textbf{\textit{#1}}}}
\newcommand{\KeywordTok}[1]{\textcolor[rgb]{0.13,0.29,0.53}{\textbf{#1}}}
\newcommand{\NormalTok}[1]{#1}
\newcommand{\OperatorTok}[1]{\textcolor[rgb]{0.81,0.36,0.00}{\textbf{#1}}}
\newcommand{\OtherTok}[1]{\textcolor[rgb]{0.56,0.35,0.01}{#1}}
\newcommand{\PreprocessorTok}[1]{\textcolor[rgb]{0.56,0.35,0.01}{\textit{#1}}}
\newcommand{\RegionMarkerTok}[1]{#1}
\newcommand{\SpecialCharTok}[1]{\textcolor[rgb]{0.81,0.36,0.00}{\textbf{#1}}}
\newcommand{\SpecialStringTok}[1]{\textcolor[rgb]{0.31,0.60,0.02}{#1}}
\newcommand{\StringTok}[1]{\textcolor[rgb]{0.31,0.60,0.02}{#1}}
\newcommand{\VariableTok}[1]{\textcolor[rgb]{0.00,0.00,0.00}{#1}}
\newcommand{\VerbatimStringTok}[1]{\textcolor[rgb]{0.31,0.60,0.02}{#1}}
\newcommand{\WarningTok}[1]{\textcolor[rgb]{0.56,0.35,0.01}{\textbf{\textit{#1}}}}
\usepackage{longtable,booktabs,array}
\usepackage{calc} % for calculating minipage widths
% Correct order of tables after \paragraph or \subparagraph
\usepackage{etoolbox}
\makeatletter
\patchcmd\longtable{\par}{\if@noskipsec\mbox{}\fi\par}{}{}
\makeatother
% Allow footnotes in longtable head/foot
\IfFileExists{footnotehyper.sty}{\usepackage{footnotehyper}}{\usepackage{footnote}}
\makesavenoteenv{longtable}
\usepackage{graphicx}
\makeatletter
\def\maxwidth{\ifdim\Gin@nat@width>\linewidth\linewidth\else\Gin@nat@width\fi}
\def\maxheight{\ifdim\Gin@nat@height>\textheight\textheight\else\Gin@nat@height\fi}
\makeatother
% Scale images if necessary, so that they will not overflow the page
% margins by default, and it is still possible to overwrite the defaults
% using explicit options in \includegraphics[width, height, ...]{}
\setkeys{Gin}{width=\maxwidth,height=\maxheight,keepaspectratio}
% Set default figure placement to htbp
\makeatletter
\def\fps@figure{htbp}
\makeatother
\setlength{\emergencystretch}{3em} % prevent overfull lines
\providecommand{\tightlist}{%
  \setlength{\itemsep}{0pt}\setlength{\parskip}{0pt}}
\setcounter{secnumdepth}{5}
% definitions for citeproc citations
\NewDocumentCommand\citeproctext{}{}
\NewDocumentCommand\citeproc{mm}{%
  \begingroup\def\citeproctext{#2}\cite{#1}\endgroup}
\makeatletter
 % allow citations to break across lines
 \let\@cite@ofmt\@firstofone
 % avoid brackets around text for \cite:
 \def\@biblabel#1{}
 \def\@cite#1#2{{#1\if@tempswa , #2\fi}}
\makeatother
\newlength{\cslhangindent}
\setlength{\cslhangindent}{1.5em}
\newlength{\csllabelwidth}
\setlength{\csllabelwidth}{3em}
\newenvironment{CSLReferences}[2] % #1 hanging-indent, #2 entry-spacing
 {\begin{list}{}{%
  \setlength{\itemindent}{0pt}
  \setlength{\leftmargin}{0pt}
  \setlength{\parsep}{0pt}
  % turn on hanging indent if param 1 is 1
  \ifodd #1
   \setlength{\leftmargin}{\cslhangindent}
   \setlength{\itemindent}{-1\cslhangindent}
  \fi
  % set entry spacing
  \setlength{\itemsep}{#2\baselineskip}}}
 {\end{list}}
\usepackage{calc}
\newcommand{\CSLBlock}[1]{\hfill\break\parbox[t]{\linewidth}{\strut\ignorespaces#1\strut}}
\newcommand{\CSLLeftMargin}[1]{\parbox[t]{\csllabelwidth}{\strut#1\strut}}
\newcommand{\CSLRightInline}[1]{\parbox[t]{\linewidth - \csllabelwidth}{\strut#1\strut}}
\newcommand{\CSLIndent}[1]{\hspace{\cslhangindent}#1}
\usepackage{booktabs}
\usepackage{geometry}
\usepackage[none]{hyphenat}
\usepackage{titlesec}
\usepackage{longtable}
\usepackage{xcolor}
\usepackage{setspace}
\usepackage{pdfpages}

\pagestyle{plain}

%%%% Set margins
\setlength{\topmargin}{-1cm}
\addtolength{\evensidemargin}{-1cm}
\addtolength{\oddsidemargin}{-1cm}
\addtolength{\textheight}{3cm}
\addtolength{\textwidth}{2cm}

% Spacing for reading guides
\newcommand{\rgs}{\vspace{12pt}} % Vertical space
\newcommand{\rgi}{\hspace{24pt}}  % Indent

\newcommand\latexcode[1]{#1}

% Format chapter titles and spacing
\renewcommand*{\chaptername}{Module}

\titleformat{\chapter}[display]
{\bfseries\Large}
{\filleft\MakeUppercase{\chaptertitlename} \Huge\thechapter}
{3ex}
{\titlerule
\vspace{1.5ex}%
\filright}
[\vspace{1.5ex}%
\titlerule]
\titlespacing*{\chapter}{0pt}{-40pt}{20pt}
\ifLuaTeX
  \usepackage{selnolig}  % disable illegal ligatures
\fi
\usepackage{bookmark}
\IfFileExists{xurl.sty}{\usepackage{xurl}}{} % add URL line breaks if available
\urlstyle{same}
\hypersetup{
  hidelinks,
  pdfcreator={LaTeX via pandoc}}

\title{\textbf{STAT 216 Coursepack}\\
\strut \\
\includegraphics[width=5in,height=\textheight]{images/msu-campus.jpg}}
\usepackage{etoolbox}
\makeatletter
\providecommand{\subtitle}[1]{% add subtitle to \maketitle
  \apptocmd{\@title}{\par {\large #1 \par}}{}{}
}
\makeatother
\subtitle{Spring 2025\\
Montana State University}
\author{Melinda Yager\\
Jade Schmidt\\
Stacey Hancock}
\date{}

\begin{document}
\maketitle

\newpage
\thispagestyle{empty}

This resource was developed by Melinda Yager, Jade Schmidt, and Stacey Hancock in 2021 to accompany the online textbook: Hancock, S., Carnegie, N., Meyer, E., Schmidt, J., and Yager, M. (2021). \emph{Montana State Introductory Statistics with R}. Montana State University. \url{https://mtstateintrostats.github.io/IntroStatTextbook/}.

This resource is released under a \href{https://creativecommons.org/licenses/by-nc-sa/4.0/}{Creative Commons BY-NC-SA 4.0} license unless otherwise noted.

\setcounter{tocdepth}{1}
\addtocontents{toc}{\protect\thispagestyle{empty}}
\tableofcontents
\thispagestyle{empty}

\newpage
\setcounter{page}{1}

\chapter*{Preface}\label{preface}
\addcontentsline{toc}{chapter}{Preface}

This coursepack accompanies the textbook for STAT 216: Montana State Introductory Statistics with R, which can be found at \url{https://mtstateintrostats.github.io/IntroStatTextbook/}. The syllabus for the course (including the course calendar), data sets, and links to D2L Brightspace, Gradescope, and the MSU RStudio server can be found on the course webpage: \url{https://math.montana.edu/courses/s216/}.
Other notes and review materials are linked in D2L.

Each of the activities in this workbook is designed to target specific learning outcomes of the course, giving you practice with important statistical concepts in a group setting with instructor guidance. In addition to the in-class activities for the course, video notes are provided to aid in taking notes while you complete the required videos. Bring this workbook with you to class each class period, and take notes in the workbook as you would your own notes. A well-written completed workbook will provide an optimal study guide for exams!

All activities and labs in this coursepack will be completed during class time. Parts of each lab will be turned in on Gradescope. To aid in your understanding, read through the introduction for each activity before attending class each day.

STAT 216 is a 3-credit in-person course. In our experience, it takes six to nine hours per week outside of class to achieve a good grade in this class. By ``good'' we mean at least a C because a grade of D or below does not count toward fulfilling degree requirements. Many of you set your goals higher than just getting a C, and we fully support that. You need roughly nine hours per week to review past activities, read feedback on previous assignments, complete current assignments, and prepare for the next day's class. A typical week in the life of a STAT 216 student looks like:

\begin{itemize}
\tightlist
\item
  \emph{Prior to class meeting}:

  \begin{itemize}
  \tightlist
  \item
    Read assigned sections of the textbook, using the provided reading guides to take notes on the material.
  \item
    Watch the provided videos, taking notes in the coursepack.
  \item
    Read through the introduction to the day's in-class activity.
  \item
    Read through the week's homework assignment and note any questions you may have on the content.
  \end{itemize}
\item
  \emph{During class meeting}:

  \begin{itemize}
  \tightlist
  \item
    Work through the guided activity, in-class activity or weekly lab with your classmates and instructor, taking detailed notes on your answers to each question in the activity.
  \end{itemize}
\item
  \emph{After class meeting}:

  \begin{itemize}
  \tightlist
  \item
    Complete any parts of the activity you did not complete in class.
  \item
    Review the activity solutions in the Math and Stat Center, and take notes on key points.
  \item
    Complete any remaining assigned readings for the week.
  \item
    Complete the week's homework assignment.
  \end{itemize}
\end{itemize}

\nocite{*}

\chapter{Basics of Data and Sampling Methods}\label{basics-of-data-and-sampling-methods}

\section{Vocabulary Review and Key Topics}\label{vocabulary-review-and-key-topics}

At the beginning of each module is a list of new vocabulary terms and key topics for that module. As you read through the material in the text book and watch the videos prior to class, look for these terms. Reference the following definitions to guide your understanding.

\subsection{Module 1 Vocabulary}\label{module-1-vocabulary}

\begin{itemize}
\item
  \textbf{Data}: observations used to answer research questions
\item
  \textbf{Observational units (cases)}: the subjects or entities on which data are collected

  \begin{itemize}
  \tightlist
  \item
    The rows in a data set represent the observational units
  \end{itemize}
\item
  \textbf{Variable}: the characteristics collected on each observational unit
\item
  \textbf{Types of variables}:

  \begin{itemize}
  \item
    \textbf{Categorical}: cases are grouped into categories
  \item
    \textbf{Quantitative}: numerical measurements, where performing arithmetic operations makes sense
  \end{itemize}
\item
  \textbf{Target population}: group of observational units of interest
\item
  \textbf{Sample}: subset of the population
\item
  \textbf{Sampling methods}:

  \begin{itemize}
  \item
    \textbf{Unbiased sampling method (e.g., a random sample)}: on average, the sample will be representative of the target population; all observational units in the target population have the same chance of being selected
  \item
    \textbf{Biased sampling method (e.g., convenience sample)}: on average, the sample will not be representative of the target population; some part of the target population will be over- or under-represented
  \end{itemize}
\item
  \textbf{Type of sampling bias}:

  \begin{itemize}
  \item
    \textbf{Selection bias}: method of sampling is biased; some part of the target population is over- or under-represented
  \item
    \textbf{Non-response bias}: part of a pre-selected sample does not respond or cannot be reached
  \item
    \textbf{Response bias}: responses are not truthful (poor/leading question phrasing, social desirability)
  \end{itemize}
\item
  \textbf{Generalization}: to what group of observational units can the results be applied to?

  \begin{itemize}
  \item
    If an unbiased method of selection was used and there is no non-response or response bias, we can generalize the results to the target population.
  \item
    If a biased method of selection was used or if non-response or response bias is present, we can only generalize the result to the sample or similar observational units.
  \end{itemize}
\end{itemize}

\newpage

\section{Activity 1: Intro to Data}\label{activity-1-intro-to-data}

\setstretch{1}

\subsection{Learning outcomes}\label{learning-outcomes}

\begin{itemize}
\tightlist
\item
  Creating a data set
\end{itemize}

\subsection{Terminology review}\label{terminology-review}

Statistics is the study of how best to collect, analyze, and draw conclusions from data. This week in class you will be introduced to the following terms:

\begin{itemize}
\item
  Observational units or cases
\item
  Variables: categorical or quantitative
\end{itemize}

For more on these concepts, read Chapter 1 in the textbook.

\subsection{General information on the Coursepack}\label{general-information-on-the-coursepack}

Information is provided throughout each activity and lab to guide students through that day's activity or lab. Be sure to read ALL the material provided at the beginning of the activity and between each question. At the end of each activity is a section called \emph{Take-home messages} that contains key points from the day's activity. Use these to review the day's activity and make sure you have a full understanding of that material.

\subsection{Steps of the statistical investigation process}\label{steps-of-the-statistical-investigation-process}

As we move through the semester we will work through the six steps of the statistical investigation process.

\begin{enumerate}
\def\labelenumi{\arabic{enumi}.}
\item
  Ask a research question.
\item
  Design a study and collect data.
\item
  Summarize and visualize the data.
\item
  Use statistical analysis methods to draw inferences from the data.
\item
  Communicate the results and answer the research question.
\item
  Revisit and look forward.
\end{enumerate}

Today we will focus on the first two steps.

\textbf{Step 1}: The first step of any statistical investigation is to \emph{ask a research question}. As stated in the textbook, ``with the rise of data science, however, we might not start with a research question, and instead start with a data set.'' Today we will create a data set by collecting responses on students in class.

\textbf{Step 2}: To answer any research question, we must \emph{design a study and collect data}. Our study will consist of answers from each student. Your responses will become our observed data that we will explore.

\textbf{Observational units} or \textbf{cases} are the subjects data are collected on. In a spreadsheet of the data set, each row will represent a single observational unit.

\newpage

\begin{enumerate}
\def\labelenumi{\arabic{enumi}.}
\tightlist
\item
  Open the Google Form linked in D2L and fill in the responses for the following questions. When creating a data set for use in R it is important to use single words or an underscore between words. Each outcome must be written the same way each time. Make sure to use all lowercase letters to create this data set to have consistency between responses. Do not give units of measure for numerical values within the data set. For \texttt{Residency} use in\_state or out\_state as the two outcomes.
\end{enumerate}

\begin{itemize}
\tightlist
\item
  Major: what is your declared major?
\end{itemize}

\vspace{0.2in}

\begin{itemize}
\tightlist
\item
  Residency: do you have in-state or out-of-state residency?
\end{itemize}

\vspace{0.2in}

\begin{itemize}
\tightlist
\item
  Num\_Credits: how many credits are you taking this semester?
\end{itemize}

\vspace{0.2in}

\begin{itemize}
\tightlist
\item
  Dominant\_hand: are you left or right-handed?
\end{itemize}

\vspace{0.2in}

\begin{itemize}
\tightlist
\item
  Hand\_span: what is the width of your dominant hand from the tip of your thumb to the tip of your pinky with your hand spread out measured in cm?
\end{itemize}

\vspace{0.2in}

\begin{itemize}
\tightlist
\item
  Grip\_dominant: what is the grip strength measured in lbs for your dominant hand?
\end{itemize}

\vspace{0.2in}

\begin{itemize}
\tightlist
\item
  Grip\_nondominant: what is the grip strength measured in lbs for your non-dominant hand?
\end{itemize}

\vspace{0.2in}

\subsection{Take-home messages}\label{take-home-messages}

\begin{enumerate}
\def\labelenumi{\arabic{enumi}.}
\item
  When creating a data set, each row will represent a single observational unit or case. Each column represents a variable collected. It is important to write each variable as a single word or use an underscore between words.
\item
  Make sure to be consistent with writing each outcome in the data set as R is case sensitive. All outcomes must be written exactly the same way.
\end{enumerate}

\subsection{Additional notes}\label{additional-notes}

Use this space to summarize your thoughts and take additional notes on today's activity and material covered, and to write down the names and contact information of your teammates.

\newpage

\section{Video Notes: Intro to data and Sampling Methods}\label{video-notes-intro-to-data-and-sampling-methods}

\setstretch{1}

Read through Sections 1.1 -- 1.3 and 2.1 in the course textbook and watch the course videos prior to coming to class. Fill in the following questions to aid in your understanding of the material. Many of the following questions are asked on the video quiz on Gradescope.

\subsection{Course Videos}\label{course-videos}

\begin{itemize}
\item
  1.2.1and1.2.2
\item
  1.2.3to1.2.4
\item
  2.1
\end{itemize}

\subsection*{Data basics: Video 1.2.1and1.2.2}\label{data-basics-video-1.2.1and1.2.2}
\addcontentsline{toc}{subsection}{Data basics: Video 1.2.1and1.2.2}

Data: \_\_\_\_\_\_\_\_\_\_\_\_\_\_\_\_\_\_\_\_\_\_\_\_\_\_\_\_\_\_\_ used to answer research questions

Observational unit or case: the people or things we \_\_\_\_\_\_\_\_\_\_\_\_\_\_\_\_\_\_\_\_\_ data from; represents the \_\_\_\_\_\_\_\_\_\_\_ in each data set

Variable: characteristics measured on each \_\_\_\_\_\_\_\_\_\_\_\_\_\_\_\_\_\_\_\_\_\_\_\_\_\_\_\_\_\_\_.

\subsubsection*{Types of variables}\label{types-of-variables}
\addcontentsline{toc}{subsubsection}{Types of variables}

\begin{itemize}
\tightlist
\item
  Categorical variable:
\end{itemize}

\vspace{0.5in}

\setstretch{1.5}

\rgi - Ordinal: levels of the variable have a natural ordering

\rgi \rgi Examples: `Scale' questions, years of schooling completed

\rgi - Nominal:levels of the variable do not have a natural ordering

\rgi \rgi Examples: hair color, eye color, zipcode

\setstretch{1}

\begin{itemize}
\tightlist
\item
  Quantitative variable:
\end{itemize}

\vspace{0.5in}

\setstretch{1.5}

\rgi - Continuous variables: value can be any value within a range.

\rgi \rgi Examples: percentage of students who are nursing majors

\rgi \rgi \rgi - average hours of exercise per week

\rgi \rgi \rgi - distance or time (measured with enough precision)

\rgi - Discrete variables: can only be specific values, with jumps between

\rgi \rgi Examples: SAT score

\rgi \rgi \rgi - number of car accidents

\setstretch{1}
\newpage

Example: The Bureau of Transportation Statistics ({``Bureau of Transportation Statistics''} 2019) collects data on all forms of public transportation. The data set seen here includes several variables collect on flights departing on a random sample of 150 US airports in December of 2019.

\vspace{1mm}

\begin{Shaded}
\begin{Highlighting}[]
\NormalTok{airport }\OtherTok{\textless{}{-}} \FunctionTok{read.csv}\NormalTok{(}\StringTok{"data/airport\_delay.csv"}\NormalTok{)}
\FunctionTok{glimpse}\NormalTok{(airport)}
\CommentTok{\#\textgreater{} Rows: 150}
\CommentTok{\#\textgreater{} Columns: 19}
\CommentTok{\#\textgreater{} $ airport             \textless{}chr\textgreater{} "ABI", "ABY", "ACV", "ACY", "ADQ", "AEX", "ALB", "\textasciitilde{}}
\CommentTok{\#\textgreater{} $ city                \textless{}chr\textgreater{} "Abilene", "Albany", "Arcata/Eureka", "Atlantic Ci\textasciitilde{}}
\CommentTok{\#\textgreater{} $ state               \textless{}chr\textgreater{} " TX", " GA", " CA", " NJ", " AK", " LA", " NY", "\textasciitilde{}}
\CommentTok{\#\textgreater{} $ airport\_name        \textless{}chr\textgreater{} " Abilene Regional", " Southwest Georgia Regional"\textasciitilde{}}
\CommentTok{\#\textgreater{} $ hub                 \textless{}chr\textgreater{} "no", "no", "no", "no", "no", "no", "no", "no", "n\textasciitilde{}}
\CommentTok{\#\textgreater{} $ international       \textless{}chr\textgreater{} "no", "no", "no", "yes", "no", "yes", "yes", "yes"\textasciitilde{}}
\CommentTok{\#\textgreater{} $ elevation\_1000      \textless{}dbl\textgreater{} 1.7906, 0.1932, 0.2223, 0.0748, 0.0787, 0.0881, 0.\textasciitilde{}}
\CommentTok{\#\textgreater{} $ latitude            \textless{}dbl\textgreater{} 32.4, 31.5, 41.0, 39.5, 57.7, 31.3, 42.7, 35.2, 45\textasciitilde{}}
\CommentTok{\#\textgreater{} $ longitude           \textless{}dbl\textgreater{} {-}99.7, {-}81.2, {-}124.1, {-}74.6, {-}152.5, {-}92.5, {-}73.8,\textasciitilde{}}
\CommentTok{\#\textgreater{} $ arr\_flights         \textless{}int\textgreater{} 195, 81, 215, 293, 54, 282, 943, 410, 53, 32314, 6\textasciitilde{}}
\CommentTok{\#\textgreater{} $ perc\_delay15        \textless{}dbl\textgreater{} 16.410256, 13.580247, 23.255814, 15.358362, 12.962\textasciitilde{}}
\CommentTok{\#\textgreater{} $ perc\_cancelled      \textless{}dbl\textgreater{} 0.5128205, 0.0000000, 4.1860465, 0.6825939, 14.814\textasciitilde{}}
\CommentTok{\#\textgreater{} $ perc\_diverted       \textless{}dbl\textgreater{} 0.00000000, 0.00000000, 2.32558139, 0.68259386, 0.\textasciitilde{}}
\CommentTok{\#\textgreater{} $ arr\_delay           \textless{}int\textgreater{} 1563, 1244, 4763, 2905, 329, 1293, 15127, 9705, 25\textasciitilde{}}
\CommentTok{\#\textgreater{} $ carrier\_delay       \textless{}int\textgreater{} 459, 890, 1613, 476, 180, 302, 5627, 2253, 439, 10\textasciitilde{}}
\CommentTok{\#\textgreater{} $ weather\_delay       \textless{}int\textgreater{} 21, 43, 549, 124, 1, 58, 2346, 168, 1236, 13331, 2\textasciitilde{}}
\CommentTok{\#\textgreater{} $ nas\_delay           \textless{}int\textgreater{} 257, 39, 154, 771, 51, 112, 2096, 616, 746, 45674,\textasciitilde{}}
\CommentTok{\#\textgreater{} $ security\_delay      \textless{}int\textgreater{} 0, 0, 0, 25, 0, 0, 44, 0, 0, 375, 0, 83, 0, 23, 0,\textasciitilde{}}
\CommentTok{\#\textgreater{} $ late\_aircraft\_delay \textless{}int\textgreater{} 826, 272, 2447, 1509, 97, 821, 5014, 6668, 108, 10\textasciitilde{}}
\end{Highlighting}
\end{Shaded}

\begin{itemize}
\tightlist
\item
  What are the observational units?
\end{itemize}

\vspace{0.2in}

\begin{itemize}
\tightlist
\item
  Identify which variables are categorical.
\end{itemize}

\vspace{0.2in}

\begin{itemize}
\tightlist
\item
  Identify which variables are quantitative.
\end{itemize}

\vspace{0.2in}

\subsubsection*{Exploratory data analysis (EDA)}\label{exploratory-data-analysis-eda}
\addcontentsline{toc}{subsubsection}{Exploratory data analysis (EDA)}

Summary statistic: a single number which \_\_\_\_\_\_\_\_\_\_\_\_\_\_\_\_\_\_\_\_\_\_\_ an entire data set

\begin{itemize}
\tightlist
\item
  Also called the point estimate.
\end{itemize}

\rgi Examples:

\rgi \rgi proportion of people who had a stroke

\vspace{0.3in}

\rgi \rgi mean (or average) age

\vspace{0.3in}

\begin{itemize}
\tightlist
\item
  The summary statistic and type of plot used depends on the type (categorical or quantitative) of variable(s)!
\end{itemize}

\newpage

\subsection*{Roles of variables: 1.2.3to1.2.4}\label{roles-of-variables-1.2.3to1.2.4}
\addcontentsline{toc}{subsection}{Roles of variables: 1.2.3to1.2.4}

Explanatory variable: predictor variable

\begin{itemize}
\item
  The variable researchers think \emph{may be} \_\_\_\_\_\_\_\_\_\_\_\_\_
  the other variable.
\item
  In an experiment, what the researchers \_\_\_\_\_\_\_\_\_\_\_\_\_ or \_\_\_\_\_\_\_\_\_\_\_\_\_\_\_\_.
\item
  The groups that we are comparing from the data set.
\end{itemize}

Response variable:

\begin{itemize}
\item
  The variable researchers think \emph{may be} \_\_\_\_\_\_\_\_\_\_\_\_\_\_\_\_\_\_\_ by the other variable.
\item
  Always simply \_\_\_\_\_\_\_\_\_\_\_\_\_\_\_\_ or \_\_\_\_\_\_\_\_\_\_\_\_\_\_\_\_\_\_; never controlled by researchers.
\end{itemize}

Examples:

Can you predict a criminal's height based on the footprint left at the scene of a crime?

\begin{itemize}
\tightlist
\item
  Identify the explanatory variable:
\end{itemize}

\vspace{0.25in}

\begin{itemize}
\tightlist
\item
  Identify the response variable:
\end{itemize}

\vspace{0.25in}

Does marking an item on sale (even without changing the price) increase the number of units sold per day, on average?

\begin{itemize}
\tightlist
\item
  Identify the explanatory variable:
\end{itemize}

\vspace{0.25in}

\begin{itemize}
\tightlist
\item
  Identify the response variable:
\end{itemize}

\vspace{0.25in}

In the Physician's Health Study ({``Physician's Health Study,''} n.d.), male physicians participated in a study to determine whether taking a daily low-dose aspirin reduced the risk of heart attacks. The male physicians were randomly assigned to the treatment groups. After five years, 104 of the 11,037 male physicians taking a daily low-dose aspirin had experienced a heart attack while 189 of the 11,034 male physicians taking a placebo had experienced a heart attack.

\begin{itemize}
\tightlist
\item
  Identify the explanatory variable:
\end{itemize}

\vspace{0.25in}

\begin{itemize}
\tightlist
\item
  Identify the response variable:
\end{itemize}

\vspace{0.25in}

\subsubsection*{Relationships between variables}\label{relationships-between-variables}
\addcontentsline{toc}{subsubsection}{Relationships between variables}

\setstretch{1.5}

\begin{itemize}
\item
  Association: the \_\_\_\_\_\_\_\_\_\_\_\_\_ between variables create a pattern; knowing something about one variable tells us about the other.

  \begin{itemize}
  \item
    Positive association: as one variable \_\_\_\_\_\_\_\_\_\_\_\_\_, the other tends to \_\_\_\_\_\_\_\_\_\_\_\_\_\_\_ also.
  \item
    Negative association: as one variable \_\_\_\_\_\_\_\_\_\_\_\_\_, the other tends to \_\_\_\_\_\_\_\_\_\_\_\_\_.
  \end{itemize}
\item
  Independent: no clear pattern can be seen between the \_\_\_\_\_\_\_\_\_\_.
\end{itemize}

\setstretch{1}

\subsection{Concept Check}\label{concept-check}

Be prepared for group discussion in the next class. One member from the table should write the answers to the following on the whiteboard.

\begin{enumerate}
\def\labelenumi{\arabic{enumi}.}
\tightlist
\item
  What is the explanatory variable in the Male Physicians study?
\end{enumerate}

\vspace{0.4in}

\begin{enumerate}
\def\labelenumi{\arabic{enumi}.}
\setcounter{enumi}{1}
\tightlist
\item
  What is the response variable in the Male Physicians study?
\end{enumerate}

\vspace{0.4in}

\setstretch{1}

\subsection*{Sampling Methods: Video 2.1}\label{sampling-methods-video-2.1}
\addcontentsline{toc}{subsection}{Sampling Methods: Video 2.1}

\setstretch{1.5}

The method used to collect data will impact

\begin{itemize}
\item
  Target population: all \_\_\_\_\_\_\_\_\_\_\_\_\_\_\_ or \_\_\_\_\_\_\_\_\_\_\_\_\_\_ of interest
\item
  Sample:\_\_\_\_\_\_\_\_\_\_\_\_\_\_\_\_ or \_\_\_\_\_\_\_\_\_\_\_\_\_\_\_\_ from which data is collected
\end{itemize}

\setstretch{1}

Example: Many high schools moved to partial or fully online schooling in Spring of 2020. Did students who graduated in 2020 tend to have a lower GPA during freshman year of college than the previous class of college freshmen? A nationally representative sample of 1000 college students who were freshmen in AY19-20 and 1000 college students who were freshmen in AY20-21 was taken to answer this question.

\begin{itemize}
\tightlist
\item
  What is the target population?
\end{itemize}

\vspace{0.2in}

\begin{itemize}
\tightlist
\item
  What is the sample?
\end{itemize}

\vspace{0.2in}

\subsubsection*{Good vs.~bad sampling}\label{good-vs.-bad-sampling}
\addcontentsline{toc}{subsubsection}{Good vs.~bad sampling}

\setstretch{1.5}

GOAL: to have a sample that is \_\_\_\_\_\_\_\_\_\_\_\_\_\_\_ of the
\_\_\_\_\_\_\_\_\_\_\_\_\_\_ \_\_\_\_\_\_\_\_\_\_\_\_\_\_\_ on the variable(s) of interest

\setstretch{1}

\begin{itemize}
\tightlist
\item
  Unbiased sample methods:
\end{itemize}

\vspace{0.5in}

\rgi \rgi Simple random sample

\begin{itemize}
\tightlist
\item
  Biased sampling method:
\end{itemize}

\vspace{0.5in}

\subsection*{Types of Sampling Bias}\label{types-of-sampling-bias}
\addcontentsline{toc}{subsection}{Types of Sampling Bias}

\begin{itemize}
\tightlist
\item
  Selection bias:
\end{itemize}

\vspace{0.5in}

Example of Selection Bias: Newspaper article from 1936 reported that Landon won the presidential election over Roosevelt based on a poll of 10 million voters. Roosevelt was the actual winner. What was wrong with this poll? Poll was completed using a telephone survey and not all people in 1936 had a telephone. Only a certain subset of the population owned a telephone so this subset was over-represented in the telephone survey. The results of the study, showing that Landon would win, did not represent the target population of all US voters.

\begin{itemize}
\tightlist
\item
  Non-response bias:
\end{itemize}

\vspace{0.5in}

\begin{itemize}
\tightlist
\item
  To calculate the non-response rate:
\end{itemize}

\[\frac{\text{number of people who do not respond}}{\text{total number of people selected for the sample}}\times 100\%\]

\begin{itemize}
\tightlist
\item
  For non-response bias to occur must first select people to participate and then they choose not to.
\end{itemize}

Example of Non-response bias: A company randomly selects buyers to complete a review of an online purchase but some choose not to respond.

\begin{itemize}
\tightlist
\item
  Response bias:
\end{itemize}

\vspace{0.5in}

Example of Response Bias: Police officer pulls you over and asks if you have been drinking. Expect people to say no, whether they have been drinking or not.

\begin{itemize}
\tightlist
\item
  Need to be able to predict how people will respond.
\end{itemize}

Words of caution:

\begin{itemize}
\tightlist
\item
  Convenience samples: gathering data for those who are easily
  accessible; online polls
\end{itemize}

\setstretch{1.5}

\rgi \rgi Selection bias?

\rgi \rgi Non-response bias?

\rgi \rgi Response bias?

\begin{itemize}
\tightlist
\item
  Random sampling reduces \_\_\_\_\_\_\_\_\_\_\_\_\_\_\_\_\_ bias, but
  has no impact on \_\_\_\_\_\_\_\_\_\_\_\_\_\_\_\_ or \_\_\_\_\_\_\_\_\_\_\_\_\_\_ bias.
\end{itemize}

\setstretch{1}

\newpage

\subsubsection*{Video Example}\label{video-example}
\addcontentsline{toc}{subsubsection}{Video Example}

A radio talk show asks people to phone in their views on whether the United States should pay off its debt to the United Nations.

\begin{itemize}
\tightlist
\item
  Selection?
\end{itemize}

\vspace{0.25in}

\begin{itemize}
\tightlist
\item
  Non-response?
\end{itemize}

\vspace{0.25in}

\begin{itemize}
\tightlist
\item
  Response?
\end{itemize}

\vspace{0.25in}

The Wall Street Journal plans to make a prediction for the US presidential election based on a survey of its readers and plans to follow-up to ensure everyone responds.

\begin{itemize}
\tightlist
\item
  Selection?
\end{itemize}

\vspace{0.25in}

\begin{itemize}
\tightlist
\item
  Non-response?
\end{itemize}

\vspace{0.25in}

\begin{itemize}
\tightlist
\item
  Response?
\end{itemize}

\vspace{0.25in}

A police detective interested in determining the extent of drug use by high school students, randomly selects a sample of high school students and interviews each one about any illegal drug use by the student during the past year.

\begin{itemize}
\tightlist
\item
  Selection?
\end{itemize}

\vspace{0.25in}

\begin{itemize}
\tightlist
\item
  Non-response?
\end{itemize}

\vspace{0.25in}

\begin{itemize}
\tightlist
\item
  Response?
\end{itemize}

\vspace{0.25in}

\subsection{Concept Check}\label{concept-check-1}

Be prepared for group discussion in the next class. One member from the table should write the answers to the following on the whiteboard.

\begin{enumerate}
\def\labelenumi{\arabic{enumi}.}
\tightlist
\item
  What are the two types of variables?
\end{enumerate}

\vspace{0.3in}

\begin{enumerate}
\def\labelenumi{\arabic{enumi}.}
\setcounter{enumi}{1}
\item
  Purpose of random selection:
  \vspace{0.6in}
\item
  Types of sampling bias:
\end{enumerate}

\vspace{0.5in}

\newpage

\section{Activity 2: Intro to Data Analysis and Sampling Bias}\label{activity-2-intro-to-data-analysis-and-sampling-bias}

\setstretch{1}

\subsection{Learning outcomes}\label{learning-outcomes-1}

\begin{itemize}
\item
  Identify observational units, variables, and variable types in a statistical study.
\item
  Creating a data set
\item
  Identify biased sampling methods.
\end{itemize}

\subsection{Terminology review}\label{terminology-review-1}

Statistics is the study of how best to collect, analyze, and draw conclusions from data. This week in class you will be introduced to the following terms:

\begin{itemize}
\item
  Observational units or cases
\item
  Variables: categorical or quantitative
\end{itemize}

For more on these concepts, read Chapter 1 in the textbook.

\subsubsection*{Further analysis of class data set}\label{further-analysis-of-class-data-set}
\addcontentsline{toc}{subsubsection}{Further analysis of class data set}

\begin{enumerate}
\def\labelenumi{\arabic{enumi}.}
\tightlist
\item
  What are the observational units or cases for the data collected in class on day 1?
\end{enumerate}

\vspace{0.3in}

\begin{enumerate}
\def\labelenumi{\arabic{enumi}.}
\setcounter{enumi}{1}
\tightlist
\item
  How many observations are reported in the data set? This is the \textbf{sample size}.
\end{enumerate}

\vspace{0.3in}

\begin{enumerate}
\def\labelenumi{\arabic{enumi}.}
\setcounter{enumi}{2}
\tightlist
\item
  The header for each column in the data set describes each variable measured on the observational unit. For each column of data, fill in the following table identifying the type of each variable, and if the variable is categorical whether the variables is binary and if the variable is quantitative the units of measure used.
\end{enumerate}

\begin{center}
\begin{tabular}{|l|p{1.5in}|p{0.5in}|p{0.5in}|} \hline
Column & Type of Variable & Binary? & Units? \\ \hline
Major & & &\\
& & & \\ \hline
Residency & & & \\
& & & \\ \hline
Num Credits & & & \\
& & & \\ \hline
Dominant hand & & & \\
& & & \\ \hline
Hand Span & & & \\
& & & \\ \hline
Grip strength dominant hand & & & \\
& & & \\ \hline
Grip strength non-dominant hand & & & \\
& & & \\ \hline
\end{tabular}
\end{center}

\newpage

\begin{enumerate}
\def\labelenumi{\arabic{enumi}.}
\setcounter{enumi}{3}
\tightlist
\item
  Review the completed data set with your table. Remember that when creating a data set for use in R it is important to use single words or an underscore between words. Each outcome must be written the same way each time to have consistency between responses. Do not include units of measure in the data set when reporting numerical values. Write down some issues found with the created class data set.
\end{enumerate}

\vspace{2.5in}

\subsection{Sampling Methods}\label{sampling-methods}

Discuss the following questions with your team.

\begin{enumerate}
\def\labelenumi{\arabic{enumi}.}
\setcounter{enumi}{4}
\tightlist
\item
  Describe how the students were selected for this study.
\end{enumerate}

\vspace{1in}

\begin{enumerate}
\def\labelenumi{\arabic{enumi}.}
\setcounter{enumi}{5}
\tightlist
\item
  Can we generalize the results of this study back to all University students? All MSU students? All Stat 216 students?
\end{enumerate}

\vspace{1in}

\begin{enumerate}
\def\labelenumi{\arabic{enumi}.}
\setcounter{enumi}{6}
\tightlist
\item
  Explain your answer to question 6.
\end{enumerate}

\newpage

\subsubsection*{Types of bias}\label{types-of-bias}
\addcontentsline{toc}{subsubsection}{Types of bias}

\begin{enumerate}
\def\labelenumi{\arabic{enumi}.}
\setcounter{enumi}{7}
\item
  To determine if the proportion of out-of-state undergraduate students at Montana State University has increased in the last 10 years, a statistics instructor sent an email survey to 500 randomly selected current undergraduate students. One of the questions on the survey asked whether they had in-state or out-of-state residency. She only received 378 responses.
  \vspace{0.1in}

  Sample size:
  \vspace{0.3in}

  Observational units sampled:
  \vspace{0.3in}

  Target population:
  \vspace{0.3in}

  Justify why there is non-response bias in this study.
  \vspace{0.5in}

  Variables measured and their types:
\end{enumerate}

\vspace{0.5in}

\begin{enumerate}
\def\labelenumi{\arabic{enumi}.}
\setcounter{enumi}{8}
\item
  A television station is interested in predicting whether or not local voters will pass a referendum to legalize marijuana for adult. The TV station asks its viewers to phone in and indicate whether they are in favor or opposed to the referendum. Of the 2241 viewers who phoned in, forty-five percent were opposed to legalizing marijuana.
  \vspace{0.1in}

  Sample size:
  \vspace{0.3in}

  Observational units sampled:
  \vspace{0.3in}

  Target population:
  \vspace{0.3in}

  Justify why there is selection bias in this study.
  \vspace{0.5in}

  Variables measured and their types:
\end{enumerate}

\vspace{0.5in}

\newpage

\begin{enumerate}
\def\labelenumi{\arabic{enumi}.}
\setcounter{enumi}{9}
\item
  To gauge the interest of Bozeman City Voters in a new swimming pool, a local organization stood outside of the Bogart Pool in Bozeman, MT, during open hours. One of the questions they asked was, ``Since the Bogart Pool is in such bad repair, don't you agree that the city should fund a new pool?''
  \vspace{0.1in}

  Sample size:
  \vspace{0.3in}

  Observational units sampled:
  \vspace{0.3in}

  Target population:
  \vspace{0.3in}

  Justify why there is response bias in this study.
  \vspace{0.5in}

  Justify why there is selection bias in this study.
  \vspace{0.5in}

  Variables measured and their types:
\end{enumerate}

\vspace{0.5in}

\subsection{Take-home messages}\label{take-home-messages-1}

\begin{enumerate}
\def\labelenumi{\arabic{enumi}.}
\item
  There are two types of variables: categorical (groups) and quantitative (numerical measures).
\item
  We will learn more about summarizing variable later in the semester. Categorical variables are summarized by calculating a proportion from the data and quantitative variables are summarized by finding the mean and the standard deviation.
\item
  There are three types of bias to be aware of when designing a sampling method: selection bias, non-response bias, and response bias.
\end{enumerate}

\subsection{Additional notes}\label{additional-notes-1}

Use this space to summarize your thoughts and take additional notes on today's activity and material covered, and to write down the names and contact information of your teammates.

\newpage

\section{Activity 3: American Indian Address}\label{activity-3-american-indian-address}

\setstretch{1}

\subsection{Learning outcomes}\label{learning-outcomes-2}

\begin{itemize}
\item
  Explain why a sampling method is unbiased or biased.
\item
  Identify biased sampling methods.
\item
  Explain the purpose of random selection and its effect on scope of inference.
\end{itemize}

\subsection{Terminology review}\label{terminology-review-2}

In this activity, we will examine unbiased and biased methods of sampling. Some terms covered in this activity are:

\begin{itemize}
\item
  Random sample
\item
  Unbiased vs biased methods of selection
\item
  Generalization
\end{itemize}

To review these concepts, see Chapter 2 in the textbook.

\subsection{Class Preparation}\label{class-preparation}

Prior to the next class, complete questions 1--3.

\subsection{American Indian Address}\label{american-indian-address}

For this activity, you will read a speech given by Jim Becenti, a member of the Navajo American Indian tribe, who spoke about the employment problems his people faced at an Office of Indian Affairs meeting in Phoenix, Arizona, on January 30, 1947 (Moquin and Van Doren 1973). His speech is below:

\textbf{It is hard for us to go outside the reservation where we meet strangers. I have been off the reservation ever since I was sixteen. Today I am sorry I quit the Santa Fe {[}Railroad{]}. I worked for them in 1912--13. You are enjoying life, liberty, and happiness on the soil the American Indian had, so it is your responsibility to give us a hand, brother. Take us out of distress. I have never been to vocational school. I have very little education. I look at the white man who is a skilled laborer. When I was a young man I worked for a man in Gallup as a carpenter's helper. He treated me as his own brother. I used his tools. Then he took his tools and gave me a list of tools I should buy and I started carpentering just from what I had seen. We have no alphabetical language.}

\textbf{We see things with our eyes and can always remember it. I urge that we help my people to progress in skilled labor as well as common labor. The hope of my people is to change our ways and means in certain directions, so they can help you someday as taxpayers. If not, as you are going now, you will be burdened the rest of your life. The hope of my people is that you will continue to help so that we will be all over the United States and have a hand with you, and give us a brotherly hand so we will be happy as you are. Our reservation is awful small. We did not know the capacity of the range until the white man come and say ``you raise too much sheep, got to go somewhere else,'' resulting in reduction to a skeleton where the Indians can't make a living on it. For eighty years we have been confused by the general public, and what is the condition of the Navajo today? Starvation! We are starving for education. Education is the main thing and the only thing that is going to make us able to compete with you great men here talking to us.}

\subsubsection*{By eye selection}\label{by-eye-selection}
\addcontentsline{toc}{subsubsection}{By eye selection}

\begin{enumerate}
\def\labelenumi{\arabic{enumi}.}
\tightlist
\item
  Circle ten words in Jim Becenti's speech which are a representative sample of the length of words in the entire text. Describe your method for selecting this sample.
\end{enumerate}

\vspace{0.3in}

\begin{enumerate}
\def\labelenumi{\arabic{enumi}.}
\setcounter{enumi}{1}
\tightlist
\item
  Fill in the table below with your selected words from the previous question and the length of each word (number of letters/digits in the word):
  \vspace{1mm}
\end{enumerate}

\begin{center}
\begin{tabular}{|l|p{3in}|p{1in}|} \hline
Observation & Word & Length  \\ \hline
1 & & \\ 
& & \\ \hline
2 & & \\ 
& & \\ \hline
3 & & \\ 
& & \\ \hline
4 & & \\ 
& & \\ \hline
5 & & \\ 
& & \\ \hline
6 & & \\ 
& & \\ \hline
7 & & \\
& & \\ \hline
8 & & \\ 
& & \\ \hline
9 & & \\ 
& & \\ \hline
10 & & \\ 
& & \\ \hline
\end{tabular}
\end{center}

\begin{enumerate}
\def\labelenumi{\arabic{enumi}.}
\setcounter{enumi}{2}
\tightlist
\item
  Calculate the mean (average) word length in your selected sample. Is this value a parameter or a statistic?\\
  \vspace{0.3in}
\end{enumerate}

\subsection{Class Activity}\label{class-activity}

\begin{enumerate}
\def\labelenumi{\arabic{enumi}.}
\tightlist
\item
  Report your mean word length in the Google sheet. Your instructor will create a visualization of the distribution of results generated by your class. Draw a picture of the plot here. Include a descriptive \(x\)-axis label. Report the mean of the sample mean word lengths.
\end{enumerate}

\vspace{2in}

\newpage

\begin{enumerate}
\def\labelenumi{\arabic{enumi}.}
\setcounter{enumi}{1}
\tightlist
\item
  Calculate how far your sample mean from Q1 is from the mean of the sample mean word lengths. Report this difference in the Google sheet. Your instructor will show you how to calculate the standard deviation.
\end{enumerate}

\vspace{1in}

~~~Interpret the standard deviation of the statistics in context of the problem.

\vspace{0.8in}

The plot created in the question 1 is a sampling distribution of statistics. This sampling distribution plots the mean word length from many samples taken from the population of words.

\begin{enumerate}
\def\labelenumi{\arabic{enumi}.}
\setcounter{enumi}{2}
\item
  Based on the plot and summary statistics of the sample mean word lengths, what is your best guess for the average word length of the population of all 359 words in the speech?
  \vspace{0.3in}
\item
  The true mean word length of the population of all 359 words in the speech is 3.95 letters. Is this value a parameter or a statistic?\\
  \vspace{0.2in}

  Where does the value of 3.95 fall in the plot given? Near the center of the distribution? In the tails of the distribution?
  \vspace{0.3in}
\item
  If the class samples were truly representative of the population of words, what proportion of sample means in the sampling distribution would you expect to be below 3.95?
  \vspace{0.5in}
\item
  Using the graph in Q1, estimate the proportion of students' computed sample means that were lower than the true mean of 3.95 letters?
  \vspace{0.5in}
\item
  Based on your answers to questions 5 and 6, would you say the sampling method used by the class is biased or unbiased? Justify your answer.\\
  \vspace{0.5in}
\item
  If the sampling method is biased, what type of sampling bias (selection, response, non-response) is present? What is the direction of the bias, i.e., does the method tend to overestimate or underestimate the population mean word length?
  \vspace{0.5in}
\item
  Should we use results from our ``by eye'' samples to make a statement about the word length in the population of words in Becenti's address? Why or why not?
  \vspace{0.6in}
\end{enumerate}

\subsubsection*{Random selection}\label{random-selection}
\addcontentsline{toc}{subsubsection}{Random selection}

Suppose instead of attempting to select a representative sample by eye (which did not work), each student used a random number generator to select a simple random sample of 10 words. A \textbf{simple random sample} relies on a random mechanism to choose a sample, without replacement, from the population, such that every sample of size 10 is equally likely to be chosen.

To use a random number generator to select a simple random sample, you first need a numbered list of all the words in the population, called a \textbf{sampling frame}. You can then generate 10 random numbers from the numbers 1 to 359 (the number of words in the population), and the chosen random numbers correspond to the chosen words in your sample.

\begin{enumerate}
\def\labelenumi{\arabic{enumi}.}
\setcounter{enumi}{9}
\tightlist
\item
  Use the random number generator at \url{https://istats.shinyapps.io/RandomNumbers/} to select a simple random sample from the population of all 359 words in the speech.
\end{enumerate}

\begin{itemize}
\item
  Set ``Choose Minimum'' to 1 and ``Choose Maximum'' to 359 to represent the 359 words in the population (the sampling frame).
\item
  Set ``How many numbers do you want to generate?'' to 10 and ensure the ``No'' option is selected under ``Sample with Replacement?''
\item
  Click ``Generate''.
\end{itemize}

Fill in the table below with the random numbers selected and use the Becenti.csv data file found on D2L to determine each number's corresponding word and word length (number of letters/digits in the word):

\begin{center}
\begin{tabular}{|l|l|p{1in}|} \hline
Observation & Number & Length  \\ \hline
1 & & \\ 
& & \\ \hline
2 & & \\ 
& & \\ \hline
3 & & \\ 
& & \\ \hline
4 & & \\ 
& & \\ \hline
5 & & \\ 
& & \\ \hline
6 & & \\ 
& & \\ \hline
7 & & \\
& & \\ \hline
8 & & \\ 
& & \\ \hline
9 & &\\ 
& & \\ \hline
10 & & \\ 
& & \\ \hline
\end{tabular}
\end{center}

\newpage

\begin{enumerate}
\def\labelenumi{\arabic{enumi}.}
\setcounter{enumi}{10}
\item
  Calculate the mean word length in your selected sample in question 10. Is this value a parameter or a statistic?
  \vspace{0.3in}
\item
  Report your mean word length in the Google sheet. Your instructor will create a visualization of the distribution of results generated by your class. Draw a picture of the plot here. Include a descriptive \(x\)-axis label. Report the mean and standard deviation of the data.
\end{enumerate}

\vspace{2.25in}

\begin{enumerate}
\def\labelenumi{\arabic{enumi}.}
\setcounter{enumi}{12}
\item
  Where does the value 3.95, the true mean word length, fall in the distribution given? Near the center of the distribution? In the tails of the distribution? Circle this value on the provided distribution.
  \vspace{0.3in}
\item
  How does the plot from Q10 compare to the plot generated in Q1?
\end{enumerate}

\rgi Is the shape similar?\\
\vspace{0.2in}

\rgi Is the range (smallest to largest values) similar?

\vspace{0.2in}

\rgi Is the mean of the distribution similar?

\vspace{0.2in}

\rgi Why didn't everyone get the same sample mean?
\vspace{0.4in}

\newpage

One set of randomly generated sample mean word lengths from a single class may not be large enough to visualize the distribution results. Let's have a computer generate 1,000 sample mean word lengths for us.

The following plot illustrates a sampling distribution of 1000 samples of size 10 selected at random from the sample.

\begin{center}\includegraphics[width=0.75\linewidth]{images/bencenti_sampling10} \end{center}

\begin{enumerate}
\def\labelenumi{\arabic{enumi}.}
\setcounter{enumi}{14}
\item
  What is the center value (mean) of the distribution displayed above?
  \vspace{0.3in}
\item
  Explain why the sampling method of using a random number generator to generate a sample is a ``better'' method than choosing 10 words ``by eye''.
  \vspace{0.8in}
\item
  Is random selection an unbiased method of selection? Explain your answer. Be sure to reference your plot from before Q15.
  \vspace{0.5in}
\end{enumerate}

\subsection*{Effect of sample size}\label{effect-of-sample-size}
\addcontentsline{toc}{subsection}{Effect of sample size}

We will now consider the impact of sample size.

\begin{enumerate}
\def\labelenumi{\arabic{enumi}.}
\setcounter{enumi}{17}
\tightlist
\item
  First, consider if each student had selected 30 words, instead of 10, by eye. Do you think this would make the plot from the previous activity centered on 3.95 (the true mean word length)? Explain your answer.
  \vspace{0.4in}
\end{enumerate}

\newpage

Now we will select 30 words instead of 10 words at random. The following plot illustrates a sampling distribution of 1000 samples of size 30 selected at random from the sample.

\begin{center}\includegraphics[width=0.75\linewidth]{images/bencenti_sampling30} \end{center}

\begin{enumerate}
\def\labelenumi{\arabic{enumi}.}
\setcounter{enumi}{18}
\tightlist
\item
  Compare the distribution displayed before question 15 to the one shown above.
\end{enumerate}

\rgi Is the shape similar?\\
\vspace{0.2in}

\rgi Is the range (smallest to largest values) similar?

\vspace{0.2in}

\rgi Is the mean of the distribution similar?

\vspace{0.2in}

\begin{enumerate}
\def\labelenumi{\arabic{enumi}.}
\setcounter{enumi}{19}
\item
  Compare the values of the standard deviation of the plots before question 15 and before question 19. Which plot shows the smallest standard deviation?
  \vspace{0.4in}
\item
  Using the evidence from your simulations, answer the following research questions:
\end{enumerate}

\rgi Does changing the sample size impact whether the sample estimates are unbiased? Explain your answer.
\vspace{0.5in}

\rgi Does changing the sample size impact the variability (spread) of sample estimates? Explain your answer
\vspace{0.5in}

\begin{enumerate}
\def\labelenumi{\arabic{enumi}.}
\setcounter{enumi}{21}
\tightlist
\item
  What is the purpose of random selection of a sample from the population?
\end{enumerate}

\vspace{0.8in}

\subsection{Take-home messages}\label{take-home-messages-2}

\begin{enumerate}
\def\labelenumi{\arabic{enumi}.}
\item
  When we use a biased method of selection, we will over or underestimate the parameter.
\item
  If the sampling method is biased, inferences made about the population based on a sample estimate will not be valid.
\item
  Random selection is an unbiased method of selection.
\item
  To determine if a sampling method is biased or unbiased, we compare the distribution of the estimates to the true value. We want our estimate to be on target or unbiased. When using unbiased methods of selection, the mean of the distribution matches or is very similar to our true parameter.
\item
  Random selection eliminates selection bias. However, random selection will not eliminate response or non-response bias.
\item
  The larger the sample size, the more similar (less variable) the statistics will be from different samples.
\item
  Sample size has no impact on whether a \emph{sampling method} is biased or not. Taking a larger sample using a biased method will still result in a sample that is not representative of the population.
\end{enumerate}

\subsection{Additional notes}\label{additional-notes-2}

Use this space to summarize your thoughts and take additional notes on today's activity and material covered.

\newpage

\chapter{Probability}\label{probability}

\section{Vocabulary Review and Key Topics}\label{vocabulary-review-and-key-topics-1}

\begin{itemize}
\item
  \textbf{Probability} (of an event): the long-run proportion of times the event would occur if the random process were repeated indefinitely (under identical conditions)
\item
  \textbf{Conditional probability} (of an event \emph{given} another event): probability of an event calculated dependent on another event having occurred
\item
  \textbf{Probability notation}:

  \begin{itemize}
  \item
    \(P(A)\): the probability of event A

    \begin{itemize}
    \tightlist
    \item
      This is the probability of a single event, \emph{unconditional} probability calculated out of the overall population
    \end{itemize}
  \item
    \(P(A^C)\): the probability of the \textbf{complement} of event A, or ``A complement''

    \begin{itemize}
    \item
      This is the probability of the opposite of event A, or ``not A''
    \item
      \(P(A^C) = 1 - P(A)\)
    \end{itemize}
  \item
    \(P(A\text{ and }B)\): the probability of event A and B

    \begin{itemize}
    \tightlist
    \item
      The is the probability of an ``and'' event, \emph{unconditional} probability calculated out of the overall population
    \end{itemize}
  \item
    \(P(A|B)\): the probability of event A given (conditional on) event B

    \begin{itemize}
    \tightlist
    \item
      This is a \emph{conditional} probability calculated out of the total population for which event B occurred
    \end{itemize}
  \end{itemize}
\end{itemize}

\newpage

\section{Video Notes: Probability}\label{video-notes-probability}

Read Chapters 22 in the course textbook. Use the following videos to complete the video notes for Module 12.

\subsection{Course Videos}\label{course-videos-1}

\begin{itemize}
\tightlist
\item
  Chapter23
\end{itemize}

\setstretch{1}

\subsection*{Probability}\label{probability-1}
\addcontentsline{toc}{subsection}{Probability}

Example: Two variables were collected on a random sample of people who had ever been married; whether a person had ever smoked and whether a person had ever been divorced. The data are displayed in the following table. This survey was based on a random sample in the United States in the early 1990s, so the data should be representative of the adult population who had ever been married at that time.

\begin{itemize}
\item
  Let event D be a person has gone through a divorce
\item
  Let event S be a person smokes
\end{itemize}

\begin{center}
\begin{tabular}{|c|c|c|c|} \hline
\hspace{0.8in} & \hspace{0.35in} Has divorced \hspace{.35in} & \hspace{0.35in} Has never divorced  \hspace{0.35in} & \hspace{0.3in} Total \hspace{0.3in} \\ 
& & & \\ \hline
Smokes & 238 & 247 & 485 \\ 
& & & \\ \hline
Does not smoke & 374 & 810 & 1184 \\ 
& & & \\ \hline
Total & 612 & 1057 & 1669 \\ 
& & & \\ \hline
\end{tabular}
\end{center}
\vspace{.1in}

\begin{itemize}
\tightlist
\item
  What is the approximate probability that the person smoked?
\end{itemize}

\vspace{0.5in}

\begin{itemize}
\tightlist
\item
  What is the approximate probability that the person had ever been divorced?
\end{itemize}

\vspace{0.5in}

\begin{itemize}
\tightlist
\item
  Given that the person had been divorced, what is the probability that he or she smoked?
\end{itemize}

\vspace{0.5in}

\begin{itemize}
\tightlist
\item
  Given that the person smoked, what is the probability that he or she had been divorced?
\end{itemize}

\vspace{0.5in}

\setstretch{1.5}

\begin{itemize}
\item
  Event: something that could occur, something we want to find the probability of

  \begin{itemize}
  \tightlist
  \item
    Getting a four when rolling a fair die
  \end{itemize}
\item
  Complement: opposite of the event

  \begin{itemize}
  \tightlist
  \item
    Getting any value but a four when rolling a fair die
  \end{itemize}
\item
  The probability of an event is the \_\_\_\_\_\_\_\_\_\_\_\_\_\_ proportion of times the event would occur if the \_\_\_\_\_\_\_\_\_\_\_\_\_\_\_\_ process were repeated indefinitely.

  \begin{itemize}
  \tightlist
  \item
    For example, the probability of getting a four when rolling a fair die is \_\_\_\_\_\_\_\_\_\_\_.
  \end{itemize}
\item
  Unconditional probabilities

  \begin{itemize}
  \item
    An \_\_\_\_\_\_\_\_\_\_\_\_\_\_\_\_\_\_\_\_probability is calculated from the entire population not\_\_\_\_\_\_\_\_\_\_\_\_\_\_\_\_\_\_\_\_\_\_\_\_\_\_\_\_\_
    on the occurrence of another event.
  \item
    Examples:

    \begin{itemize}
    \item
      The probability of a single event

      \begin{itemize}
      \tightlist
      \item
        The probability a selected Stat 216 student is a computer science major.
      \end{itemize}
    \item
      An ``And'' probability

      \begin{itemize}
      \tightlist
      \item
        The probability a selected Stat 216 student is a computer science major and a freshman.
      \end{itemize}
    \end{itemize}
  \end{itemize}
\item
  Conditional probabilities

  \begin{itemize}
  \item
    A \_\_\_\_\_\_\_\_\_\_\_\_\_\_\_\_\_\_\_\_\_ probability is calculated
    \_\_\_\_\_\_\_\_\_\_\_\_\_\_\_\_\_\_\_\_\_\_\_ on the occurrence of another event.
  \item
    Examples:

    \begin{itemize}
    \item
      The probability of event A given B

      \begin{itemize}
      \tightlist
      \item
        The probability a selected freshman Stat 216 student is a computer science major.
      \end{itemize}
    \item
      The probability of event B given A

      \begin{itemize}
      \tightlist
      \item
        The probability a selected computer science Stat 216 student is a freshman
      \end{itemize}
    \end{itemize}
  \end{itemize}
\end{itemize}

\setstretch{1}

\begin{itemize}
\item
  Let event D be a person has gone through a divorce
\item
  Let event S be a person smokes
\end{itemize}

\begin{center}
\begin{tabular}{|c|c|c|c|} \hline
\hspace{0.8in} & \hspace{0.35in} Has divorced \hspace{.35in} & \hspace{0.35in} Has never divorced  \hspace{0.35in} & \hspace{0.3in} Total \hspace{0.3in} \\ 
& & & \\ \hline
Smokes & 238 & 247 & 485 \\ 
& & & \\ \hline
Does not smoke & 374 & 810 & 1184 \\ 
& & & \\ \hline
Total & 612 & 1057 & 1669 \\ 
& & & \\ \hline
\end{tabular}
\end{center}
\vspace{.1in}

Calculate and interpret each of the following:

\setstretch{1.5}

\begin{itemize}
\tightlist
\item
  \(P(S^C)=\)
\end{itemize}

\vspace{0.6in}

\begin{itemize}
\tightlist
\item
  \(P(D^C|S^C)=\)
\end{itemize}

\vspace{0.6in}

\setstretch{1}

\setstretch{1}

\subsubsection*{Creating a hypothetical two-way table}\label{creating-a-hypothetical-two-way-table}
\addcontentsline{toc}{subsubsection}{Creating a hypothetical two-way table}

Steps:

\begin{itemize}
\item
  Start with a large number like 100000.
\item
  Then use the unconditional probabilities to fill in the row or column totals.
\item
  Now use the conditional probabilities to begin filling in the interior cells.
\item
  Use subtraction to find the remaining interior cells.
\item
  Add the column values together for each row to find the row totals.
\item
  Add the row values together for each column to find the column totals.
\end{itemize}

Example: An airline has noticed that 30\% of passengers pre-pay for checked bags at the time the ticket is purchased. The no-show rate among customers that pre-pay for checked bags is 5\%, compared to 15\% among customers that do not pre-pay for checked bags.

\begin{itemize}
\tightlist
\item
  Let event B = customer pre-pays for checked bag
\item
  Let event N = customer no shows
\end{itemize}

\setstretch{1.5}

Start by identifying the probability notation for each value given.

\begin{itemize}
\tightlist
\item
  0.30 =
\end{itemize}

\vspace{0.1in}

\begin{itemize}
\tightlist
\item
  0.05 =
\end{itemize}

\vspace{0.1in}

\begin{itemize}
\tightlist
\item
  0.15 =
\end{itemize}

\vspace{0.1in}

\setstretch{1}

\begin{center}
\begin{tabular}{|c|c|c|c|} \hline
\hspace{0.8in} & \hspace{0.35in} $B$ \hspace{.35in} & \hspace{0.35in} $B^C$  \hspace{0.35in} & \hspace{0.3in} Total \hspace{0.3in} \\ 
& & & \\ \hline
$N$& & & \\ 
& & & \\ \hline
$N^C$& & & \\ 
& & & \\ \hline
Total & & & 100,000 \\ 
& & & \\ \hline
\end{tabular}
\end{center}
\vspace{.1in}

\begin{itemize}
\tightlist
\item
  What is the probability that a randomly selected customer who shows for the flight, pre-purchased checked bags?
\end{itemize}

\vspace{1in}

\subsubsection*{Diagnostic tests}\label{diagnostic-tests}
\addcontentsline{toc}{subsubsection}{Diagnostic tests}

\begin{itemize}
\tightlist
\item
  Sensitivity:
\end{itemize}

\vspace{0.3in}

\begin{itemize}
\tightlist
\item
  Specificity:
\end{itemize}

\vspace{0.3in}

\begin{itemize}
\tightlist
\item
  Prevalence:
\end{itemize}

\vspace{0.3in}

\subsection{Concept Check}\label{concept-check-2}

Be prepared for group discussion in the next class. One member from the table should write the answers to the following on the whiteboard.

\begin{enumerate}
\def\labelenumi{\arabic{enumi}.}
\tightlist
\item
  Calculate and interpret the following: \(P(D^C|S^C)=\).
\end{enumerate}

\vspace{1in}

\begin{enumerate}
\def\labelenumi{\arabic{enumi}.}
\setcounter{enumi}{1}
\tightlist
\item
  What is the probability notation for 0.15 in the airline example?
\end{enumerate}

\vspace{1in}

\newpage

\section{Activity 4: Probability Studies}\label{activity-4-probability-studies}

\setstretch{1}

\subsection{Learning outcomes}\label{learning-outcomes-3}

\begin{itemize}
\item
  Recognize and simulate probabilities as long-run frequencies.
\item
  Construct two-way tables to evaluate conditional probabilities.
\end{itemize}

\subsection{Terminology review}\label{terminology-review-3}

In today's activity, we will cover two-way tables and probability. Some terms covered in this activity are:

\begin{itemize}
\item
  Proportions
\item
  Probability
\item
  Conditional probability
\item
  Two-way tables
\end{itemize}

To review these concepts, see Chapter 23 in the textbook.

\subsection{Overview of probabiliy}\label{overview-of-probabiliy}

The probability of an event is the long-run proportion of times the event would occur if the random process were repeated indefinitely (under identical conditions).

To calculate the probability of an event happening:

\[\text{probability} = \frac{\text{number of ways an event can happen}}{\text{total number of possible outcomes}}\]

For example, to calculate the probability of a coin flip landing on heads; there are only two outcomes (heads or tails) and only one possibility way to land on heads.

\[P(heads) = \frac{1}{2} = 0.5\]
The figure below shows the long-run proportion of times a simulated coin flip lands on heads on the y-axis, and the number of tosses on the x-axis. Notice how the long-run proportion starts converging to 0.5 as the number of tosses increases.

\begin{figure}

{\centering \includegraphics[width=0.65\linewidth]{images/coinsim} 

}

\end{figure}

In today's activity we will discuss the probability of a single event, the probability of an ``and'' event, and the probability of a conditional event.

\subsubsection*{Probability notation}\label{probability-notation}
\addcontentsline{toc}{subsubsection}{Probability notation}

We will use the notation P(event) to represent the probability of an event and use letters to represent events. The following are notations for different probabilities where we are discussing event A and event B:

\begin{itemize}
\item
  \(P(A)\) represents the probability of event A
\item
  \(P(A^C)\) represents the probability of the complement of event A

  \begin{itemize}
  \tightlist
  \item
    \(P(A^C) = 1 - P(A)\)
  \end{itemize}
\item
  \(P(A and B)\) represents the probability of events A and B
\item
  \(P(A|B)\) represents the probability of event A given event B
\item
  \(P(B|A)\) represents the probability of event B given event A
\end{itemize}

\subsubsection{Probability questions}\label{probability-questions}

For the beginning of this activity we will start with discussing the probabilities associated with drawing a card from a standard card deck. In a card deck there are:

\begin{itemize}
\item
  52 cards
\item
  Half are red, half are black
\item
  Four suits: spades, hearts, diamonds, and clubs
\item
  Each suit has 13 cards: cards 2\$-\$10, ace, jack, queen, and king
\item
  Let A represent the event that a card is an ace
\item
  Let B represent the event that a card is red
\end{itemize}

To find the probability of selecting an ace, first start with determining how many aces are possible (four) and how many cards will we select from (total of 52).

\vspace{1in}

Find the probability of selecting a card that is not an ace. This is the complement of event A.

\vspace{1in}

Find the probability of selecting a red ace. There are only two red aces and a total of 52 cards.

\vspace{1in}

Find the probability of selecting an ace given that the card is red. There are two red aces but only \(\frac{52}{2} = 26\) red cards

\vspace{1in}

If a card drawn is an ace, what is the probability the card drawn is red. There are four aces but only two that are red.

\vspace{1in}

\subsection{Calculating probabilities from a two-way table}\label{calculating-probabilities-from-a-two-way-table}

\begin{enumerate}
\def\labelenumi{\arabic{enumi}.}
\tightlist
\item
  In 2014, the website FiveThirtyEight examined the works of Bob Ross to see what trends could be found. They determined that of all the paintings he created, 95\% of them contained at least one ``happy tree.'' Of those works with a happy tree, 43\% contained at least one ``almighty mountain.'' Of the paintings that did not have at least one happy tree, only 10\% contained at least one almighty mountain.
  \vspace{1mm}
\end{enumerate}

Let \(A\) = Bob Ross painting contains a happy tree, and \(B\) = Bob Ross painting contains an almighty mountain
\vspace{0.1in}

\begin{center}
\begin{tabular}{|c|c|c|c|} \hline
\hspace{0.8in} & \hspace{0.25in}  $A$ \hspace{.25in} & \hspace{0.25in} $A^C$ \hspace{0.25in} & \hspace{0.25in} Total \hspace{0.25in} \\ \hline
 $B$ & 40850 & 500 & 41350 \\ \hline
 $B^C$ & 54150 & 4500 & 58650 \\ \hline
Total & 95000 & 5000 & 100000 \\ \hline
\end{tabular}
\end{center}
\vspace{.1in}

\begin{enumerate}
\def\labelenumi{\alph{enumi}.}
\tightlist
\item
  What is the probability that a randomly selected Bob Ross painting contains both a ``happy tree'' and an ``almighty mountain''? Use appropriate probability notation.
\end{enumerate}

\vspace{0.5in}

\begin{enumerate}
\def\labelenumi{\alph{enumi}.}
\setcounter{enumi}{1}
\tightlist
\item
  What is the probability that a selected Bob Ross painting without an ``almighty mountain'' contains a ``happy tree.'' Use appropriate probability notation.
\end{enumerate}

\vspace{0.5in}

\begin{enumerate}
\def\labelenumi{\alph{enumi}.}
\setcounter{enumi}{2}
\tightlist
\item
  What is the probability that a selected Bob Ross painting does not contain a ``happy tree'' given it does not contain an ``almighty mountain''. Use appropriate probability notation.
\end{enumerate}

\vspace{0.55in}

\newpage

\begin{enumerate}
\def\labelenumi{\arabic{enumi}.}
\setcounter{enumi}{1}
\item
  A recent study of population decline of white-tailed deer in Wyoming due to chronic wasting disease (Edmunds 2016) (CWD) reported the prevalence of CWD to be 35.4\%. The survival rate of CWD positive deer was 39.6\% and the survival rate of CWD negative deer was 80.1\%.\\
  \vspace{1mm}

  Let \(A\) = the event a deer has CWD, and \(B\) = the event the deer survived.
  \vspace{0.1in}

  \begin{enumerate}
  \def\labelenumii{\alph{enumii}.}
  \item
    Identify what each numerical value given in the problem represents in probability notation.
    \vspace{.1in}

    0.354 =\\
    \vspace{.1in}

    0.396 =\\
    \vspace{.1in}

    0.801 =\\
    \vspace{.1in}
  \item
    Create a hypothetical two-way table to represent the situation.
  \end{enumerate}
\end{enumerate}

\begin{center}
\begin{tabular}{|c|c|c|c|} \hline
\hspace{0.8in} & \hspace{0.35in} $A$ \hspace{.35in} & \hspace{0.35in} $A^C$  \hspace{0.35in} & \hspace{0.3in} Total \hspace{0.3in} \\ 
& & & \\ \hline
$B$& & & \\ 
& & & \\ \hline
$B^C$& & & \\ 
& & & \\ \hline
Total & & & 100,000 \\ 
& & & \\ \hline
\end{tabular}
\end{center}
\vspace{.1in}

\begin{enumerate}
\def\labelenumi{\alph{enumi}.}
\setcounter{enumi}{2}
\item
  Find \(P(A \mbox{ and } B)\). What does this probability represent in the context of the problem?
  \vspace{.8in}
\item
  Find the probability that a deer that has CWD does not survive. What is the notation used for this probability?
  \vspace{.8in}
\item
  What is the probability that a deer does not survive given they do not have CWD? What is the notation used for this probability?
\end{enumerate}

\newpage

\subsection{Take home messages}\label{take-home-messages-3}

\begin{enumerate}
\def\labelenumi{\arabic{enumi}.}
\item
  Conditional probabilities are calculated dependent on a second variable. In probability notation, the variable following \texttt{\textbar{}} is the variable on which we are conditioning. The denominator used to calculate the probability will be the total for the variable on which we are conditioning.
\item
  When creating a two-way table we typically want to put the explanatory variable on the columns of the table and the response variable on the rows.
\item
  To fill in the two-way table, always start with the unconditional variable in the total row or column and then use the conditional probabilities to fill in the interior cells.
\end{enumerate}

\subsection{Additional notes}\label{additional-notes-3}

Use this space to summarize your thoughts and take additional notes on today's activity and material covered.

\newpage

\section{Activity 5: What's the probability?}\label{activity-5-whats-the-probability}

\setstretch{1}

\subsection{Learning outcomes}\label{learning-outcomes-4}

\begin{itemize}
\item
  Recognize and simulate probabilities as long-run frequencies.
\item
  Construct two-way tables to evaluate conditional probabilities.
\end{itemize}

\subsection{Terminology review}\label{terminology-review-4}

In today's activity, we will cover two-way tables and probability. Some terms covered in this activity are:

\begin{itemize}
\item
  Proportions
\item
  Probability
\item
  Conditional probability
\item
  Two-way tables
\end{itemize}

To review these concepts, see Chapter 23 in the textbook.

\subsection{Probability}\label{probability-2}

\begin{enumerate}
\def\labelenumi{\arabic{enumi}.}
\item
  A dataset was collected on all NBA basketball players from inception of the league. The probability that an NBA player is above average height is 59.7\%. Of NBA players that are above average height, 46.4\% averaged at least four rebounds a game. The probability that an NBA player averages less than four rebounds in a game given they are below average height is 13.3\%.
  \vspace{1mm}

  Let \(A\) = player is above average height, and \(B\) = player averages at least four rebounds a game.
  \vspace{0.1in}
\end{enumerate}

\begin{center}
\begin{tabular}{|c|c|c|c|} \hline
\hspace{0.8in} & \hspace{0.25in} $A$ \hspace{.25in} & \hspace{0.25in} $A^C$ \hspace{0.25in} & \hspace{0.25in} Total \hspace{0.25in} \\ \hline $B$ & 27700.8 & 34940.1 & 62640.9 \\ \hline
 $B^C$ & 31999.2 & 5359.9 & 37359.1 \\ \hline
Total & 59700 & 40300 & 100000 \\ \hline
\end{tabular}
\end{center}
\vspace{.1in}

\begin{enumerate}
\def\labelenumi{\alph{enumi}.}
\tightlist
\item
  What is the probability that a randomly selected NBA player averages at least 4 rebounds a game? Use appropriate probability notation.
\end{enumerate}

\vspace{0.5in}

\begin{enumerate}
\def\labelenumi{\alph{enumi}.}
\setcounter{enumi}{1}
\tightlist
\item
  What is the probability that a randomly selected NBA player is both above average height and averages at least 4 rebounds a game. Use appropriate probability notation.
\end{enumerate}

\vspace{0.5in}

\begin{enumerate}
\def\labelenumi{\alph{enumi}.}
\setcounter{enumi}{2}
\tightlist
\item
  What is the probability that a randomly selected NBA player is not above average height given they do not average at least 4 rebounds a game. Use appropriate probability notation.
\end{enumerate}

\vspace{0.55in}

\newpage

\begin{enumerate}
\def\labelenumi{\arabic{enumi}.}
\setcounter{enumi}{1}
\item
  Since the early 1980s, the rapid antigen detection test (RADT) of group A \emph{streptococci} has been used to detect strep throat. A recent study of the accuracy of this test shows that the \textbf{sensitivity}, the probability of a positive RADT given the person has strep throat, is 86\% in children, while the \textbf{specificity}, the probability of a negative RADT given the person does not have strep throat, is 92\% in children. The \textbf{prevalence}, the probability of having group A strep, is 37\% in children. (Stewart et al. 2014)
  \vspace{1mm}

  Let \(A\) = the event the child has strep throat, and \(B\) = the event the child has a positive RADT.
  \vspace{0.1in}

  \begin{enumerate}
  \def\labelenumii{\alph{enumii}.}
  \item
    Identify what each numerical value given in the problem represents in probability notation.
    \vspace{.1in}

    0.86 =\\
    \vspace{.1in}

    0.92 =\\
    \vspace{.1in}

    0.37 =\\
    \vspace{.1in}
  \item
    Create a hypothetical two-way table to represent the situation.
  \end{enumerate}
\end{enumerate}

\begin{center}
\begin{tabular}{|c|c|c|c|} \hline
\hspace{0.8in} & \hspace{0.35in} $A$ \hspace{.35in} & \hspace{0.35in} $A^C$  \hspace{0.35in} & \hspace{0.3in} Total \hspace{0.3in} \\ 
& & & \\ \hline
$B$& & & \\ 
& & & \\ \hline
$B^C$& & & \\ 
& & & \\ \hline
Total & & & 100,000 \\ 
& & & \\ \hline
\end{tabular}
\end{center}
\vspace{.1in}

\begin{enumerate}
\def\labelenumi{\alph{enumi}.}
\setcounter{enumi}{2}
\item
  Find \(P(A \mbox{ and } B)\). What does this probability represent in the context of the problem?
  \vspace{.8in}
\item
  Find the probability that a child with a positive RADT actually has strep throat. What is the notation used for this probability?
  \vspace{.8in}
\item
  What is the probability that a child does not have strep given that they have a positive RADT? What is the notation used for this probability?
\end{enumerate}

\newpage

\newpage

\subsection{Take home messages}\label{take-home-messages-4}

\begin{enumerate}
\def\labelenumi{\arabic{enumi}.}
\item
  Conditional probabilities are calculated dependent on a second variable. In probability notation, the variable following \texttt{\textbar{}} is the variable on which we are conditioning. The denominator used to calculate the probability will be the total for the variable on which we are conditioning.
\item
  When creating a two-way table we typically want to put the explanatory variable on the columns of the table and the response variable on the rows.
\item
  To fill in the two-way table, always start with the unconditional variable in the total row or column and then use the conditional probabilities to fill in the interior cells.
\end{enumerate}

\subsection{Additional notes}\label{additional-notes-4}

Use this space to summarize your thoughts and take additional notes on today's activity and material covered.

\newpage

\chapter{Exploring Categorical Data: Exploratory Data Analysis and Inference using Simulation-based Methods}\label{exploring-categorical-data-exploratory-data-analysis-and-inference-using-simulation-based-methods}

\section{Vocabulary Review and Key Topics}\label{vocabulary-review-and-key-topics-2}

Review the Golden Ticket posted in the resources at the end of the coursepack for a summary of a single categorical variable.

\begin{itemize}
\item
  \textbf{Summary statistic (point estimate)}: the value of a numerical summary measure computed from \emph{sample} data

  \begin{itemize}
  \item
    Summary measures covered in STAT 216 include: single proportion, difference in proportions, single mean, paired mean difference, difference in means, correlation, and slope of a regression line
  \item
    For a single categorical variable, a proportion is calculated
  \item
    To interpret in context include:

    \begin{itemize}
    \item
      Summary measure (in context)
    \item
      Value of the statistic
    \end{itemize}
  \end{itemize}
\item
  \textbf{Parameter of interest}: a numerical summary measure of the entire \emph{population} in which we are interested

  \begin{itemize}
  \item
    The value of the parameter of interest is unknown (unless we have access to the entire population)
  \item
    To write in context:

    \begin{itemize}
    \item
      Population word (true, long-run, population)
    \item
      Summary measure (depends on the type of data)
    \item
      Context

      \begin{itemize}
      \item
        Observational units
      \item
        Variable(s)
      \end{itemize}
    \end{itemize}
  \end{itemize}
\item
  \textbf{Frequency bar plot}: plots the count (frequency) of observational units in each level of a categorical variable
\item
  \textbf{Relative frequency bar plot}: plots the proportion (relative frequency) of observational units in each level of a categorical variable
\item
  \textbf{Hypothesis testing}: a formal statistical technique for evaluating two competing possibilities about a population: the null hypothesis and alternative hypothesis

  \begin{itemize}
  \item
    When we observe an effect in a sample, we would like to determine if this observed effect represents an actual effect in the population, or whether it was simply due to random chance.
  \item
    A hypothesis test helps us answer the following question about the population: How strong is the \emph{evidence} of an effect?
  \end{itemize}
\item
  \textbf{Null hypothesis}: typically represents a statement of ``no difference'', ``no effect'', or the status quo

  \begin{itemize}
  \tightlist
  \item
    The null hypothesis is what we assume is true when calculating the p-value. Thus, we can never have evidence \emph{for} the null hypothesis---we cannot ``accept'' a null hypothesis---we can only find evidence \emph{against} the null hypothesis if the observed data is very unlikely to have occurred under the assumption that the null hypothesis is true.
  \end{itemize}
\item
  \textbf{Alternative hypothesis}: represents an alternative claim under consideration and is often represented by a range of possible values for the parameter of interest.

  \begin{itemize}
  \tightlist
  \item
    The alternative hypothesis is determined by the research question.
  \end{itemize}
\end{itemize}

Hypotheses:

\[H_0: \pi = \pi_0\]
\[H_A: \pi \left\{
\begin{array}{ll}
< \\
\ne \\
< \\
\end{array}
\right\}
\pi_0 \]

\begin{itemize}
\item
  \textbf{Null Distribution}: a distribution of simulated sample statistics created under the assumption that the null hypothesis is true
\item
  \textbf{Simulation methods to create the null distribution}: a process of using a computer program (e.g., R) to simulate many samples that we would expect based on the null hypothesis.

  R code to use simulation methods for one categorical variable to find the p-value, \texttt{one\_proportion\_test}, is shown below.

\begin{Shaded}
\begin{Highlighting}[]
\FunctionTok{one\_proportion\_test}\NormalTok{(}\AttributeTok{probability\_success =}\NormalTok{ xx, }\CommentTok{\# Null hypothesis value}
      \AttributeTok{sample\_size =}\NormalTok{ xx, }\CommentTok{\# Enter sample size}
      \AttributeTok{number\_repetitions =} \DecValTok{1000}\NormalTok{, }\CommentTok{\# Enter number of simulations}
      \AttributeTok{as\_extreme\_as =}\NormalTok{ xx, }\CommentTok{\# Observed statistic}
      \AttributeTok{direction =} \StringTok{"xx"}\NormalTok{, }\CommentTok{\# Specify direction of alternative hypothesis}
      \AttributeTok{summary\_measure =} \StringTok{"proportion"}\NormalTok{) }\CommentTok{\# Reporting proportion or number of successes?}
\end{Highlighting}
\end{Shaded}
\item
  \textbf{P-value}: the probability of the value of the observed sample statistic or a value more extreme, if the null hypothesis were true

  \begin{itemize}
  \item
    To write in context include:

    \begin{itemize}
    \item
      Statement about probability or proportion of samples
    \item
      Statistic (summary measure and value)
    \item
      Direction of the alternative
    \item
      Null hypothesis (in context)
    \end{itemize}
  \end{itemize}
\item
  \textbf{Strength of evidence}: the p-value indicates the amount of evidence there is against the null hypothesis. The smaller the p-value the more evidence there is against the null hypothesis.
\end{itemize}

\begin{center}\includegraphics[width=0.9\linewidth]{images/soe_gradient_gray} \end{center}

\begin{itemize}
\item
  \textbf{Conclusion} (to a hypothesis test): answers the research question. How much evidence is there in support of the alternative hypothesis?

  \begin{itemize}
  \item
    To write in context include:

    \begin{itemize}
    \item
      Amount of evidence
    \item
      Parameter of interest
    \item
      Direction of the alternative hypothesis
    \end{itemize}
  \end{itemize}
\item
  \textbf{Confidence interval}: an interval estimate for the parameter of interest

  \begin{itemize}
  \item
    A confidence interval helps us answer the following question about the population: How \emph{large} is the effect?
  \item
    To write in context include:

    \begin{itemize}
    \item
      How confident you are (e.g., 90\%, 95\%, 98\%, 99\%)
    \item
      Parameter of interest
    \item
      Calculated interval
    \end{itemize}
  \end{itemize}
\item
  \textbf{Bootstrapping}: creating a simulated sample of the same size as the original sample by sampling with replacement from the original sample
\item
  \textbf{Simulation methods to create the bootstrap distribution}: a process of using a computer program to simulate many bootstrapped samples.

  R code to use simulation methods for one categorical variable to find a confidence interval, \texttt{one\_proportion\_bootstrap\_CI}, is shown below.

\begin{Shaded}
\begin{Highlighting}[]
\FunctionTok{one\_proportion\_bootstrap\_CI}\NormalTok{(}\AttributeTok{sample\_size =}\NormalTok{ xx, }\CommentTok{\# Sample size}
                \AttributeTok{number\_successes =}\NormalTok{ xx, }\CommentTok{\# Observed number of successes}
                \AttributeTok{number\_repetitions =} \DecValTok{1000}\NormalTok{, }\CommentTok{\# Number of bootstrap samples to use}
                \AttributeTok{confidence\_level =} \FloatTok{0.95}\NormalTok{) }\CommentTok{\# Confidence level as a decimal}
\end{Highlighting}
\end{Shaded}
\item
  \textbf{Percentile method}: process to find the confidence interval from the bootstrap distribution

  \begin{itemize}
  \item
    A 90\% confidence interval will be found between the 5th and 95th percentiles
  \item
    A 95\% confidence interval will be found between the 2.5th and 97.5th percentiles
  \item
    A 99\% confidence interval will be found between the 0.5th and 99.5th percentiles
  \end{itemize}
\end{itemize}

\subsection{Key topics}\label{key-topics}

\subsubsection*{Exploratory data analysis}\label{exploratory-data-analysis}
\addcontentsline{toc}{subsubsection}{Exploratory data analysis}

At the end of this module, you should understand how to calculate a summary statistic and plot a single categorical variable.

\begin{itemize}
\item
  Notation for a sample proportion: \(\hat{p}\)
\item
  Notation for a population proportion: \(\pi\)
\item
  Types of plots for a single categorical variable:

  \begin{itemize}
  \item
    Frequency bar plot
  \item
    Relative frequency bar plot
  \end{itemize}
\end{itemize}

\subsection*{Inference}\label{inference}
\addcontentsline{toc}{subsection}{Inference}

Additionally, we will use simulation methods \textbf{to find evidence of an effect by finding a p-value} and \textbf{estimating how large the effect is by creating a confidence interval}.

This is steps 4 and 5 from the steps of the statistical investigation process.

\subsection*{Steps of the statistical investigation process}\label{steps-of-the-statistical-investigation-process-1}
\addcontentsline{toc}{subsection}{Steps of the statistical investigation process}

As we move through the semester we will work through the six steps of the statistical investigation process.

\begin{enumerate}
\def\labelenumi{\arabic{enumi}.}
\item
  Ask a research question.
\item
  Design a study and collect data.
\item
  Summarize and visualize the data.
\item
  Use statistical analysis methods to draw inferences from the data.
\item
  Communicate the results and answer the research question.
\item
  Revisit and look forward.
\end{enumerate}

\newpage

\section{Video Notes: Exploratory Data Analysis of Categorical Variables}\label{video-notes-exploratory-data-analysis-of-categorical-variables}

Read Chapter 3, 4, 9, 10 and Sections 14.1 and 14.2 in the course textbook. Use the following videos to complete the video notes for Module 4.

\subsection{Course Videos}\label{course-videos-2}

\begin{itemize}
\item
  4.1
\item
  4.2
\item
  Chapter9
\item
  14.1
\item
  Chapter10
\item
  14.2
\end{itemize}

\setstretch{1}

\subsection*{Summarizing categorical data - Video 4.1}\label{summarizing-categorical-data---video-4.1}
\addcontentsline{toc}{subsection}{Summarizing categorical data - Video 4.1}

\begin{itemize}
\item
  A \_\_\_\_\_\_\_\_\_\_\_\_\_\_ is calculated on data from a sample
\item
  The parameter of interest is what we want to know from the population.
\item
  Includes:

  \begin{itemize}
  \item
    Population word (true, long-run, population)
  \item
    Summary measure (depends on the type of data)
  \item
    Context

    \begin{itemize}
    \item
      Observational units
    \item
      Variable(s)
    \end{itemize}
  \end{itemize}
\end{itemize}

Categorical data can be numerically summarized by calculating a \_\_\_\_\_\_\_\_\_\_\_\_\_\_\_ from the data set.

Notation used for the population proportion:

\begin{itemize}
\tightlist
\item
  Single categorical variable:
\end{itemize}

\vspace{0.2in}

\begin{itemize}
\tightlist
\item
  Two categorical variables:
\end{itemize}

\vspace{0.2in}

\rgi \rgi - Subscripts represent the \_\_\_\_\_\_\_\_\_\_\_ variable groups

Notation used for the sample proportion:

\begin{itemize}
\tightlist
\item
  Single categorical variable:
\end{itemize}

\vspace{0.2in}

\begin{itemize}
\tightlist
\item
  Two categorical variables
\end{itemize}

\vspace{0.2in}

\setstretch{1.5}

Categorical data can be reported in a \_\_\_\_\_\_\_\_\_\_\_\_table,
which plots counts or a \_\_\_\_\_\_\_\_\_\_\_\_\_\_
frequency table, which plots the proportion.

When we have two categorical variables we report the data in a \_\_\_\_\_\_\_\_\_\_\_\_\_\_\_ or two-way table with the \_\_\_\_\_\_\_\_\_\_\_\_\_\_\_ variable on the columns and the \_\_\_\_\_\_\_\_\_\_\_\_ variable on the rows.

\setstretch{1}

\vspace{2mm}

Example from the Video: Gallatin Valley is the fastest growing county in Montana. You'll often hear Bozeman residents complaining about the `out-of-staters' moving in. A local real estate agent recorded data on a random sample of 100 home sales over the last year at her company and noted where the buyers were moving from as well as the age of the person or average age of a couple buying a home. The variable age was binned into two categories, ``Under30'' and ``Over30.'' Additionally, the variable, state the buyers were moving from, was created as a binary variable, ``Out'' for a location out of state and ``In'' for a location in state.

The following code reads in the data set, \texttt{moving\_to\_mt} and names the object moving.

\begin{Shaded}
\begin{Highlighting}[]
\NormalTok{moving }\OtherTok{\textless{}{-}} \FunctionTok{read.csv}\NormalTok{(}\StringTok{"data/moving\_to\_mt.csv"}\NormalTok{)}
\end{Highlighting}
\end{Shaded}

The \texttt{R} function \texttt{glimpse} was used to give the following output.

\begin{Shaded}
\begin{Highlighting}[]
\FunctionTok{glimpse}\NormalTok{(moving)}
\end{Highlighting}
\end{Shaded}

\begin{verbatim}
#> Rows: 100
#> Columns: 4
#> $ From      <chr> "CA", "CA", "CA", "CA", "CA", "CA", "CA", "CA", "CA", "CA", ~
#> $ Age_Group <chr> "Under30", "Under30", "Under30", "Under30", "Under30", "Unde~
#> $ Age       <int> 25, 26, 27, 27, 29, 29, 35, 37, 49, 63, 65, 77, 22, 24, 24, ~
#> $ InOut     <chr> "Out", "Out", "Out", "Out", "Out", "Out", "Out", "Out", "Out~
\end{verbatim}

\begin{itemize}
\tightlist
\item
  What are the observational units in this study?
\end{itemize}

\vspace{0.3in}

\begin{itemize}
\tightlist
\item
  What type of variable is \texttt{Age}?
\end{itemize}

\vspace{0.3in}

\begin{itemize}
\tightlist
\item
  What type of variable is \texttt{Age\_Group}?
\end{itemize}

To further analyze the categorical variable, \texttt{From}, we can create either a frequency table:

\begin{verbatim}
#>   From  n
#> 1   CA 12
#> 2   CO  8
#> 3   MT 61
#> 4   WA 19
\end{verbatim}

Or a relative frequency table:

\begin{verbatim}
#>   From  n freq
#> 1   CA 12 0.12
#> 2   CO  8 0.08
#> 3   MT 61 0.61
#> 4   WA 19 0.19
\end{verbatim}

\begin{itemize}
\tightlist
\item
  How many home sales have buyers from WA?
\end{itemize}

\vspace{0.2in}

\begin{itemize}
\tightlist
\item
  What proportion of sampled home sales have buyers from WA?
\end{itemize}

\vspace{0.2in}

\begin{itemize}
\tightlist
\item
  What notation is used for the proportion of home sale buyers that that are from WA?
\end{itemize}

\vspace{0.2in}

\newpage

\subsubsection*{Displaying categorical variables - Video 4.2}\label{displaying-categorical-variables---video-4.2}
\addcontentsline{toc}{subsubsection}{Displaying categorical variables - Video 4.2}

\begin{itemize}
\tightlist
\item
  Types of plots for a single categorical variable
\end{itemize}

\vspace{0.4in}

The following code in \texttt{R} will create a frequency bar plot of the variable, \texttt{From}.

\begin{Shaded}
\begin{Highlighting}[]
\NormalTok{moving }\SpecialCharTok{\%\textgreater{}\%}
    \FunctionTok{ggplot}\NormalTok{(}\FunctionTok{aes}\NormalTok{(}\AttributeTok{x =}\NormalTok{ From))}\SpecialCharTok{+} \CommentTok{\#Enter the variable to plot}
    \FunctionTok{geom\_bar}\NormalTok{(}\AttributeTok{stat =} \StringTok{"count"}\NormalTok{) }\SpecialCharTok{+} 
    \FunctionTok{labs}\NormalTok{(}\AttributeTok{title =} \StringTok{"Frequency Bar Plot of State of Origin for}
\StringTok{         Gallatin County Home Sales"}\NormalTok{, }
         \CommentTok{\#Title your plot (type of plot, observational units, variable)}
       \AttributeTok{y =} \StringTok{"Frequency"}\NormalTok{, }\CommentTok{\#y{-}axis label}
       \AttributeTok{x =} \StringTok{"State of Origin"}\NormalTok{) }\CommentTok{\#x{-}axis label}
\end{Highlighting}
\end{Shaded}

\begin{center}\includegraphics[width=0.65\linewidth]{03-VN03-EDA_OneCatSimulation_files/figure-latex/unnamed-chunk-5-1} \end{center}

\begin{itemize}
\tightlist
\item
  What can we see from this plot?
\end{itemize}

\vspace{0.3in}

Additionally, we can create a relative frequency bar plot.

\begin{Shaded}
\begin{Highlighting}[]
\NormalTok{moving }\SpecialCharTok{\%\textgreater{}\%}
  \FunctionTok{ggplot}\NormalTok{(}\FunctionTok{aes}\NormalTok{(}\AttributeTok{x =}\NormalTok{ From))}\SpecialCharTok{+} \CommentTok{\#Enter the variable to plot}
  \FunctionTok{geom\_bar}\NormalTok{(}\FunctionTok{aes}\NormalTok{(}\AttributeTok{y =} \FunctionTok{after\_stat}\NormalTok{(prop), }\AttributeTok{group =} \DecValTok{1}\NormalTok{)) }\SpecialCharTok{+}
  \FunctionTok{labs}\NormalTok{(}\AttributeTok{title =} \StringTok{"Relative Frequency Bar Plot of State of Origin }
\StringTok{       for Gallatin County Home Sales"}\NormalTok{, }
       \CommentTok{\#Title your plot}
       \AttributeTok{y =} \StringTok{"Relative Frequency"}\NormalTok{, }\CommentTok{\#y{-}axis label}
       \AttributeTok{x =} \StringTok{"State of Origin"}\NormalTok{) }\CommentTok{\#x{-}axis label}
\end{Highlighting}
\end{Shaded}

\begin{center}\includegraphics[width=0.65\linewidth]{03-VN03-EDA_OneCatSimulation_files/figure-latex/unnamed-chunk-6-1} \end{center}

\setstretch{1.5}

\begin{itemize}
\tightlist
\item
  Note: the x-axis is the \_\_\_\_\_\_\_\_\_\_\_\_\_\_\_ between the frequency bar plot and the relative frequency bar plot. However, the \_\_\_\_\_\_\_\_\_\_\_\_\_\_ differs. The scale for the frequency bar plot goes from \_\_\_\_\_\_\_\_\_\_\_\_\_\_\_\_\_\_\_\_\_\_\_\_\_\_\_\_\_\_\_ and the scale for the relative frequency bar plot is from \_\_\_\_\_\_\_\_\_\_\_\_\_\_\_\_\_\_\_\_\_\_\_\_\_\_\_\_\_\_.
\end{itemize}

\setstretch{1}

\subsection*{Hypothesis Testing - Video Chapter9}\label{hypothesis-testing---video-chapter9}
\addcontentsline{toc}{subsection}{Hypothesis Testing - Video Chapter9}

Purpose of a hypothesis test:

\begin{itemize}
\item
  Use data collected on a sample to give information about the population.
\item
  Determines \_\_\_\_\_\_\_\_\_\_\_\_\_\_\_\_\_\_ of \_\_\_\_\_\_\_\_\_\_\_\_\_\_\_\_\_\_\_\_\_ of an effect
\end{itemize}

General steps of a hypothesis test

\begin{enumerate}
\def\labelenumi{\arabic{enumi}.}
\item
  Write a research question and hypotheses.
\item
  Collect data and calculate a summary statistic.
\item
  Model a sampling distribution which assumes the null hypothesis is true.
\item
  Calculate a p-value.
\item
  Draw conclusions based on a p-value.
\end{enumerate}

\subsection*{Hypothesis Testing/Justice System}\label{hypothesis-testingjustice-system}
\addcontentsline{toc}{subsection}{Hypothesis Testing/Justice System}

\setstretch{1.5}

\begin{itemize}
\item
  Two possible outcomes if the observed statistic is unusual:

  \begin{itemize}
  \item
    Strong evidence against \_\_\_\_\_\_\_\_\_\_\_\_\_\_\_\_\_\_ -\textgreater{} \_\_\_\_\_\_\_\_\_\_\_\_\_\_\_\_\_\_\_\_
  \item
    Not enough evidence against \_\_\_\_\_\_\_\_\_\_\_\_\_\_\_\_\_\_\_\_\_ -\textgreater{} \_\_\_\_\_\_\_\_\_\_\_\_\_\_\_\_\_\_\_\_\_\_
  \end{itemize}
\item
  Always written about the \_\_\_\_\_\_\_\_\_\_\_\_\_\_\_\_\_\_ (population)
\end{itemize}

\setstretch{1}

\subsubsection*{Null hypothesis}\label{null-hypothesis}
\addcontentsline{toc}{subsubsection}{Null hypothesis}

\begin{itemize}
\item
  Skeptical perspective, no difference, no effect, random chance
\item
  What the researcher hopes is \_\_\_\_\_\_\_\_\_\_\_\_\_\_\_.
\end{itemize}

Notation:

\vspace{0.2in}

\subsubsection*{Alternative hypothesis}\label{alternative-hypothesis}
\addcontentsline{toc}{subsubsection}{Alternative hypothesis}

\begin{itemize}
\item
  New perspective, a chance, a difference, an effect
\item
  What the researcher hopes is \_\_\_\_\_\_\_\_\_\_\_\_\_\_\_\_.
\end{itemize}

Notation:

\vspace{0.2in}

\subsection*{Simulation vs.~Theory-based Methods}\label{simulation-vs.-theory-based-methods}
\addcontentsline{toc}{subsection}{Simulation vs.~Theory-based Methods}

\subsubsection*{Simulation-based method}\label{simulation-based-method}
\addcontentsline{toc}{subsubsection}{Simulation-based method}

\setstretch{1.5}

Creation of the null distribution

\begin{itemize}
\tightlist
\item
  Simulate many samples assuming
\end{itemize}

\vspace{0.2in}

\begin{itemize}
\item
  Find the proportion of \_\_\_\_\_\_\_\_\_\_\_\_\_\_\_\_\_\_\_ at least as extreme as the observed sample \_\_\_\_\_\_\_\_\_\_\_\_
\item
  The null distribution estimates the sample to sample variability expected in the population
\end{itemize}

\setstretch{1}

\subsubsection*{Theory-based method}\label{theory-based-method}
\addcontentsline{toc}{subsubsection}{Theory-based method}

\begin{itemize}
\item
  Use a mathematical model to determine a distribution under the null hypothesis
\item
  Compare the observed sample statistic to the model to calculate a probability
\item
  \emph{Theory-based methods will be discussed in the next module}
\end{itemize}

\subsubsection*{P-value}\label{p-value}
\addcontentsline{toc}{subsubsection}{P-value}

\setstretch{1.5}

\begin{itemize}
\item
  What does the p-value measure?

  \begin{itemize}
  \tightlist
  \item
    Probability of observing the sample \_\_\_\_\_\_\_\_\_\_\_\_\_\_\_\_\_\_\_ or more \_\_\_\_\_\_\_\_\_\_
    assuming the \_\_\_\_\_\_\_\_ hypothesis is \_\_\_\_\_\_\_\_\_\_.
  \end{itemize}
\item
  How much evidence does the p-value provide against the null hypothesis?
\end{itemize}

\begin{center}\includegraphics[width=0.75\linewidth]{images/soe_gradient_gray} \end{center}

\rgi \rgi - The \_\_\_\_\_\_\_\_\_\_\_\_\_\_\_\_\_\_the p-value, the \_\_\_\_\_\_\_\_\_\_\_\_\_\_\_\_\_\_\_ the evidence against the null hypothesis.

\begin{itemize}
\tightlist
\item
  Write a conclusion based on the p-value.
\end{itemize}

\rgi \rgi - Answers the \_\_\_\_\_\_\_\_\_\_\_\_\_\_\_\_ question.

\rgi \rgi - Amount of \_\_\_\_\_\_\_\_\_\_\_\_\_\_\_\_\_ in support of the \_\_\_\_\_\_\_\_\_\_\_\_\_\_\_\_\_ hypothesis.

\begin{itemize}
\tightlist
\item
  Decision: can we reject or fail to reject the null hypothesis?
\end{itemize}

\rgi - Significance level: cut-off of ``small'' vs ``large'' p-value

\rgi \rgi - \(\text{p-value} \le \alpha\)

\rgi \rgi \rgi - Strong enough evidence against the null hypothesis

\rgi \rgi \rgi - Decision:

\vspace{0.2in}

\rgi \rgi \rgi - Results are \_\_\_\_\_\_\_\_\_\_\_\_\_\_\_\_\_\_\_\_\_\_\_ significant.

\rgi \rgi - \(\text{p-value} > \alpha\)

\rgi \rgi \rgi - Not enough evidence against the null hypothesis

\rgi \rgi \rgi - Decision:

\vspace{0.17in}

\rgi \rgi \rgi - Results are not \_\_\_\_\_\_\_\_\_\_\_\_\_\_\_\_\_\_\_\_\_ significant.

\setstretch{1}

\subsection*{One proportion test}\label{one-proportion-test}
\addcontentsline{toc}{subsection}{One proportion test}

\begin{itemize}
\item
  Reminder: review summary measures and plots discussed in the Week 3 material and Chapter 4 of the textbook.
\item
  The summary measure for a single categorical variable is a \_\_\_\_\_\_\_\_\_\_\_\_\_\_.
\end{itemize}

Notation:

\begin{itemize}
\item
  Population proportion:
\item
  Sample proportion:
\end{itemize}

Parameter of Interest:

\begin{itemize}
\item
  Include:

  \begin{itemize}
  \item
    Reference of the population (true, long-run, population, all)
  \item
    Summary measure
  \item
    Context

    \begin{itemize}
    \item
      Observational units/cases
    \item
      Response variable (and explanatory variable if present)

      \begin{itemize}
      \tightlist
      \item
        If the response variable is categorical, define a `success' in context
      \end{itemize}
    \end{itemize}
  \end{itemize}
\end{itemize}

\subsubsection*{Hypothesis testing}\label{hypothesis-testing}
\addcontentsline{toc}{subsubsection}{Hypothesis testing}

Conditions:

\begin{itemize}
\tightlist
\item
  Independence:
\end{itemize}

\vspace{0.3in}

Null hypothesis assumes ``no effect'', ``no difference'', ``nothing interesting happening'', etc.

\rgi Always of form: ``parameter'' = null value

\(H_0:\)

\vspace{0.5in}

\(H_A:\)

\vspace{0.5in}

\begin{itemize}
\tightlist
\item
  Research question determines the direction of the alternative hypothesis.
\end{itemize}

Video 14.1 Example: A 2007 study published in the Behavioral Ecology and Sociobiology Journal was titled ``Why do blue-eyed men prefer blue-eyed women?'' (Laeng 2007) In this study, conducted in Norway, 114 volunteer heterosexual blue-eyed males rated the attractiveness of 120 pictures of females. The researchers recorded which eye-color (blue, green, or brown) was rated the highest, on average. In the sample, 51 of the volunteers rated the blue-eyed women the most attractive. Do blue-eyed heterosexual men tend to find blue-eyed women the most attractive?

Parameter of interest:

\vspace{0.5in}

Write the null and alternative hypotheses for the blue-eyed study:

In notation:

\vspace{1mm}

\(H_0:\)

\vspace{0.2in}

\(H_A:\)

\vspace{0.2in}

Statistic:

\vspace{0.4in}

Is the independence condition met to analyze these data using a simulation-based approach?

\vspace{0.2in}

\newpage

\subsubsection*{Simulation-based method}\label{simulation-based-method-1}
\addcontentsline{toc}{subsubsection}{Simulation-based method}

\begin{itemize}
\item
  Simulate many samples assuming \(H_0: \pi = \pi_0\)

  \begin{itemize}
  \item
    Create a spinner with that represents the null value
  \item
    Spin the spinner \(n\) times
  \item
    Calculate and plot the simulated sample proportion from each simulation
  \item
    Repeat 1000 times (simulations) to create the null distribution
  \item
    Find the proportion of simulations at least as extreme as \(\hat{p}\)
  \end{itemize}
\end{itemize}

\begin{Shaded}
\begin{Highlighting}[]
\FunctionTok{set.seed}\NormalTok{(}\DecValTok{216}\NormalTok{)}
\FunctionTok{one\_proportion\_test}\NormalTok{(}\AttributeTok{probability\_success =} \FloatTok{0.333}\NormalTok{, }\CommentTok{\# Null hypothesis value}
          \AttributeTok{sample\_size =} \DecValTok{114}\NormalTok{, }\CommentTok{\# Enter sample size}
          \AttributeTok{number\_repetitions =} \DecValTok{1000}\NormalTok{, }\CommentTok{\# Enter number of simulations}
          \AttributeTok{as\_extreme\_as =} \FloatTok{0.447}\NormalTok{, }\CommentTok{\# Observed statistic}
          \AttributeTok{direction =} \StringTok{"greater"}\NormalTok{, }\CommentTok{\# Specify direction of alternative hypothesis}
          \AttributeTok{summary\_measure =} \StringTok{"proportion"}\NormalTok{) }\CommentTok{\# Reporting proportion or number of successes?}
\end{Highlighting}
\end{Shaded}

\begin{center}\includegraphics[width=0.7\linewidth]{03-VN03-EDA_OneCatSimulation_files/figure-latex/unnamed-chunk-8-1} \end{center}

Explain why the null distribution is centered at the value of approximately 0.333:

\vspace{0.5in}

Interpretation of the p-value:

\begin{itemize}
\item
  Statement about probability or proportion of samples
\item
  Statistic (summary measure and value)
\item
  Direction of the alternative
\item
  Null hypothesis (in context)
\end{itemize}

\vspace{0.8in}

\newpage

Conclusion:

\begin{itemize}
\item
  Amount of evidence
\item
  Parameter of interest
\item
  Direction of the alternative hypothesis
\end{itemize}

\vspace{0.6in}

Generalization:

\begin{itemize}
\tightlist
\item
  Can the results of the study be generalized to the target population?
\end{itemize}

\vspace{0.4in}

\subsection*{Confidence interval - Video Chapter10}\label{confidence-interval---video-chapter10}
\addcontentsline{toc}{subsection}{Confidence interval - Video Chapter10}

\rgi \(\text{statistic} \pm \text{margin of error}\)

Vocabulary:

\begin{itemize}
\tightlist
\item
  Point estimate:
\end{itemize}

\vspace{0.3in}

\begin{itemize}
\tightlist
\item
  Margin of error:
\end{itemize}

\vspace{0.3in}

\setstretch{1.5}

Purpose of a confidence interval

\begin{itemize}
\item
  To give an \_\_\_\_\_\_\_\_\_\_\_\_\_\_\_\_\_\_\_\_ \_\_\_\_\_\_\_\_\_\_\_\_\_\_\_\_\_\_\_ for the parameter of interest
\item
  Determines how \_\_\_\_\_\_\_\_\_\_\_\_\_\_ an effect is
\end{itemize}

\setstretch{1}

\subsubsection*{Sampling distribution}\label{sampling-distribution}
\addcontentsline{toc}{subsubsection}{Sampling distribution}

\setstretch{1.5}

\begin{itemize}
\item
  Ideally, we would take many samples of the same \_\_\_\_\_\_\_\_\_\_\_ from the same population to create a sampling distribution
\item
  But only have 1 sample, so we will \_\_\_\_\_\_\_\_\_\_\_\_\_\_\_\_\_ with \_\_\_\_\_\_\_\_\_\_\_\_\_\_\_\_\_ from the one sample.
\item
  Need to estimate the sampling distribution to see the \_\_\_\_\_\_\_\_\_\_\_\_\_\_\_\_\_ in the sample
\end{itemize}

\setstretch{1}

\subsubsection*{Simulation-based methods}\label{simulation-based-methods}
\addcontentsline{toc}{subsubsection}{Simulation-based methods}

Bootstrap distribution:

\begin{itemize}
\item
  Write the response variable values on cards
\item
  Sample with replacement \(n\) times (bootstrapping)
\item
  Calculate and plot the simulated difference in sample means from each simulation
\item
  Repeat 1000 times (simulations) to create the bootstrap distribution
\item
  Find the cut-offs for the middle X\% (confidence level) in a bootstrap distribution.
\end{itemize}

What is bootstrapping?

\begin{itemize}
\item
  Assume the ``population'' is many, many copies of the original sample.
\item
  Randomly sample with replacement from the original sample \(n\) times.
\end{itemize}

\subsubsection*{Video 14.2}\label{video-14.2}
\addcontentsline{toc}{subsubsection}{Video 14.2}

Let's revisit the blue-eyed male study to estimate the \emph{proportion of ALL heterosexual blue-eyed males who tend to find blue-eyed women the most attractive} by creating a 90\% confidence interval.

Bootstrap distribution:

\begin{Shaded}
\begin{Highlighting}[]
\FunctionTok{set.seed}\NormalTok{(}\DecValTok{216}\NormalTok{)}
\FunctionTok{one\_proportion\_bootstrap\_CI}\NormalTok{(}\AttributeTok{sample\_size =} \DecValTok{114}\NormalTok{, }\CommentTok{\# Sample size}
                    \AttributeTok{number\_successes =} \DecValTok{51}\NormalTok{, }\CommentTok{\# Observed number of successes}
                    \AttributeTok{number\_repetitions =} \DecValTok{1000}\NormalTok{, }\CommentTok{\# Number of bootstrap samples to use}
                    \AttributeTok{confidence\_level =} \FloatTok{0.90}\NormalTok{) }\CommentTok{\# Confidence level as a decimal}
\end{Highlighting}
\end{Shaded}

\begin{center}\includegraphics[width=0.7\linewidth]{03-VN03-EDA_OneCatSimulation_files/figure-latex/unnamed-chunk-9-1} \end{center}

Confidence interval interpretation:

\begin{itemize}
\item
  How confident you are (e.g., 90\%, 95\%, 98\%, 99\%)
\item
  Parameter of interest
\item
  Calculated interval
\item
  Order of subtraction when comparing two groups
\end{itemize}

\vspace{0.8in}

\newpage

How does changing the confidence level impact the width of the confidence interval?

95\% Confidence Interval:

\begin{Shaded}
\begin{Highlighting}[]
\FunctionTok{set.seed}\NormalTok{(}\DecValTok{216}\NormalTok{)}
\FunctionTok{one\_proportion\_bootstrap\_CI}\NormalTok{(}\AttributeTok{sample\_size =} \DecValTok{114}\NormalTok{, }\CommentTok{\# Sample size}
                    \AttributeTok{number\_successes =} \DecValTok{51}\NormalTok{, }\CommentTok{\# Observed number of successes}
                    \AttributeTok{number\_repetitions =} \DecValTok{1000}\NormalTok{, }\CommentTok{\# Number of bootstrap samples to use}
                    \AttributeTok{confidence\_level =} \FloatTok{0.95}\NormalTok{) }\CommentTok{\# Confidence level as a decimal}
\end{Highlighting}
\end{Shaded}

\begin{center}\includegraphics[width=0.7\linewidth]{03-VN03-EDA_OneCatSimulation_files/figure-latex/unnamed-chunk-10-1} \end{center}

99\% Confidence Interval:

\begin{Shaded}
\begin{Highlighting}[]
\FunctionTok{set.seed}\NormalTok{(}\DecValTok{216}\NormalTok{)}
\FunctionTok{one\_proportion\_bootstrap\_CI}\NormalTok{(}\AttributeTok{sample\_size =} \DecValTok{114}\NormalTok{, }\CommentTok{\# Sample size}
                    \AttributeTok{number\_successes =} \DecValTok{51}\NormalTok{, }\CommentTok{\# Observed number of successes}
                    \AttributeTok{number\_repetitions =} \DecValTok{1000}\NormalTok{, }\CommentTok{\# Number of bootstrap samples to use}
                    \AttributeTok{confidence\_level =} \FloatTok{0.99}\NormalTok{) }\CommentTok{\# Confidence level as a decimal}
\end{Highlighting}
\end{Shaded}

\begin{center}\includegraphics[width=0.7\linewidth]{03-VN03-EDA_OneCatSimulation_files/figure-latex/unnamed-chunk-11-1} \end{center}

\subsection{Concept Check}\label{concept-check-3}

Be prepared for group discussion in the next class. One member from the table should write the answers to the following on the whiteboard.

\begin{enumerate}
\def\labelenumi{\arabic{enumi}.}
\tightlist
\item
  What is the summary measure calculated from a single categorical variable?
\end{enumerate}

\vspace{0.3in}

\begin{enumerate}
\def\labelenumi{\arabic{enumi}.}
\setcounter{enumi}{1}
\tightlist
\item
  Write the alternative hypothesis for this study in notation? How was the direction of the alternative hypothesis determined?
\end{enumerate}

\vspace{0.4in}

\begin{enumerate}
\def\labelenumi{\arabic{enumi}.}
\setcounter{enumi}{2}
\tightlist
\item
  Do the results of the confidence interval \emph{match} the results based on the p-value?
\end{enumerate}

\vspace{0.5in}

\newpage

\section{Activity 6: Helper-Hinderer Part 1 --- Simulation-based Hypothesis Test}\label{activity-6-helper-hinderer-part-1-simulation-based-hypothesis-test}

\setstretch{1}

\subsection{Learning outcomes}\label{learning-outcomes-5}

\begin{itemize}
\item
  Identify the two possible explanations (one assuming the null hypothesis and one assuming the alternative hypothesis) for a relationship seen in sample data.
\item
  Given a research question involving a single categorical variable, construct the null and alternative hypotheses
  in words and using appropriate statistical symbols.
\item
  Describe and perform a simulation-based hypothesis test for a single proportion.
\end{itemize}

\subsection{Terminology review}\label{terminology-review-5}

In today's activity, we will work through a simulation-based hypothesis testing for a single categorical variable. Some terms covered in this activity are:

\begin{itemize}
\item
  Parameter of interest
\item
  Null hypothesis
\item
  Alternative hypothesis
\item
  Simulation
\end{itemize}

To review these concepts, see Chapters 9 \& 14 in your textbook.

\subsection{Steps of the statistical investigation process}\label{steps-of-the-statistical-investigation-process-2}

We will work through a five-step process to complete a hypothesis test for a single proportion, first introduced in the activity in week 1.

\begin{itemize}
\item
  \textbf{Ask a research question} that can be addressed by collecting data. What are the researchers trying to show?
\item
  \textbf{Design a study and collect data}. This step involves selecting the people or objects to be studied and how to gather relevant data on them.
\item
  \textbf{Summarize and visualize the data}. Calculate summary statistics and create graphical plots that best represent the research question.
\item
  \textbf{Use statistical analysis methods to draw inferences from the data}. Choose a statistical inference method appropriate for the data and identify the p-value and/or confidence interval after checking assumptions. In this study, we will focus on using randomization to generate a simulated p-value.
\item
  \textbf{Communicate the results and answer the research question}. Using the p-value and confidence interval from the analysis, determine whether the data provide statistical evidence against the null hypothesis. Write a conclusion that addresses the research question.
\end{itemize}

\newpage

\subsection{Helper-Hinderer}\label{helper-hinderer}

A study by Hamblin, Wynn, and Bloom reported in Nature (Hamblin, Wynn, and Bloom 2007) was intended to check young kids' feelings about helpful and non-helpful behavior. Non-verbal infants ages 6 to 10 months were shown short videos with different shapes either helping or hindering the climber. As a class we will watch this short video to see how the experiment was run: \url{https://youtu.be/anCaGBsBOxM}. Researchers were hoping to assess: Are infants more likely to choose the helper toy over the hinderer toy? In the study, of the 16 infants age 6 to 10 months, 14 chose the \emph{helper} toy and 2 chose the \emph{hinderer} toy.

In this study, the \textbf{observational units are the infants ages 6 to 10 months}. The \textbf{variable measured on each observational unit (infant) is whether they chose the helper or the hinderer toy}. This is a categorical variable so we will be assessing the proportion of infants ages 6 to 10 months that choose the helper toy. Choosing the helper toy in this study will be considered a success.

\subsubsection*{Ask a research question}\label{ask-a-research-question}
\addcontentsline{toc}{subsubsection}{Ask a research question}

\begin{enumerate}
\def\labelenumi{\arabic{enumi}.}
\tightlist
\item
  Identify the research question for this study. What are the researchers hoping to show?
\end{enumerate}

\vspace{0.6in}

\subsubsection*{Design a study and collect data}\label{design-a-study-and-collect-data}
\addcontentsline{toc}{subsubsection}{Design a study and collect data}

Before using statistical inference methods, we must check that the cases are independent. The sample observations are independent if the outcome of one observation does not influence the outcome of another. One way this condition is met is if data come from a simple random sample of the target population.

\begin{enumerate}
\def\labelenumi{\arabic{enumi}.}
\setcounter{enumi}{1}
\tightlist
\item
  Are the cases independent? Justify your answer.
\end{enumerate}

\vspace{0.8in}

\subsubsection*{R code}\label{r-code}
\addcontentsline{toc}{subsubsection}{R code}

For almost all activities and labs it will be necessary to upload the provided R script file from D2L for that day. Your instructor will highlight a few steps in uploading files to and using RStudio.

The following are the steps to upload the necessary R script file for this activity:

\begin{itemize}
\item
  Download the Activity R script file from D2L.
\item
  Click ``Upload'' in the ``Files'' tab in the bottom right window of RStudio. In the pop-up window, click ``Choose File'', and navigate to the folder where the Activity R script file is saved (most likely in your downloads folder). Click ``Open''; then click ``Ok''.
\item
  You should see the uploaded file appear in the list of files in the bottom right window. Click on the file name to open the file in the Editor window (upper left window).
\end{itemize}

Notice that the first threelines of code contain a prompt called \texttt{library}. Packages needed to run functions in R are stored in directories called libraries. When using the MSU RStudio server, all the packages needed for the class are already installed. We simply must tell R which packages we need for each R script file. We use the prompt \texttt{library} to load each \textbf{package} (or library) needed for each activity. Note, these \texttt{library} lines MUST be run each time you open a R script file in order for the functions in R to work.

\begin{itemize}
\tightlist
\item
  Highlight and run lines 1--3 to load the packages needed for this activity. Notice the use of the \# symbol in the R script file. This symbol is not part of the R code. It is used by these authors to add comments to the R code and explain what each call is telling the program to do.
\end{itemize}

R will ignore everything after a \# symbol when executing the code. Refer to the instructions following the \# symbol to understand what you need to enter in the code.

\begin{Shaded}
\begin{Highlighting}[]
\FunctionTok{library}\NormalTok{(tidyverse)}
\FunctionTok{library}\NormalTok{(ggplot2)}
\FunctionTok{library}\NormalTok{(catstats)}
\end{Highlighting}
\end{Shaded}

Throughout activities, we will often include the R code you would use in order to produce output or plots. These ``code chunks'' appear in gray. In the code chunk below, we demonstrate how to read the data set into R using the \texttt{read.csv()} function. The line of code shown below (line 7 in the R script file) reads in the data set and names the data set \texttt{infants}.

\subsubsection*{Summarize and visualize the data}\label{summarize-and-visualize-the-data}
\addcontentsline{toc}{subsubsection}{Summarize and visualize the data}

The following code reads in the data set and gives the number of infants in each level of the variable, whether the infant chose the helper or the hinderer.

\begin{itemize}
\tightlist
\item
  Highlight and run lines 7 and 8 to check that you get the same counts as shown below
\end{itemize}

\begin{Shaded}
\begin{Highlighting}[]
 \CommentTok{\# Read in data set}
\NormalTok{infants }\OtherTok{\textless{}{-}} \FunctionTok{read.csv}\NormalTok{(}\StringTok{"https://math.montana.edu/courses/s216/data/infantchoice.csv"}\NormalTok{)}
\NormalTok{infants }\SpecialCharTok{\%\textgreater{}\%} \FunctionTok{count}\NormalTok{(choice)  }\CommentTok{\# Count number in each choice category}
\end{Highlighting}
\end{Shaded}

\begin{verbatim}
#>     choice  n
#> 1   helper 14
#> 2 hinderer  2
\end{verbatim}

The following formula is used to calculate the proportion of successes in the sample.

\[\hat{p} = \frac{\mbox{number of successes}}{\mbox{total number of observational units}}\]

\begin{enumerate}
\def\labelenumi{\arabic{enumi}.}
\setcounter{enumi}{2}
\tightlist
\item
  Using the R output and the formula given, calculate the summary statistic (sample proportion) to represent the research question. Recall that \texttt{choosing\ the\ helper\ toy} is a considered a success. Use appropriate notation.
\end{enumerate}

\vspace{0.5in}

To visually display this data we can use either a frequency bar plot or a relative frequency bar plot.

\begin{itemize}
\item
  Enter the name of the variable name \texttt{choice} for \texttt{variable} in the R code to create the frequency bar plot.
\item
  Note the name of the title is given in line 16 and includes the type of plot, observational units, and variable name
\item
  Highlight and run lines 13--19 to create the plot
\end{itemize}

\begin{Shaded}
\begin{Highlighting}[]
\NormalTok{infants }\SpecialCharTok{\%\textgreater{}\%} \CommentTok{\# Data set piped into...}
    \FunctionTok{ggplot}\NormalTok{(}\FunctionTok{aes}\NormalTok{(}\AttributeTok{x =}\NormalTok{ variable)) }\SpecialCharTok{+}   \CommentTok{\# This specifies the variable}
    \FunctionTok{geom\_bar}\NormalTok{(}\AttributeTok{stat =} \StringTok{"count"}\NormalTok{) }\SpecialCharTok{+}  \CommentTok{\# Tell it to make a bar plot}
    \FunctionTok{labs}\NormalTok{(}\AttributeTok{title =} \StringTok{"Frequency Bar Plot of Toy Choice for Pre{-}verbal Infants"}\NormalTok{,  }
       \CommentTok{\# Give your plot a title}
       \AttributeTok{x =} \StringTok{"Toy Choice"}\NormalTok{,   }\CommentTok{\# Label the x axis}
       \AttributeTok{y =} \StringTok{"Frequency"}\NormalTok{)  }\CommentTok{\# Label the y axis}
\end{Highlighting}
\end{Shaded}

\begin{enumerate}
\def\labelenumi{\arabic{enumi}.}
\setcounter{enumi}{3}
\tightlist
\item
  Sketch the frequency bar plot created below.
\end{enumerate}

\vspace{1.8in}

We could also choose to display the data as a proportion in a \textbf{relative frequency} bar plot. To find the relative frequency, the count in each level of \texttt{choice} is divided by the sample size. This calculation is the sample proportion for each level of choice. Notice that in the following code we told R to create a bar plot with proportions.

\begin{itemize}
\tightlist
\item
  In the R script file, highlight and run lines 23--29 to create the relative frequency bar plot.
\end{itemize}

\begin{Shaded}
\begin{Highlighting}[]
\NormalTok{infants }\SpecialCharTok{\%\textgreater{}\%} \CommentTok{\# Data set piped into...}
    \FunctionTok{ggplot}\NormalTok{(}\FunctionTok{aes}\NormalTok{(}\AttributeTok{x =}\NormalTok{ choice)) }\SpecialCharTok{+}   \CommentTok{\# This specifies the variable}
    \FunctionTok{geom\_bar}\NormalTok{(}\FunctionTok{aes}\NormalTok{(}\AttributeTok{y =} \FunctionTok{after\_stat}\NormalTok{(prop), }\AttributeTok{group =} \DecValTok{1}\NormalTok{)) }\SpecialCharTok{+}  \CommentTok{\# Tell it to make a bar plot with proportions}
    \FunctionTok{labs}\NormalTok{(}\AttributeTok{title =} \StringTok{"Relative Frequency Bar Plot of Toy Choice for Pre{-}verbal Infants"}\NormalTok{,  }
       \CommentTok{\# Give your plot a title}
       \AttributeTok{x =} \StringTok{"Toy Choice"}\NormalTok{,   }\CommentTok{\# Label the x axis}
       \AttributeTok{y =} \StringTok{"Relative Frequency"}\NormalTok{)  }\CommentTok{\# Label the y axis}
\end{Highlighting}
\end{Shaded}

\begin{center}\includegraphics[width=0.5\linewidth]{03-A06-inference-1cat_test-simulation_files/figure-latex/unnamed-chunk-4-1} \end{center}

\begin{enumerate}
\def\labelenumi{\arabic{enumi}.}
\setcounter{enumi}{4}
\tightlist
\item
  Which features in the relative frequency bar plot are the same as the frequency bar plot? Which are different?
\end{enumerate}

\vspace{0.5in}

We cannot assess whether infants are more likely to choose the helper toy based on the statistic and plot alone. The next step is to analyze the data by using a hypothesis test to discover if there is evidence against the null hypothesis.

\subsubsection*{Use statistical analysis methods to draw inferences from the data}\label{use-statistical-analysis-methods-to-draw-inferences-from-the-data}
\addcontentsline{toc}{subsubsection}{Use statistical analysis methods to draw inferences from the data}

When performing a hypothesis test, we must first identify the null hypothesis. The null hypothesis is written about the parameter of interest, or the value that summarizes the variable in the population.

The parameter of interest is a statement about what we want to find about the population. The following must be included when writing the parameter of interest.

\begin{itemize}
\item
  Population word (true, long-run, population)
\item
  Summary measure (depends on the type of data)
\item
  Context

  \begin{itemize}
  \item
    Observational units
  \item
    Variable(s)
  \end{itemize}
\end{itemize}

For this study, the parameter of interest, \(\pi\), represents the \textbf{true or population proportion of infants ages 6--10 months who will choose the helper toy}.

If the children are just randomly choosing the toy, we would expect half (0.5) of the infants to choose the helper toy. This is the null value for our study.

\begin{enumerate}
\def\labelenumi{\arabic{enumi}.}
\setcounter{enumi}{5}
\tightlist
\item
  Using the parameter of interest given above, write out the null hypothesis in words. That is, what do we assume to be true about the parameter of interest when we perform our simulation?
  \vspace{0.8in}
\end{enumerate}

The notation used for a population proportion (or probability, or true proportion) is \(\pi\). Since this summarizes a population, it is a parameter. When writing the \textbf{null hypothesis} in notation, we set the parameter equal to the null value, \(H_0: \pi = \pi_0\).

\begin{enumerate}
\def\labelenumi{\arabic{enumi}.}
\setcounter{enumi}{6}
\tightlist
\item
  Write the null hypothesis in notation using the null value of 0.5 in place of \(\pi_0\) in the equation given on the previous page.
\end{enumerate}

\vspace{0.5in}

The \textbf{alternative hypothesis} is the claim to be tested and the direction of the claim (less than, greater than, or not equal to) is based on the research question.

\begin{enumerate}
\def\labelenumi{\arabic{enumi}.}
\setcounter{enumi}{7}
\tightlist
\item
  Based on the research question from question 1, are we testing that the parameter is greater than 0.5, less than 0.5 or different than 0.5?
\end{enumerate}

\vspace{0.2in}

\begin{enumerate}
\def\labelenumi{\arabic{enumi}.}
\setcounter{enumi}{8}
\tightlist
\item
  Write out the alternative hypothesis in notation.
\end{enumerate}

\vspace{0.5in}

Remember that when utilizing a hypothesis test, we are evaluating two competing possibilities. For this study the \textbf{two possibilities} are either\ldots{}

\begin{itemize}
\item
  The true proportion of infants who choose the helper is 0.5 and our results just occurred by random chance; or,
\item
  The true proportion of infants who choose the helper is greater than 0.5 and our results reflect this.
\end{itemize}

Notice that these two competing possibilities represent the null and alternative hypotheses.

We will now simulate one sample of a \textbf{null distribution} of sample proportions. The null distribution is created under the assumption the null hypothesis is true. In this case, we assume the true proportion of infants who choose the helper is 0.5, so we will create 1000 (or more) different simulations of 16 infants under this assumption.

Let's think about how to use a coin to create one simulation of 16 infants under the assumption the null hypothesis is true. Let heads equal infant chose the helper toy and tails equal infant chose the hinderer toy.

\begin{enumerate}
\def\labelenumi{\arabic{enumi}.}
\setcounter{enumi}{9}
\tightlist
\item
  How many times would you flip a coin to simulate the sample of infants?
\end{enumerate}

\vspace{0.2in}

\begin{enumerate}
\def\labelenumi{\arabic{enumi}.}
\setcounter{enumi}{10}
\tightlist
\item
  Flip a coin 16 times recording the number of times the coin lands on heads. This represents one simulated sample of 16 infants randomly choosing the toy. Calculate the proportion of coin flips that resulted in heads.
\end{enumerate}

\vspace{0.2in}

\begin{enumerate}
\def\labelenumi{\arabic{enumi}.}
\setcounter{enumi}{11}
\tightlist
\item
  Is the value from question 9 closer to 0.5, the null value, or closer to the sample proportion, 0.875?
\end{enumerate}

\vspace{0.2in}

Report the number of coin flips you got in the Google sheet on D2L.

\begin{enumerate}
\def\labelenumi{\arabic{enumi}.}
\setcounter{enumi}{12}
\tightlist
\item
  Sketch the graph created by your instructor of the proportion of heads out of 16 coin flips.
\end{enumerate}

\vspace{2in}

\begin{enumerate}
\def\labelenumi{\arabic{enumi}.}
\setcounter{enumi}{13}
\tightlist
\item
  Circle the observed statistic (value from question 3) on the distribution shown above. Where does this statistic fall in this distribution: Is it near the center of the distribution (near 0.5) or in one of the tails of the distribution?
\end{enumerate}

\vspace{0.2in}

\begin{enumerate}
\def\labelenumi{\arabic{enumi}.}
\setcounter{enumi}{14}
\tightlist
\item
  Is the observed statistic likely to happen or unlikely to happen if the true proportion of infants who choose the helper is 0.5? Explain your answer using the plot.
\end{enumerate}

\vspace{0.8in}

In the next class, we will continue to assess the strength of evidence against the null hypothesis by using a computer to simulate 1000 samples when we assume the null hypothesis is true.

\subsection{Take-home messages}\label{take-home-messages-5}

\begin{enumerate}
\def\labelenumi{\arabic{enumi}.}
\item
  Two types of plots are used for plotting categorical variables: frequency bar plots, relative frequency bar plots.
\item
  In a hypothesis test we have two competing hypotheses, the null hypothesis and the alternative hypothesis. The null hypothesis represents either a skeptical perspective or a perspective of no difference or no effect. The alternative hypothesis represents a new perspective such as the possibility that there has been a change or that there is a treatment effect in an experiment.
\item
  In a simulation-based test, we create a distribution of possible simulated statistics for our sample if the null hypothesis is true. Then we see if the calculated observed statistic from the data is likely or unlikely to occur when compared to the null distribution.
\item
  To create one simulated sample on the null distribution for a sample proportion, spin a spinner with probability equal to \(\pi_0\) (the null value), \(n\) times or draw with replacement \(n\) times from a deck of cards created to reflect \(\pi_0\) as the probability of success. Calculate and plot the proportion of successes from the simulated sample.
\end{enumerate}

\subsection{Additional notes}\label{additional-notes-5}

Use this space to summarize your thoughts and take additional notes on today's activity and material covered.

\newpage

\section{Activity 7: Helper-Hinderer (continued)}\label{activity-7-helper-hinderer-continued}

\setstretch{1}

\subsection{Learning outcomes}\label{learning-outcomes-6}

\begin{itemize}
\item
  Describe and perform a simulation-based hypothesis test for a single proportion.
\item
  Interpret and evaluate a p-value for a simulation-based hypothesis test for a single proportion.
\item
  Explore what a p-value represents
\end{itemize}

\subsection{Steps of the statistical investigation process}\label{steps-of-the-statistical-investigation-process-3}

In today's activity we will continue with steps 4 and 5 in the statistical investigation process. We will continue to assess the Helper-Hinderer study from last class.

\begin{itemize}
\item
  \textbf{Ask a research question} that can be addressed by collecting data. What are the researchers trying to show?
\item
  \textbf{Design a study and collect data}. This step involves selecting the people or objects to be studied and how to gather relevant data on them.
\item
  \textbf{Summarize and visualize the data}. Calculate summary statistics and create graphical plots that best represent the research question.
\item
  \textbf{Use statistical analysis methods to draw inferences from the data}. Choose a statistical inference method appropriate for the data and identify the p-value and/or confidence interval after checking assumptions. In this study, we will focus on using randomization to generate a simulated p-value.
\item
  \textbf{Communicate the results and answer the research question}. Using the p-value and confidence interval from the analysis, determine whether the data provide statistical evidence against the null hypothesis. Write a conclusion that addresses the research question.
\end{itemize}

\subsection{Helper-Hinderer}\label{helper-hinderer-1}

A study by Hamblin, Wynn, and Bloom reported in Nature (Hamblin, Wynn, and Bloom 2007) was intended to check young kids' feelings about helpful and non-helpful behavior. Non-verbal infants ages 6 to 10 months were shown short videos with different shapes either helping or hindering the climber. As a class we will watch this short video to see how the experiment was run: \url{https://youtu.be/anCaGBsBOxM}. Researchers were hoping to assess: Are infants more likely to choose the helper toy over the hinderer toy? In the study, of the 16 infants age 6 to 10 months, 14 chose the \emph{helper} toy and 2 chose the \emph{hinderer} toy.

\begin{enumerate}
\def\labelenumi{\arabic{enumi}.}
\tightlist
\item
  Report the sample proportion (summary statistic) calculated in the previous activity.
\end{enumerate}

\vspace{0.3in}

\begin{enumerate}
\def\labelenumi{\arabic{enumi}.}
\setcounter{enumi}{1}
\tightlist
\item
  Write the alternative hypothesis in words in context of the problem. Remember the direction we are testing is dependent on the research question.
\end{enumerate}

\vspace{0.8in}

Today, we will use the computer to simulate a null distribution of 1000 different samples of 16 infants, plotting the proportion who chose the helper in each sample, based on the assumption that the true proportion of infants who choose the helper is 0.5 (or that the null hypothesis is true).

\newpage

To use the computer simulation, we will need to enter the

\begin{itemize}
\tightlist
\item
  assumed ``probability of success'' (\(\pi_0\)),
\item
  ``sample size'' (the number of observational units or cases in the sample),
\item
  ``number of repetitions'' (the number of samples to be generated - typically we use 10000),
\item
  ``as extreme as'' (the observed statistic), and
\item
  the ``direction'' (matches the direction of the alternative hypothesis).
\end{itemize}

\begin{enumerate}
\def\labelenumi{\arabic{enumi}.}
\setcounter{enumi}{2}
\tightlist
\item
  What values should be entered for each of the following into the one proportion test to create 1000 simulations?
\end{enumerate}

\vspace{1mm}

\begin{itemize}
\tightlist
\item
  Probability of success:
\end{itemize}

\vspace{.2in}

\begin{itemize}
\tightlist
\item
  Sample size:
\end{itemize}

\vspace{.2in}

\begin{itemize}
\tightlist
\item
  Number of repetitions:
\end{itemize}

\vspace{.2in}

\begin{itemize}
\tightlist
\item
  As extreme as:
\end{itemize}

\vspace{.2in}

\begin{itemize}
\tightlist
\item
  Direction (\texttt{"greater"}, \texttt{"less"}, or \texttt{"two-sided"}):
\end{itemize}

We will use the \texttt{one\_proportion\_test()} function in \texttt{R} (in the \texttt{catstats} package) to simulate the null distribution of sample proportions and compute a p-value. Using the provided \texttt{R} script file, fill in the values/words for each \texttt{xx} with your answers from question 3 in the one proportion test to create a null distribution with 1000 simulations. Then highlight and run lines 1--16.

\begin{Shaded}
\begin{Highlighting}[]
\FunctionTok{one\_proportion\_test}\NormalTok{(}\AttributeTok{probability\_success =}\NormalTok{ xx, }\CommentTok{\# Null hypothesis value}
          \AttributeTok{sample\_size =}\NormalTok{ xx, }\CommentTok{\# Enter sample size}
          \AttributeTok{number\_repetitions =} \DecValTok{10000}\NormalTok{, }\CommentTok{\# Enter number of simulations}
          \AttributeTok{as\_extreme\_as =}\NormalTok{ xx, }\CommentTok{\# Observed statistic}
          \AttributeTok{direction =} \StringTok{"xx"}\NormalTok{, }\CommentTok{\# Specify direction of alternative hypothesis}
          \AttributeTok{summary\_measure =} \StringTok{"proportion"}\NormalTok{) }\CommentTok{\# Reporting proportion or number of successes?}
\end{Highlighting}
\end{Shaded}

\begin{enumerate}
\def\labelenumi{\arabic{enumi}.}
\setcounter{enumi}{3}
\tightlist
\item
  Sketch the null distribution created from the \texttt{R} code here.
\end{enumerate}

\vspace{1.8in}

\begin{enumerate}
\def\labelenumi{\arabic{enumi}.}
\setcounter{enumi}{4}
\tightlist
\item
  Around what value is the null distribution centered? Why does that make sense?
\end{enumerate}

\vspace{1in}

\begin{enumerate}
\def\labelenumi{\arabic{enumi}.}
\setcounter{enumi}{5}
\tightlist
\item
  Circle the observed statistic (value from question 1) on the distribution you drew in question 4. Where does this statistic fall in the null distribution: Is it near the center of the distribution (near 0.5) or in one of the tails of the distribution?
\end{enumerate}

\vspace{0.2in}

\begin{enumerate}
\def\labelenumi{\arabic{enumi}.}
\setcounter{enumi}{6}
\tightlist
\item
  Is the observed statistic likely to happen or unlikely to happen if the true proportion of infants who choose the helper is 0.5? Explain your answer using the plot.
\end{enumerate}

\vspace{0.5in}

\begin{enumerate}
\def\labelenumi{\arabic{enumi}.}
\setcounter{enumi}{7}
\tightlist
\item
  Using the simulation, what is the proportion of simulated samples that generated a sample proportion at the observed statistic or greater, if the true proportion of infants who choose the helper is 0.5? \emph{Hint}: Look under the simulation.
\end{enumerate}

\vspace{0.2in}

The value in question 8 is the \textbf{p-value}. The smaller the p-value, the more evidence we have against the null hypothesis.

\begin{enumerate}
\def\labelenumi{\arabic{enumi}.}
\setcounter{enumi}{8}
\tightlist
\item
  Using the following guidelines for the strength of evidence, how much evidence do the data provide against the null hypothesis? (Circle one of the five descriptions.)
\end{enumerate}

\begin{center}\includegraphics[width=0.9\linewidth]{images/soe_gradient_gray} \end{center}

\subsubsection*{Interpret the p-value}\label{interpret-the-p-value}
\addcontentsline{toc}{subsubsection}{Interpret the p-value}

The p-value measures the probability that we observe a sample proportion as extreme as what was seen in the data or more extreme (matching the direction of the Ha) IF the null hypothesis is true. This is a conditional probability, calculated dependent on the null hypothesis being true. Represented in probability notaton:

\[P(\text{statistic or more extreme|the null hypothesis is true})\]

\begin{enumerate}
\def\labelenumi{\arabic{enumi}.}
\setcounter{enumi}{9}
\tightlist
\item
  What did we assume to create the null distribution? Write the null hypothesis is context.
\end{enumerate}

\vspace{0.7in}

\begin{enumerate}
\def\labelenumi{\arabic{enumi}.}
\setcounter{enumi}{10}
\tightlist
\item
  What value did we compare to the null distribution to find the p-value? What is the value of the summary statistic (sample proportion)?
\end{enumerate}

\vspace{0.3in}

\begin{enumerate}
\def\labelenumi{\arabic{enumi}.}
\setcounter{enumi}{11}
\tightlist
\item
  In what direction (greater than or less than) did we count from the statistic to find the number of simulations?
  \vspace{0.3in}
\end{enumerate}

\newpage

\begin{enumerate}
\def\labelenumi{\arabic{enumi}.}
\setcounter{enumi}{12}
\tightlist
\item
  Fill in the blanks below to interpret the p-value.
\end{enumerate}

\setstretch{1.5}

We would observe a sample proportion of \hrulefill  

or (greater, less, more extreme) \hrulefill   

with a probability of \hrulefill  

IF we assume (\(H_0\) in context) \hrulefill.

\hrulefill

\setstretch{1}
\vspace{12pt}

\subsubsection*{Communicate the results and answer the research question}\label{communicate-the-results-and-answer-the-research-question}
\addcontentsline{toc}{subsubsection}{Communicate the results and answer the research question}

When we write a conclusion we answer the research question by stating how much evidence there is for the alternative hypothesis.

\begin{enumerate}
\def\labelenumi{\arabic{enumi}.}
\setcounter{enumi}{13}
\tightlist
\item
  Write a conclusion in context of the study. How much evidence does the data provide in support of the alternative hypothesis?
\end{enumerate}

\vspace{0.6in}

\setstretch{1.5}

\setstretch{1}

\subsection{Take-home messages}\label{take-home-messages-6}

\begin{enumerate}
\def\labelenumi{\arabic{enumi}.}
\item
  The null distribution is created based on the assumption the null hypothesis is true. We compare the sample statistic to the distribution to find the likelihood of observing this statistic.
\item
  The p-value measures the probability of observing the sample statistic or more extreme (in direction of the alternative hypothesis) is the null hypothesis is true.
\end{enumerate}

\subsection{Additional notes}\label{additional-notes-6}

Use this space to summarize your thoughts and take additional notes on today's activity and material covered.

\newpage

\section{Activity 8: Helper-Hinderer --- Simulation-based Confidence Interval}\label{activity-8-helper-hinderer-simulation-based-confidence-interval}

\setstretch{1}

\subsection{Learning outcomes}\label{learning-outcomes-7}

\begin{itemize}
\item
  Use bootstrapping to find a confidence interval for a single proportion.
\item
  Interpret a confidence interval for a single proportion.
\end{itemize}

\subsection{Terminology review}\label{terminology-review-6}

In today's activity, we will introduce simulation-based confidence intervals for a single proportion. Some terms covered in this activity are:

\begin{itemize}
\item
  Parameter of interest
\item
  Bootstrapping
\item
  Confidence interval
\end{itemize}

To review these concepts, see Chapters 10 \& 14 in your textbook.

\subsection{Helper-Hinderer}\label{helper-hinderer-2}

In the last class, we found very strong evidence that the true proportion of infants who will choose the helper character is greater than 0.5. But what \emph{is} the true proportion of infants who will choose the helper character? We will use this same study to estimate this parameter of interest by creating a confidence interval.

As a reminder: A study by Hamblin, Wynn, and Bloom reported in Nature (Hamblin, Wynn, and Bloom 2007) was intended to check young kids' feelings about helpful and non-helpful behavior. Non-verbal infants ages 6 to 10 months were shown short videos with different shapes either helping or hindering the climber. Researchers were hoping to assess: Are infants more likely to preferentially choose the helper toy over the hinderer toy? In the study, of the 16 infants age 6 to 10 months, 14 chose the \emph{helper} toy and 2 chose the \emph{hinderer} toy.

A \textbf{point estimate} (our observed statistic) provides a single plausible value for a parameter. However, a point estimate is rarely perfect; usually there is some error in the estimate. In addition to supplying a point estimate of a parameter, a next logical step would be to provide a plausible \emph{range} of values for the parameter. This plausible range of values for the population parameter is called an \textbf{interval estimate} or \textbf{confidence interval}.

\subsubsection*{Activity intro}\label{activity-intro}
\addcontentsline{toc}{subsubsection}{Activity intro}

\begin{enumerate}
\def\labelenumi{\arabic{enumi}.}
\tightlist
\item
  What is the value of the point estimate?
\end{enumerate}

\vspace{0.3in}

\begin{enumerate}
\def\labelenumi{\arabic{enumi}.}
\setcounter{enumi}{1}
\tightlist
\item
  If we took another random sample of 16 infants, would we get the exact same point estimate? Explain why or why not.
\end{enumerate}

\vspace{0.5in}

In today's activity, we will use bootstrapping to find a 95\% confidence interval for \(\pi\), the parameter of interest.

\begin{enumerate}
\def\labelenumi{\arabic{enumi}.}
\setcounter{enumi}{2}
\tightlist
\item
  In your own words, explain the bootstrapping process.
  \vspace{0.5in}
\end{enumerate}

\subsubsection*{Use statistical analysis methods to draw inferences from the data}\label{use-statistical-analysis-methods-to-draw-inferences-from-the-data-1}
\addcontentsline{toc}{subsubsection}{Use statistical analysis methods to draw inferences from the data}

\begin{enumerate}
\def\labelenumi{\arabic{enumi}.}
\setcounter{enumi}{3}
\tightlist
\item
  Write out the parameter of interest in words, in context of the study. \emph{Hint: this is the same as in Activity 6 and 7.}
\end{enumerate}

\vspace{0.5in}

To create the null distribution we flipped a coin 16 times to simulate infants randomly choosing the helper toy with a probability of 50\%.

\begin{enumerate}
\def\labelenumi{\arabic{enumi}.}
\setcounter{enumi}{4}
\tightlist
\item
  Why can't we use a coin to simulate the bootstrap distribution.
\end{enumerate}

\vspace{0.7in}

To create the bootstrap distribution.

\begin{itemize}
\item
  First we would label the cards to represent the sample statistic: 14 helper and 2 hinderer.
\item
  Sample with replacement 16 times
\end{itemize}

\begin{enumerate}
\def\labelenumi{\arabic{enumi}.}
\setcounter{enumi}{5}
\tightlist
\item
  Using the cards provided by your instructor, create one bootstrap sample. Report your simulated sample proportion on the whiteboard.
\end{enumerate}

\vspace{0.3in}

To use the computer simulation to create a bootstrap distribution, we will need to enter the

\begin{itemize}
\tightlist
\item
  ``sample size'' (the number of observational units or cases in the sample),
\item
  ``number of successes'' (the number of cases that choose the helper character),
\item
  ``number of repetitions'' (the number of samples to be generated), and
\item
  the ``confidence level'' (which level of confidence are we using to create the confidence interval).
\end{itemize}

\begin{enumerate}
\def\labelenumi{\arabic{enumi}.}
\setcounter{enumi}{6}
\tightlist
\item
  What values should be entered for each of the following into the simulation to create the bootstrap distribution of sample proportions to find a 95\% confidence interval?
  \vspace{1mm}
\end{enumerate}

\begin{itemize}
\tightlist
\item
  Sample size:
\end{itemize}

\vspace{.1in}

\begin{itemize}
\tightlist
\item
  Number of successes:
\end{itemize}

\vspace{.1in}

\begin{itemize}
\tightlist
\item
  Number of repetitions:
\end{itemize}

\vspace{.1in}

\begin{itemize}
\tightlist
\item
  Confidence level (as a decimal):
\end{itemize}

\vspace{.1in}

We will use the \texttt{one\_proportion\_bootstrap\_CI()} function in R (in the \texttt{catstats} package) to simulate the bootstrap distribution of sample proportions and calculate a confidence interval. Using the provided R script file, fill in the values/words for each \texttt{xx} with your answers from question 5 in the one proportion bootstrap confidence interval (CI) code to create a bootstrap distribution with 1000 simulations. Then highlight and run lines 1--9.

\begin{Shaded}
\begin{Highlighting}[]
\FunctionTok{one\_proportion\_bootstrap\_CI}\NormalTok{(}\AttributeTok{sample\_size =}\NormalTok{ xx, }\CommentTok{\# Sample size}
                    \AttributeTok{number\_successes =}\NormalTok{ xx, }\CommentTok{\# Observed number of successes}
                    \AttributeTok{number\_repetitions =} \DecValTok{10000}\NormalTok{, }\CommentTok{\# Number of bootstrap samples to use}
                    \AttributeTok{confidence\_level =}\NormalTok{ xx) }\CommentTok{\# Confidence level as a decimal}
\end{Highlighting}
\end{Shaded}

\newpage

\begin{enumerate}
\def\labelenumi{\arabic{enumi}.}
\setcounter{enumi}{7}
\tightlist
\item
  Sketch the bootstrap distribution created below.
\end{enumerate}

\vspace{1.8in}

\begin{enumerate}
\def\labelenumi{\arabic{enumi}.}
\setcounter{enumi}{8}
\item
  What is the value at the center of this bootstrap distribution? Why does this make sense?
  \vspace{.8in}
\item
  Explain why the two vertical lines are at the 2.5th percentile and the 97.5th percentile.
\end{enumerate}

\vspace{.4in}

\begin{enumerate}
\def\labelenumi{\arabic{enumi}.}
\setcounter{enumi}{10}
\tightlist
\item
  Report the 95\% bootstrapped confidence interval for \(\pi\). Use interval notation: (lower value, upper value).
\end{enumerate}

\vspace{0.2in}

\begin{enumerate}
\def\labelenumi{\arabic{enumi}.}
\setcounter{enumi}{11}
\tightlist
\item
  Interpret the 95\% confidence interval in context.
\end{enumerate}

\vspace{.6in}

\subsubsection*{Communicate the results and answer the research question}\label{communicate-the-results-and-answer-the-research-question-1}
\addcontentsline{toc}{subsubsection}{Communicate the results and answer the research question}

\begin{enumerate}
\def\labelenumi{\arabic{enumi}.}
\setcounter{enumi}{12}
\tightlist
\item
  Is the value 0.5 (the null value) in the 95\% confidence interval?
\end{enumerate}

\vspace{.2in}

~~~Explain how this indicates that the p-value provides strong evidence against the null.

\vspace{0.5in}

\subsubsection*{Effect of confidence level}\label{effect-of-confidence-level}
\addcontentsline{toc}{subsubsection}{Effect of confidence level}

\begin{enumerate}
\def\labelenumi{\arabic{enumi}.}
\setcounter{enumi}{13}
\tightlist
\item
  Suppose instead of finding a 95\% confidence interval, we found a 90\% confidence interval. Would you expect the 90\% confidence interval to be narrower or wider? Explain your answer.
\end{enumerate}

\vspace{0.4in}

\begin{enumerate}
\def\labelenumi{\arabic{enumi}.}
\setcounter{enumi}{14}
\tightlist
\item
  The following R code produced the bootstrap distribution with 1000 simulations that follows. Circle the value that changed in the code.
\end{enumerate}

\begin{Shaded}
\begin{Highlighting}[]
\FunctionTok{one\_proportion\_bootstrap\_CI}\NormalTok{(}\AttributeTok{sample\_size =} \DecValTok{16}\NormalTok{, }\CommentTok{\# Sample size}
                    \AttributeTok{number\_successes =} \DecValTok{14}\NormalTok{, }\CommentTok{\# Observed number of successes}
                    \AttributeTok{number\_repetitions =} \DecValTok{1000}\NormalTok{, }\CommentTok{\# Number of bootstrap samples to use}
                    \AttributeTok{confidence\_level =} \FloatTok{0.90}\NormalTok{) }\CommentTok{\# Confidence level as a decimal}
\end{Highlighting}
\end{Shaded}

\begin{center}\includegraphics[width=0.7\linewidth]{03-A08-inference-1cat_CI-simulation_files/figure-latex/unnamed-chunk-2-1} \end{center}

\begin{enumerate}
\def\labelenumi{\arabic{enumi}.}
\setcounter{enumi}{15}
\tightlist
\item
  Report both the 95\% confidence interval (question 9) and the 90\% confidence interval (question 13). Is the 90\% confidence interval narrower or wider than the 95\% confidence interval?
\end{enumerate}

\vspace{0.5in}

\begin{enumerate}
\def\labelenumi{\arabic{enumi}.}
\setcounter{enumi}{16}
\tightlist
\item
  Explain why the upper value of the confidence interval is truncated at 1.
\end{enumerate}

\vspace{0.3in}

\setstretch{1.5}

\begin{enumerate}
\def\labelenumi{\arabic{enumi}.}
\setcounter{enumi}{17}
\tightlist
\item
  Fill in the blanks below to write a paragraph summarizing the results of the study as if writing a press release.
\end{enumerate}

Researchers were interested if infants observe social cues and would be more likely to choose the helper toy over the hinderer toy. In a sample of (sample size) \_\_\_\_\_\_\_\_\_\_\_\_\_infants, (number of successes) \_\_\_\_\_\_\_\_\_\_\_\_\_\_\_chose the helper toy. A simulation null distribution with 1000 simulations was created in RStudio. The p-value was found by calculating the proportion of simulations in the null distribution at the sample statistic of 0.875 and greater. This resulted in a p-value of (value of p-value)\_\_\_\_\_\_\_\_\_\_\_\_\_\_\_. We would observe a sample proportion of (value of the sample proportion) \_\_\_\_\_\_\_\_\_\_\_\_\_\_\_\_\_\_\_\_\_\_ or (greater, less, more extreme) \_\_\_\_\_\_\_\_\_\_\_\_\_\_\_\_\_\_\_\_\_ with a probability of (value of p-value)\_\_\_\_\_\_\_\_\_\_\_\_\_\_\_\_\_\_\_\\
IF we assume (\(H_0\) in context) \_\_\_\_\_\_\_\_\_\_\_\_\_\_\_\_\_\_\_\_\_\_\_\_\_\_\_\_\_\_\_\_\_\_\_\_\_\_\_\_\_\_\_\_.
Based on this p-value, there is (very strong/little to no) \_\_\_\_\_\_\_\_\_\_\_\_\_\_\_\_\_\_\_\_\_\_ evidence that the (sample/true)\_\_\_\_\_\_\_\_\_\_\_\_\_\_\_\_\_\_\_\_\_ proportion of infants age 6 to 10 months who will choose the helper toy is (greater than, less than, not equal to) \_\_\_\_\_\_\_\_\_\_\_\_\_\_\_\_\_\_\_\_\_ 0.5. In addition, a 95\% confidence interval was found for the parameter of interest. We are 95\% confident that the (true/sample)\_\_\_\_\_\_\_\_\_\_\_\_\_\_\_\_\_\_\_\_\_\_\_\_\_ proportion of infants age 6 to 10 months who will choose the helper toy is between (lower value)\_\_\_\_\_\_\_\_\_\_\_\_\_\_\_\_ and (upper value)\_\_\_\_\_\_\_\_\_\_\_\_\_\_\_\_\_\_\_\_. The results of this study can be generalized to (all infants age 6 to 10 months/infants similar to those in this study)\_\_\_\_\_\_\_\_\_\_\_\_\_\_\_\_\_\_\_\_\_\_\_\_\_\_\_ as the researchers (did/did not)\_\_\_\_\_\_\_\_\_\_\_\_\_\_\_\_\_\_\_\_\_ select a random sample.

\setstretch{1}

\subsection{Take-home messages}\label{take-home-messages-7}

\begin{enumerate}
\def\labelenumi{\arabic{enumi}.}
\item
  The goal in a hypothesis test is to assess the strength of evidence for an effect, while the goal in creating a confidence interval is to determine how large the effect is. A \textbf{confidence interval} is a range of \emph{plausible} values for the parameter of interest.
\item
  A confidence interval is built around the point estimate or observed calculated statistic from the sample. This means that the sample statistic is always the center of the confidence interval. A confidence interval includes a measure of sample to sample variability represented by the \textbf{margin of error}.
\item
  In simulation-based methods (bootstrapping), a simulated distribution of possible sample statistics is created showing the possible sample-to-sample variability. Then we find the middle \(X\) percent of the distribution around the sample statistic using the percentile method to give the range of values for the confidence interval. This shows us that we are \(X\)\% confident that the parameter is within this range, where \(X\) represents the level of confidence.
\item
  When the null value is within the confidence interval, it is a plausible value for the parameter of interest; thus, we would find a larger p-value for a hypothesis test of that null value. Conversely, if the null value is NOT within the confidence interval, we would find a small p-value for the hypothesis test and strong evidence against this null hypothesis.
\item
  To create one simulated sample on the bootstrap distribution for a sample proportion, label \(n\) cards with the original responses. Draw with replacement \(n\) times. Calculate and plot the resampled proportion of successes.
\end{enumerate}

\subsection{Additional notes}\label{additional-notes-7}

Use this space to summarize your thoughts and take additional notes on today's activity and material covered.

\newpage

\chapter{Inference for a Single Categorical Variable: Theory-based Methods}\label{inference-for-a-single-categorical-variable-theory-based-methods}

\section{Vocabulary Review and Key Topics}\label{vocabulary-review-and-key-topics-3}

Review the Golden Ticket posted in the resources at the end of the coursepack for a summary of a single categorical variable.

\begin{itemize}
\item
  \textbf{Theory-based methods}: when specific conditions are met, a data can be fit with a theoretical distribution
\item
  \textbf{Conditions for the sampling distribution of \(\hat{p}\) to follow an approximate normal distribution}:

  \begin{itemize}
  \item
    \textbf{Independence}: The sample's observations are independent, e.g., are from a simple random sample. (\emph{Remember}: This also must be true to use simulation methods!)
  \item
    \textbf{Large enough sample size: Success-failure condition}: We \emph{expect} to see at least 10 successes and 10 failures in the sample, \(n\hat{p}≥10\) and \(n(1-\hat{p})≥10\).
  \end{itemize}
\item
  \textbf{Standardized statistic}: calculation to standardize the sample statistic in order to compare the standardized value to the theoretical distribution

  \begin{itemize}
  \tightlist
  \item
    Measures the number of standard errors the sample statistic is from the null value.
  \end{itemize}
\item
  \textbf{Standard normal distribution}: a theoretical distribution that is symmetric centered on the mean of zero with a standard deviation of one
\end{itemize}

\[N(0,1)\]

\begin{itemize}
\tightlist
\item
  \textbf{Standardized sample proportion}: standardized statistic for a single categorical variable calculated using:
\end{itemize}

\[
Z = \frac{\hat{p} - \pi_0}{SE_0(\hat{p})},
\]

\begin{itemize}
\tightlist
\item
  \textbf{Standard error of the sample proportion assuming the null is true}: measures the how far each possible sample proportion is from the true proportion, on average and is calculated using the null value:
\end{itemize}

\[SE_0(\hat{p})=\sqrt{\frac{\pi_0\times(1-\pi_0)}{n}}\].

\begin{itemize}
\item
  The p-value can be found by using the pnorm function.

  \begin{itemize}
  \tightlist
  \item
    Enter the value of the standardized statistic for xx
  \end{itemize}
\end{itemize}

\begin{Shaded}
\begin{Highlighting}[]
\FunctionTok{pnorm}\NormalTok{(xx, }\AttributeTok{lower.tail=}\ConstantTok{TRUE}\NormalTok{)}
\end{Highlighting}
\end{Shaded}

\begin{itemize}
\tightlist
\item
  \textbf{Margin of error}: half the width of the confidence interval
\end{itemize}

\[ME = z^* \times SE(\hat{p})\]

\begin{itemize}
\tightlist
\item
  \textbf{Standard error of the sample proportion for a confidence interval}
\end{itemize}

\[SE(\hat{p}) = \sqrt{\frac{\hat{p}\times (1-\hat{p})}{n}}\]

\begin{itemize}
\tightlist
\item
  To find the confidence interval add and subtract the margin of error to the sample statistic
\end{itemize}

\[\hat{p} \pm ME\]

\begin{itemize}
\item
  R code to find the multiplier for the confidence interval using theory-based methods.

  \begin{itemize}
  \item
    qnorm will give you the multiplier using the standard normal distribution
  \item
    Enter the percentile for the given level of confidence
  \end{itemize}
\end{itemize}

\begin{Shaded}
\begin{Highlighting}[]
\FunctionTok{qnorm}\NormalTok{(percentile, }\AttributeTok{lower.tail=}\ConstantTok{FALSE}\NormalTok{)}
\end{Highlighting}
\end{Shaded}

\subsection{Key topics}\label{key-topics-1}

\begin{itemize}
\item
  Theory-based methods should give the same results as simulation based methods if the sample size is large enough (success-failure condition is met).
\item
  If repeat samples of the same size are taken from the population, 95\% of samples will create a 95\% confidence interval that contains the parameter of interest.
\end{itemize}

\newpage

\section{Video Notes: Inference for One Categorical Variable using Theory-based Methods}\label{video-notes-inference-for-one-categorical-variable-using-theory-based-methods}

Read Chapters 11 and 13 and Sections 14.3 and 14.4 in the course textbook. Use the following videos to complete the video notes for Module 4.

\subsection{Course Videos}\label{course-videos-3}

\begin{itemize}
\item
  Chapter11
\item
  14.3TheoryTests
\item
  14.3TheoryIntervals
\end{itemize}

\setstretch{1}

\subsection*{Theory-based methods}\label{theory-based-methods}
\addcontentsline{toc}{subsection}{Theory-based methods}

\subsubsection*{Central limit theorem - Video Chapter11}\label{central-limit-theorem---video-chapter11}
\addcontentsline{toc}{subsubsection}{Central limit theorem - Video Chapter11}

The Central Limit Theorem tells us that the \_\_\_\_\_\_\_\_\_\_\_\_\_\_ distribution of a sample proportion (and sample mean and sample differences) will be approximately \_\_\_\_\_\_\_\_\_\_\_\_\_\_ if the sample size is \_\_\_\_\_\_\_\_\_\_\_\_\_\_ \_\_\_\_\_\_\_\_\_\_\_\_\_\_\_\_.

The \_\_\_\_\_\_\_\_\_\_\_\_\_\_ of the distribution of sample proportions (sampling distribution) from thousands of samples will be bell-shaped/symmetric (Normal), if the sample size is large enough and the observations are \_\_\_\_\_\_\_\_\_\_\_\_\_\_\_\_.

\begin{itemize}
\tightlist
\item
  \(\hat{p} \sim N (\pi, \sqrt{\frac{\pi \times (1-\pi)}{n}})\)
\end{itemize}

Conditions of the CLT:

\begin{itemize}
\tightlist
\item
  Independence (\emph{also must be met to use simulation methods}): the response for one observational unit will not influence another observational unit
\end{itemize}

\vspace{1mm}

\begin{itemize}
\tightlist
\item
  Large enough sample size:
\end{itemize}

\vspace{0.3in}

Normal distribution:

\begin{itemize}
\item
  Bell-shaped and \_\_\_\_\_\_\_\_\_\_\_\_\_\_
\item
  Standard normal distribution: \(N(0,1)\)
\end{itemize}

\begin{center}\includegraphics[width=0.45\linewidth]{04-VN04-1cat_theory_files/figure-latex/simpleNormalc-1} \end{center}

\newpage

Standardized statistic: Z - score

\vspace{1mm}

\begin{itemize}
\tightlist
\item
  \(Z = \frac{\mbox{statistic} - \mbox{null value}}{\mbox{standard error of the statistic}}\)
\end{itemize}

\vspace{0.5in}

\begin{itemize}
\tightlist
\item
  Measures the \_\_\_\_\_\_\_\_\_\_\_ of standard \_\_\_\_\_\_\_\_\_\_\_\_\_ the statistic is from the null value
\end{itemize}

Example(s): Heights of Caucasian American adult males are roughly Normally distributed with a mean of 1.72 m and a standard deviation of 0.28 m. Find and interpret the z-score for a man who is 5'4'' (1.626 m) tall. Round your answer to three decimal places.

\vspace{0.6in}

Heights of Caucasian American adult females are roughly Normally distributed with a mean of 1.59 meters and a standard deviation of 0.22 meters. Which is more unusual: a 5'4'' (1.626 m) tall male or a 5'9'' (1.753 m) tall female?

\vspace{0.6in}

In a Normal curve, the area under the curve is equal to 1, representing a probability. Therefore the shaded area represents the probability of a man being under 1.626 meters tall.

\begin{Shaded}
\begin{Highlighting}[]
\FunctionTok{library}\NormalTok{(openintro)}
\FunctionTok{normTail}\NormalTok{(}\AttributeTok{m =} \FloatTok{1.72}\NormalTok{, }\AttributeTok{s =} \FloatTok{0.28}\NormalTok{, }\AttributeTok{L =} \FloatTok{1.626}\NormalTok{)}
\FunctionTok{pnorm}\NormalTok{(}\AttributeTok{mean =} \FloatTok{1.72}\NormalTok{, }\AttributeTok{sd =} \FloatTok{0.28}\NormalTok{, }\AttributeTok{q =} \FloatTok{1.626}\NormalTok{)}
\CommentTok{\#\textgreater{} [1] 0.3685432}
\end{Highlighting}
\end{Shaded}

\begin{center}\includegraphics[width=0.6\linewidth]{04-VN04-1cat_theory_files/figure-latex/unnamed-chunk-1-1} \end{center}

\vspace{1mm}

We can also reverse that order. Given a percentage, we can find the associated percentile, or quantile. Here we display calculating the value that cuts off the lower 0.75 proportion of male adult Caucasian heights using the qnorm() function.

\begin{Shaded}
\begin{Highlighting}[]
\FunctionTok{qnorm}\NormalTok{(}\AttributeTok{mean =} \FloatTok{1.72}\NormalTok{, }\AttributeTok{sd =} \FloatTok{0.28}\NormalTok{, }\AttributeTok{p =} \FloatTok{0.75}\NormalTok{)}
\CommentTok{\#\textgreater{} [1] 1.908857}
\FunctionTok{normTail}\NormalTok{(}\AttributeTok{m =} \FloatTok{1.72}\NormalTok{, }\AttributeTok{s =} \FloatTok{0.28}\NormalTok{, }\AttributeTok{L =} \FloatTok{1.909}\NormalTok{)}
\end{Highlighting}
\end{Shaded}

\begin{center}\includegraphics[width=0.6\linewidth]{04-VN04-1cat_theory_files/figure-latex/unnamed-chunk-2-1} \end{center}

\subsection*{68-95-99.7 Rule}\label{rule}
\addcontentsline{toc}{subsection}{68-95-99.7 Rule}

\begin{itemize}
\item
  68\% of Normal distribution within 1 SD of the mean (mean -- SD, mean + SD)
\item
  95\% within (mean -- 2SD, mean + 2SD)
\item
  99.7\% within (mean -- 3SD, mean + 3SD)
\end{itemize}

\begin{center}\includegraphics[width=0.65\linewidth]{images/Empirical_Rule_Mark_bw} \end{center}

General steps of a hypothesis test

\begin{enumerate}
\def\labelenumi{\arabic{enumi}.}
\item
  Write a research question and hypotheses.
\item
  Collect data and calculate a summary statistic.
\item
  Model a sampling distribution which assumes the null hypothesis is true.
\item
  Calculate a p-value.
\item
  Draw conclusions based on a p-value.
\end{enumerate}

\newpage

\subsubsection*{Example in Video 14.3TheoryTests}\label{example-in-video-14.3theorytests}
\addcontentsline{toc}{subsubsection}{Example in Video 14.3TheoryTests}

Example: The American Red Cross reports that 10\% of US residents eligible to donate blood actually do donate. A poll conducted on a representative of 200 Montana residents eligible to donate blood found that 33 had donated blood sometime in their life. Do Montana residents donate at a different rate than US population?

Hypotheses:

In notation:

\(H_0:\)

\vspace{0.2in}

\(H_A:\)

\vspace{0.2in}

Parameter of interest:

\vspace{0.6in}

Conditions for inference using theory-based methods:

\begin{itemize}
\item
  Independence:

  \begin{itemize}
  \tightlist
  \item
    The outcome of one observation does not influence the outcome of another.
  \item
    Taking a random sample is one way to satisfy this condition.
  \end{itemize}
\item
  Large enough sample size:
\end{itemize}

\vspace{1in}

Are the conditions met to analyze the blood donations data using theory-based methods?

\vspace{1in}

To use theory-based methods to perform a hypothesis test:

\begin{itemize}
\item
  1st: Calculate the standardized statistic
\item
  2nd: Find the area under the standard normal distribution at least as extreme as the standardized statistic
\end{itemize}

Equation for the standard error of the sample proportion assuming the null hypothesis is true:

\vspace{0.5in}

\setstretch{1.5}

\begin{itemize}
\tightlist
\item
  This value measures how far each possible sample statistic is from the null value, on average.
\end{itemize}

\setstretch{1}

Equation for the standardized sample proportion:

\vspace{0.5in}

\setstretch{1.5}

\begin{itemize}
\tightlist
\item
  This value measures how many standard deviations the sample proportion is above/below the null value.
\end{itemize}

\setstretch{1}

Calculate the standardized sample proportion of Montana residents that have donated blood sometime in their life.

\begin{itemize}
\tightlist
\item
  First calculate the standard error of the sample proportion assuming the null hypothesis is true
\end{itemize}

\vspace{0.5in}

\begin{itemize}
\tightlist
\item
  Then calculate the Z score.
\end{itemize}

\vspace{0.5in}

\begin{center}\includegraphics[width=0.5\linewidth]{04-VN04-1cat_theory_files/figure-latex/standNormalc-1} \end{center}

Interpret the standardized statistic

\vspace{0.5in}

To find the p-value, find the area under the standard normal distribution at the standardized statistic and more extreme.

\begin{Shaded}
\begin{Highlighting}[]
\FunctionTok{pnorm}\NormalTok{(}\FloatTok{3.064}\NormalTok{, }\AttributeTok{lower.tail =} \ConstantTok{FALSE}\NormalTok{)}\SpecialCharTok{*}\DecValTok{2}
\end{Highlighting}
\end{Shaded}

\begin{verbatim}
#> [1] 0.002183989
\end{verbatim}

Interpretation of the p-value:

\begin{itemize}
\item
  Statement about probability or proportion of samples
\item
  Statistic (summary measure and value)
\item
  Direction of the alternative
\item
  Null hypothesis (in context)
\end{itemize}

\vspace{0.6in}

Conclusion:

\begin{itemize}
\item
  Amount of evidence
\item
  Parameter of interest
\item
  Direction of the alternative hypothesis
\end{itemize}

\vspace{0.5in}

Decision at a significance level of 0.05 \((\alpha = 0.05)\):

\vspace{0.3in}

Generalization:

\begin{itemize}
\tightlist
\item
  Can the results of the study be generalized to the target population?
\end{itemize}

\vspace{0.4in}

\subsection*{Confidence interval - 14.3TheoryIntervals}\label{confidence-interval---14.3theoryintervals}
\addcontentsline{toc}{subsection}{Confidence interval - 14.3TheoryIntervals}

\begin{itemize}
\item
  Interval of \_\_\_\_\_\_\_\_\_\_ values for the parameter of interest
\item
  \(CI = \text{statistic} \pm \text{margin of error}\)
\end{itemize}

\vspace{0.5in}

\subsubsection*{Theory-based method for a single categorical variable}\label{theory-based-method-for-a-single-categorical-variable}
\addcontentsline{toc}{subsubsection}{Theory-based method for a single categorical variable}

\begin{itemize}
\item
  \(CI = \hat{p} \pm (z^* \times SE(\hat{p}))\)
\item
  Multiplier (\(z^*\)) is the value at a certain \_\_\_\_\_\_\_\_\_\_\_\_ under the standard normal distribution
\end{itemize}

\begin{center}\includegraphics[width=0.5\linewidth]{04-VN04-1cat_theory_files/figure-latex/standardNormalcur-1} \end{center}

For a 95\% confidence interval:

\begin{Shaded}
\begin{Highlighting}[]
\FunctionTok{qnorm}\NormalTok{(}\FloatTok{0.975}\NormalTok{, }\AttributeTok{lower.tail=}\ConstantTok{TRUE}\NormalTok{)}
\end{Highlighting}
\end{Shaded}

\begin{verbatim}
#> [1] 1.959964
\end{verbatim}

\setstretch{1.5}

\begin{itemize}
\tightlist
\item
  When creating a confidence interval, we no longer assume the \_\_\_\_\_\_\_\_\_\_\_\_\_ hypothesis is true. Use \_\_\_\_\_\_\_\_ to calculate the sample to sample variability, rather than \(\pi_0\).
\end{itemize}

\setstretch{1}

Equation for the standard error of the sample proportion \emph{NOT} assuming the null is true:

\vspace{0.5in}

\newpage

Example: Estimate the true proportion of Montana residents that have donated blood at least once in their life.

Find a 95\% confidence interval:

\vspace{1in}

Confidence interval interpretation:

\begin{itemize}
\item
  How confident you are (e.g., 90\%, 95\%, 98\%, 99\%)
\item
  Parameter of interest
\item
  Calculated interval
\item
  Order of subtraction when comparing two groups
\end{itemize}

\vspace{0.8in}

\subsubsection*{Interpreting confidence level}\label{interpreting-confidence-level}
\addcontentsline{toc}{subsubsection}{Interpreting confidence level}

\setstretch{1.5}

What does it mean to be 95\% confident in a created confidence interval?

\begin{itemize}
\item
  Our goal is to only take one sample from the population to create a confidence interval.
\item
  Based on the 68-95-99.7 rule, we know that approximately \_\_\_\_\_\_\% of sample \_\_\_\_\_\_\_\_\_\_\_\_\_\_ will fall within \_\_\_\_\_\_\_\_\_\_ from the parameter.
\item
  If we create 95\% confidence intervals, \_\_\_\_\_\_\_\_\% of samples will create a 95\% \_\_\_\_\_\_\_\_\_\_\_\_\_\_ interval that will contain the \_\_\_\_\_\_\_\_\_\_\_\_\_ of interest.
\item
  95\% of samples accurately \_\_\_\_\_\_\_\_\_\_\_\_\_\_ the parameter of interest

  \begin{itemize}
  \tightlist
  \item
    When we create one confidence interval, we are 95\% \_\_\_\_\_\_\_\_\_\_\_\_\_\_\_\_ that we have a ``good'' sample that created a confidence interval that contains the \_\_\_\_\_\_\_\_\_\_\_ of interest.
  \end{itemize}
\end{itemize}

\setstretch{1}

Interpret the confidence \textbf{level} for the blood donation study.

\vspace{0.5in}

\newpage

\subsection{Concept Check}\label{concept-check-4}

Be prepared for group discussion in the next class. One member from the table should write the answers to the following on the whiteboard.

\begin{enumerate}
\def\labelenumi{\arabic{enumi}.}
\tightlist
\item
  What conditions must be met to use the Normal Distribution to approximate the sampling distribution of sampling proportions?
\end{enumerate}

\vspace{0.6in}

\begin{enumerate}
\def\labelenumi{\arabic{enumi}.}
\setcounter{enumi}{1}
\tightlist
\item
  Should the conclusion include a population word like \emph{true} or \emph{long-run}? Explain your answer.
\end{enumerate}

\vspace{0.6in}

\newpage

\section{Activity 9: Handedness of Male Boxers}\label{activity-9-handedness-of-male-boxers}

\setstretch{1}

\subsection{Learning outcomes}\label{learning-outcomes-8}

\begin{itemize}
\item
  Describe and perform a theory-based hypothesis test for a single proportion.
\item
  Check the appropriate conditions to use a theory-based hypothesis test.
\item
  Calculate and interpret the standardized sample proportion.
\item
  Interpret and evaluate a p-value for a theory-based hypothesis test for a single proportion.
\item
  Use the normal distribution to find the p-value.
\end{itemize}

\subsection{Terminology review}\label{terminology-review-7}

In this activity, we will introduce theory-based hypothesis tests for a single categorical variable. Some terms covered in this activity are:

\begin{itemize}
\item
  Parameter of interest
\item
  Standardized statistic
\item
  Normal distribution
\item
  p-value
\end{itemize}

To review these concepts, see Chapter 11 \& 14 in your textbook.

Activities from module 5 covered simulation-based methods for hypothesis tests involving a single categorical variable. This activity covers theory-based methods for testing a single categorical variable.

\subsection{Handedness of male boxers}\label{handedness-of-male-boxers}

Left-handedness is a trait that is found in about 10\% of the general population. Past studies have shown that left-handed men are over-represented among professional boxers (Richardson and Gilman 2019). The fighting claim states that left-handed men have an advantage in competition. In this random sample of 500 male professional boxers, we want to see if there is an over-prevalence of left-handed fighters. In the sample of 500 male boxers, 81 were left-handed.

\subsection{Summary statistics review}\label{summary-statistics-review}

\begin{itemize}
\item
  Download the R file for today's activity from D2L
\item
  Upload the file to the R server
\item
  Run lines 1--15 to load the needed packages and the data set and create a plot of the data
\end{itemize}

\begin{Shaded}
\begin{Highlighting}[]
 \CommentTok{\# Read in data set}
\NormalTok{boxers }\OtherTok{\textless{}{-}} \FunctionTok{read.csv}\NormalTok{(}\StringTok{"https://math.montana.edu/courses/s216/data/Male\_boxers\_sample.csv"}\NormalTok{)}
\NormalTok{boxers }\SpecialCharTok{\%\textgreater{}\%} \FunctionTok{count}\NormalTok{(Stance)  }\CommentTok{\# Count number in each Stance category}
\end{Highlighting}
\end{Shaded}

\begin{verbatim}
#>         Stance   n
#> 1  left-handed  81
#> 2 right-handed 419
\end{verbatim}

\newpage

\begin{Shaded}
\begin{Highlighting}[]
\NormalTok{boxers }\SpecialCharTok{\%\textgreater{}\%} \CommentTok{\# Data set piped into...}
    \FunctionTok{ggplot}\NormalTok{(}\FunctionTok{aes}\NormalTok{(}\AttributeTok{x =}\NormalTok{ Stance)) }\SpecialCharTok{+}   \CommentTok{\# This specifies the variable}
    \FunctionTok{geom\_bar}\NormalTok{(}\FunctionTok{aes}\NormalTok{(}\AttributeTok{y =} \FunctionTok{after\_stat}\NormalTok{(prop), }\AttributeTok{group =} \DecValTok{1}\NormalTok{)) }\SpecialCharTok{+}  \CommentTok{\# Tell it to make a bar plot with proportions}
    \FunctionTok{labs}\NormalTok{(}\AttributeTok{title =} \StringTok{"\_\_\_\_\_\_\_\_\_\_\_\_\_\_\_\_\_\_\_\_ of Handedness of Male Professional Boxers"}\NormalTok{,  }
       \CommentTok{\# Give your plot a title}
       \AttributeTok{x =} \StringTok{"Handedness"}\NormalTok{,   }\CommentTok{\# Label the x axis}
       \AttributeTok{y =} \StringTok{"Relative Frequency"}\NormalTok{)  }\CommentTok{\# Label the y axis}
\end{Highlighting}
\end{Shaded}

\begin{center}\includegraphics[width=0.5\linewidth]{04-A09-inference-1cat-theory_files/figure-latex/unnamed-chunk-2-1} \end{center}

\begin{enumerate}
\def\labelenumi{\arabic{enumi}.}
\tightlist
\item
  What type of plot was created of these data?
\end{enumerate}

\vspace{0.2in}

\subsection*{Hypotheses and summary statistics}\label{hypotheses-and-summary-statistics}
\addcontentsline{toc}{subsection}{Hypotheses and summary statistics}

\begin{enumerate}
\def\labelenumi{\arabic{enumi}.}
\setcounter{enumi}{1}
\tightlist
\item
  Write out the parameter of interest in words, in context of the study.
\end{enumerate}

\vspace{0.8in}

\begin{enumerate}
\def\labelenumi{\arabic{enumi}.}
\setcounter{enumi}{2}
\item
  Write out the null hypothesis in words.
  \vspace{0.8in}
\item
  Write out the alternative hypothesis in notation.
  \vspace{0.3in}
\item
  Give the value of the summary statistic (sample proportion) for this study. Use proper notation.
\end{enumerate}

\vspace{0.3in}

\subsection*{Theory-based methods}\label{theory-based-methods-1}
\addcontentsline{toc}{subsection}{Theory-based methods}

The sampling distribution of a single proportion --- how that proportion varies from sample to sample --- can be mathematically modeled using the normal distribution if certain conditions are met.

Conditions for the sampling distribution of \(\hat{p}\) to follow an approximate normal distribution:

\begin{itemize}
\item
  \textbf{Independence}: The sample's observations are independent, e.g., are from a simple random sample. (\emph{Remember}: This also must be true to use simulation methods!)
\item
  \textbf{Success-failure condition}: We \emph{expect} to see at least 10 successes and 10 failures in the sample, \(n\hat{p}≥10\) and \(n(1-\hat{p})≥10\).
\end{itemize}

\begin{enumerate}
\def\labelenumi{\arabic{enumi}.}
\setcounter{enumi}{5}
\tightlist
\item
  Verify that the independence condition is satisfied.
\end{enumerate}

\vspace{0.5in}

\begin{enumerate}
\def\labelenumi{\arabic{enumi}.}
\setcounter{enumi}{6}
\tightlist
\item
  Is the success-failure condition met to model the data with the normal distribution? Explain your answer in context of the problem.
\end{enumerate}

\vspace{0.8in}

To calculate the standardized statistic we use the general formula

\[
Z = \frac{\text{point estimate} - \text{null value}}{SE_0(\text{point estimate})}.
\]
For a single categorical variable the standardized sample proportion is calculated using

\[
Z = \frac{\hat{p} - \pi_0}{SE_0(\hat{p})},
\]
where the standard error is calculated using the null value:

\[SE_0(\hat{p})=\sqrt{\frac{\pi_0\times(1-\pi_0)}{n}}\].

For this study, the null standard error of the sample proportion is calculated using the null value, 0.1.

\[SE_0(\hat{p})=\sqrt{\frac{0.1\times(1-0.1)}{500}} = 0.013\].

Each sample proportion of male boxers that are left-handed is 0.013 from the true proportion of male boxers that are left-handed, on average.

\newpage

\begin{enumerate}
\def\labelenumi{\arabic{enumi}.}
\setcounter{enumi}{7}
\tightlist
\item
  Label the standard normal distribution shown below with the null value as the center value (below the value of zero). Label the tick marks to the right of the null value by adding 1 standard error to the null value to represent 1 standard error, 2 standard errors, and 3 standard errors from the null. Repeat this process to the left of the null value by subtracting 1 standard error for each tick mark.
\end{enumerate}

\vspace{2mm}

\begin{figure}

{\centering \includegraphics[width=0.5\linewidth]{04-A09-inference-1cat-theory_files/figure-latex/Normcur-1} 

}

\caption{Standard Normal Curve}\label{fig:Normcur}
\end{figure}

\begin{enumerate}
\def\labelenumi{\arabic{enumi}.}
\setcounter{enumi}{8}
\tightlist
\item
  Using the null standard error of the sample proportion, calculate the standardized sample proportion (Z). Mark this value on the standard normal distribution above.
\end{enumerate}

\vspace{0.6in}

The standardized statistic is used as a ruler to measure how far the sample statistic is from the null value. Essentially, we are converting the sample proportion into a measure of standard errors to compare to the standard normal distribution.

The standardized statistic measures the \emph{number of standard errors the sample statistic is from the null value}.

\begin{enumerate}
\def\labelenumi{\arabic{enumi}.}
\setcounter{enumi}{9}
\tightlist
\item
  Interpret the standardized sample proportion from question 9 in context of the problem.
\end{enumerate}

\vspace{.8in}

We will use the \texttt{pnorm()} function in \texttt{R} to find the p-value. The value for Z was entered into the code below to get the p-value. Check that this answer matches what you calculated in question 7. Notice that we used \texttt{lower.tail\ =\ FALSE} to find the p-value. \texttt{R} will calculate the p-value \emph{greater} than the value of the standardized statistic.

Notes:

\begin{itemize}
\item
  Use \texttt{lower.tail\ =\ TRUE} when doing a left-sided test.
\item
  Use \texttt{lower.tail\ =\ FALSE} when doing a right-sided test.
\item
  To find a two-sided p-value, use a left-sided test for negative Z or a right-sided test for positive Z, then multiply the value found by 2 to get the p-value.
\item
  Enter the value of the standardized statistic for xx
\item
  Highlight and run lines 21--23
\end{itemize}

\begin{Shaded}
\begin{Highlighting}[]
\FunctionTok{pnorm}\NormalTok{(xx, }\CommentTok{\# Enter value of standardized statistic}
      \AttributeTok{m=}\DecValTok{0}\NormalTok{, }\AttributeTok{s=}\DecValTok{1}\NormalTok{, }\CommentTok{\# Using the standard normal mean = 0, sd = 1}
      \AttributeTok{lower.tail=}\ConstantTok{FALSE}\NormalTok{) }\CommentTok{\# Gives a p{-}value greater than the standardized statistic}
\end{Highlighting}
\end{Shaded}

\begin{enumerate}
\def\labelenumi{\arabic{enumi}.}
\setcounter{enumi}{10}
\item
  Report the p-value obtained from the \texttt{R} output.
  \vspace{0.3in}
\item
  Write a conclusion based on the value of the p-value.
\end{enumerate}

\vspace{0.6in}

\subsection*{Impacts on the P-value}\label{impacts-on-the-p-value}
\addcontentsline{toc}{subsection}{Impacts on the P-value}

Suppose that we want to show that the true proportion of male boxers \textbf{differs} from that in the general population.

\begin{enumerate}
\def\labelenumi{\arabic{enumi}.}
\setcounter{enumi}{12}
\tightlist
\item
  Write out the alternative hypothesis in notation for this new research question.
\end{enumerate}

\vspace{0.5in}

\begin{enumerate}
\def\labelenumi{\arabic{enumi}.}
\setcounter{enumi}{13}
\tightlist
\item
  How would this impact the p-value?
\end{enumerate}

\vspace{0.2in}

\begin{enumerate}
\def\labelenumi{\arabic{enumi}.}
\setcounter{enumi}{14}
\tightlist
\item
  Suppose instead of 500 male boxers the researchers only took a sample of 300 male boxers and found the same proportion (\(\hat{p}=0.162\)) of male boxers that are left-handed. Since we are still assuming the same null value, 0.1, the standard error would be calculated as below:
\end{enumerate}

\[SE_0(\hat{p})=\sqrt{\frac{0.1(1-0.1)}{300}} = 0.017\].

The standardized statistic for this new sample is calculated below:

\[t = \frac{0.162-0.1}{0.017} = 3.64\]

\begin{enumerate}
\def\labelenumi{\arabic{enumi}.}
\setcounter{enumi}{15}
\tightlist
\item
  Mark the value of the original standardized statistic from question 9 and the value of the standardized statistic from the smaller sample size on the standard normal distribution below.
\end{enumerate}

\begin{figure}

{\centering \includegraphics[width=0.5\linewidth]{04-A09-inference-1cat-theory_files/figure-latex/Norcur-1} 

}

\caption{Standard Normal Curve}\label{fig:Norcur}
\end{figure}

\newpage

\begin{enumerate}
\def\labelenumi{\arabic{enumi}.}
\setcounter{enumi}{16}
\tightlist
\item
  How does the decrease in sample size affect the p-value?
\end{enumerate}

\vspace{0.3in}

\begin{enumerate}
\def\labelenumi{\arabic{enumi}.}
\setcounter{enumi}{17}
\tightlist
\item
  Suppose another sample of 500 male boxers was taken and 68 were found to be left-handed. Since we are still assuming the same null value, 0.1, the standard error would be calculated as before:
\end{enumerate}

\[SE_0(\hat{p})=\sqrt{\frac{0.1(1-0.1)}{500}} = 0.013\].

The standardized statistic for this new sample is calculated below:

\[t = \frac{0.136-0.1}{0.013} = 2.769\]
19. Mark the t-value of the original standardized statistic from question 9 and the value of the standardized statistic calculated with \(\hat{p}=0.136\) on the standard normal distribution below.

\begin{figure}

{\centering \includegraphics[width=0.5\linewidth]{04-A09-inference-1cat-theory_files/figure-latex/Norcuv-1} 

}

\caption{Standard Normal Curve}\label{fig:Norcuv}
\end{figure}

\begin{enumerate}
\def\labelenumi{\arabic{enumi}.}
\setcounter{enumi}{19}
\tightlist
\item
  How does a statistic closer to the null value affect the p-value?
\end{enumerate}

\vspace{0.3in}

\begin{enumerate}
\def\labelenumi{\arabic{enumi}.}
\setcounter{enumi}{20}
\tightlist
\item
  Summarize how each of the following affected the p-value:
\end{enumerate}

\begin{enumerate}
\def\labelenumi{\alph{enumi})}
\tightlist
\item
  Switching to a two-sided test.
\end{enumerate}

\vspace{0.4in}

\begin{enumerate}
\def\labelenumi{\alph{enumi})}
\setcounter{enumi}{1}
\tightlist
\item
  Using a smaller sample size.
\end{enumerate}

\vspace{0.4in}

\begin{enumerate}
\def\labelenumi{\alph{enumi})}
\setcounter{enumi}{2}
\tightlist
\item
  Using a sample statistic closer to the null value.
\end{enumerate}

\vspace{0.4in}

\subsection{Take-home messages}\label{take-home-messages-8}

\begin{enumerate}
\def\labelenumi{\arabic{enumi}.}
\item
  Both simulation and theory-based methods can be used to find a p-value for a hypothesis test. In order to use theory-based methods we need to check that both the independence and the success-failure conditions are met.
\item
  The standardized statistic measures how many standard errors the statistic is from the null value. The larger the standardized statistic the more evidence there is against the null hypothesis.
\item
  The p-value for a two-sided test is approximately two times the value for a one-sided test. A two-sided test provides less evidence against the null hypothesis.
\item
  The larger the sample size, the smaller the sample to sample variability. This will result in a larger standardized statistic and more evidence against the null hypothesis.
\item
  The farther the statistic is from the null value, the larger the standardized statistic. This will result in a smaller p-value and more evidence against the null hypothesis.
\end{enumerate}

\subsection{Additional notes}\label{additional-notes-8}

Use this space to summarize your thoughts and take additional notes on today's activity and material covered.

\newpage

\section{Activity 10: Confidence interval and what confidence means}\label{activity-10-confidence-interval-and-what-confidence-means}

\setstretch{1}

\subsection{Learning outcomes}\label{learning-outcomes-9}

\begin{itemize}
\item
  Explore what confidence means
\item
  Interpret the confidence level
\item
  Explore impact of sample size, direction of the alternative hypothesis, and value of the sample statistic on the p-value.
\end{itemize}

\subsection{Terminology review}\label{terminology-review-8}

In this activity, we will explore what being 95\% confidence means. Some terms covered in this activity are:

\begin{itemize}
\item
  Parameter of interest
\item
  Two-sided vs.~one-sided tests
\item
  Confidence level
\end{itemize}

\subsection{Handedness of male boxers continued}\label{handedness-of-male-boxers-continued}

We will use the male boxer study to look at what confidence means.

Left-handedness is a trait that is found in about 10\% of the general population. Past studies have shown that left-handed men are over-represented among professional boxers (Richardson and Gilman 2019). The fighting claim states that left-handed men have an advantage in competition. In this random sample of 500 male professional boxers, we want to see if there is an over-prevalence of left-handed fighters. In the sample of 500 male boxers, 81 were left-handed.

\begin{Shaded}
\begin{Highlighting}[]
 \CommentTok{\# Read in data set}
\NormalTok{boxers }\OtherTok{\textless{}{-}} \FunctionTok{read.csv}\NormalTok{(}\StringTok{"https://math.montana.edu/courses/s216/data/Male\_boxers\_sample.csv"}\NormalTok{)}
\NormalTok{boxers }\SpecialCharTok{\%\textgreater{}\%} \FunctionTok{count}\NormalTok{(Stance)  }\CommentTok{\# Count number in each Stance category}
\end{Highlighting}
\end{Shaded}

\begin{verbatim}
#>         Stance   n
#> 1  left-handed  81
#> 2 right-handed 419
\end{verbatim}

\subsection*{\texorpdfstring{What does \emph{confidence} mean?}{What does confidence mean?}}\label{what-does-confidence-mean}
\addcontentsline{toc}{subsection}{What does \emph{confidence} mean?}

In the interpretation of a 95\% confidence interval, we say that we are 95\% confident that the parameter is within the confidence interval. Why are we able to make that claim? What does it mean to say ``we are 95\% confident''?

\begin{enumerate}
\def\labelenumi{\arabic{enumi}.}
\tightlist
\item
  In the last activity we found very strong evidence that the true proportion of male professional boxers that are left-handed is greater than 0.1. As a class, determine a plausible value for the true proportion of male boxers that are left-handed. \emph{Note: we are making assumptions about the population here. This is not based on our calculated data, but we will use this applet to better understand what happens when we take many, many samples from this believed population.}
\end{enumerate}

\vspace{0.2in}

\begin{enumerate}
\def\labelenumi{\arabic{enumi}.}
\setcounter{enumi}{1}
\tightlist
\item
  Go to this website, \url{http://www.rossmanchance.com/ISIapplets.html} and choose `Simulating Confidence Intervals'. In the input on the left-hand side of the screen enter the value from question 1 for \(\pi\) (the true value), 500 for \(n\), and 100 for `Number of intervals'. Click `sample'.
\end{enumerate}

\vspace{1mm}

\begin{itemize}
\tightlist
\item
  In the graph on the bottom right, click on a green dot. Write down the confidence interval for this sample given on the graph on the left. Does this confidence interval contain the true value chosen in question 1?
\end{itemize}

\vspace{0.4in}

\begin{itemize}
\item
  Now click on a red dot. Write down the confidence interval for this sample. Does this confidence interval contain the true value chosen in question 1?
  \vspace{0.5in}
\item
  How many intervals out of 100 contain \(\pi\), the true value chosen in question 1? \emph{Hint}: This is given to the left of the graph of green and red intervals.
  \vspace{0.4in}
\end{itemize}

\begin{enumerate}
\def\labelenumi{\arabic{enumi}.}
\setcounter{enumi}{2}
\tightlist
\item
  Click on `sample' nine more times. Write down the `Running Total' for the proportion of intervals that contain \(\pi\).
\end{enumerate}

\vspace{0.5in}

\begin{enumerate}
\def\labelenumi{\arabic{enumi}.}
\setcounter{enumi}{3}
\tightlist
\item
  Change the confidence level to 90\%. What happened to the width of the intervals?
\end{enumerate}

\vspace{0.2in}

\begin{enumerate}
\def\labelenumi{\arabic{enumi}.}
\setcounter{enumi}{4}
\tightlist
\item
  Write down the \texttt{Running\ Total} for the proportion of intervals that contain \(\pi\) using a 90\% confidence level.
\end{enumerate}

\vspace{0.4in}

\begin{enumerate}
\def\labelenumi{\arabic{enumi}.}
\setcounter{enumi}{5}
\tightlist
\item
  Interpret the level of confidence. \emph{Hint}: What proportion of samples would we expect to give a confidence interval that contains the parameter of interest?
\end{enumerate}

\vspace{0.8in}

\subsubsection*{Theory-based confidence interval}\label{theory-based-confidence-interval}
\addcontentsline{toc}{subsubsection}{Theory-based confidence interval}

To calculate a theory-based 95\% confidence interval for \(\pi\), we will first find the \textbf{standard error} of \(\hat{p}\) by plugging in the value of \(\hat{p}\) for \(\pi\) in \(SD(\hat{p})\):

\[SE(\hat{p}) = \sqrt{\frac{\hat{p}\times (1-\hat{p})}{n}}\]

Note that we do not include a ``0'' subscript, since we are not assuming a null hypothesis.

\begin{enumerate}
\def\labelenumi{\arabic{enumi}.}
\setcounter{enumi}{6}
\tightlist
\item
  Calculate the standard error of the sample proportion to find a 95\% confidence interval.
\end{enumerate}

\vspace{0.5in}

We will calculate the margin of error and confidence interval in questions 10 and 11 of this activity. \textbf{The margin of error (ME)} is the value of the \(z^*\) multiplier times the standard error of the statistic.

\[ME = z^* \times SE(\hat{p})\]
The \(z^*\) multiplier is the percentile of a standard normal distribution that corresponds to our confidence level. If our confidence level is 95\%, we find the Z values that encompass the middle 95\% of the standard normal distribution. If 95\% of the standard normal distribution should be in the middle, that leaves 5\% in the tails, or 2.5\% in each tail.

The \texttt{qnorm()} function in R will tell us the \(z^*\) value for the desired percentile (in this case, 95\% + 2.5\% = 97.5\% percentile).

\begin{itemize}
\item
  Enter the value of 0.975 for xx in the provided R script file.
\item
  Highlight and run line 12. This will give the value of the multiplier for a 95\% confidence interval.
\end{itemize}

\begin{Shaded}
\begin{Highlighting}[]
\FunctionTok{qnorm}\NormalTok{(xx, }\AttributeTok{lower.tail =} \ConstantTok{TRUE}\NormalTok{) }\CommentTok{\# Multiplier for 95\% confidence interval}
\end{Highlighting}
\end{Shaded}

\begin{enumerate}
\def\labelenumi{\arabic{enumi}.}
\setcounter{enumi}{7}
\item
  Report the value of the multiplier needed to calculate the 95\% confidence interval for the true proportion of male boxers that are left-handed.
  \vspace{0.2in}
\item
  Fill in the normal distribution shown below to show how R found the \(z^*\) multiplier.
\end{enumerate}

\begin{figure}

{\centering \includegraphics[width=0.45\linewidth]{04-A10-confidenceLevel_files/figure-latex/Normalcur-1} 

}

\caption{Standard Normal Curve}\label{fig:Normalcur}
\end{figure}

\begin{enumerate}
\def\labelenumi{\arabic{enumi}.}
\setcounter{enumi}{9}
\tightlist
\item
  Calculate the margin of error for the 95\% confidence interval.
  \vspace{0.6in}
\end{enumerate}

To find the confidence interval, we will add and subtract the \textbf{margin of error} to the point estimate:
\[\text{point estimate}\pm\text{margin of error}\]
\[\hat{p}\pm z^* \times SE(\hat{p})\]

\begin{enumerate}
\def\labelenumi{\arabic{enumi}.}
\setcounter{enumi}{10}
\item
  Calculate the 95\% confidence interval for the parameter of interest.
  \vspace{0.6in}
\item
  Interpret the 95\% confidence \textbf{interval} in the context of the problem.
  \vspace{1in}
\end{enumerate}

\subsection{Take-home messages}\label{take-home-messages-9}

\begin{enumerate}
\def\labelenumi{\arabic{enumi}.}
\item
  If repeat samples of the same size are selected from the population, approximately 95\% of samples will create a 95\% confidence interval that contains the parameter of interest.
\item
  The calculation of the confidence interval uses the standard error calculated using the sample proportion rather than the null value.
\end{enumerate}

\subsection{Additional notes}\label{additional-notes-9}

Use this space to summarize your thoughts and take additional notes on today's activity and material covered.

\newpage

\section{Module 3 and 4 Lab: Mixed Breed Dogs in the U.S.}\label{module-3-and-4-lab-mixed-breed-dogs-in-the-u.s.}

\setstretch{1}

\subsection{Learning outcomes}\label{learning-outcomes-10}

\begin{itemize}
\item
  Determine whether simulation or theory-based methods of inference can be used.
\item
  Analyze and interpret a study involving a single categorical variable.
\end{itemize}

\subsection{Mixed Breed Dogs in the U.S.}\label{mixed-breed-dogs-in-the-u.s.}

The American Veterinary Medical Association estimated in 2010 that approximately 49\% of dog owners in the U.S. own dogs that are classified as ``mixed breed.'' As part of a larger 2022 international study (Banton 2022) about overall dog health, survey participants were asked, among other things, to report whether their dog was purebred or a mixed breed. Seven hundred and fifty (750) dog owners from the U.S. were recruited to complete an online survey via an email indicating they had been randomly selected by Qualtrics (an ``experience management'' company that specializes in surveys). Three hundred sixty-four (364) out of 675 respondents from the U.S. reported they owned a mixed breed dog. Is there evidence that, in the last decade, the proportion of dog owners in the U.S. that own a mixed breed dog has changed from the value reported in 2010?

\subsubsection*{Activity intro}\label{activity-intro-1}
\addcontentsline{toc}{subsubsection}{Activity intro}

\begin{itemize}
\item
  Download the R script file and the data file (US\_dogs.csv) from D2L
\item
  Upload both files to D2L and open the R script file
\item
  Enter the name of the dataset for datasetname.csv.
\item
  Highlight and run lines 1 - 6
\end{itemize}

\begin{enumerate}
\def\labelenumi{\arabic{enumi}.}
\tightlist
\item
  What is the value of the point estimate?
\end{enumerate}

\vspace{0.3in}

\begin{enumerate}
\def\labelenumi{\arabic{enumi}.}
\setcounter{enumi}{1}
\tightlist
\item
  Create a plot of the data using the R code. Make sure to include an appropriate title with type of plot, observational units, and variable. \textbf{Upload the plot to Gradescope}.
\end{enumerate}

\begin{Shaded}
\begin{Highlighting}[]
\NormalTok{dogs }\SpecialCharTok{\%\textgreater{}\%} \CommentTok{\# Data set piped into...}
    \FunctionTok{ggplot}\NormalTok{(}\FunctionTok{aes}\NormalTok{(}\AttributeTok{x =}\NormalTok{ variable)) }\SpecialCharTok{+}   \CommentTok{\# This specifies the variable}
    \FunctionTok{geom\_bar}\NormalTok{(}\FunctionTok{aes}\NormalTok{(}\AttributeTok{y =} \FunctionTok{after\_stat}\NormalTok{(prop), }\AttributeTok{group =} \DecValTok{1}\NormalTok{)) }\SpecialCharTok{+}  \CommentTok{\# Tell it to make a bar plot with proportions}
    \FunctionTok{labs}\NormalTok{(}\AttributeTok{title =} \StringTok{"Don\textquotesingle{}t forget to title your plot"}\NormalTok{,  }
       \CommentTok{\# Give your plot a title}
       \AttributeTok{x =} \StringTok{"Breed of Dog"}\NormalTok{,   }\CommentTok{\# Label the x axis}
       \AttributeTok{y =} \StringTok{"Relative Frequency"}\NormalTok{)  }\CommentTok{\# Label the y axis}
\end{Highlighting}
\end{Shaded}

\subsubsection*{Use statistical analysis methods to draw inferences from the data}\label{use-statistical-analysis-methods-to-draw-inferences-from-the-data-2}
\addcontentsline{toc}{subsubsection}{Use statistical analysis methods to draw inferences from the data}

\begin{enumerate}
\def\labelenumi{\arabic{enumi}.}
\setcounter{enumi}{2}
\tightlist
\item
  \textbf{Write out the parameter of interest in words, in context of the study.}
\end{enumerate}

\vspace{0.5in}

\begin{enumerate}
\def\labelenumi{\arabic{enumi}.}
\setcounter{enumi}{3}
\tightlist
\item
  Write out the null and alternative hypotheses in notation.
\end{enumerate}

\vspace{1mm}

\(H_0:\)

\vspace{0.3in}

\(H_A:\)

\vspace{0.3in}

\begin{enumerate}
\def\labelenumi{\arabic{enumi}.}
\setcounter{enumi}{4}
\tightlist
\item
  \textbf{Will theory-based methods give the sample results as simulation based methods? Explain your answer.}
\end{enumerate}

\vspace{0.6in}

To use the computer simulation, we will need to enter the
* assumed ``probability of success'' (\(\pi_0\)),
* ``sample size'' (the number of observational units or cases in the sample),
* ``number of repetitions'' (the number of samples to be generated),
* ``as extreme as'' (the observed statistic), and
* the ``direction'' (matches the direction of the alternative hypothesis).

We will use the \texttt{one\_proportion\_test()} function in \texttt{R} (in the \texttt{catstats} package) to simulate the null distribution of sample proportions and compute a p-value. Using the provided \texttt{R} script file, fill in the values/words for each \texttt{xx} with your answers from question 5 in the one proportion test to create a null distribution with 1000 simulations. Then highlight and run lines 21--26.

\begin{Shaded}
\begin{Highlighting}[]
\FunctionTok{one\_proportion\_test}\NormalTok{(}\AttributeTok{probability\_success =}\NormalTok{ xx, }\CommentTok{\# Null hypothesis value}
          \AttributeTok{sample\_size =}\NormalTok{ xx, }\CommentTok{\# Enter sample size}
          \AttributeTok{number\_repetitions =} \DecValTok{1000}\NormalTok{, }\CommentTok{\# Enter number of simulations}
          \AttributeTok{as\_extreme\_as =}\NormalTok{ xx, }\CommentTok{\# Observed statistic}
          \AttributeTok{direction =} \StringTok{"xx"}\NormalTok{, }\CommentTok{\# Specify direction of alternative hypothesis}
          \AttributeTok{summary\_measure =} \StringTok{"proportion"}\NormalTok{) }\CommentTok{\# Reporting proportion or number of successes?}
\end{Highlighting}
\end{Shaded}

\begin{enumerate}
\def\labelenumi{\arabic{enumi}.}
\setcounter{enumi}{5}
\tightlist
\item
  Report the p-value from the study.
\end{enumerate}

\vspace{0.2in}

The \(z^*\) multiplier is the percentile of a standard normal distribution that corresponds to our confidence level.

\begin{itemize}
\item
  Enter the value of the appropriate percentile for xx in the provided R script file to find the multiplier for a 90\% confidence interval.
\item
  Highlight and run line 17
\end{itemize}

\begin{Shaded}
\begin{Highlighting}[]
\FunctionTok{qnorm}\NormalTok{(xx. }\AttributeTok{lower.tail =} \ConstantTok{TRUE}\NormalTok{) }\CommentTok{\# Multiplier for 90\% confidence interval}
\end{Highlighting}
\end{Shaded}

\begin{enumerate}
\def\labelenumi{\arabic{enumi}.}
\setcounter{enumi}{6}
\tightlist
\item
  \textbf{Calculate the margin of error for a 90\% confidence interval.}
\end{enumerate}

\vspace{0.5in}

\begin{enumerate}
\def\labelenumi{\arabic{enumi}.}
\setcounter{enumi}{7}
\tightlist
\item
  Calculate a 90\% confidence interval.
\end{enumerate}

\vspace{0.6in}

\subsubsection*{Summarize the results of the study}\label{summarize-the-results-of-the-study}
\addcontentsline{toc}{subsubsection}{Summarize the results of the study}

\begin{enumerate}
\def\labelenumi{\arabic{enumi}.}
\setcounter{enumi}{8}
\tightlist
\item
  Write a paragraph summarizing the results of the study. Be sure to describe:
\end{enumerate}

\begin{itemize}
\item
  Summary statistic and interpretation

  \begin{itemize}
  \item
    Summary measure (in context)
  \item
    Value of the statistic
  \item
    Order of subtraction when comparing two groups
  \end{itemize}
\item
  P-value and interpretation

  \begin{itemize}
  \item
    Statement about probability or proportion of samples
  \item
    Statistic (summary measure and value)
  \item
    Direction of the alternative
  \item
    Null hypothesis (in context)
  \end{itemize}
\item
  Confidence interval and interpretation

  \begin{itemize}
  \item
    How confident you are (e.g., 90\%, 95\%, 98\%, 99\%)
  \item
    Parameter of interest
  \item
    Calculated interval
  \item
    Order of subtraction when comparing two groups
  \end{itemize}
\item
  Conclusion (written to answer the research question)

  \begin{itemize}
  \item
    Amount of evidence
  \item
    Parameter of interest
  \item
    Direction of the alternative hypothesis
  \end{itemize}
\item
  Scope of inference

  \begin{itemize}
  \item
    To what group of observational units do the results apply (target population or observational units similar to the sample)?
  \item
    What type of inference is appropriate (causal or non-causal)?
  \end{itemize}
\end{itemize}

\textbf{Upload a copy of your group's paragraph to Gradescope.}

\newpage

Paragraph (continued):

\newpage

\chapter{Unit 1 Review}\label{unit-1-review}

The following module contains both a list of key topics covered in Unit 1 as well as Module Review Worksheets that will be covered in Weekly Review Sessions.

\section{Module Review}\label{module-review}

\setstretch{1}

The following worksheets review each of the modules. These worksheets will be completed during Melinda's Study Sessions each week. Solutions will be posted on D2L in the Unit 1 Review folder after the study sessions.

\section{Key Topics}\label{key-topics-2}

Review the key topics for Unit 1 prior to the first exams. All of these topics will be covered in Modules 1 - 4.

\newpage

\section{Module 1 Review}\label{module-1-review}

\begin{enumerate}
\def\labelenumi{\arabic{enumi}.}
\tightlist
\item
  Suppose that the proportion of all American adults that fit the medical definition of being obese is 0.23. A large medical clinic would like to determine if the proportion of their patients that are obese is higher than that of all American adults. The clinic takes a simple random sample of 30 of their patients and finds that 9 patients in the sample are obese.
\end{enumerate}

\begin{enumerate}
\def\labelenumi{\alph{enumi}.}
\tightlist
\item
  What is the target population?
\end{enumerate}

\vspace{0.4in}

\begin{enumerate}
\def\labelenumi{\alph{enumi}.}
\setcounter{enumi}{1}
\tightlist
\item
  What are the observational units?
\end{enumerate}

\vspace{0.4in}

\begin{enumerate}
\def\labelenumi{\alph{enumi}.}
\setcounter{enumi}{2}
\tightlist
\item
  What variable is being studied?
\end{enumerate}

\vspace{0.4in}

\begin{enumerate}
\def\labelenumi{\alph{enumi}.}
\setcounter{enumi}{3}
\tightlist
\item
  Is the variable identified in part (c) categorical or quantitative?
\end{enumerate}

\vspace{0.4in}

\begin{enumerate}
\def\labelenumi{\arabic{enumi}.}
\setcounter{enumi}{1}
\tightlist
\item
  Martha works in Macy's advertising department. She is interested in the shopping experience of all Macy's shoppers in the U.S. Every Saturday morning for a month she stands outside of the Bozeman Macy's asking people about their experience. One of the questions she uses is: ``As a huge fan of Macy's, I believe Macy's has the best choices of clothing in Bozeman. Don't you agree?'' Every person that was asked, responded.
\end{enumerate}

\begin{enumerate}
\def\labelenumi{\alph{enumi}.}
\tightlist
\item
  Identify the target population.
\end{enumerate}

\vspace{0.4in}

\begin{enumerate}
\def\labelenumi{\alph{enumi}.}
\setcounter{enumi}{1}
\tightlist
\item
  Identify the sample.
\end{enumerate}

\vspace{0.4in}

\begin{enumerate}
\def\labelenumi{\alph{enumi}.}
\setcounter{enumi}{2}
\tightlist
\item
  Which of the three types of sampling bias (selection, non-response, response) may be present? Explain your choice(s).
\end{enumerate}

\newpage

\section{Module 2 Review}\label{module-2-review}

\begin{enumerate}
\def\labelenumi{\arabic{enumi}.}
\tightlist
\item
  Spelling errors in a text can either be non-word errors (teh instead of the) or word errors (lose instead of loose). It was found that non-word errors make up about 25\% of all errors. A human proofreader will catch 92\% of non-word errors and 75\% of word errors.
\end{enumerate}

Let N represent non-word errors and C represent that a human proofreader will catch the error.

\begin{enumerate}
\def\labelenumi{\alph{enumi}.}
\item
  Identify the following values with appropriate probability notation.
  \vspace{2mm}

  \(0.25\)
  \vspace{2mm}

  \(0.92\)
  \vspace{2mm}

  \(0.75\)
  \vspace{2mm}
\item
  Fill in the table below to represent the situation:
\end{enumerate}

\begin{center}
\begin{tabular}{|c|c|c|c|} \hline
\hspace{0.8in} & \hspace{0.25in}  $N$ \hspace{.25in} & \hspace{0.25in} $N^C$ \hspace{0.25in} & \hspace{0.25in} Total \hspace{0.25in} \\ \hline
 $C$ &  &  &  \\ \hline
 $C^C$ &  & &  \\ \hline
Total &  &  & 100000 \\ \hline
\end{tabular}
\end{center}
\vspace{.1in}

\begin{enumerate}
\def\labelenumi{\alph{enumi}.}
\setcounter{enumi}{2}
\tightlist
\item
  Using your table calculate the probability that a randomly selected error caught by a human proofreader is a non-word error. Use appropriate probability notation.
\end{enumerate}

\vspace{1in}

\begin{enumerate}
\def\labelenumi{\alph{enumi}.}
\setcounter{enumi}{3}
\tightlist
\item
  Find the probability a selected error is a non-word error and was not caught by a human proofreader. Use appropriate probability notation.
\end{enumerate}

\vspace{1in}

\begin{enumerate}
\def\labelenumi{\alph{enumi}.}
\setcounter{enumi}{4}
\tightlist
\item
  Find the value of \(P(N|C)\). What does this probability mean?
\end{enumerate}

\vspace{1in}

\begin{enumerate}
\def\labelenumi{\arabic{enumi}.}
\setcounter{enumi}{1}
\tightlist
\item
  A private college report contains these statistics:
\end{enumerate}

\begin{itemize}
\item
  70\% of incoming freshmen attended public schools
\item
  75\% of public-school students who enroll as freshmen eventually graduate
\item
  90\% of other freshmen eventually graduate
\end{itemize}

Let A represent the event that a freshman attended public school and B the event that a freshman eventually graduates.

\begin{enumerate}
\def\labelenumi{\alph{enumi}.}
\tightlist
\item
  Identify the following values with appropriate probability notation.
\end{enumerate}

\vspace{2mm}

\begin{verbatim}
 = 0.70
\end{verbatim}

\vspace{2mm}

\begin{verbatim}
= 0.75
\end{verbatim}

\vspace{2mm}

\begin{verbatim}
= 0.90
\end{verbatim}

\vspace{2mm}

\begin{enumerate}
\def\labelenumi{\alph{enumi}.}
\setcounter{enumi}{1}
\tightlist
\item
  Fill in the table below to represent the situation:
\end{enumerate}

\begin{center}
\begin{tabular}{|c|c|c|c|} \hline
\hspace{0.8in} & \hspace{0.25in}  $A$ \hspace{.25in} & \hspace{0.25in} $A^C$ \hspace{0.25in} & \hspace{0.25in} Total \hspace{0.25in} \\ \hline
 $B$ &  &  &  \\ \hline
 $B^C$ &  & &  \\ \hline
Total &  &  & 100000 \\ \hline
\end{tabular}
\end{center}
\vspace{.1in}

\begin{enumerate}
\def\labelenumi{\alph{enumi}.}
\setcounter{enumi}{2}
\tightlist
\item
  Calculate the probability a selected freshman attended public school given they did not graduate. Use appropriate probability notation.
\end{enumerate}

\vspace{1in}

\begin{enumerate}
\def\labelenumi{\alph{enumi}.}
\setcounter{enumi}{3}
\tightlist
\item
  Calculate the probability a selected freshman does not graduate. Use appropriate probability notation.
\end{enumerate}

\vspace{1in}

\begin{enumerate}
\def\labelenumi{\alph{enumi}.}
\setcounter{enumi}{4}
\tightlist
\item
  Of the population of freshman that attended public school, what is the probability they do not graduate. Use appropriate probability notation.
\end{enumerate}

\vspace{1in}

\begin{enumerate}
\def\labelenumi{\alph{enumi}.}
\setcounter{enumi}{5}
\tightlist
\item
  Find the value of \(P(A \text{and} B^C)\). Write this probability in context of the problem.
\end{enumerate}

\newpage

\section{Module 3 Review - Simulation Methods}\label{module-3-review---simulation-methods}

\begin{Shaded}
\begin{Highlighting}[]
\NormalTok{hearing }\OtherTok{\textless{}{-}} \FunctionTok{read.csv}\NormalTok{(}\StringTok{"data/hearing\_loss.csv"}\NormalTok{)}
\end{Highlighting}
\end{Shaded}

A recent study examined hearing loss data for 1753 U.S. teenagers. In this sample, 328 were found to have some level of hearing loss. News of this study spread quickly, with many news articles blaming the prevalence of hearing loss on the higher use of ear buds by teens. At MSNBC.com (8/17/2010), Carla Johnson summarized the study with the headline: ``1 in 5 U.S. teens has hearing loss, study says.'' Is this an appropriate or a misleading headline?

\begin{enumerate}
\def\labelenumi{\arabic{enumi}.}
\tightlist
\item
  Write the parameter of interest in context of the study.
\end{enumerate}

\vspace{0.8in}

\begin{enumerate}
\def\labelenumi{\arabic{enumi}.}
\setcounter{enumi}{1}
\tightlist
\item
  Write the null hypothesis in words and notation in context of the problem.
\end{enumerate}

\vspace{1in}

\begin{enumerate}
\def\labelenumi{\arabic{enumi}.}
\setcounter{enumi}{2}
\tightlist
\item
  Based on the research questions, choose the direction for the alternative hypothesis.
\end{enumerate}

\vspace{0.3in}

\begin{enumerate}
\def\labelenumi{\arabic{enumi}.}
\setcounter{enumi}{3}
\tightlist
\item
  Write the alternative hypothesis in words and notation in context of the problem.
\end{enumerate}

\vspace{1in}

\begin{enumerate}
\def\labelenumi{\arabic{enumi}.}
\setcounter{enumi}{4}
\tightlist
\item
  Calculate the summary statistic. Use proper notation.
\end{enumerate}

\vspace{0.3in}
\newpage

\begin{enumerate}
\def\labelenumi{\arabic{enumi}.}
\setcounter{enumi}{5}
\tightlist
\item
  What values should be entered for each of the following into the one proportion test to create 1000 simulations?
\end{enumerate}

\begin{itemize}
\tightlist
\item
  Probability of success:
\end{itemize}

\vspace{0.2in}

\begin{itemize}
\tightlist
\item
  Sample size:
\end{itemize}

\vspace{0.2in}

\begin{itemize}
\tightlist
\item
  Number of repetitions:
\end{itemize}

\vspace{0.2in}

\begin{itemize}
\tightlist
\item
  As extreme as:
\end{itemize}

\vspace{0.2in}

\begin{itemize}
\tightlist
\item
  Direction (``greater'', ``less'', or ``two-sided''):
\end{itemize}

\vspace{0.2in}

\begin{center}\includegraphics[width=0.7\linewidth]{05-UR-module3_review_files/figure-latex/unnamed-chunk-2-1} \end{center}

\begin{enumerate}
\def\labelenumi{\arabic{enumi}.}
\setcounter{enumi}{6}
\tightlist
\item
  Interpret the p-value in context of the problem.
\end{enumerate}

\vspace{1in}

\begin{enumerate}
\def\labelenumi{\arabic{enumi}.}
\setcounter{enumi}{7}
\tightlist
\item
  How much evidence does the data provide against the null hypothesis?
\end{enumerate}

\begin{center}\includegraphics[width=0.9\linewidth]{images/soe_gradient_gray} \end{center}

\vspace{0.2in}

\begin{enumerate}
\def\labelenumi{\arabic{enumi}.}
\setcounter{enumi}{8}
\tightlist
\item
  Write a conclusion to the study in context of the problem.
\end{enumerate}

\vspace{0.8in}

\begin{enumerate}
\def\labelenumi{\arabic{enumi}.}
\setcounter{enumi}{9}
\tightlist
\item
  Would a 95\% confidence interval contain the null value of 0.2? Explain.
\end{enumerate}

\vspace{0.8in}

\begin{enumerate}
\def\labelenumi{\arabic{enumi}.}
\setcounter{enumi}{10}
\tightlist
\item
  What values should be entered for each of the following into the simulation to create the bootstrap distribution of sample proportions to find a 95\% confidence interval?
  \vspace{1mm}
\end{enumerate}

\begin{itemize}
\tightlist
\item
  Sample size:
\end{itemize}

\vspace{.1in}

\begin{itemize}
\tightlist
\item
  Number of successes:
\end{itemize}

\vspace{.1in}

\begin{itemize}
\tightlist
\item
  Number of repetitions:
\end{itemize}

\vspace{.1in}

\begin{itemize}
\tightlist
\item
  Confidence level (as a decimal):
\end{itemize}

\vspace{.1in}

\newpage

\begin{Shaded}
\begin{Highlighting}[]
\FunctionTok{set.seed}\NormalTok{(}\DecValTok{216}\NormalTok{)}
\FunctionTok{one\_proportion\_bootstrap\_CI}\NormalTok{(}\AttributeTok{sample\_size =} \DecValTok{1753}\NormalTok{, }\CommentTok{\# Sample size}
                    \AttributeTok{number\_successes =} \DecValTok{328}\NormalTok{, }\CommentTok{\# Observed number of successes}
                    \AttributeTok{number\_repetitions =} \DecValTok{10000}\NormalTok{, }\CommentTok{\# Number of bootstrap samples to use}
                    \AttributeTok{confidence\_level =} \FloatTok{0.95}\NormalTok{) }\CommentTok{\# Confidence level as a decimal}
\end{Highlighting}
\end{Shaded}

\begin{center}\includegraphics[width=0.7\linewidth]{05-UR-module3_review_files/figure-latex/unnamed-chunk-4-1} \end{center}

\begin{enumerate}
\def\labelenumi{\arabic{enumi}.}
\setcounter{enumi}{11}
\tightlist
\item
  Explain how to use cards to create one bootstrap sample.
\end{enumerate}

\vspace{1in}

\begin{enumerate}
\def\labelenumi{\arabic{enumi}.}
\setcounter{enumi}{12}
\tightlist
\item
  Report the 95\% confidence interval in interval notation.
\end{enumerate}

\vspace{0.2in}

\begin{enumerate}
\def\labelenumi{\arabic{enumi}.}
\setcounter{enumi}{13}
\tightlist
\item
  Interpret the 95\% confidence interval in context of the problem.
\end{enumerate}

\vspace{0.8in}

\newpage

\section{Module 4 Review}\label{module-4-review}

Statistician Jessica Utts has conducted an extensive analysis of Ganzfeld studies that have investigated psychic functioning. Ganzfeld studies involve a ``sender'' and a ``receiver.'' Two people are placed in separate rooms. The sender looks at a ``target'' image on a television screen and attempts to transmit information about the target to the receiver. The receiver is then shown four possible choices or targets, one of which is the correct target and the other three are ``decoys.'' The receiver must choose the one he or she thinks best matches the description transmitted by the sender. If the correct target is chosen by the receiver, the session is a ``hit.'' Otherwise, it is a miss. Utts reported that her analysis considered a total of 2,124 sessions and found a total of 709 ``hits'' (Utts, 2010). Is there evidence of psychic ability?

\begin{enumerate}
\def\labelenumi{\arabic{enumi}.}
\tightlist
\item
  Write the parameter of interest in context of the study.
\end{enumerate}

\vspace{0.6in}

\begin{enumerate}
\def\labelenumi{\arabic{enumi}.}
\setcounter{enumi}{1}
\tightlist
\item
  Calculate the point estimate. Use proper notation.
\end{enumerate}

\vspace{0.3in}

\begin{enumerate}
\def\labelenumi{\arabic{enumi}.}
\setcounter{enumi}{2}
\tightlist
\item
  Write the null hypothesis in words.
\end{enumerate}

\vspace{0.6in}

\begin{enumerate}
\def\labelenumi{\arabic{enumi}.}
\setcounter{enumi}{3}
\tightlist
\item
  Write the alternative hypothesis in notation.
\end{enumerate}

\vspace{0.2in}

A single proportion can be mathematically modeled using the normal distribution if certain conditions are met.

Conditions for the sample distribution of \(\hat{p}\).

\begin{itemize}
\item
  Independence: The sample's observations are independent, e.g., are from a simple random sample
\item
  Large enough sample size:

  \begin{itemize}
  \tightlist
  \item
    Success-Failure Condition: There are at least 10 successes and 10 failures in the sample
  \end{itemize}
\end{itemize}

\[n \times \hat{p} \ge 10\] and \[n \times (1-\hat{p}) \ge 10\]

\begin{enumerate}
\def\labelenumi{\arabic{enumi}.}
\setcounter{enumi}{4}
\tightlist
\item
  Are the conditions met to model the data with the Normal distribution?
\end{enumerate}

\vspace{0.6in}

Standardized sample proportion.

The standardized statistic for theory-based methods for one proportion is:

\[Z = \frac{\hat{p}-\pi_0}{SE_0(\hat{p})}\]

Where \[SE_0(\hat{p})=\sqrt\frac{\pi_0\times (1-\pi_0)}{n}\]

\begin{enumerate}
\def\labelenumi{\arabic{enumi}.}
\setcounter{enumi}{5}
\tightlist
\item
  Calculate the null standard error of the sample proportion
\end{enumerate}

\vspace{0.6in}

\begin{enumerate}
\def\labelenumi{\arabic{enumi}.}
\setcounter{enumi}{6}
\tightlist
\item
  Calculate the standardized statistic for the sample proportion.
\end{enumerate}

\vspace{0.4in}

\begin{figure}

{\centering \includegraphics[width=0.5\linewidth]{05-UR-module4_review_files/figure-latex/simpleNormalcurve-1} 

}

\caption{A standard normal curve.}\label{fig:simpleNormalcurve}
\end{figure}

\begin{center}\includegraphics[width=0.5\linewidth]{05-UR-module4_review_files/figure-latex/Normcurve-1} \end{center}

\newpage

\begin{enumerate}
\def\labelenumi{\arabic{enumi}.}
\setcounter{enumi}{7}
\tightlist
\item
  Interpret the standardized statistic in context of the problem.
\end{enumerate}

\vspace{1in}

We will use the \texttt{pnorm()} function in \texttt{R} to find the p-value. The value of the standardized statistic calculated in question 8 is entered into the \texttt{R} code. We used \texttt{lower.tail\ =\ FALSE} to find the p-value so that \texttt{R} will calculate the p-value \emph{greater} than the value of the standardized statistic.

Notes:

\begin{itemize}
\tightlist
\item
  Use \texttt{lower.tail\ =\ TRUE} when doing a left-sided test.
\item
  Use \texttt{lower.tail\ =\ FALSE} when doing a right-sided test.
\item
  To find a two-sided p-value, use a left-sided test for negative Z or a right-sided test for positive Z, then multiply the value found by 2 to get the p-value.
\end{itemize}

\begin{Shaded}
\begin{Highlighting}[]
\FunctionTok{pnorm}\NormalTok{(}\FloatTok{9.333}\NormalTok{, }\CommentTok{\# Enter value of standardized statistic}
      \AttributeTok{m=}\DecValTok{0}\NormalTok{, }\AttributeTok{s=}\DecValTok{1}\NormalTok{, }\CommentTok{\# Using the standard normal mean = 0, sd = 1}
      \AttributeTok{lower.tail=}\ConstantTok{FALSE}\NormalTok{) }\CommentTok{\# Gives a p{-}value greater than the standardized statistic}
\CommentTok{\#\textgreater{} [1] 5.145792e{-}21}
\end{Highlighting}
\end{Shaded}

\begin{enumerate}
\def\labelenumi{\arabic{enumi}.}
\setcounter{enumi}{8}
\tightlist
\item
  Report the value of the p-value.
\end{enumerate}

\vspace{0.1in}

Simulation Method:

\begin{center}\includegraphics[width=0.85\linewidth]{05-UR-module4_review_files/figure-latex/unnamed-chunk-2-1} \end{center}

\begin{enumerate}
\def\labelenumi{\arabic{enumi}.}
\setcounter{enumi}{9}
\tightlist
\item
  Interpret the p-value in context of the study.
\end{enumerate}

\vspace{0.8in}

Next we will use theory-based methods to estimate the parameter of interest.

To calculate a theory-based 95\% confidence interval for \(\pi\), we will first find the \textbf{standard error} of \(\hat{p}\) by plugging in the value of \(\hat{p}\) for \(\pi\) in \(SD(\hat{p})\):

\[SE(\hat{p}) = \sqrt{\frac{\hat{p}\times(1-\hat{p})}{n}}.\]
Note that we do not include a ``0'' subscript, since we are not assuming a null hypothesis.

\begin{enumerate}
\def\labelenumi{\arabic{enumi}.}
\setcounter{enumi}{10}
\tightlist
\item
  Calculate the standard error of the sample proportion to find a 95\% confidence interval.
\end{enumerate}

\vspace{0.5in}

To find the confidence interval, we will add and subtract the \textbf{margin of error} to the point estimate:

\[\text{point estimate}\pm\text{margin of error}\]
\[\hat{p}\pm z^* SE(\hat{p})\]

The \(z^*\) multiplier is the percentile of a standard normal distribution that corresponds to our confidence level. If our confidence level is 95\%, we find the Z values that encompass the middle 95\% of the standard normal distribution. If 95\% of the standard normal distribution should be in the middle, that leaves 5\% in the tails, or 2.5\% in each tail. The \texttt{qnorm()} function in \texttt{R} will tell us the \(z^*\) value for the desired percentile (in this case, 95\% + 2.5\% = 97.5\% percentile).

\begin{figure}

{\centering \includegraphics[width=0.5\linewidth]{05-UR-module4_review_files/figure-latex/Ncurve-1} 

}

\caption{A standard normal curve.}\label{fig:Ncurve}
\end{figure}

\begin{Shaded}
\begin{Highlighting}[]
\FunctionTok{qnorm}\NormalTok{(}\FloatTok{0.975}\NormalTok{) }\CommentTok{\# Multiplier for 95\% confidence interval}
\end{Highlighting}
\end{Shaded}

\begin{verbatim}
#> [1] 1.959964
\end{verbatim}

\begin{enumerate}
\def\labelenumi{\arabic{enumi}.}
\setcounter{enumi}{11}
\tightlist
\item
  Calculate the margin of error for a 95\% confidence interval for the true proportion of sessions that will result in a hit.
\end{enumerate}

\vspace{0.6in}

\begin{enumerate}
\def\labelenumi{\arabic{enumi}.}
\setcounter{enumi}{12}
\tightlist
\item
  Calculate the 95\% confidence interval for the true proportion of sessions that will result in a hit.
\end{enumerate}

\vspace{1in}

\newpage

Simulation Methods:

\begin{center}\includegraphics[width=0.85\linewidth]{05-UR-module4_review_files/figure-latex/unnamed-chunk-4-1} \end{center}

\begin{enumerate}
\def\labelenumi{\arabic{enumi}.}
\setcounter{enumi}{13}
\item
  Interpret the 95\% confidence interval in context of the problem.
  \vspace{0.6in}
\item
  Write a conclusion based on the p-value and the 95\% confidence interval.
\end{enumerate}

\vspace{0.6in}

\newpage

\section{Unit 1 Review}\label{unit-1-review-1}

\section{Key Topics Exam 1}\label{key-topics-exam-1}

Descriptive statistics and study design:

\begin{enumerate}
\def\labelenumi{\arabic{enumi}.}
\item
  Identify the observational units.
\item
  Identify the types of variables (categorical or quantitative).
\item
  Identify the explanatory variable (if present) and the response variable (roles of variables).
\item
  Identify the appropriate type of graph and summary measure.
\item
  Identify if a given value is a statistic or a parameter. Identify the appropriate notation.
\item
  Identify the study design (observational study or randomized experiment).
\item
  Identify the sampling method and potential types of sampling bias (non-response, response, selection).
\item
  Identify and interpret the summary statistic
\item
  Identify the target population
\item
  Identify the types of sampling bias (response, non-response, selection, none)
\item
  Identify the type(s) of graph(s) that could be used to plot the given variable(s).
\end{enumerate}

Hypothesis testing:

\begin{enumerate}
\def\labelenumi{\arabic{enumi}.}
\setcounter{enumi}{11}
\item
  Write the parameter of interest in context of the problem.
\item
  State the null and alternative hypotheses in both words and notation
\item
  Verify the validity condition is met to use simulation-based methods to find a p-value.
\item
  Verify the validity conditions are met to use theory-based methods to find a p-value from the theoretical distribution.
\item
  In a simulation-based hypothesis test, describe how to create one dot on a dotplot of the null distribution using coins, cards, or spinners.
\item
  Explain where the null distribution is centered and why.
\item
  Describe and illustrate how R calculates the p-value for a simulation-based test.
\item
  Describe and illustrate how R calculates the p-value for a theory-based test.
\item
  Type of theoretical distribution (standard normal distribution or t-distribution with appropriate degrees of freedom) used to model the standardized statistic in a theory-based hypothesis test.
\item
  Calculate and interpret the standard error of the statistic under the null using the correct formula on the Golden ticket.
\item
  Calculate and interpret the appropriate standardized statistic using the correct formula on the Golden ticket.
\item
  Interpret the p-value in context of the study: it is the probability of \_\_\_\_, assuming \_\_\_\_.
\item
  Evaluate the p-value for strength of evidence against the null: how much evidence does the p-value provide against the null?
\item
  Write a conclusion about the research question based on the p-value.
\item
  Describe which features of the study impact the p-value and how.
\end{enumerate}

Confidence interval:

\begin{enumerate}
\def\labelenumi{\arabic{enumi}.}
\setcounter{enumi}{26}
\item
  Describe how to simulate one bootstrapped sample using cards.
\item
  Explain where the bootstrap distribution is centered and why.
\item
  Find an appropriate percentile confidence interval using a bootstrap distribution from R output.
\item
  Verify the validity condition is met to use simulation-based methods to find the confidence interval.
\item
  Verify the validity conditions are met to use theory-based methods to calculate a confidence interval.
\item
  Describe and illustrate how the bootstrap distribution is used to find the confidence interval for a given confidence level.
\item
  Describe and illustrate how the standard normal distribution or t-distribution is used to find the multiplier for a given confidence level.
\item
  Calculate and interpret the standard error of the statistic (not assuming the null hypothesis) using the correct formula on the Golden ticket
\item
  Calculate the appropriate margin of error and confidence interval using theory-based methods.
\item
  Interpret the confidence interval in context of the study.
\item
  Based on the interval, what decision can you make about the null hypothesis? Does the confidence interval agree with the results of the hypothesis test? Justify your answer.
\item
  Interpret the confidence level in context of the study. What does ``confidence'' mean?
\item
  Describe which features of the study have an effect on the width of the confidence interval and how.
\end{enumerate}

\newpage

\chapter*{References}\label{references}
\addcontentsline{toc}{chapter}{References}

\phantomsection\label{refs}
\begin{CSLReferences}{1}{0}
\bibitem[\citeproctext]{ref-pga}
{``Average Driving Distance and Fairway Accuracy.''} 2008. \href{https://www.pga.com/\%20and\%20https://www.lpga.com/}{https://www.pga.com/ and https://www.lpga.com/}.

\bibitem[\citeproctext]{ref-banton2022}
Banton, et al, S. 2022. {``Jog with Your Dog: Dog Owner Exercise Routines Predict Dog Exercise Routines and Perception of Ideal Body Weight.''} \emph{PLoS ONE} 17(8).

\bibitem[\citeproctext]{ref-bhavsar2022}
Bhavsar, et al, A. 2022. {``Increased Risk of Herpes Zoster in Adults ≥50 Years Old Diagnosed with COVID-19 in the United States.''} \emph{Open Forum Infectious Diseases} 9(5).

\bibitem[\citeproctext]{ref-islands}
Bulmer, M. n.d. {``Islands in Schools Project.''} \url{https://sites.google.com/site/islandsinschoolsprojectwebsite/home}.

\bibitem[\citeproctext]{ref-bts}
{``Bureau of Transportation Statistics.''} 2019. \url{https://www.bts.gov/}.

\bibitem[\citeproctext]{ref-babies}
{``Child Health and Development Studies.''} n.d. \url{https://www.chdstudies.org/}.

\bibitem[\citeproctext]{ref-darley1973}
Darley, J. M., and C. D. Batson. 1973. {``"From Jerusalem to Jericho": A Study of Situational and Dispositional Variables in Helping Behavior.''} \emph{Journal of Personality and Social Psychology} 27: 100--108.

\bibitem[\citeproctext]{ref-davis2020}
Davis, Smith, A. K. 2020. {``A Poor Substitute for the Real Thing: Captive-Reared Monarch Butterflies Are Weaker, Paler and Have Less Elongated Wings Than Wild Migrants.''} \emph{Biology Letters} 16.

\bibitem[\citeproctext]{ref-doit2015}
Du Toit, et al, G. 2015. {``Randomized Trial of Peanut Consumption in Infants at Risk for Peanut Allergy.''} \emph{New England Journal of Medicine} 372.

\bibitem[\citeproctext]{ref-edmunds2016}
Edmunds, et al, D. 2016. {``Chronic Wasting Disease Drives Population Decline of White-Tailed Deer.''} \emph{PLoS ONE} 11(8).

\bibitem[\citeproctext]{ref-ipeds}
Education Statistics, National Center for. 2018. {``IPEDS.''} \url{https://nces.ed.gov/ipeds/}.

\bibitem[\citeproctext]{ref-gbmarried}
{``Great Britain Married Couples: Great Britain Office of Population Census and Surveys.''} n.d. \url{https://discovery.nationalarchives.gov.uk/details/r/C13351}.

\bibitem[\citeproctext]{ref-zeitler2012}
Group, TODAY Study. 2012. {``\href{https://www.ncbi.nlm.nih.gov/pubmed/22540912}{A Clinical Trial to Maintain Glycemic Control in Youth with Type 2 Diabetes}.''} \emph{New England Journal of Medicine} 366: 2247--56.

\bibitem[\citeproctext]{ref-hamblin2007}
Hamblin, J. K., K. Wynn, and P. Bloom. 2007. {``Social Evaluation by Preverbal Infants.''} \emph{Nature} 450 (6288): 557--59.

\bibitem[\citeproctext]{ref-hirschfelder2018}
Hirschfelder, A., and P. F. Molin. 2018. {``I Is for Ignoble: Stereotyping Native Americans.''} \href{Retrieved\%20from\%20https://www.ferris.edu/HTMLS/news/jimcrow/native/homepage.htm.}{Retrieved from https://www.ferris.edu/HTMLS/news/jimcrow/native/homepage.htm.}

\bibitem[\citeproctext]{ref-hutchison2013}
Hutchison, R. L., and M. A. Hirthler. 2013. {``\href{https://www.ncbi.nlm.nih.gov/pubmed/23932117}{Upper Extremity Injuies in Homer's Iliad}.''} \emph{Journal of Hand Surgery (American Volume)} 38: 1790--93.

\bibitem[\citeproctext]{ref-imdb}
{``{IMDb} Movies Extensive Dataset.''} 2016. \url{https://kaggle.com/stefanoleone992/imdb-extensive-dataset}.

\bibitem[\citeproctext]{ref-kalra2022}
Kalra, et al., Dl. 2022. {``Trustworthiness of Indian Youtubers.''} Kaggle. \url{https://doi.org/10.34740/KAGGLE/DSV/4426566}.

\bibitem[\citeproctext]{ref-keating2021}
Keating, D., N. Ahmed, F. Nirappil, Stanley-Becker I., and L. Bernstein. 2021. {``Coronavirus Infections Dropping Where People Are Vaccinated, Rising Where They Are Not, Post Analysis Finds.''} \emph{Washington Post}. \url{https://www.washingtonpost.com/health/2021/06/14/covid-cases-vaccination-rates/}.

\bibitem[\citeproctext]{ref-laeng2007}
Laeng, Mathisen, B. 2007. {``Why Do Blue-Eyed Men Prefer Women with the Same Eye Color?''} \emph{Behavioral Ecology and Sociobiology} 61(3).

\bibitem[\citeproctext]{ref-levin2000}
Levin, D. T. 2000. {``Race as a Visual Feature: Using Visual Search and Perceptual Discrimination Tasks to Understand Face Categories and the Cross-Race Recognition Deficit.''} \emph{Journal of Experimental Psychology} 129(4).

\bibitem[\citeproctext]{ref-madden2020}
Madden, et al, J. 2020. {``Ready Student One: Exploring the Predictors of Student Learning in Virtual Reality.''} \emph{PLoS ONE} 15(3).

\bibitem[\citeproctext]{ref-miller1956}
Miller, G. A. 1956. {``The Magical Number Seven, Plus or Minus Two: Some Limits on Our Capacity for Processing Information.''} \emph{Psychological Review} 63(2).

\bibitem[\citeproctext]{ref-becentispeech}
Moquin, W., and C. Van Doren. 1973. {``Great Documents in American Indian History.''} Praeger.

\bibitem[\citeproctext]{ref-pew2022}
{``More Americans Are Joining the 'Cashless' Economy.''} 2022. \url{https://www.pewresearch.org/short-reads/2022/10/05/more-americans-are-joining-the-cashless-economy/.}

\bibitem[\citeproctext]{ref-weather}
National Weather Service Corporate Image Web Team. n.d. {``National Weather Service -- {NWS} Billings.''} \url{https://w2.weather.gov/climate/xmacis.php?wfo=byz}.

\bibitem[\citeproctext]{ref-obrien2019}
O'Brien, Lynch, H. D. 2019. {``Crocodylian Head Width Allometry and Phylogenetic Prediction of Body Size in Extinct Crocodyliforms.''} \emph{Integrative Organismal Biology} 1.

\bibitem[\citeproctext]{ref-ocean}
{``Ocean Temperature and Salinity Study.''} n.d. \url{https://calcofi.org/}.

\bibitem[\citeproctext]{ref-WashPost2022}
{``Older People Who Get Covid Are at Increased Risk of Getting Shingles.''} 2022. \url{https://www.washingtonpost.com/health/2022/04/19/shingles-and-covid-over-50/.}

\bibitem[\citeproctext]{ref-physhealth}
{``Physician's Health Study.''} n.d. \url{https://phs.bwh.harvard.edu/}.

\bibitem[\citeproctext]{ref-porath2017}
Porath, Erez, C. 2017. {``Does Rudeness Really Matter? The Effects of Rudeness on Task Performance and Helpfulness.''} \emph{Academy of Management Journal} 50.

\bibitem[\citeproctext]{ref-quinn1999}
Quinn, G. E., C. H. Shin, M. G. Maguire, and R. A. Stone. 1999. {``Myopia and Ambient Lighting at Night.''} \emph{Nature} 399 (6732): 113--14. \url{https://doi.org/10.1038/20094}.

\bibitem[\citeproctext]{ref-ramachandran2007}
Ramachandran, V. 2007. {``3 Clues to Understanding Your Brain.''} \url{https://www.ted.com/talks/vs_ramachandran_3_clues_to_understanding_your_brain}.

\bibitem[\citeproctext]{ref-cdchospitalization}
{``Rates of Laboratory-Confimed COVID-19 Hospitalizations by Vaccination Status.''} 2021. CDC. \url{https://covid.cdc.gov/covid-data-tracker/\#covidnet-hospitalizations-vaccination}.

\bibitem[\citeproctext]{ref-richardson2019}
Richardson, T., and R. T. Gilman. 2019. {``Left-Handedness Is Associated with Greater Fighting Success in Humans.''} \emph{Scientific Reports} 9 (1): 15402. \url{https://doi.org/10.1038/s41598-019-51975-3}.

\bibitem[\citeproctext]{ref-stephens2020}
Stephens, R., and O. Robertson. 2020. {``Swearing as a Response to Pain: Assessing Hypoalgesic Effects of Novel "Swear" Words.''} \emph{Frontiers in Psychology} 11: 643--62.

\bibitem[\citeproctext]{ref-stewart2014}
Stewart, E. H., B. Davis, B. L. Clemans-Taylor, B. Littenberg, C. A. Estrada, and R. M. Centor. 2014. {``Rapid Antigen Group a Streptococcus Test to Diagnose Pharyngitis: A Systematic Review and Meta-Analysis''} 9 (11). \url{https://doi.org/10.1371/journal.pone.0111727}.

\bibitem[\citeproctext]{ref-stroop1935}
Stroop, J. R. 1935. {``Studies of Interference in Serial Verbal Reactions.''} \emph{Journal of Experimental Psychology} 18: 643--62.

\bibitem[\citeproctext]{ref-subach2022}
Subach, et al, A. 2022. {``Foraging Behaviour, Habitat Use and Population Size of the Desert Horned Viper in the Negev Desert.''} \emph{Soc.Open Sci} 9.

\bibitem[\citeproctext]{ref-sulheim2017}
Sulheim, S., A. Ekeland, I. Holme, and R. Bahr. 2017. {``Helmet Use and Risk of Head Injuries in Alpine Skiers and Snowboarders: Changes After an Interval of One Decade''} 51 (1): 44--50. \url{https://doi.org/10.1136/bjsports-2015-095798}.

\bibitem[\citeproctext]{ref-titanic}
{``Titanic.''} n.d. \url{http://www.encyclopedia-titanica.org}.

\bibitem[\citeproctext]{ref-covidvaccinetracker}
{``US COVID-19 Vaccine Tracker: See Your State's Progress.''} 2021. Mayo Clinic. \url{https://www.mayoclinic.org/coronavirus-covid-19/vaccine-tracker}.

\bibitem[\citeproctext]{ref-usepa2020}
US Environmental Protection Agency. n.d. {``Air Data -- Daily Air Quality Tracker.''} \url{https://www.epa.gov/outdoor-air-quality-data/air-data-daily-air-quality-tracker}.

\bibitem[\citeproctext]{ref-wahlstrom2014}
Wahlstrom, et al, K. 2014. {``Examining the Impact of Later School Start Times on the Health and Academic Performance of High School Students: A Multi-Site Study.''} \emph{Center for Applied Research and Educational Improvement}.

\bibitem[\citeproctext]{ref-Weiss1988}
Weiss, R. D. 1988. {``Relapse to Cocaine Abuse After Initiating Desipramine Treatment.''} \emph{JAMA} 260(17).

\bibitem[\citeproctext]{ref-navajo2011}
{``Welcome to the Navajo Nation Government: Official Site of the Navajo Nation.''} 2011.\href{\%20Retrieved\%20from\%20https://www.navajo-nsn.gov/.}{Retrieved from https://www.navajo-nsn.gov/.}

\bibitem[\citeproctext]{ref-wilson2016}
Wilson, Woodruff, J. P. 2016. {``Vertebral Adaptations to Large Body Size in Theropod Dinosaurs.''} \emph{PLoS ONE} 11(7).

\end{CSLReferences}

\end{document}
